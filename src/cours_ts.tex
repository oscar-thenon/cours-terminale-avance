\documentclass{report}

% --- LES PAQUETS

\usepackage[utf8]{inputenc} % Français
\usepackage[T1]{fontenc} % Français

\usepackage{amsmath} % Formules mathématiques, ex: DeclareMathOperator
\usepackage{amssymb} % Symboles spéciaux, ex: square
\usepackage{amsthm} % Théorèmes, ex: theoremstyle
\usepackage{enumerate} % Options sur les listes, ex: [(i)]
\usepackage{textcomp} % Symboles texte pour gensymb
\usepackage{gensymb} % Quelques unités, ex: degree
\usepackage{graphicx} % Insère des graphiques
\usepackage{hyperref} % Permet les hyperliens
\usepackage[left=2.25cm,right=2.25cm,top=2.25cm,bottom=2.25cm]{geometry} % Mise en page
\usepackage{xcolor} % Gère les couleurs, ex: fcolorbox
\usepackage{diagbox} % Perlet de faire des diagonales dans les tableaux
\usepackage{mathrsfs} % Police lettres attachées
\usepackage{pifont} % check et croix
\usepackage{tcolorbox} % Encadrage arrondi et couleurs

% --- LES THÉORÈMES

\theoremstyle{definition}
\newtheorem{definition}{Définition}[section] % Définition
\newtheorem{exemple}{Exemple}[section] % Exemple
\newtheorem{exercice}{Exercice}[section] % Exercice
\newtheorem{lem}{Lemme}[section] % Exercice
\newtheorem{notation}{Notation}[section] % Théorème 
\newtheorem{prop}{Proposition}[section] % Proposition
\newtheorem{coro}{Corollaire}[section] % Corollaire
\newtheorem{remarque}{Remarque}[section] % Remarque
\newtheorem{theorem}{Théorème}[section] % Théorème 

% --- NOUVELLES COMMANDES

% --- ENVIRONNEMENTS

% Majeurs

\newcommand{\defi}[1]{\begin{tcolorbox}[boxrule=1.85pt,arc=1ex, colback=cbdefi, colframe=cfdefi, left=3pt, right=3pt, top=3pt, bottom=2pt]\begin{definition}#1\end{definition}\end{tcolorbox}} % Définitions

\newcommand{\nota}[1]{\begin{tcolorbox}[boxrule=1.85pt,arc=1ex, colback=cbnota, colframe=cfnota, left=3pt, right=3pt, top=3pt, bottom=2pt]\begin{notation}#1\end{notation}\end{tcolorbox}} % Notations

\newcommand{\pro}[1]{\begin{tcolorbox}[boxrule=1.85pt,arc=1ex, colback=cbpro, colframe=cfpro, left=3pt, right=3pt, top=3pt, bottom=2pt]\begin{prop}#1\end{prop}\end{tcolorbox}} % Propositions

\newcommand{\lemme}[1]{\begin{tcolorbox}[boxrule=1.85pt,arc=1ex, colback=cblemme, colframe=cflemme, left=3pt, right=3pt, top=3pt, bottom=2pt]\begin{lem}#1\end{lem}\end{tcolorbox}} % Lemmes

\newcommand{\theo}[1]{\begin{tcolorbox}[boxrule=1.85pt,arc=1ex, colback=cbtheo, colframe=cftheo, left=3pt, right=3pt, top=3pt, bottom=2pt]\begin{theorem}#1\end{theorem}\end{tcolorbox}} % Théorèmes

\newcommand{\cor}[1]{\begin{tcolorbox}[boxrule=1.85pt,arc=1ex, colback=cbcor, colframe=cfcor, left=3pt, right=3pt, top=3pt, bottom=2pt]\begin{coro}#1\end{coro}\end{tcolorbox}} % Corollaires

% Mineurs

\newcommand{\ex}[1]{\begin{exemple}#1\end{exemple}} % Exemple

\newcommand{\exo}[1]{\begin{exercice}#1\end{exercice}} % Exercice

\newcommand{\preuve}[1]{\begin{proof}#1\end{proof}} % Preuve

\newcommand{\rem}[1]{\begin{remarque}#1\end{remarque}} % Remarque

\newcommand{\rems}[1]{\begin{remarque}\begin{itemize}#1\end{itemize}\end{remarque}} % Remarques multiples

% Notation mathématiques

\newcommand{\C}{\mathbf{C}} % Ensemble des complexes
\newcommand{\chaine}[1]{\[
\begin{array}{r c l}
#1
\end{array}
\]}
\newcommand{\cjg}[1]{\overline{#1}}
\newcommand{\cmark}{\textcolor{brown}{\ding{51}}} % check
\newcommand{\contient}{\supseteq} % Contenance
\newcommand{\doubleimp}[2]{\begin{itemize}\item[($\Rightarrow$)] #1 \item[($\Leftarrow$)] #2 \end{itemize}} % Double implication
\newcommand{\doubleimpi}[2]{\begin{itemize}\item[($\Leftarrow$)] #1 \item[($\Rightarrow$)] #2 \end{itemize}} % Double implication inverse
\newcommand{\dt}{\mathscr{D}}
\newcommand{\eqv}{\Longleftrightarrow}
\newcommand{\estdans}{\subseteq} % Appartenance
\newcommand{\fons}[5]{$\begin{array}{lrcl}
#1 : & #2 & \longrightarrow & #3 \\
    & #4 & \longmapsto & #5 \end{array}$} % Fonction standard
    \newcommand{\fonsn}[4]{$\begin{array}{lrcl}
& #1 & \longrightarrow & #2 \\
    & #3 & \longmapsto & #4 \end{array}$} % Fonction standard sans nom
\newcommand{\fone}[3]{$#1: #2\to #3$} % Fonction ensembliste
\newcommand{\fonen}[2]{$#1\longrightarrow #2$} % Fonction ensembliste sans nom
\newcommand{\fonl}[5]{$#1: #2\to #3,#4\mapsto #5$} % Fonction ligne
\newcommand{\imp}{\Longrightarrow} % Implication
\newcommand{\Lim}{\lim\limits_} % Limite
\newcommand{\limc}[2]{\lim\limits_{#1\to #2}} % Limite
\newcommand{\la}{\left|} % Valeur absolue gauche
\newcommand{\lc}{\left[} % Crochet gauche
\newcommand{\lint}{[\![} % [[
\newcommand{\lp}{\left(} % Parenthèse gauche
\newcommand{\N}{\mathbf{N}} % Ensemble des naturels
\newcommand{\norm}[1]{\left\parallel #1 \right\parallel}
\newcommand{\pc}{\mathscr{P}} % Plan complexe
\newcommand{\plan}{\mathscr{P}} % Plan
\newcommand{\poly}{\mathbf{C}[z]} % Ensemble des polynômes
\newcommand{\polyd}[1]{\mathbf{C}_{#1}[z]} % Ensemble des polynômes
\newcommand{\produit}[2]{\prod\limits_{#1}^{#2}} % Somme
\newcommand{\programme}{\textcolor{blue}{$\circledast$ }} % Au programme
\newcommand{\Q}{\mathbf{Q}} % Ensemble des rationnels
\newcommand{\R}{\mathbf{R}} % Ensemble des réels
\newcommand{\Rp}{\mathbf{R}} % Ensemble des réels
\newcommand{\Rm}{\mathbf{R_{-}}} 
\newcommand{\Ret}{\mathbf{R^*}}
\newcommand{\Rep}{\mathbf{R^*_{+}}} 
\newcommand{\Rem}{\mathbf{R^*_{-}}} 

\newcommand{\ra}{\right|} % Valeur absolue droite
\newcommand{\rc}{\right]} % Crochet droit
\newcommand{\rep}{\mathscr{R}} % repère
\newcommand{\rint}{]\! ]} % [[
\newcommand{\rp}{\right)} % Parenthèse droite
\newcommand{\sph}{\mathscr{S}}
\newcommand{\sauf}[1]{\backslash\left\{#1\right\}}
\newcommand{\somme}[2]{\sum\limits_{#1}^{#2}} % Somme
\newcommand{\suite}[1]{$(#1_n)_{n\in\N}$} % Définition suite
\newcommand{\suiten}[1]{$(#1_n)_{n\in\N^*}$} % Définition suite commençant à 0
\newcommand{\suiteq}[2]{$(#1_n)_{n\in#2}$} % Définition suite commençant n'importe où.
\newcommand{\sys}[1]{\[
\left\{
\begin{array}{r c l}
#1
\end{array}
\right.
\]}
\newcommand{\U}{\mathbf{U}} % Ensemble des naturels
\newcommand{\vr}[1]{\overrightarrow{#1}}
\newcommand{\vre}[3]{\parent{\begin{matrix} #1 \\ #2 \\ #3 \end{matrix}}} % vecteur c
\newcommand{\xmark}{\textcolor{red}{\ding{55}}} % croix
\newcommand{\Z}{\mathbf{Z}} % Ensemble des entiers

\newcommand{\intent}[1]{[\![#1]\!]}

% Délimitations

\newcommand{\abs}[1]{\left|#1\right|} % ||
\newcommand{\cro}[1]{\left[#1\right]} % []
\newcommand{\ens}[1]{\left\{#1\right\}} % {}
\newcommand{\parent}[1]{\left(#1\right)} % ()
\newcommand{\spoiler}[1]{$\blacktriangleright$\textcolor{white}{ #1 }$\blacktriangleleft$}

% Expressions

\newcommand{\cad}{c'est-à-dire } % C'est-à-dire
\newcommand{\ie}{\emph{i.e} } % id est
\newcommand{\iex}{il existe } % il existe
\newcommand{\pt}{pour tout } % C'est-à-dire
\newcommand{\ssi}{si et seulement si } % SSi
\newcommand{\tq}{tel que } % SSi
\newcommand{\gug}{ « } % Guillement gauche
\newcommand{\gud}{ » } % Guillement droit

% --- RENOUVELLEMENT DE COMMANDES

\renewcommand{\abstractname}{Avant-propos}
\renewcommand{\appendixname}{Annexe}
\renewcommand{\subset}{\subseteq}
\renewcommand{\chaptername}{Chapitre}
\renewcommand{\contentsname}{Sommaire}
\renewcommand{\partname}{Partie}
\renewcommand*{\proofname}{Preuve}
\renewcommand{\epsilon}{\varepsilon}
\renewcommand{\phi}{\varphi}
\newcommand{\intg}[3]{\int_{#1}^{#2} #3 \, \mathrm{d}x}
\newcommand{\intgv}[4]{\int_{#1}^{#2} #3 \, \mathrm{d}#4}

% --- LES DÉFINITIONS

\DeclareMathOperator{\card}{card}
\DeclareMathOperator{\dis}{d}
\DeclareMathOperator{\I}{I}
\DeclareMathOperator{\id}{Id}
\DeclareMathOperator{\im}{Im}
\DeclareMathOperator{\re}{Re}

% Couleurs

\definecolor{cbdefi}{RGB}{248,226,255} % Définition fond
\definecolor{cfdefi}{RGB}{169,160,172} % Définition bordure

\definecolor{cbnota}{RGB}{224,220,255} % Notation fond
\definecolor{cfnota}{RGB}{146,145,182} % Notation bordure

\definecolor{cbpro}{RGB}{235,235,235} % Proposition fond
\definecolor{cfpro}{RGB}{157,157,157} % Proposition bordure

\definecolor{cblemme}{RGB}{255,237,188} % Lemme fond
\definecolor{cflemme}{RGB}{182,149,116} % Lemme bordure

\definecolor{cbtheo}{RGB}{255,210,210} % Théorème fond
\definecolor{cftheo}{RGB}{157,129,129} % Théorème bordure

\definecolor{cbcor}{RGB}{255,210,180} % Corollaire fond
\definecolor{cfcor}{RGB}{157,139,127} % Corollaire bordure

% --- SIMPLIFICATIONS LATEX

\newcommand{\tend}{\\ \hline}

% --- EN-TÊTE

\title{Cours de Mathématiques \\ Terminale Scientifique}
\author{Oscar \textsc{Thenon}}
\date{Dernière mise à jour : août 2019}

\begin{document}
\begin{large}
\maketitle

\begin{abstract}
Ce cours couvre une partie du programme de l'enseignement spécifique de la classe de mathématiques de Terminale Scientifique, tel que cadré par le \href{https://www.education.gouv.fr/pid285/bulletin_officiel.html?pid_bo=25847}{bulletin officiel spécial numéro 8 du 13 octobre 2011}. Ce programme n'est actuellement plus à jour, il a été modifié conformément au \href{https://www.education.gouv.fr/pid285/bulletin_officiel.html?pid_bo=39051}{bulletin officiel spécial numéro 8 du 25 juillet 2019}. Les parties traitées ici sont les suites réelles, les nombres complexes, la géométrie dans l'espace et enfin les fonctions.

Ce cours est destiné au premier chef aux élèves souhaitant poursuivre des études de mathématiques dans le supérieur. Il y a en effet de très nombreuses preuves et théories hors-programme dont la complexité - toute relative - peut rebuter les élèves ne souhaitant que se cantonner au programme officiel. Toutefois cette complexité n'est pas poussée à l'extrême, il ne s'agit pas d'une préparation aux CPGE, ce cours a simplement l'ambition d'un cours de terminale très approfondi, ce qui nous semble profitable pour le public visé.

Les éléments explicitement au programme officiel sont clairement signalés par le sigle \programme. J'espère que le lecteur prendra autant de plaisir à feuilleter ce cours que j'en ai eu à le rédiger.
\end{abstract}

\tableofcontents


\chapter{Les fondamentaux}

\section{La démonstration}

La démonstration consiste à \textit{prouver une proposition}, c'est-à-dire un énoncé qui a pour valeur de vérité soit \textit{vrai} soit \textit{faux}, à l'aide d'une suite de raisonnements logiques. En mathématiques, il y a beaucoup de manières d'établir des raisonnements, chacune étant plus adaptée à certaines situations. Nous allons dans cette section décrire quelques raisonnements très classiques.

A chaque fois que le lecteur rédige ou lit une démonstration, nous l'encourageons à détecter le raisonnement précis sur lequel se base cette démonstration : cela aide à la compréhension et donc à l'apprentissage.

Bien évidemment, les raisonnements qui suivent sont loin d'être exhaustifs : souvent une démonstration utilise un mixte de tous ceux-là. C'est pourquoi il est utile d'avoir un répertoire le plus large possible, d'autant que plus le niveau avance, moins les démonstrations diront explicitement le raisonnement qu'elles utilisent.

\subsection{Raisonnements sur les réels et les fonctions réelles}

\subsubsection{\textcolor{brown}{Prouver une égalité}}

Il s'agit de prouver une proposition de type $a=b$ où $a,b\in\R$.

\begin{enumerate}
\item \textbf{Chaîne d'égalités}. On prouve que $a=r_1$ puis que $r_1=r_2$ et ainsi de suite jusqu'à trouver un $n$ tel que $r_n=b$.

\item \textbf{Double inégalité}. On prouve $a\le b$ puis $b\le a$. C'est une méthode à laquelle on ne pense pas souvent, elle peut pourtant parfois totalement débloquer le problème. En général, un sens est immédiat et l'autre est plus difficile.

\item \textbf{Par retranchement}. On prouve que $a-b=0$.

\item \textbf{Par quotient}. On prouve que $\dfrac{a}{b}=1$.

\item \textbf{Par l'absurde}. On suppose que $a\neq b$ et on en déduit une absurdité.
\end{enumerate}

\subsubsection{\textcolor{brown}{Prouver qu'une fonction est constante}}

On veut prouver que \fone{f}{I}{\R} est constante sur $I$ avec $I$ un intervalle de $\R$.

\begin{enumerate}
\item \textbf{Par dérivation}. Montrer que $f$ est dérivable sur $I$ et constater que pour tout $x\in I$, $f'(x)=0$.

\item \textbf{Par primitive}. Montrer que $f$ admet une primitive $F$ sur $I$ et constater que $F$ est linéaire sur $I$.
\end{enumerate}

\subsubsection{\textcolor{brown}{Prouver une propriété sur les entiers}}

Soit $P(n)$ un prédicat sur les entiers naturels. On souhaite prouver une propriété sur $P(n)$. \textit{Note : le raisonnement par récurrence fera l'objet de toute une section ultérieure.}

\begin{enumerate}
\item \textbf{Par récurrence simple}. On prouve que $P(0)$ est vraie : c'est l'\textit{initialisation}. On suppose qu'il existe $n\in \N$ tel que $P(n)$ est vraie et on en déduit que $P(n+1)$ est vraie : c'est l'\textit{hérédité}.
\item \textbf{Par récurrence double}. On prouve que $P(0)$ et $P(1)$ sont vraies : c'est l'\textit{initialisation}. On suppose qu'il existe $n\in \N$ tel que $P(n-1)$  et $P(n)$ sont vraies et on en déduit que $P(n+1)$ est vraie : c'est l'\textit{hérédité}. On peut de la même manière faire une récurrence triple, quadruple etc. si besoin.
\item \textbf{Par récurrence forte}. On prouve que $P(0)$ est vraie : c'est l'\textit{initialisation}. On suppose qu'il existe $n\in \N$ tel que pour tout entier $0\le m\le n$, $P(m)$ est vraie et on en déduit que $P(n+1)$ est vraie : c'est l'\textit{hérédité forte}. Ce type de raisonnement est particulièrement adapté lorsque $P(n)$ est une somme notamment.
\end{enumerate}

\subsection{Raisonnements de type implication}

\subsubsection{\textcolor{brown}{Prouver une implication}}

Il s'agit de prouver une proposition de type «~$A\Rightarrow B$~».

\begin{enumerate}
\item \textbf{Prouver que $A$ est faux}. En effet, le faux impliquant n'importe quoi, ce type de proposition sera toujours vraie si $A$ est faux. En général, c'est très rare qu'on ait affaire à ce cas de figure : $A$ est en général vraie ou alors vraie dans certains cas (prédicats).

\item \textbf{Démonstration directe}. On suppose que $A$ est vraie, et on en déduit que $B$ est vraie aussi.

\item \textbf{Par l'absurde}. On suppose que $A$ est vraie. On suppose que $B$ est faux, et on trouve une absurdité (par exemple $1=0$).

\item \textbf{Par contraposée}. On prouve la proposition $\neg B \Rightarrow \neg A $ en utilisant l'un des 3 raisonnements précédent.
\end{enumerate}

\subsubsection{\textcolor{brown}{Prouver qu'une implication est fausse}}

Il s'agit de prouver une proposition de type «$\neg(A\Rightarrow B)$~».

\begin{enumerate}
\item \textbf{Démonstration directe}. On suppose que $A$ est vraie, et on en déduit que $B$ est faux.
\item \textbf{Par l'absurde}. On suppose que $A$ est vraie. On suppose que $B$ est vraie, et on trouve une absurdité (par exemple $1=0$).
\item \textbf{Par contraposée}. On prouve la proposition $\neg(\neg B \Rightarrow \neg A )$ en utilisant l'un des 2 raisonnements précédent.
\end{enumerate}

\subsubsection{\textcolor{brown}{Prouver une équivalence}}

Il s'agit de prouver une proposition de type «$P=A\Longleftrightarrow B$~».

\begin{enumerate}
\item \textbf{Démonstration directe}. On suppose que $P$ est vraie, et par une chaîne d'équivalences, on en déduit que c'est équivalent à une autre équivalence $P'$ dont on sait qu'elle est vraie.

\item \textbf{Double implication}. On prouve $A\Rightarrow B$ et $B\Rightarrow A$ en utilisant l'un des raisonnements précédent. En général, un sens est immédiat et l'autre est plus difficile.
\end{enumerate}

\subsubsection{\textcolor{brown}{Prouver une chaîne d'équivalences}}

Soit $P_1,...,P_n$ des propositions. Il s'agit de prouver une proposition de type «$P_1\Longleftrightarrow...\Longleftrightarrow P_n$~».

\begin{enumerate}
\item \textbf{Démonstration directe}. On prouve successivement $P_1\Longleftrightarrow P_2$ puis $P_2\Longleftrightarrow P_3$ et ainsi de suite jusqu'à $P_{n-1}\Longleftrightarrow P_n$ (peu importe l'ordre).

\item \textbf{Chaîne d'implications}. On prouve $P_1\Rightarrow P_2$ puis $P_2\Rightarrow P_3$ et ainsi de suite jusqu'à $P_n\Rightarrow P_1$ (il est important de fermer la boucle). L'ordre a ici une importance.
\end{enumerate}

\subsection{Raisonnements sur les ensembles}

\subsubsection{\textcolor{brown}{Prouver une appartenance}}

Soit $E$ un ensemble. On veut prouver qu'un certain $x\in E$.

\begin{enumerate}
\item \textbf{Par caractérisation}. Si $E$ est défini par une certaine caractérisation (par exemple, une propriété ou une équation), montrer que $x$ respecte cette caractérisation.

\item \textbf{Par sous-ensemble}. Si on sait que $F\subset E$, montrer que $x\in F$.

\item \textbf{Par l'absurde}. Supposer que $x\not\in F$ et en déduire une absurdité.
\end{enumerate}

\subsubsection{\textcolor{brown}{Prouver une inclusion}}

Soit $E,F$ deux ensembles. On veut prouver $E\subset F$.

\begin{enumerate}
\item \textbf{Par élément}. Supposer que $x\in E$ puis montrer que $x\in F$.

\item \textbf{Par chaîne d'inclusions}. Prouver que $E\subset E_1$ puis que $E_1\subset E_2$ et ainsi de suite jusqu'à montrer qu'il existe $n$ tel que $E_n\subset F$.
\end{enumerate}

\subsubsection{\textcolor{brown}{Prouver une égalité d'ensembles}}

Soit $E,F$ deux ensembles. On veut prouver $E= F$.

\begin{enumerate}
\item \textbf{Par double inclusion}. Montrer que $E\subset F$ puis que $F\subset E$. En général, un sens est immédiat et l'autre est plus difficile.

\item \textbf{Par chaîne d'égalités}. Prouver que $E= E_1$ puis que $E_1=E_2$ et ainsi de suite jusqu'à montrer qu'il existe $n$ tel que $E_n= F$.
\end{enumerate}

\subsubsection{\textcolor{brown}{Prouver le cardinal d'un ensemble}}

Soit $E$ un ensemble. On souhaite prouver que $E$ a pour cardinal $n\in N$.

\begin{enumerate}
\item \textbf{Par comptage direct}. On énumère un à un tous les éléments de $E$ et on constate qu'il y en a $n$. Bien entendu, dans le majorité des cas cela n'est pas convenable.

\item \textbf{Par bijection}. On prouve l'existence de n'importe quelle bijection \fone{f}{E}{\lint 1,n\rint} ou \fone{f}{\lint 1,n\rint} {E}. C'est la méthode usuelle. Dans le cas où on veut prouver que le cardinal d'un ensemble est infini, il suffit de trouver n'importe quelle bijection \fone{f}{E}{\N}.
\end{enumerate}




\subsection{Raisonnements existence-unicité}

\subsubsection{\textcolor{brown}{Prouver une existence}}

On souhaite prouver l'existence d'un objet répondant à un critère $C$.

\begin{enumerate}
\item \textbf{Par vérification directe}. On conjecture que $X$ est candidat, et on vérifie qu'il vérifie $C$.

\item \textbf{Par analyse-synthèse}. On suppose l'existence d'un $Y$ vérifiant $C$. On en déduit des propriétés sur $Y$, et on finit par isoler un objet $X_1$ qu'on suppose être candidat : c'est l'\textit{analyse}. Et on vérifie que $X_1$ convient en utilisant le point précédent : c'est la \textit{synthèse}. Si $X_1$ ne marche pas, on continue à trouver de nouvelles propriétés sur $Y$ jusqu'à trouver un $X_2$ qui semble convenir, et ainsi de suite. 
\end{enumerate}

\subsubsection{\textcolor{brown}{Prouver une unicité}}

On sait qu'il existe un objet $X$ répondant à un critère $C$ et on souhaite prouver qu'il est unique.

\begin{enumerate}
\item \textbf{Par identification}. On suppose l'existence d'un $Y$ répondant au critère $C$ et on prouve qu'alors $Y=X$. Dans le cas particulier où $X$ est une fonction réelle, on pourra poser $f=Y-X$ et prouver que $f$ est constante à 0 en utilisant la technique dédiée.
\item \textbf{Par l'absurde}. On suppose l'existence d'un $Y\neq X$ répondant au critère $C$ et on en déduit une absurdité.
\end{enumerate}



\section{Raisonnement par récurrence}

\subsection{Définitions et théorème}
Le raisonnement par récurrence est un raisonnement qui permet de démontrer des propositions dépendant d'entiers naturels. Extrêmement efficace dans beaucoup de cas, il faut toujours y penser quand on doit démontrer ce genre de choses. Cependant il faut \textit{toujours} chercher avant s'il n'y a pas une manière plus simple pour s'y prendre (inutile par exemple d'utiliser un raisonnement par récurrence pour montrer que $n^2-6n+9$ est positif pour tout $n\in\N$).

Commençons par deux définitions afin de poser les bases.

\defi{(proposition) une \textit{proposition} est un énoncé qui peut prendre l'une des deux valeurs de vérité : \textit{vrai} ou \textit{faux}. Lorsque une proposition dépend de variables et que selon leur valeur, sa valeur de vérité change, on l'appelle \textit{prédicat}.}

\rem{On rencontre aussi le terme \textit{assertion} qui est synonyme de \textit{proposition}.}

\ex{\begin{itemize}
\item "2<8" est une proposition, vraie.
\item "51 est premier" est une proposition, fausse.
\item "$n^2-6n+9> 0$" est un prédicat dont la variable est ici $n$. Si $n\in\N\neq 3$ elle est vraie, si $n=3$ elle est fausse.
\end{itemize}}

\defi{\programme (hérédité) Soit $P(n)$ un prédicat. 

On dit que $P(n)$ est \textit{héréditaire} sur $\N$ si pour tout $n\in\N$, $P(n)\Longrightarrow P(n+1)$.}

\exo{\begin{itemize}
\item Soit $P(n)$ le prédicat "$\sum\limits_{k=0}^n k = n^2$". $P$ est-il héréditaire ?
\item Trouver un prédicat non héréditaire.
\end{itemize}}

\theo{\programme Soit $P(n)$ un prédicat. Si $P(0)$ est vraie et que $P(n)$ est héréditaire, alors pour tout $n\in\N$, $P(n)$ est vrai.}

\rem{\begin{itemize}
\item L'étape consistant à montrer que $P(0)$ est vrai s'appelle \textit{initialisation}.
\item Le théorème formel ne sera pas donné tel quel en cours. Notamment la notion de prédicat n'est pas du tout au programme.
\end{itemize}}

\preuve{Tout sous-ensemble de $\N$ non vide admet un minimum car minoré par $0$, donc tout sous-ensemble de $\N$ qui n'admet aucun minimum est l'ensemble vide. Soit $P(n)$ un prédicat héréditaire et tel que $P(0)$ est vrai. Soit $F=\{n\in\N|\urcorner P(n)\}$. Supposons que $F$ admet un minimum $m$. $m>0$ puisque $P(0)$ est vrai. $P(m)$ est donc faux et pour tout $n\in\lint 0; m\lint$, $P(n)$ est vrai. Donc $P(m-1)$ est vrai or par hérédité, $P(m)$ aussi ce qui est absurde. $F$ n'admet donc pas de minimum donc $F=\emptyset$.}

On peut prendre l'exemple des dominos pour illustrer le raisonnement par récurrence : supposons une infinité de dominos alignés numérotés en commençant à 0. Soit $P(n)$ le prédicat "le domino $n$ tombe". $P(0)$ est traduit par le fait que le domino 0 tombe (par exemple en le poussant manuellement) et $P(n)$ héréditaire par le fait que quand un domino tombe le suivant tombe aussi (par exemple parce qu'ils sont suffisament peu espacés). On voit bien intuitivement que pour être certain que les tous les dominos tombent il faut absolument que les deux conditions soient respectées.

\subsection{Utilisation}

Au lycée, les enseignants ne sont pas très regardant sur la rédaction des raisonnements par récurrence. Mais ce n'est pas le cas en prépa et il est extrêmement facile d'écrire une rédaction incorrecte, par conséquent autant prendre les bonnes habitudes dès la terminale. Voici un exemple de rédaction correcte d'un raisonnement par récurrence, il est fortement recommandé au lecteur de s'en servir comme modèle.

\ex{ Soit $n\in\N$. Montrons que $\sum\limits_{k=0}^n k = \dfrac{n(n+1)}{2}$.
Pour tout $n\in\N$, notons $P(n)$ la proposition « $\sum\limits_{k=0}^n k = \dfrac{n(n+1)}{2}$ » (on évitera d'utiliser le terme plus précis de "prédicat" au lycée pour éviter aux enseignants d'avoir à chercher ce que cela signifie). Montrons par récurrence que pour tout $n\in\N$, $P(n)$ est vrai.

\begin{description}
\item[Initialisation] Montrons que $P(0)$ est vraie (la formulation "montrons $P(0)$" est strictement équivalente mais on l'évitera au lycée). Pour $n=0$, $\sum\limits_{k=0}^n k=\sum\limits_{k=0}^0 k=0$ et $\dfrac{n(n+1)}{2}=\dfrac{0(0+1)}{2}=0$ or $0=0$ donc $P(0)$ est vraie.
\item[Hérédité] Montrons que $P(n)$ est héréditaire. Soit $n\in\N$ et supposons que $P(n)$ est vraie (on dit que $P(n)$ est l'\textit{hypothèse de récurrence}). Montrons que $P(n+1)$ est vraie, c'est-à-dire que $\sum\limits_{k=0}^{n+1} k = \dfrac{(n+1)(n+2)}{2}$.

$\sum\limits_{k=0}^{n+1} k=\sum\limits_{k=0}^{n} k + (n+1)$ donc par hypothèse de récurrence, $\sum\limits_{k=0}^{n+1} k = \dfrac{n(n+1)}{2} + (n+1)=(n+1)\lp\dfrac{n}{2}+1\rp=\dfrac{(n+1)(n+2)}{2}$.

On a donc $P(n+1)$ donc $P(n)$ est héréditaire.

\item[Conclusion] $P(0)$ est vraie et $P(n)$ est héréditaire donc pour tout $n\in\N$, $P(n)$ est vraie. Donc pour tout $n\in\N$, $\sum\limits_{k=0}^n k = \dfrac{n(n+1)}{2}$.
\end{description}} 

\rem{Cette rédaction peut sembler très lourde, mais elle a le mérite d'être adaptée au lycée et d'être parfaitement claire et rigoureuse. En prépa certains raccourcis pourront être pris (avec prudence toutefois). Nous conseillons au lecteur d'utiliser mot pour mot cette rédaction par défaut car elle est irréprochable, cependant nous rappelons que la rédaction qui prime est toujours celle de l'enseignant, s'il vous demande explicitement de rédiger d'une certaine façon, forcez-vous y (même si elle risque du coup d'être moins rigoureuse).}

\exo{\label{exrecu}\begin{itemize}
\item Soit $(U_n)_{n\in\N}$ la suite réelle définie par $U_0=\dfrac{\pi}{4}$ et telle que pour tout $n\in\N$, $U_{n+1}=6U_n-4$. Montrer que pour tout $n\in\N$, $U_n\le \dfrac{4}{5}$.
\item Soit $P(n)$ un prédicat. Nier "$P(n)$ est héréditaire". 
\item Soit $P(n)$ un prédicat tel que $P(0)$ est vrai et qui n'est pas héréditaire. Déterminer la valeur de vérité de la proposition "pour tout $n\in\N^*$, $P(n)$ est faux".
\end{itemize}}
\chapter{Suites réelles}

\section{Fondamentaux sur les suites}

\subsection{Qu'est-ce qu'une suite ?}

\defi{\programme On appelle \textit{suite de réels} toute famille d'éléments réels indicée par des entiers naturels. On appelle \emph{termes de la suite} les éléments de cette suite.}
\ex{La famille $(a_0,a_1,a_2)$ telle que $a_0=5$, $a_1=\pi$, $a_2=\sqrt{2}$ est une suite de réels, comportant donc 3 éléments. Elle est indicée par les entiers naturels 0, 1 et 2.}

Quelques conventions :
\begin{itemize}
\item Les indices sont généralement successifs.
\item Il est d'usage que le premier indice égale 0 (mais ce n'est pas une obligation).
\item La plupart des suites que nous allons étudier compteront un nombre infini d'éléments, si bien qu'en vertu des points précédents, la plupart de nos suites auront des indices parcourant tout $\N$, dans l'ordre.
\end{itemize}

\nota{\programme On note $(u_n)_{n\in\N}$ pour désigner la suite $(u_0,u_1,...)$.}
\rems{\item En adaptant cette notation, on peut désigner la suite $(w_1,w_2,...)$ par $(w_n)_{n\in\N\backslash\{0\}}$ si on ne veut pas commencer à 0 par exemple.
\item Chez les élèves les notations sont souvent mal employées~:
\begin{enumerate}
\item $(u_n)_{n\in\N}$ est la façon \emph{rigoureuse} de désigner la suite $(u_0,u_1,...)$.
\item $(u_n)$ est une \emph{notation raccourcie pour désigner la suite} $(u_n)_{n\in\N}$ qu'on peut utiliser lorsqu'il n'y aucune ambiguité (notamment par rapport aux indices) et \emph{seulement lorsqu'on a déjà définie la suite à l'aide de la notation précédente}, donc en particulier on a pas le droit d'écrire "soit $(u_n)$ la suite...".
\item Enfin, $u_n$ désigne le \emph{terme} de rang $n$ de la suite $(u_n)_{n\in\N}$, $u_n$ est donc une \textit{quantité réelle}, ce n'est pas une suite. Par conséquent écrire "la suite $u_n$ est croissante" n'a pas de sens.
\end{enumerate}
\item Attention : le $n$-ième terme d'une suite n'est pas la même chose que le terme d'indice $n$ de cette même suite. Par exemple le premier terme de la suite $(u_n)_{n\in\N}$ est $u_0$ et son terme d'indice 1 est $u_1$.}

Définir un à un les termes d'une suite pour la définir (comme on l'a fait pour le premier exemple) n'est pas pratique. Il existe essentiellement deux manières (ce ne sont pas les seules) de définir d'un coup tous les termes d'une suite : par \textit{récurrence} ou de manière \textit{explicite}.

\begin{description}
\item[Définition par récurrence] On donne la valeur du premier terme de la suite puis on donne de façon générique la valeur de $u_{n+1}$ en fonction de $u_n$, de sorte qu'on peut calculer tout terme de la suite grâce à son précédent. Voici un exemple de définition d'une suite par récurrence valide~:

"Soit $(v_n)_{n\in\N}$ la suite définie par $v_0=1$ et tel que pour tout $n\in\N$, $v_{n+1}=v_n+2$."

On peut par conséquent déterminer tous les termes de $(v_n)$ : $v_1=v_0+2=1+2=3$, $v_2=v_1+2=3+2=5$ et ainsi de suite (haha).

Si tout terme de la suite ne dépend \textbf{que} du terme précédent, on dit que la suite est \textit{récurrente d'ordre 1}. Si elle ne dépend \textbf{que} des $n\in\N$ termes précédents on dit qu'elle est \textit{récurrente d'ordre} $n$.


\item[Définition explicite] On donne une relation entre $u_n$ et son rang $n$ (et seulement son rang~$n$). Voici un exemple de définition explicite d'une suite valide :

"Soit $(w_n)_{n\in\N}$ la suite définie pour tout $n\in\N$ par $w_n=2n+1$".

On peut ainsi déterminer tous les termes de $(w_n)$ : $w_0=2\times0+1=1$, $w_1=2\times 1+1=3$ et ainsi de suite.
\end{description}

\rem{Il est aussi possible de définir une suite par récurrence en donnant une relation entre un terme et ses \textit{deux} précédents, dans ce cas il faut renseigner les \textit{deux} premiers termes de la suite (voir la suite de Fibonacci définie en-dessous).}

Il y a un inconvénient évident à la définition par récurrence : si $(u_n)_{n\in\N}$ est une suite définie par récurrence, pour déterminer $u_{100}$ par exemple il faut connaître impérativement $u_{99}$, donc $u_{98}$ et ainsi de suite jusqu'à redescendre à $u_0$ qu'on connaît. Cela nous oblige donc à établir 100 calculs pour déterminer $u_{100}$, contre 1 seul (!) si $(u_n)$ est définie de manière explicite.
Par conséquent, il faut toujours privilégier les définitions explicites si possible.

Parfois, il est facile de déterminer de manière explicite une suite définie par récurrence, par exemple la suite $(w_n)$ telle que définie précédemment n'est que la définition explicite de la suite $(v_n)$ définie auparavant. Mais parfois c'est très difficile, par exemple la suite de Fibonacci $(f_n)_{n\in\N}$ définie par $f_0=0$, $f_1=1$ et tel que pour tout $n\in\N$, $f_{n+2}=f_{n+1}+f_n$, dont on connait une définition explicite mais qui n'est pas du tout évidente à trouver (pour information elle utilise le nombre d'or).

Nous conseillons fortement au lecteur de prendre pour modèle les exemples donnés au-dessus pour définir ses suites afin d'éviter les maladresses.

\exo{Expliquer pourquoi les suites suivantes sont mal définies.
\begin{enumerate}
\item Soit $(v_n)$ la suite définie par $v_0=1$ et tel que pour tout $n\in\N$, $v_{n+1}=v_n+2$.
\item Soit $(a_n)_{n\in\N}$ la suite définie par $a_0=1$ et tel que $a_{n+1}=a_n+2$.
\item Soit $(b_n)_{n\in\N^*}$ la suite définie pour $n\in\N$ par $b_n=0$.
\item Soit $(d_n)_{n\in\N}$ la suite définie par $d_0=\pi$ et tel que pour tout $n\in\N$, $d_{n+2}=10 d_{n+1}-d_n$.
\item Soit $(f_n)_{n\in\N}$ la suite définie pour tout $n\in\N$ par $f_{n}=4+3f_{n-1}$
\end{enumerate}}

\subsection{Définitions}

Dans cette sous-section, \suite{a} est une suite réelle.

\defi{\programme (croissance, décroissance)
\begin{itemize}
\item Si pour tout $n\in\N$, $a_{n}\le a_{n+1}$ (resp. $a_{n}< a{n+1}$) alors on dit que $(a_n)$ est \textit{croissante} (sur $\N$) (resp. \textit{strictement croissante}).
\item Si pour tout $n\in\N$, $a_{n}\ge a_{n+1}$ (resp. $a_{n}> a_{n+1}$) alors on dit que $(a_n)$ est \textit{croissante} (sur $\N$) (resp. \textit{strictement croissante}).
\end{itemize}}

\defi{(croissance, décroissance sur un intervalle) Soit $n_0$ et $n_1$ deux entiers naturels tels que $n_0<n_1$.

\begin{itemize}
\item On dit que $(a_n)$ est \textit{croissante} (resp. \textit{strictement croissante}) \textit{sur} $\lint n_0;n_1 \rint$ si pour tout $n\in \lint n_0;n_1 \lint$, $a_n\le u_{n+1}$ (resp. $a_n< u_{n+1}$).
\item On dit que $(a_n)$ est \textit{décroissante} (resp. \textit{strictement décroissante}) \textit{sur} $\lint n_0;n_1 \rint$ si pour tout $n\in \lint n_0;n_1 \lint$, $a_n\ge u_{n+1}$ (resp. $a_n > a_{n+1}$).
\end{itemize}}

\defi{\programme (constante) On dit que $(a_n)$ est \textit{constante} si tous ses termes sont égaux, \ie si pour tout $n\in\N$, $a_{n+1}-a_n=0$.}

\rem{Si $(a_n)$ ne s'annule pas, on peut aussi montrer que $\dfrac{a_{n+1}}{a_n}=1$ pour tout $n\in\N$.}

\defi{(suite extraite) Soit $(s_n)_{n\in\N}$. On dit que $(s_n)$ est une \textit{suite extraite} (ou \textit{sous-suite}) de $(a_n)$ s'il existe \fone{\varphi}{\N}{\N} une fonction \textit{strictement croissante} telle que pour tout $n\in\N$, $s_n=a_{\varphi(n)}$.}

\rem{Cette définition un peu compliquée correspond exactement à l'intuition que l'on se fait : une sous-suite est une suite fabriquée en ne prenant que certains termes d'une autre. La fonction n'a pour rôle que de garantir qu'on les prend de façon ordonnée.}

\defi{\programme (à partir d'un certain rang) Soit $P$ une propriété sur les suites. Soit $n_r\in\N$, \fons{\phi}{\N}{\N}{n}{n_r+n} et \suite{s} la suite extraite de $(a_n)$ définie pour tout $n\in\N$ par $s_n=a_{\phi(n)}$.

\begin{itemize}
\item On dit que $(a_n)$ \textit{vérifie $P$ à partir du rang $n_r$} si $(s_n)$ vérifie $P$.
\item On dit que $(a_n)$ \textit{vérifie $P$ à partir d'un certain rang} s'il existe $n_r\in\N$ tel que $(a_n)$ vérifie $P$ à partir du rang $n_r$.
\end{itemize}}

\rem{Cette définition formelle n'est pas au programme. Mais l'intuition de la notion d'«~à partir d'un certain rang~» l'est.}

\defi{\programme (stationnaire) On dit que $(a_n)$ est \textit{stationnaire} si elle est constante à partir d'un certain rang.}

\defi{(majorant, minorant) On dit que $M\in\R$ (resp. $m\in\R$) est un \textit{majorant} (resp. \textit{minorant}) de $(a_n)$ si pour tout $n\in\N$, $a_n\le M$ (resp. $a_n\ge m$). On dit alors que $(a_n)$ est \textit{majorée} (resp. \textit{minorée}).}

\rem{$(a_n)$ est majorée \ssi il existe $M\in\R_+$ tel que pour tout $n\in\N$, $a_n\le M$ (et on a un résultat similaire pour la minoration). Cette propriété peut être utile dans certaines preuves.}

\ex{\begin{itemize}
\item La suite $(U_n)$ de l'exercice \ref{exrecu} est majorée par $\dfrac{4}{5}$. Elle n'est en revanche pas minorée. 
\item La suite $(v_n)_{n\in\N}$ définie pour tout $n\in\N$ par $v_n=n(-1)^n$ n'est ni majorée ni minorée. Prouvons qu'elle n'est pas majorée. Supposons qu'elle est majorée et soit $M$ un de ses majorants. $M\ge0$ puisque $u_0=0$. Soit $n_M$ l'entier pair le plus proche de $M$ et strictement supérieur à $M$. Alors $v_{n_M}=\lceil M \rceil$ ou $v_{n_M}=\lceil M \rceil +1$ ou $v_{n_M}=\lceil M \rceil +2$. Dans tous les cas $v_{n_M}>M$, ce qui est absurde puisqu'on a supposé que $M$ majore $(v_n)$. Donc cette suite n'est pas majorée.
\end{itemize}}

\defi{(suite bornée) On dit que $(a_n)$ est \textit{bornée} si elle est à la fois minorée et majorée.}

\rem{\label{caraborn} $(a_n)$ est bornée si et seulement si il existe $M\in\R_+$ tel que pour tout $n\in\N$, $|a_n|<M$. $M$ majore alors $(a_n)$ et $-M$ minore $(a_n)$. Cette propriété peut être utile dans certaines preuves.}

\rem{\label{bornesup} $\bigoplus$ Les trois définitions ci-dessous s'étendent aux sous-ensembles de réels en général. La \textit{propriété de la borne supérieure} énonce en plus un résultat important : toute partie $P$ (sous-ensemble) non vide et majorée de $\R$ possède un plus petit majorant, c'est-à-dire que l'ensemble des majorants de $P$ admet un minimum. On l'appelle \textit{borne supérieure} de $P$ et on le note $\sup P$. De même, toute partie $P$ non vide et minorée de $\R$ admet un plus grand minorant, appelé \textit{borne inférieure de $P$} que l'on note $\inf P$. Par exemple, $\sup ]0;1[=1$ et $\inf ]0;1[ = 0$. Cette propriété peut être vue comme un des axiomes nécessaires à la construction de $\R$, c'est pourquoi ne nous nous aventurerons pas dans une tentative de démonstration.}





\section{Comportement à l'infini}

Soit $(a_n)_{n\in\N}$ une suite réelle. On s'intéresse au comportement de $(a_n)$ quand $n$ tend vers $+\infty$. 

\subsection{Définitions}



\defi{\programme (convergence) \begin{itemize}
\item Soit $l\in\R$. On dit que $(u_n)$ \textit{converge vers} (ou \textit{tend vers}) $l$ si pour tout réel $r>0$, l'intervalle $]l-r;l+r[$ contient tous les termes de $(u_n)$ à partir d'un certain rang, \ie si pour tout réel $r>0$, il existe $n_r\in\N$ tel que pour tout $n\ge n_r$, $a_n\in]l-r;l+r[$. On dit alors que $l$ est \textit{la limite} de $(a_n)$. 
\item On dit que $(a_n)$ \textit{converge} s'il existe $l\in\R$ tel que $(a_n)$ converge vers $l$.
\end{itemize}}

\rem{\begin{itemize}
\item Voici une vision intuitive : prenons n'importe quel «~tuyau~» qui entoure $l$. Si à partir d'un certain rang tous les termes de la suite sont dans le tuyau, c'est qu'elle converge vers $l$.
\item Officiellement, la définition de la convergence est la suivante : «~$(a_n)$ converge vers $l\in\R$ si tout intervalle ouvert contenant $l$ contient toutes les valeurs de $(a_n)$ à partir d'un certain rang~». Elle est équivalente à celle que nous donnons, la nôtre a l'avantage d'être beaucoup plus pratique dans les preuves.
\item Dans la définition, $n_r$ dépend de $r$ (c'est pourquoi je l'écris ainsi).
\item $a_n\in]l-r;l+r[$ peut s'écrire aussi $l-r<a_n<l+r$ mais aussi $|a_n-l|< r$ (qui est peut être moins facile à voir). Selon les démonstrations, l'une de ces trois écritures équivalentes peut être plus commode que les autres, c'est pourquoi il faut savoir reconnaître, comprendre et utiliser les trois.
\item Traditionnellement, dans le supérieur on note plutôt $\epsilon$ au lieu de $r$. J'ai choisi cette notation qui se rapproche plus de l'image du tuyau, $r$ étant en quelque sorte son rayon.
\end{itemize}}

\defi{(divergence)

\begin{itemize}
\item $(a_n)$ ne \textit{converge pas vers} (ou \textit{ne tend pas vers}) $l\in\R$ s'il existe un réel $r>0$ tel que pour tout $n_s\in\N$, il existe un entier $n_{\sigma}\ge n_s$ tel que $a_{\sigma}\not\in ]l-r;l+r[$.
\item $(a_n)$ \textit{diverge} (ou \textit{ne converge pas}) si pour tout $l\in\R$, $(a_n)$ ne converge pas vers $l$.
\end{itemize}}

\rem{\begin{itemize}
\item La définition de «~ne converge pas vers~» est en fait la stricte négation de celle de «~converge vers~». Intuitivement elle se comprend comme suit : $(a_n)$ ne converge pas vers $l\in\R$ s'il existe \textit{un tuyau particulier} entourant $l$ tel que pour toute barre verticale, on peut trouver \textit{un terme particulier} de la suite qui est à la fois à droite de cette barre et qui est en-dehors du tuyau.
\item Attention : dans la définition de «~ne converge pas vers~», $n_s$ \textit{ne dépend pas de} $r$, d'où le nom, et $n_{\sigma}$ \textit{dépend} de $n_s$, d'où le nom. De manière générale en mathématiques, nier une proposition implique d'inverser les dépendances des objets.
\end{itemize}}

\ex{Nier "$(u_n)_{n\in\N}$ ne converge pas".}

\ex{\label{exsuitegeo} Montrons que la suite $(g_n)_{n\in\N}$ définie par $g_0=42$ et pour tout $n\in\N$ par $g_{n+1}=\dfrac{g_n}{3}$ converge vers $0$. On montre par récurrence que pour tout $n\in\N$, $g_n> 0$. Soit $n\in\N$. $\dfrac{g_{n+1}}{g_n}=\dfrac{1}{3}$. $(g_n)$ est une suite géométrique de premier terme $g_0=42$ et de raison $q=\dfrac{1}{3}$. Donc pour tout $n\in\N$, $g_n=\dfrac{42}{3^n}$. Soit un réel $r>0$. On veut montrer qu'il existe $n_r\in\N$ tel que pour tout $n\ge n_r$, $g_n\in]-r;r[$. Résolvons l'équation $g_n<r$ sur $\N$. Après une série d'inégalités on trouve que l'ensemble des solutions est $S=\{n\in\N|n>\dfrac{\ln(42)-\ln(r)}{\ln(3)}\}$. Posons $n_r=\left\lceil \dfrac{\ln(42)-\ln(r)}{\ln(3)} \right\rceil+1$. $n_r\in S$ donc $g_{n_r}<r$. Par récurrence on prouve que $(g_n)$ est décroissante sur $\N$ et de plus pour tout $n\in\N$, $g_n> 0$. Soit un entier $n\ge n_r$. Alors $g_n\in ]0;r[$ donc $g_n\in ]-r;r[$. Conclusion : $(g_n)$ converge vers 0.}

\nota{\programme \label{conv}\textit{Dans le cas} où $(u_n)_{n\in\N}$ converge vers $l\in\R$ et \textit{seulement dans ce cas}, alors on note $\limc{n}{+\infty} u_n = l$ (lire "la limite de $u_n$ quand $n$ tend vers plus infini est $l$")}
\textbf{Attention} : il est interdit d'utiliser la notation $\limc{n}{+\infty} u_n$ \textit{avant} d'avoir prouvé la convergence de la suite car l'écrire suppose \textit{a priori} l'existence d'une limite !
\rem{On trouve parfois la notation $u_n\underset{n\rightarrow +\infty}{\longrightarrow} l$ (lire "$u_n$ tend vers $l$ quand $n$ tend vers plus infini").}

\defi{\programme (divergence) Soit $(u_n)_{n\in\N}$ une suite réelle. Si $(u_n)$ ne converge pas, on dit que $(u_n)$ \textit{diverge}.}

\ex{\begin{itemize}
\item La suite $(u_n)_{n\in\N}$ définie pour tout $n\in\N$ par $u_n=n$ diverge.
\item La suite $(u_n)_{n\in\N}$ définie pour tout $n\in\N$ par $u_n=(-1)^n$ diverge.
\item La suite $(u_n)_{n\in\N}$ définie pour tout $n\in\N$ par $u_n=n(-1)^n$ diverge.
\end{itemize}}

\defi{\programme Soit $(u_n)_{n\in\N}$ une suite réelle. Si pour tout $M\in\R$ (resp. $m\in\R$), il existe $n_M\in\N$ (resp. $n_m\in\N$) tel que pour tout entier $n\ge n_M$, $u_n>M$ (resp. $u_n < m$) alors on dit que $(u_n)$ \textit{diverge vers} $+\infty$ (resp. \textit{diverge vers} $-\infty$).}

\rem{\begin{itemize}
\item Officiellement la définition de la divergence vers $+\infty$ est la suivante : «~$(u_n)$ diverge vers $+\infty$ si tout intervalle de la forme $]A;+\infty[$ contient toutes les valeurs de $(u_n)$ à partir d'un certain rang~». Elle est équivalente à celle que nous donnons.
\item De façon intuitive : on fixe n'importe quelle "barre". Si au bout d'un moment tous les termes de la suite se trouvent au-dessus, c'est qu'elle diverge vers $+\infty$.
\end{itemize}}

\nota{\programme \textit{Dans le cas} où $(u_n)_{n\in\N}$ diverge vers $+\infty$ (resp. vers $-\infty$) et \textit{seulement dans ce cas}, alors on note $\limc{n}{+\infty} u_n = +\infty$ (resp. $\limc{n}{+\infty} u_n = -\infty$).}
\rem{La mise en garde et la remarque suivants la notation \ref{conv} sont toujours valables ici.}

\subsection{Propositions générales}

Démontrer des convergences ou des divergences est fastidieux en utilisant les définitions. Nous allons ici donner tout un arsenal de propositions afin d'aller beaucoup plus vite. En effet, selon les caractéristiques de la suite qu'on étudie (monotonie, majorée...) on peut souvent déduire des choses sur son comportement à l'infini.

\subsubsection{Convergence}
\pro{Si une suite converge alors sa limite est unique.}

\preuve{Soit $(u_n)_{n\in\N}$ convergente et supposons qu'elle converge vers les deux limites $l_1\in\R$ et $l_2\in\R$ avec $l_1<l_2$. Soit $r=\dfrac{l_2-l_1}{2}$. Posons $n_1\in\N$ (resp. $n_2\in\N$) tel que pour tout entier $n\ge n_1$ (resp. $n\ge n_2$), $u_n\in]l_1-r;l_1+r[$ (resp. $u_n\in]l_2-r;l_2+r[$). Posons $n_0=\max(n_1,n_2)$ et soit un entier $n\ge n_0$. Alors on a à la fois $l_1-r<u_n<l_1+r$ et $l_2-r<u_n<l_2+r$. Or $l_1+r=l_2-r$ donc on a $l_1+r<u_n<l_1+r$ ce qui est absurde. Donc $l_1=l_2$.}

\pro{Toute suite convergente est bornée.}

\preuve{Soit $(u_n)_{n\in\N}$ une suite convergent vers $l\in\R$. Soit un réel $r>0$. Alors il existe $n_r\in\N$ tel que pour tout entier $n\ge n_r$, $u_n\in]l-r;l+r[$. Posons un tel $n_r$. l'ensemble $E=\{u_n|n\in\lint 0;n_r\lint\}$ est fini donc possède un minimum $m_E$ et un maximum $M_E$. Posons $m=\min(m_E,l-r)$ et $M=\max(M_E,l+r)$. $M$ majore $(u_n)$ et $m$ la minore, donc $(u_n)$ est bornée.}

\rem{\textbf{Attention} : la réciproque est fausse. Par exemple la suite $(u_n)_{n\in\N}$ définie pour tout $n\in\N$ par $u_n=(-1)^n$ est bornée mais ne converge pas.}

\cor{Une suite non majorée ou non minorée diverge.}

\preuve{C'est la contraposée de la proposition précédente.}

\rem{Par contre, ce n'est pas parce qu'une suite est non majorée qu'elle diverge nécessairement vers $+\infty$. Par exemple la suite $(u_n)_{n\in\N}$ défnie pour tout $n\in\N$ par $u_n=n(-1)^n$ est non majorée mais ne diverge pas vers $\pm\infty$.}

\theo{(de la convergence monotone) Toute suite croissante (strictement ou non) majorée converge. Toute suite décroissante (strictement ou non) minorée converge.}

\preuve{Montrons le cas décroissant (pour changer). Soit $(u_n)_{n\in\N}$ une suite décroissante minorée. L'ensemble $U=\{u_n|n\in\N\}$ est non vide et minorée donc par la propriété de la borne inférieure (voir remarque \ref{bornesup}), $U$ possède une borne inférieure. Soit $l=\inf U$. Montrons que $(u_n)$ converge vers $l$. Soit un réel $r>0$. Si $[l;l+r[$ ne contenait aucun terme de $(u_n)$, alors $l+r$ serait un de ses minorants ce qui est absurde puisque $l$ est le plus grand d'entre eux. Donc il existe $n_r\in\N$ tel que $u_{n_r}\in[l;l+r[$. Soit un entier $n\ge n_r$. Comme $(u_n)$ est décroissante et minorée par $l$ alors $u_n\in[l;l+r[$ donc $u_n\in]l-r;l+r[$, donc $(u_n)$ converge vers $l$.}

\pro{Soit $(a_n)_{n\in\N}$, $(b_n)_{n\in\N}$ et $(p_n)_{n\in\N}$ la suite définie pour tout $n\in\N$ par $p_n=a_n b_n$. Si $(a_n)$ est bornée et si $(b_n)$ converge vers $0$, alors $(p_n)$ converge vers 0.}

\preuve{D'après la remarque \ref{caraborn}, posons $M>0$ tel que pour tout $n\in\N$, $|a_n|<M$. Soit un réel $r>0$ et posons $\varepsilon=\dfrac{r}{M}>0$. Posons $n_r\in\N$ tel que pour tout entier $n\ge n_r$, $b_n\in]-\varepsilon;\varepsilon[$ et soit un entier $n\ge n_r$. Alors $|a_n|<M$ et $|b_n|<\varepsilon$ donc par multiplication, $|a_n b_n|<M\varepsilon$ \cad $|p_n|<r$ donc $(p_n)$ converge vers 0.}

\cor{\label{pnul} Soit $(a_n)_{n\in\N}$, $(b_n)_{n\in\N}$ et $(p_n)_{n\in\N}$ la suite définie pour tout $n\in\N$ par $p_n=a_n b_n$. Si $(a_n)$ converge et si $(b_n)$ converge vers $0$, alors $(p_n)$ converge vers 0.}

\preuve{Toute suite convergente est bornée donc la proposition précédente s'applique.}

\pro{\programme Soit $(a_n)_{n\in\N}$ une suite croissante et convergente vers $l\in\R$. Alors pour tout $n\in\N$, $a_n\le l$.}

\preuve{\programme Supposons qu'il existe $n_s\in\N$ tel que $a_{n_s}>l$. Montrons alors que $(a_n)$ ne converge pas vers $l$. Explicitons la définition de \gug~$(a_n)$ ne converge pas vers $l$~\gud : \gug il existe un réel $r>0$ tel que pour tout entier $n_0\in\N$, il existe un entier $n_{\sigma}\ge n_0$ tel que $a_{n_{\sigma}}\not\in]l-r;l+r[$~».  Posons $r=a_{n_s}-l>0$ et soit $n_0\in\N$. Posons $n_{\sigma}=\max(n_0, n_s)$. Donc on a d'une part $n_{\sigma}\ge n_0$ et d'autre part comme $n_{\sigma}\ge n_s$ et que $(a_n)$ est croissante alors $a_{n_{\sigma}}\ge a_{n_{s}}$. On a $]l-r;l+r[=]2l-a_{n_{s}};a_{n_{s}}[$ donc $a_{n_{\sigma}}\not\in]l-r;l+r[$. Donc $(a_n)$ ne converge pas vers $l$ ce qui est absurde.}

\subsubsection{Divergence}

\pro{Une suite qui diverge vers $\pm\infty$ diverge.}

\preuve{Nous allons montrer le cas $+\infty$, l'autre étant similaire. Soit $(u_n)_{n\in\N}$ une suite divergeant vers $+\infty$. Supposons qu'il existe $l\in\R$ tel que $(u_n)$ converge vers $l$. Soit un réel $r>0$. Alors il existe $n_r\in\N$ tel que pour tout entier $n\ge n_r$, $u_n\in]l-r;l+r[$. Posons un tel $n_r$. Posons $M=l+r$. Comme $(u_n)$ diverge vers $+\infty$, il existe $n_M\in\N$ tel que pour tout entier $n\ge n_M$, $u_n>M$. Posons un tel $n_M$. Soit enfin $n_0=\max(n_r,n_M)$. On a donc à la fois $u_{n_0}\in]l-r;l+r[$ et $u_{n_0}>l+r$, c'est absurde. Donc $(u_n)$ ne converge pas.}

\pro{Toute suite divergente vers $+\infty$ est non majorée.}

\preuve{Soit $(u_n)_{n\in\N}$ une suite divergente vers $+\infty$. Supposons que $(u_n)$ est majorée et soit $M\in\R$ un de ses majorants. Il existe $n_M\in\N$ tel que $u_{n_M}>M$ donc $M$ ne majore pas $(u_n)$, c'est absurde.}

\pro{La croissance ou la décroissance d'une suite n'est pas une condition suffisante à la divergence.}

\preuve{La suite de l'exemple \ref{exsuitegeo} est décroissante strictement mais converge vers $0$.}

\pro{Toute suite croissante non majorée diverge vers $+\infty$. Toute suite décroissante non minorée diverge vers $-\infty$.}

\preuve{Montrons le cas croissant. Soit $(u_n)_{n\in\N}$ une suite croissante non majorée. Soit $M\in\R$. $(u_n)$ est non majorée donc il existe $n_M\in\N$ tel que $u_{n_M}>M$. Soit un entier $n\ge n_M$. $(u_n)$ étant croissante, $u_n\ge u_{n_M}>M$, donc $(u_n)$ diverge vers $+\infty$.}

\pro{Toute suite croissante non convergente diverge vers $+\infty$. Toute suite décroissante non convergente diverge vers $-\infty$.}

\preuve{Montrons le cas croissant. Soit $(u_n)_{n\in\N}$ une suite croissante qui ne converge pas. La contraposée de la proposition précédente nous indique que toute suite divergente est non croissante ou non majorée. Or $(u_n)$ est croissante donc elle est non majorée. Or on sait qu'une suite croissante et non majorée diverge vers $+\infty$.}

\pro{Soit \suite{a} une suite convergente vers $l\in\R$. Alors toute sous-suite de $(a_n)$ converge également vers $l$.}

\preuve{Préambule : on montre facilement (par récurrence par exemple) que si \fone{\phi}{\N}{\N} est strictement croissante alors pour tout $n\in\N$, $\phi(n)\ge n$. Soit \suite{s} une sous-suite de $(a_n)$ et soit \fone{\phi}{\N}{\N} strictement croissante telle que pour tout $n\in\N$, $s_n=a_{\phi(n)}$. Soit un réel $r>0$ et posons un entier $n_r$ tel que pour tout entier $n\ge n_r$, $a_n\in]l-r;l+r[$. Soit un entier $n\ge n_r$. Puisque $\phi$ est strictement croissante, $\phi(n)\ge \phi(n_r)\ge n_r$. Donc $a_{\phi(n)}\in]l-r;l+r[$ \cad $s_n\in]l-r;l+r[$. Donc $(s_n)$ converge vers $l$.}

\rem{Ce sont surtout les deux corollaires ci-dessous de cette proposition qu'on utilise en pratique.}

\cor{Soit \suite{a}. S'il existe une sous-suite de $(a_n)$ divergente, alors $(a_n)$ diverge.}

\preuve{Supposons que $(a_n)$ converge vers $l\in\R$ et qu'il existe une sous-suite de $(a_n)$ divergente. Alors d'après la proposition précédente, toute sous-suite de $(a_n)$ converge vers $l$, or il en existe une qui diverge. Donc $(a_n)$ ne converge pas, donc diverge.}

\cor{Soit \suite{a}. S'il existe deux sous-suites de $(a_n)$ qui convergent vers des limites différentes, alors $(a_n)$ diverge.}

\preuve{Similaire à la précédente.}

\subsection{Comparaison de suites}

\pro{\programme Soit $(a_n)_{n\in\N}$ et $(b_n)_{n\in\N}$ telles que :

\begin{enumerate}[i)]
\item $(a_n)$ diverge vers $+\infty$,
\item Pour tout $n\in\N$, $b_n\ge a_n$.
\end{enumerate}
Alors $(b_n)$ diverge vers $+\infty$.}

\rem{On a la même chose avec le cas $(a_n)$ divergente vers $-\infty$ et $b_n\le a_n$.}

\preuve{\programme Soit un réel $M\in\R$. Soit $n_r\in\N$ tel que pour tout entier $n\ge n_r$, $a_n>M$. Soit un entier $n\ge n_r$. Alors puisque $b_n\ge a_n$ on a $b_n>M$ donc $(b_n)$ diverge vers $+\infty$.}

\rem{La proposition est aussi vraie si $b_n\ge a_n$ seulement à partir d'un certain rang $n_0$. C'est d'ailleurs la proposition officielle. La démonstration s'adapte très simplement en prenant $n_r\ge n_0$.}

\pro{Soit $(a_n)_{n\in\N}$ et $(b_n)_{n\in\N}$ deux suites. Soit $(d_n)_{n\in\N}$ définie pour tout $n\in\N$ par $d_n=|a_n-b_n|$ et supposons que $(d_n)$ converge vers 0. Alors $(a_n)$ et $(b_n)$ ont le même comportement, \cad que soit elles convergent toutes les deux auquel cas leur limite est la même, soit elles divergent toutes les deux vers $\pm \infty$ soit elles divergent toutes les deux.}

\defi{\label{adj} (adjacence) Soit deux suites $(u_n)_{n\in\N}$ et $(v_n)_{n\in\N}$. Soit $(d_n)_{n\in\N}$ définie pour tout $n\in\N$ par $d_n=|u_n-v_n|$. On dit que $(u_n)$ et $(v_n)$ sont adjacentes si :

\begin{enumerate}[i)]
\item $(u_n)$ est croissante,
\item $(v_n)$ est décroissante,
\item $(d_n)$ converge vers 0.
\end{enumerate}}

\rem{La définition est équivalente en posant $d_n=u_n-v_n$ ou $d_n=v_n-u_n$ et cela a l'avantage de rendre la démonstration qui va suivre moins pénible en évitant des distinctions de cas. Toutefois, il m'a semblé que garder la valeur absolue correspond mieux à l'intuition~: c'est la distance entre les deux suites qui tend à s'annuler.}

\pro{Si deux suites sont adjacentes alors elles convergent et leur limite est la même.}

\preuve{Soit $(u_n)_{n\in\N}$ et $(v_n)_{n\in\N}$ adjacentes et soit $(d_n)_{n\in\N}$ définie pour tout $n\in\N$ par $d_n=|u_n-v_n|$. Supposons que $(u_n)$ ne converge pas. On sait qu'une suite croissante non convergente diverge vers $+\infty$ donc $(u_n)$ diverge vers $+\infty$. Soit un réel $r>0$. $(d_n)$ converge vers $0$ donc il existe $n_r\in\N$ tel que pour tout entier $n\ge n_r$, $d_n\in]-r;r[$. Posons un tel $n_r$. $(u_n)$ diverge vers $+\infty$ et n'est donc pas majorée, donc en particulier pas par $v_0+r$. Il existe donc un entier $n_0\ge n_r$ tel que $u_{n_0}>v_0+r$. Or $(v_n)$ est décroissante donc $u_{n_0}>v_{n_0}+r$ c'est-à-dire $u_{n_0}-v_{n_0}>r$ \cad $|u_{n_0}-v_{n_0}|>r$ \cad $d_{n_0}>r$ ce qui est absurde puisque $d_{n_0}\in]-r;r[$. Donc $(u_n)$ converge et de même, $(v_n)$ aussi. Appelons $l_u$ et $l_v$ leurs limites respectives et supposons $l_u< l_v$. Donc pour tout $n\in\N$, $v_n>u_n$ et donc pour tout $n\in\N$, $d_n=v_n-u_n$. Soit $r=\dfrac{l_v-l_u}{3}$. Il existe $n_u$ (resp. $n_v$) tel que pour tout entier $n\ge n_u$ (resp. $n\ge n_v$), $u_n\in]l_u-r;l_u+r[$ (resp. $v_n\in]l_v-r;l_v+r[$). Posons $n_0=\max(n_u,n_v)$ et soit un entier $n\ge n_0$. On a donc à la fois $u_n\in]l_u-r;l_u+r[$ et $v_n\in]l_v-r;l_v+r[$, \cad $|u_n-l_u|<r$ et $|v_n-l_v|<r$ \cad $l_u-u_n<r$ et $v_n-l_v<r$. En additionant ces deux inégalités on obtient $v_n-u_n<2r-l_u+l_v$ \cad $v_n-u_n<-r$. Mais alors $d_n=v_n-u_n<-r$ ce qui est absurde puisque $(d_n)$ tend vers $0$. De même en supposant cette fois $l_u> l_v$ on trouve une contradiction (attention cependant : dans ce cas $d_n=u_n-v_n$ seulement à partir d'un certain rang et pas pour tout $n\in\N$, une fois ce détail réglé la démarche est la même). Donc $l_u=l_v$.}

\theo{\programme (des gendarmes, ou des sandwichs) Soit $(a_n)_{n\in\N}$, $(b_n)_{n\in\N}$ et $(c_n)_{n\in\N}$ trois suites telles que :
\begin{enumerate}[i)]
\item $(a_n)$ et $(c_n)$ sont adjacentes,
\item pour tout $n\in\N$, $a_n\le b_n \le c_n$.
\end{enumerate}
Alors ces trois suites convergent vers la même limite.}

\preuve{$(a_n)$ et $(c_n)$ sont adjacentes donc convergent vers la même limite. Notons-la $l$. Soit un réel $r>0$. Posons un entier $n_a$ (resp. $n_c$) tel que pour tout entier $n\ge n_a$ (resp. $n\ge n_c$), $a_n\in]l-r;l+r[$ (resp. $c_n\in]l-r;l+r[$). Posons $n_0=\max(n_a,n_c)$ et soit un entier $n\ge n_0$. Donc on a à la fois $l-r<a_n<l+r$ et $l-r<b_n<l+r$. On sait que $b_n\ge a_n$ donc $l-r<b_n$. On sait que $b_n\le c_n$ donc $b_n< l+r$. Donc $b_n\in]l-r;l+r[$ donc $(b_n)$ converge vers $l$.}
\rem{Ce théorème est également valide si on a $a_n\le b_n \le c_n$ seulement à partir d'un certain rang $n_b$. En ce cas il faudra simplement poser $n_0=\max(n_a,n_b,n_c)$ dans la démonstration.}

\subsection{Par la valeur absolue}

\pro{Soit $(u_n)_{n\in\N}$ et $(a_n)_{n\in\N}$ définie pour tout $n\in\N$ par $a_n=|u_n|$. Si $(u_n)$ converge alors $(a_n)$ aussi (et sa limite est alors la valeur absolue de celle de $(u_n)$).}

\preuve{Préambule : l'\textit{inégalité triangulaire inverse} nous affirme que pour tout $x,y\in\R$, \newline $||x|-|y||<|x-y|$. Soit $l\in\R$ la limite de $(u_n)$ et soit $r>0$. Soit $n_r$ un entier tel que pour tout entier $n\ge n_r$, $|u_n-l|<r$ et soit un entier $n\ge n_r$. Alors $|u_n-l|<r$ donc par l'inégalité triangulaire inverse, $||u_n|-|l||<|u_n-l|<r$ \cad $|a_n-|l||<r$. Donc $(a_n)$ converge vers $|l|$.}

\rem{\textbf{Attention} : la réciproque n'est pas vraie. Par exemple si $u_n=(-1)^n$ alors $a_n=1$, et alors $(a_n)$ converge mais pas $(u_n)$. Comme nous le verrons dans la proposition suivante, la réciproque ne marche en général que si $(a_n)$ converge vers 0.}

\cor{\label{divabs} Soit $(u_n)_{n\in\N}$ et $(a_n)_{n\in\N}$ définie pour tout $n\in\N$ par $a_n=|u_n|$. Si $(a_n)$ diverge alors $(u_n)$ aussi.}

\preuve{C'est la contraposée de la proposition précédente.}

\pro{\label{convabsz}Soit $(u_n)_{n\in\N}$ et $(a_n)_{n\in\N}$ définie pour tout $n\in\N$ par $a_n=|u_n|$. Si $(a_n)$ converge vers 0 alors $(u_n)$ aussi.}

\preuve{Soit $r>0$. Soit $n_r$ un entier tel que pour tout entier $n\ge n_r$, $-r<|a_n|<r$ et soit un entier $n\ge n_r$. Si $a_n\ge0$ alors $|a_n|=a_n$ donc $-r<a_n<r$. Si $a_n\le0$ alors $|a_n|=-a_n$ donc $-r<-a_n<r$ donc $-r<a_n<r$. Dans tous les cas $-r<a_n<r$ et donc $(a_n)$ converge vers 0.}

\subsection{Par les fonctions}

\theo{\label{caraseq} (caractérisation séquentielle des limites) Soit $l\in\R$, \suite{u} une suite qui converge vers $l$, $E\subset \R$ tel que tous les termes de $(u_n)$ sont dans $E$ et \fone{f}{E}{\R}. Alors $f$ admet une limite en $l$ \ssi la suite \suite{f(u_n)} converge vers $f(l)$.}

\rem{On peut remplacer $l$ par $\pm\infty$.}

\preuve{La démonstration est au programme du chapitre d'analyse réelle.}

\ex{Soit \suite{u} la suite définie pour tout $n\in\N$ par $u_n=\sqrt{\dfrac{n^2+3}{2n^2+n+1}}$. Étudions son comportement. Posons \suite{v} la suite définie pour tout $n\in\N$ par $v_n=\dfrac{n^2+3}{2n^2+n+1}$. On a donc, pour tout $n\in\N$, $u_n=\sqrt{v_n}$. On montre en factorisant que $v_n$ converge vers $\dfrac{1}{2}$. Or $\limc{x}{1/2}\sqrt{x}=\dfrac{\sqrt{2}}{2}$. D'où $u_n$ converge vers $\dfrac{\sqrt{2}}{2}$.}

\defi{Soit \suite{a} une suite récurrente d'ordre 1, c'est-à-dire qu'il existe une \fone{f}{\R}{\R} telle que pour tout $n\in\N$, $f(a_n)=a_{n+1}$. Une telle fonction est appelée \textit{fonction associée à} $(a_n)$.}

\theo{\label{ptfixe} (du point fixe) Soit \suite{a} une suite récurrente d'ordre 1, \fone{f}{\R}{\R} une fonction associée à $(a_n)$ et $F=\{x\in\R|f(x)=x\}$ \cad l'ensemble des points fixes de $f$. Si $(a_n)$ converge vers $l\in\R$ et si $f$ est continue en $l$, alors $l\in F$.}

\preuve{$f$ est continue en $l$ donc $\limc{x}{l} f(x)=f(l)$. Par la caractérisation séquentielle des limites, puisque $\limc{n}{+\infty} a_{n}=l$ alors $\limc{n}{+\infty} f(a_n)=f(l)$. Or $\limc{n}{+\infty} f(a_n) = \limc{n}{+\infty} a_{n+1}=\limc{n}{+\infty} a_n=l$. Par conséquent on a $f(l)=l$ \cad que $l\in F$.}

\cor{Soit \suite{a} une suite récurrente d'ordre 1, \fone{f}{\R}{\R} une fonction associée à $(a_n)$ et $F=\{x\in\R|f(x)=x\}$. Soit $l\not\in F$. Alors $(a_n)$ ne converge pas vers $l$ ou bien $f$ n'est pas continue en $l$.}

\rem{Ce corollaire permet d'établir les limites candidates. C'est particulièrment efficace si les fonctions associées ne sont pas continues seulement en quelques points (et c'est le cas de la plupart des fonctions usuelles).}

\preuve{C'est la contraposée de la proposition précédente.}

\ex{\begin{itemize}
\item Supposons que $a_0\in\R^*$ et pour tout $n\in\N$, $a_{n+1}=\dfrac{1}{a_n}$. \fons{f}{\R^*}{\R}{x}{\dfrac{1}{x}} est une fonction associée à $(a_n)$. Dans ce cas $F=\{-1;1\}$. $f$ est continue sur $x\in\R\backslash\{0\}$ et par conséquent, si $(a_n)$ converge vers $l\in\R$ alors $l\in\{-1;0;1\}$.

\item Supposons que $a_0\in\R$ et pour tout $n\in\N$, $a_{n+1}=a_n^2-a_n+2$. \fons{f}{\R}{\R}{x}{x^2-x+2} est une fonction associée à $(a_n)$. Dans ce cas $F=\emptyset$. $f$ est continue sur $\R$ donc $(a_n)$ diverge.
\end{itemize}}

\pro{Soit \suite{a} une suite qui converge vers  $\alpha\in\R$. Soit $T=\{a_n|n\in\N\}$, $I$ une partie de $\R$ tel que $T\subseteq I$ et \fone{f}{I}{\R}. Alors si $\limc{x}{\alpha} f(x) = \beta$, alors la suite \suite{F} définie pour tout $n\in\N$ par $F_n=f(u_n)$ converge vers $\beta$.}

%\defi{(fonction associée) Soit \suite{a} une suite récurrente d'ordre 1. Soit $T=\{a_n|n\in\N\}$. On appelle \textit{fonction associée} à $(a_n)$ toute fonction \fone{f}{I}{\R} telle que $T\subseteq I\subseteq\R$ et telle que pour tout $n\in\N$, $f(a_n)=a_{n+1}$.}

%\ex{Si pour tout $n\in\N$, $a_{n+1}=2a_n-3$ alors la fonction \fone{f}{\R}{\R} telle que pour tout $x\in\R$, $f(x)=2x-3$ est une fonction associée à $(a_n)$.}

%\rem{L'existence d'une telle fonction est garantie en réalité par la définition même de suite récurrente d'ordre 1.}

\defi{Soit \suite{a} une suite telle que chacun de ses termes ne dépendent que de leur rang. On appelle \textit{fonction associée} à $(a_n)$ toute fonction \fone{f}{I}{\R} telle que $I$ est un intervalle de $\R$ et telle que pour tout $n\in\N\cap I$, $f(n)=a_{n}$.}

\ex{Si pour tout $n\in\N$, $a_{n}=\dfrac{1}{n+1}$ alors la fonction \fone{f}{\R_+}{\R} telle que pour tout $x\in\R$, $f(x)=\dfrac{1}{x+1}$ est une fonction associée à $(a_n)$.}

\pro{Soit \suite{a} une suite telle que chacun de ses termes ne dépendent que de leur rang. Soit $I$ un intervalle de $\R$ contenant au moins deux entiers naturels consécutifs et soit \fone{f}{I}{\R} une fonction associée à $(a_n)$. Soit enfin $n_0$ et $n_1$ le plus petit et le plus grand entier naturel de $I$ respectivement. Si $f$ est croissante (resp. décroissante) sur $I$ alors $(a_n)$ est croissante (resp. décroissante) sur $\lint n_0;n_1 \rint$.}

\preuve{Supposons $f$ croissante sur $I$, \cad que pour tout $x,y\in I$ avec $x<y$, $f(x)\le f(y)$.  Soit $n\in \lint n_0; n_1 \lint$. On a $n<n+1$ et comme $n, n+1\in I$ alors $f(n)\le f(n+1)$ \cad $a_n \le a_{n+1}$, donc $(a_n)$ est croissante sur $I$.}

\rem{Cette proposition fonctionne aussi avec le "strictement".}

\pro{Soit \suite{a} une suite telle que chacun de ses termes ne dépendent que de leur rang. Soit $I$ un intervalle de $\R$ contenant $\N$ (typiquement $\R_+$) et soit \fone{f}{I}{\R} une fonction associée à $(a_n)$. Si $f$ est croissante (resp. décroissante) sur $I$ alors $(a_n)$ est croissante (resp. décroissante).}

\preuve{Immédiat en se servant de la preuve précédente comme modèle.}

\rem{Les deux propositions précédentes sont extrêmement pratiques pour prouver la monotonie d'une fonction, il ne faut donc pas hésiter à les utiliser au lieu de rédiger une longue récurrence.}

\subsection{Opérations sur les limites}
\label{oplim}

Connaissant le comportement de deux suites, que peut-on dire du comportement d'une opération sur elles deux ? C'est ce que nous allons voir ici. Dans toute cette section, $(a_n)_{n\in\N}$ et $(b_n)_{n\in\N}$ sont deux suites réelles.

\subsubsection{Multiplication par un scalaire}

\pro{\programme Si $(a_n)$ converge vers $\alpha\in\R$ alors pour tout $\lambda\in\R$, la suite $(c_n)_{n\in\N}$ définie pour tout $n\in\N$ par $c_n=\lambda a_n$ converge vers $\lambda\alpha$.}

\preuve{Supposons $\lambda> 0$ et soit un réel $r>0$. Posons $q=\dfrac{r}{\lambda}>0$. Posons $n_r\in\N$ tel que pour tout entier $n\ge n_r$, $a_n\in]\alpha-q;\alpha+q[$. Soit un entier $n\ge n_r$. Alors $\alpha-q<a_n<\alpha+q$ \cad $\lambda(\alpha-q)<\lambda a_n<\lambda(\alpha+q)$ \cad $\lambda\alpha-r<c_n<\lambda\alpha+r$ ce qui prouve que $(c_n)$ converge vers $\lambda\alpha$. Le cas $\lambda< 0$ est similaire en posant $q=-\dfrac{r}{\lambda}$. Le cas $\lambda=0$ nous donne immédiatement que pour tout $n\in\N$, $c_n=0$ donc $(c_n)$ converge vers $0=\lambda\alpha$.}

\pro{\programme Si $(a_n)$ diverge alors pour tout réel $\lambda\neq 0$, la suite $(c_n)_{n\in\N}$ définie pour tout $n\in\N$ par $c_n=\lambda a_n$ diverge. Si $\lambda=0$ alors $(c_n)$ converge vers $0$.}

\preuve{Le cas $\lambda=0$ est immédiat. Supposons donc $\lambda> 0$. Supposons que $(c_n)$ converge vers $l\in\R$. Soit un réel $r>0$ et $q=\lambda r$. Posons $n_r\in\N$ tel que pour tout entier $n\ge n_r$, $c_n\in]l-q;l+q[$. Soit un entier $n\ge n_r$. Alors $l-q<c_n<l+q$ \cad $\dfrac{l-q}{\lambda}<\dfrac{c_n}{\lambda}<\dfrac{l+q}{\lambda}$ \cad $\dfrac{l}{\lambda}-r<a_n<\dfrac{l}{\lambda}+r$. Donc $(a_n)$ converge vers $\dfrac{l}{\lambda}$, c'est absurde. Le cas $\lambda<0$ est similaire en posant $q=-\lambda r$.}

\pro{\programme Soit $(a_n)$ divergente vers $+\infty$. Soit $\lambda\in\R$ et soit la suite $(c_n)_{n\in\N}$ définie pour tout $n\in\N$ par $c_n=\lambda a_n$.

\begin{enumerate}
\item Si $\lambda=0$ alors $(c_n)$ converge vers $0$.
\item Si $\lambda>0$ alors $(c_n)$ diverge vers $+\infty$.
\item Si $\lambda<0$ alors $(c_n)$ diverge vers $-\infty$.
\end{enumerate}}
\rem{Comme d'habitude, on a la même chose pour $(a_n)$ divergente vers $-\infty$ en adaptant.}

\preuve{Le cas $\lambda=0$ est immédiat. Soit $\lambda>0$. Soit un réel $M>0$ et $M'=\dfrac{M}{\lambda}>0$. Posons un entier $n_M$ tel que pour tout entier $n\ge n_M$, $a_n > M'$. Soit un entier $n\ge n_M$. Alors $a_n > M'$ \cad $\lambda a_n > \lambda M'$ \cad $c_n > M$. Donc $(c_n)$ diverge vers $+\infty$. Le cas $\lambda<0$ est similaire en posant $M'=-\dfrac{M}{\lambda}$.}

Voici un tableau récapitulatif.

\begin{tabular}{|c|c|c|c|}
\hline
$(a_n)_{n\in\N}$ & $\lambda < 0$ & $\lambda = 0$ & $\lambda > 0$ \\ \hline
Converge vers $l\in\R$ & $\lambda l$ & 0 & $\lambda l$ \\ \hline
Diverge vers $+\infty$ & $-\infty$ &  0 & $+\infty$ \\ \hline
Diverge vers $-\infty$ & $+\infty$ & 0 & $-\infty$ \\ \hline
Diverge & Diverge & 0 & Diverge \\ \hline
\end{tabular}

\subsubsection{Addition de suites}

Dans cette sous-section $(s_n)_{n\in\N}$ est la suite définie pour tout $n\in\N$ par $s_n=a_n+b_n$. Bien entendu, ce que nous allons voir ici s'applique à la soustraction en considérant par exemple l'opposé de la suite $(b_n)$. 

\pro{\programme Si $(a_n)$ converge vers $\alpha$ et $(b_n)$ vers $\beta$ alors $(s_n)$ converge vers $\alpha+\beta$.}

\preuve{Posons $\sigma=\alpha+\beta$. Soit un réel $r>0$ et posons $n_a$ (resp. $n_b$) un entier tel que pour tout entier $n\ge n_a$ (resp. $n\ge n_b$), $a_n\in]\alpha-r;\alpha+r[$ (resp. $b_n\in]\beta-r;\beta+r[$). Soit $n_0=\max(n_a,n_b)$ et soit un entier $n\ge n_0$. On a donc à la fois $a_n\in]\alpha-r;\alpha+r[$ et $b_n\in]\beta-r;\beta+r[$. Donc $s_n=a_n+b_n\in]\sigma-2r;\sigma+2r[$ donc $s_n\in]\sigma-r;\sigma+r[$ et donc $(s_n)$ converge vers $\sigma$.}

\pro{\programme Si $(a_n)$ converge vers $\alpha$ et $(b_n)$ diverge vers $+\infty$ (resp. $-\infty$) alors $(s_n)$ diverge vers $+\infty$ (resp. $-\infty$).}

\preuve{Traitons le cas $+\infty$. Soit un réel $M>0$ et $M'=2M-\alpha$. Posons un entier $n_a$ (resp. $n_b$) tel que pour tout entier $n\ge n_a$ (resp. $n\ge n_b$), $a_n\in]\alpha-M;\alpha+M[$ (resp. $b_n>M'$). Soit $n_0=\max(n_a,n_b)$ et soit un entier $n\ge n_0$. On a donc à la fois $a_n>\alpha-M$ et $b_n>M'$ donc par addition, $s_n>\alpha-M+M'$ \cad $s_n>M$. Donc $(s_n)$ diverge vers $+\infty$ (le lecteur pourrait reprocher à juste titre qu'on a pris seulement $M>0$ et non $M\in\R$ qui est exigé par la définition mais dans le cas de la divergence vers $+\infty$ c'est la même chose).}

\pro{\programme Si $(a_n)$ et $(b_n)$ divergent vers $+\infty$ (resp. $-\infty$) alors $(s_n)$ diverge vers $+\infty$ (resp. $-\infty$).}

\preuve{Traitons le cas $-\infty$ pour changer. Soit $m<0$. Posons un entier $n_a$ (resp. $n_b$) tel que pour tout entier $n\ge n_a$ (resp. $n\ge n_b$), $a_n<m$ (resp. $b_n<m$). Soit $n_0=\max(n_a,n_b)$ et soit un entier $n\ge n_0$. On a donc à la fois $a_n<m$ et $b_n<m$ donc par addition, $s_n<2m$ \cad $s_n<m$. Donc $(s_n)$ diverge vers $-\infty$ (même remarque que précédemment : prendre $m<0$ dans le cas de la divergence vers $-\infty$ suffit).}

\pro{\programme Si $(a_n)$ diverge vers $+\infty$ et $(b_n)$ diverge vers $-\infty$, $(s_n)$ peut soit converger (et pas nécessairement vers $0$) soit diverger (et pas nécessairement vers $\pm\infty$). Bref : \textbf{on ne peut rien dire}.}

\preuve{\begin{itemize}
\item Supposons que $(a_n)$ et $(b_n)$ soient définis pour tout $n\in\N$ par $a_n=n$ et $b_n=1-n$. $(a_n)$ est croissante et non majorée donc divergente vers $+\infty$ et de même, $(b_n)$ est divergente vers $-\infty$. Pour tout $n\in\N$, $s_n=1$ donc $(s_n)$ converge vers 1.
\item Supposons que $(a_n)$ et $(b_n)$ soient définis pour tout naturel $n$ pair par $a_n=n$ et $b_n=-n$ et pour tout naturel $n$ impair par $a_n=n-1$ et $b_n=-n-1$. Alors on montre par récurrence que $(a_n)$ est croissante et non majorée donc divergente vers $+\infty$, que $(b_n)$ est décroissante et non minorée donc divergente vers $-\infty$ et que que pour tout $n\in\N$, $s_n=2\times(-1)^n$ qui est divergente (et pas en $\pm\infty$ puisqu'elle est bornée).
\end{itemize}}
Finalement, voici une table récapitulative concernant la somme des limites :

\begin{tabular}{|c|c|c|c|}
\hline
\backslashbox{$\limc{n}{+\infty} b_n$}{$\limc{n}{+\infty} a_n$} & $\alpha\in\R$ & $+\infty$ & $-\infty$ \\ \hline
$\beta\in\R$ & $\alpha+\beta$ & $+\infty$ & $-\infty$ \\ \hline
$+\infty$ & $+\infty$ & $+\infty$ & Ind \\ \hline
$-\infty$ & $-\infty$ & Ind & $-\infty$ \\ \hline
\end{tabular}

\subsubsection{Multiplication de suites}

Dans cette sous-section $(p_n)_{n\in\N}$ est la suite définie pour tout $n\in\N$ par $p_n=a_nb_n$. Bien entendu, ce que nous allons voir ici s'applique à la division en considérant par exemple l'inverse de la suite $(b_n)$ \textit{pourvu qu'elle ne s'annule pas}.

\pro{\programme Si $(a_n)$ converge vers $\alpha\in\R$ et $(b_n)$ converge vers $\beta\in\R$ alors $(p_n)$ converge vers $\alpha\beta$.}



\preuve{Soit $(u_n)_{n\in\N}$ la suite définie pour tout $n\in\N$ par $u_n=a_n(b_n -\beta)$ et $(v_n)_{n\in\N}$ la suite définie pour tout $n\in\N$ par $v_n=(a_n-\alpha)\beta$. Montrons que $(u_n)$ converge vers 0. $(b_n)$ converge vers $\beta$ donc par addition, $\limc{n}{+\infty} (b_n-\beta)=\beta-\beta=0$. On sait de plus que $(a_n)$ converge donc le corollaire \ref{pnul} s'applique, donc $(u_n)$ converge vers 0. De même on montre que $(v_n)$ converge également vers 0. Donc par addition, $\limc{n}{+\infty}(u_n+v_n)=0$. Or pour tout $n\in\N$, $u_n+v_n=p_n-\alpha\beta$ donc $\limc{n}{+\infty}(p_n-\alpha\beta)=0$ donc par addition, $(p_n)$ converge vers $\alpha\beta$.}

\pro{\programme Si $(a_n)$ converge vers $\alpha\in\R_+^*$ et $(b_n)$ diverge vers $+\infty$ alors $(p_n)$ diverge vers $+\infty$.}

\preuve{Soit un réel $M>0$. Soit $r\in]0;\alpha[$. Posons $n_r\in\N$ \tq \pt entier $n\ge n_r$, $\alpha-r<a_n<\alpha+r$. Soit $q=\dfrac{M}{\alpha-r}>0$. Posons $n_q\in\N$ \tq \pt entier $n\ge n_q$, $q<b_n$. Posons $n_0=\max(n_r,n_q)$. Soit un entier $n\ge n_0$. Alors on a à la fois $0<\alpha-r<a_n$ et $0<q<b_n$ donc par multiplication termes à termes de ces deux inégalités, on obtient $M<p_n$, conclusion $p_n$ diverge vers $+\infty$.}

\pro{\programme Si $(a_n)$ diverge vers $+\infty$ et $(b_n)$ diverge vers $-\infty$ alors $(p_n)$ diverge vers $-\infty$.}

\preuve{Soit un réel $m<0$, un réel $M_a>0$ et $m_b=\dfrac{m}{M_q}<0$. Posons $n_a\in\N$ \tq \pt entier $n\ge n_a$, $a_n>M_a$. Posons $n_b\in\N$ \tq \pt entier $n\ge n_b$, $b_n<m_b$. Posons $n_0=\max(n_a,n_b)$. Soit un entier $n\ge n_0$. Alors on a à la fois $0<M_a<a_n$ et $b_n<m_b<0$ \ie $0<-m_b<-b_n$ donc par multiplication termes à termes de ces deux inégalités, on obtient $p_n<m$, conclusion $p_n$ diverge vers $-\infty$.}

\pro{\programme Si $(a_n)$ converge vers $0$ et $(b_n)$ diverge vers $+\infty$ alors \textbf{on ne peut rien dire} sur le comportement de $(p_n)$ en général.}

\ex{\begin{itemize}
\item Si pour tout $n\in\N$, $a_n=0$ et $b_n=n$ alors $(a_n)$ converge vers 0, $(b_n)$ diverge vers $+\infty$ et $(p_n)$ qui est constante à 0 \textcolor{blue}{converge vers 0}. 
\item Si pour tout $n\in\N$, $a_n=\dfrac{1}{n+1}$ et $b_n=n+1$ alors $(a_n)$ converge vers 0, $(b_n)$ diverge vers $+\infty$ et $(p_n)$ qui est constante à 1 \textcolor{blue}{converge vers 1}. 
\item Si pour tout $n\in\N$, $a_n=\dfrac{1}{n+1}$ et $b_n=(n+1)^2$ alors $(a_n)$ converge vers 0, $(b_n)$ diverge vers $+\infty$ et $(p_n)$ \textcolor{blue}{diverge vers $+\infty$}.
\item Si pour tout $n\in\N$, $a_n=\dfrac{(-1)^n}{n+1}$ et $b_n=n+1$ alors $(a_n)$ converge vers 0 (car produit d'une suite bornée et d'une suite convergente vers 0), $(b_n)$ diverge vers $+\infty$ et $(p_n)$ \textcolor{blue}{diverge}.
\end{itemize}}

En adaptant ces démonstrations aux autres cas, nous obtenons ainsi le tableau récapitulatif~:

\begin{tabular}{|c|c|c|c|c|c|}
\hline
\backslashbox{$\limc{n}{+\infty} b_n$}{$\limc{n}{+\infty} a_n$} & $\alpha<0$ & $\alpha=0$ & $\alpha>0$ & $-\infty$ & $+\infty$ \\ \hline
$\beta<0$ & $\alpha\beta$ & 0 & $\alpha\beta$ & $+\infty$ & $-\infty$ \\ \hline
$\beta=0$ & 0 & 0 & 0 & Ind & Ind \\ \hline
$\beta>0$ & $\alpha\beta$ & 0 & $\alpha\beta$ & $-\infty$ & $+\infty$ \\ \hline
$-\infty$ & $+\infty$ & Ind & $-\infty$ & $+\infty$ & $-\infty$ \\ \hline
$+\infty$ & $-\infty$ & Ind & $+\infty$ & $-\infty$ & $+\infty$ \\ \hline
\end{tabular}

\subsubsection{Quotient de suites}

\section{Suites remarquables}

En classe de Première vous avez rencontré deux types de suites : les suites \emph{arithmétiques} et les suites \emph{géométriques}. Nous allons rappeler leurs définitions et leurs propriétés. Mise en garde toutefois : les preuves ne peuvent être comprises qu'après avoir lu la section "comportement à l'infini des suites".

\subsubsection{Suites arithmétiques}

\defi{\programme (suites arithmétiques) Soit $r\in\R$ et $\alpha\in\R$. On appelle \emph{suite arithmétique de premier terme $\alpha$ et de raison $r$} la suite $(a_n)_{n\in\N}$ définie par $a_0=\alpha$ et tel que pour tout $n\in\N$, $a_{n+1}=a_n+r$.}

Il s'agit fondamentalement d'une suite dont la différence entre un terme et son précédent est constante et égale à $r$. La définition ci-dessus est bien sûr donnée par récurrence mais une telle suite peut aussi être définie explicitement comme suit.

\pro{\programme Soit $(a_n)_{n\in\N}$. $(a_n)$ est une suite arithmétique de premier terme $\alpha\in\R$ et de raison $r\in\R$ \ssi pour tout $n\in\N$, $a_n=\alpha+rn$.}

\preuve{Cela se montre par simple récurrence.}

Dans la pratique, pour déterminer si une suite donnée $(u_n)_{n\in\N}$ est arithmétique, on calcule pour tout $n\in\N$ la quantité $u_{n+1}-u_n$ et si cette dernière égale $r$ alors on peut affirmer que $(u_n)$ est arithmétique de raison $r$ et de premier terme $u_0$. Et pour déterminer un terme quelconque d'une telle suite, on va utiliser la définition explicite.

\exo{Soit $(v_n)_{n\in\N}$ la suite définie par $v_0=0$ et tel que pour $n\in\N$, $v_{n+1}=v_n+n+1$. Soit $(w_n)_{n\in\N}$ la suite définie pour tout $n\in\N$ par $w_n=\dfrac{n^2-1}{2}$. Enfin soit $(y_n)_{n\in\N}$ la suite définie pour tout $n\in\N$ par $y_n=v_n-w_n$. Montrer que $(y_n)$ est arithmétique (on précisera sa raison et son premier terme) puis montrer que $y_{99}=50$.}

\pro{\programme Soit $(a_n)_{n\in\N}$ une suite arithmétique de premier terme $\alpha\in\R$ et de raison $r\in\R$. Alors :

\begin{enumerate}[i)]
\item si $r=0$ alors $(a_n)$ est stationnaire et converge vers $\alpha$,
\item si $r>0$ alors $(a_n)$ est strictement croissante et diverge vers $+\infty$,
\item si $r<0$ alors $(a_n)$ est strictement décroissante et diverge vers $-\infty$
\end{enumerate}}

\rem{Les termes de convergence et divergence seront étudiées plus tard.}

\preuve{Cette preuve ne sera accessible qu'une fois la section "limite de suite" étudiée. Si $r=0$ alors pour tout $n\in\N$, $a_n=\alpha$ donc $(a_n)$ est stationnaire donc converge vers $\alpha$. Si $r>0$ alors pour tout $n\in\N$, $a_{n+1}-a_n=\alpha+r(n+1)-\alpha-rn=1$ donc $(a_n)$ est strictement croissante. Supposons que $(a_n)$ est majorée par $M\in\R$. L'équation $\alpha+rn>M$ a pour ensemble de solutions $S=\{n\in\N|n>\dfrac{M-\alpha}{r}\}$. Posons donc $n_0=\left\lceil \dfrac{M-\alpha}{r} \right\rceil+1$. Alors $n_0\in S$ donc $a_{n_0}>M$ ce qui est absurde. Donc $(a_n)$ n'est pas majorée et est de plus croissante, donc $(a_n)$ diverge vers $+\infty$. Le cas $r<0$ est similaire.}

Soit $(a_n)_{n\in\N}$ une suite et soit $(s_n)_{n\in\N}$ définie pour tout $n\in\N$ par $s_n=\somme{k=0}{n} a_n$.

\pro{\programme Si $(a_n)$ est une suite arithmétique de premier terme $\alpha\in\R$ et de raison $r\in\R$ alors pour tout $n\in\N$, $s_n=\dfrac{(n+1)(\alpha+a_n)}{2}$.}

\rem{\begin{itemize}
\item Pour le retenir, retenir la phrase \gug nombre de termes fois la moyenne des termes extrêmes\gud.
\item En effet, aussi surprenant que cela puisse paraître, $r$ n'apparaît pas dans la formule. 
\end{itemize}}

\preuve{Par récurrence simple.}

\subsubsection{Suites géométriques}

defi{\programme Soit $q\in\R$ et $\alpha\in\R$. On appelle \emph{suite géométrique de premier terme $\alpha$ et de raison $q$} la suite $(a_n)_{n\in\N}$ définie par $a_0=\alpha$ et tel que pour tout $n\in\N$, $a_{n+1}=q a_n$.}

\pro{\programme Soit $(a_n)_{n\in\N}$ géométrique de premier terme $\alpha$ et de raison $q$. Alors pour tout $n\in\N$, $a_n=\alpha q^n$.}

\preuve{Cela se montre par simple récurrence.}

\pro{\programme Soit $(a_n)_{n\in\N}$ une suite géométrique de premier terme $\alpha\in\R$ et de raison $q\in\R$. Alors :

\begin{enumerate}[i)]
\item Si $\alpha=0$ alors $(a_n)$ est constante et converge vers 0.
\item Si $\alpha > 0$ et :
\begin{enumerate}
\item si $|q|<1$ alors $(a_n)$ est décroissante (strictement si $q\neq0$) et converge vers 0,
\item si $q=1$ alors $(a_n)$ est constante et converge vers $\alpha$,
\item si $q\le -1$ alors $(a_n)$ diverge.
\item si $q>1$ alors $(a_n)$ est strictement croissante et diverge vers $+\infty$.
\end{enumerate}
\item Si $\alpha < 0$ et :
\begin{enumerate}
\item si $|q|<1$ alors $(a_n)$ converge vers 0 (\textbf{attention} : elle n'est croissante que si $q\in[0;1|$~!),
\item si $q=1$ alors $(a_n)$ est constante et converge vers $\alpha$,
\item si $q\le -1$ alors $(a_n)$ diverge.
\item si $q>1$ alors $(a_n)$ est strictement décroissante et diverge vers $-\infty$.
\end{enumerate}
\end{enumerate}}

\rem{Il n'est pas question d'apprendre tout ça par coeur. Ce qu'il faut retenir c'est que quel que soit le premier terme, si $q\in]-1;1]$ la suite converge et autrement elle diverge. Le reste se retrouve facilement en calculant les deux premiers termes.}

\preuve{Nous n'allons donner que les idées de la preuve qui serait beaucoup trop longue si nous détaillions.

\begin{enumerate}[i)]
\item C'est immédiat.
\item
\begin{enumerate}
\item On montre que la suite est décroissante et minorée par 0 donc par le théorème de convergence monotone, elle converge puis on montre par l'absurde que sa limite ne peut être supérieure à 0.
\item C'est immédiat.
\item Si $q=-1$ on montre qu'elle admet deux sous-suites de limites différentes, si $q<-1$ on montre que la valeur absolue de la suite n'est pas majorée, donc par le corollaire \ref{divabs} elle diverge.
\item On montre que la suite est strictement croissante et non majorée.
\end{enumerate}
\item
\begin{enumerate}
\item Si $q\in[0;1[$ on montre que la suite est croissante et majorée par 0 donc par le théorème de convergence monotone, elle converge puis on montre par l'absurde que sa limite ne peut être inférieure à 0. Si $q\in]-1;0[$, on montre que la suite $(|q^n|)_{n\in\N}$ converge vers 0 (par le théorème de convergence monotone...) donc par la proposition \ref{convabsz} et par multiplication par un scalaire, la suite converge vers 0.
\item C'est immédiat.
\item Si $q=-1$ on montre qu'elle admet deux sous-suites de limites différentes, si $q<-1$ on montre que la valeur absolue de la suite n'est pas majorée, donc par le corollaire \ref{divabs} elle diverge.
\item On montre que la suite est strictement décroissante et non minorée.
\end{enumerate}
\end{enumerate}}

\pro{\programme Si $(a_n)$ est une suite géométrique de premier terme $\alpha\in\R$ et de raison $q\in\R$ alors pour tout $n\in\N$, $s_n=\alpha\dfrac{(1-q)^{n+1}}{1-q}$ si $q\neq 1$ et $s_n=\alpha(n+1)$ si $q=1$.}

\rem{Pour le retenir, retenir la phrase... Heu...}

\preuve{Par récurrence simple.}
\chapter{Nombres complexes}

\defi{\programme L'ensemble des \textit{complexes}, noté $\C$, est l'ensemble des couples de réels $(x,y)$ que l'on muni des opérations suivantes :

\begin{itemize}
\item \textit{Addition} : si $(x',y')\in\C,(x,y)+(x',y')=(x+x',y+y')$
\item \textit{Multiplication} : si $(x',y')\in\C, (x,y)\times (x',y')=(xx'-yy',xy'+x'y)$
\end{itemize}}
\rems{\item La multiplication des complexes est donc finalement la seule opération différente de ce qu'on connaît pour les réels. Attention : même si elle est différente, on utilise quand même le même signe $\times$, prendre garde à ne pas confondre !
\item L'égalité de deux nombres complexes est vérifiée \ssi leurs composantes sont égales.
\item Donc quand on écrit «~soit $z\in\C$~», on écrit en réalité «~soit $x,y\in\R$ et notons $z$ le complexe $(x,y)$~». Attention, si on écrit simplement «~soit $x,y\in\R$ et notons $z=(x,y)$~», \textit{a priori} $z$ est juste un couple de réels, nous n'avons ici aucun moyen de savoir qu'il est complexe (donc muni des opérations ci-dessus).}

\rem{En classe de Terminale, les complexes seront probablement directement introduits avec la forme algébrique.}

\section{\texorpdfstring{$\C$ est un corps}{C est un corps}}

\pro{(sur l'addition) Soit $z,z',z''\in\C$. Alors :

\begin{enumerate}[i)]
\item $z+ (z' + z'')=(z+ z')+ z''$. On dit qu'il y a \textit{associativité} de l'addition.
\item $(0,0)+ z = z+ (0,0) = z$. On dit que $(0,0)$ est l'\textit{élément neutre} de $\C$ pour l'addition.
\item Pour tout $z\in\C$, il existe un unique $z_i\in\C$ tel que $z+ z_i = z_i+z = (0,0)$. On dit qu'il y a \textit{inversibilité} des éléments de $\C$ pour l'addition et que $z_i$ est \textit{l'inverse} de $z$ pour l'addition. De plus on a $z_i=(-x,-y)$.
\item $z+ z' = z'+ z$. On dit qu'il y a \textit{commutativité} de l'addition.
\end{enumerate}}

\rems{\item Les deux premiers points nous disent que $\C$ muni de l'addition est un \textit{monoïde}. Les trois premiers points nous disent que $\C$ muni de l'addition est un \textit{groupe}. Le quatrième nous dit que ce groupe est en plus \textit{commutatif}. Le Rubik's Cube peut être modélisé aussi par un groupe, mais qui lui n'est pas commutatif.
\item Si $z\in\C$, l'inverse de $z$ pour l'addition est généralement appelé «~opposé de $z$~» par analogie avec les réels.}

\preuve{Tous les points sauf le troisième se démontrent immédiatement en revenant à la définition de la multiplication de complexes. Montrons donc le point 3. Soit $z=(x,y)\in\C,z_i=(x',y')\in\C$. Résolvons $z+ z_i=(0,0)$ d'inconnu $z_i$, \cad le système

\sys{x+x' &=& 0 \\
y+y' &=& 0}

d'inconnues $x'$ et $y'$. Nous trouvons alors que $z_i$ doit nécessairement valoir $(-x,-y)$. Réciproquement, avec $z_i$ valant ce complexe, nous montrons que $z+ z_i = z_i+ z = (0,0)$.}

\pro{(sur la multiplication) Soit $z,z',z''\in\C$. Alors :

\begin{enumerate}[i)]
\item $z\times (z' \times z'')=(z\times z')\times z''$. On dit qu'il y a \textit{associativité} de la multiplication.
\item $(1,0)\times z = z\times (1,0) = z$. On dit que $(1,0)$ est l'\textit{élément neutre} de $\C$ pour la multiplication.
\item $z\times z' = z'\times z$. On dit qu'il y a \textit{commutativité} de la multiplication.
\end{enumerate}}

\rems{$\C$ muni de la multiplication est donc un monoïde. Mais attention : ce n'est \textit{pas} un groupe ! En effet, comme nous le verrons, le complexe $(0,0)$ n'a pas d'inverse. On ne peut donc \textit{pas} dire qu'il y a inversibilité des éléments de $\C$ pour la multiplication.}

\preuve{Se démontre immédiatement en revenant à la définition de la multiplication de complexes. }


\pro{(distributivité) Soit $z,z',z''\in\C$. Alors on a $z\times (z'+z'')=z\times z' + z\times z''$ et $(z'+z'')\times z=z'\times z + z''\times z$. On dit que la multiplication est \textit{ditributive} par rapport à l'addition.}
\preuve{En revenant la définition et en utilisant la commutativité de la multiplication (à noter qu'il n'est pas obligatoire d'utiliser la commutativité de la multiplication, mais cela permet d'aller plus vite).}
\rem{On a donc les éléments suivants :

\begin{itemize}
\item $\C$ muni de l'addition est un groupe commutatif.
\item $\C$ muni de la multiplication est un monoïde.
\item La multiplication est distributive par rapport à l'addition.
\end{itemize}
Ces éléments nous disent que $\C$ est un \textit{anneau}. Comme de plus la multiplication est commutative, on dit que $\C$ est un \textit{anneau commutatif}.}

\pro{\programme (inverse d'un complexe) \item Pour tout $z\in\C^*=\C\backslash\{(0,0)\}$, il existe un unique $z_i\in\C$ tel que $z\times z_i = z_i\times z = (1,0)$. On dit que $z_i$ est \textit{l'inverse} de $z$. De plus on a $z_i=\left(\dfrac{x}{x^2+y^2},-\dfrac{y}{x^2+y^2}\right)$. $(0,0)$ est le seul complexe à ne pas avoir d'inverse.}
\preuve{Soit $z=(x,y)\in\C^*,z_i=(x',y')\in\C$. Résolvons $z\times z_i=(1,0)$ d'inconnu $z_i$, \cad le système

\sys{xx'-yy' &=& 1 \\
xy'+x'y&=& 0}

d'inconnues $x'$ et $y'$. Nous trouvons alors que $z_i$ doit nécessairement valoir $\left(\dfrac{x}{x^2+y^2},-\dfrac{y}{x^2+y^2}\right)$. Réciproquement, avec $z_i$ valant ce complexe, nous montrons que $z\times z_i = z_i\times z = (1,0)$. Enfin, $(0,0)$ n'a pas d'inverse car pour tout $z_i\in\C$, $(0,0)\times z_i = (0,0)$, il n'existe donc pas de $z_i\in\C$ tel que $(0,0)\times z_i = (1,0)$.}

\rem{Cette dernière proposition, avec le fait que $\C$ est un anneau commutatif, nous dit que $\C$ est un \textit{corps}. Intuitivement, cela signifie que toutes les opérations entre les réels sont valables entre les complexes (en effet, $\R$ est également un corps). C'est pour cela qu'il y a un fort lien entre ces deux ensembles.}

\subsection{Simplication des notations}
Soit $z,z'\in\C$. De même que les réels, nous procédons à des raccourcis de notation en ce qui concerne la multiplication. Ainsi :

\begin{itemize}
\item On peut noter $zz'$ pour $z\times z'$.
\item Pour tout $n\in\N$, on peut noter $z^n$ pour $n$ multiplications de $z$ avec $z^0=(1,0)$.
\item Si $z\neq (0,0)$, on note $\dfrac{1}{z}$ ou encore $z^{-1}$ pour l'inverse de $z$.
\item Si $z\neq (0,0)$ et si $n\in\N$ on peut noter $z^{-n}$ pour $\left(\dfrac{1}{z}\right)^n$.
\end{itemize}
Remarquons que ces notations sont cohérentes avec celle que l'on utilise pour les réels. Ainsi, pour tout $z\in\C^*$, tout $a,b\in\Z$, $z^a z^b = z^{a+b}$. Cela se montre simplement par associativité de la multiplication.

\section{\texorpdfstring{Identification avec les réels et $i$}{Identification avec les reels et i}}

\pro{Soit $z=(x,0)\in\C$. Alors $z$ peut être identifié au réel $x$, on note alors $z=x$.}

\preuve{Nous ne donnons ici que deux arguments pour avoir une idée intuitive. Premier argument : soit $z=(a,0)\in\C$ et $z'=(b,0)\in\C$. Alors on peut vérifier que $z\times z'=(ab,0)$ et que $z+z'=(a+b,0)$, \cad que les complexes dont la partie imaginaire est nulle se comportent exactement comme les réels par rapport à l'addition et la multiplication. L'autre argument, c'est que les couples de la forme $(x,0)$ avec $x\in\R$ sont de dimension 1 (se représenter une droite) tout comme l'ensemble des réels, et conceptuellement il y a toujours moyen de passer d'une droite à une autre en identifiant leurs éléments.}

\pro{$\R\subset\C$}

\preuve{Soit $x\in\R$ et soit $z=(x,0)\in\C$. Par la proposition précédente, $z=x$ donc $x\in\C$.}

\rem{Cela signifie que tout réel \textit{est} un complexe (rigoureusement : peut être \textit{identifié} à un complexe). La réciproque n'est bien entendu pas vraie : $(1,1)\in\C$ est un couple de réels et n'est donc pas un réel.}

\pro{(multiplication par un scalaire) Soit $\lambda\in\R$ et $z=(x,y)\in\C$. Alors $\lambda z = (\lambda x, \lambda y)$.}

\preuve{$\lambda z=(\lambda,0)\times (x,y)=(\lambda x,\lambda y)$.}


\nota{On note $i$ le complexe $(0,1)$. On le nomme \textit{nombre imaginaire}.}

\theo{\programme $i^2=-1$}

\preuve{$i^2=i\times i=(0,1)\times (0,1)=(-1,0)=-1$}

\rems{\item On pourrait se dire que quelque chose cloche : un carré strictement négatif, ce n'est pas possible. Oui, mais il ne faut pas oublier que $i$ est un complexe et ne peut pas être identifié à un réel, or un complexe positif ou négatif... Cela n'a aucun sens~: on a défini aucune relation d'ordre sur les complexes. Donc en toute généralité, pour $z\in\C$, on ne peut surtout pas écrire $z^2\ge 0$ !

\item Deuxième remarque, aurait-on alors  $i=\sqrt{-1}$ ? Certains enseignants introduisent ainsi le nombre imaginaire mais j'estime que c'est une grosse erreur, à la fois pédagogique et mathématique. Pour commencer, il ne faut pas oublier que la racine carrée est une fonction $\R_+ \longrightarrow \R_+$. Donc le passage à la racine carrée n'est \textit{pas autorisée} si l'un des deux membres est négatif ! $\sqrt{-1}$ n'a aucun sens. D'autre part, même en admettant que cela soit possible, on aurait $\sqrt{i^2}=|i|$. Mais que signifie la valeur absolue d'un complexe (il s'agit bien ici d'une \textit{valeur absolue} et non d'un \textit{module} pour ceux qui connaîtraient le terme) ? Comparer un complexe avec un réel (0 pour la valeur absolue) ne veut rien dire non plus. En quel nom $i$ serait-il «~positif~» pour pouvoir écrire $i=\sqrt{-1}$ ? En conclusion, on voit bien que cette écriture pose énormément de problèmes mathématiques, c'est pourquoi je recommande chaudement à tout élève un temps soit peut attaché à sa crédibilité de ne jamais, au grand jamais, écrire quelque chose d'aussi abominablement affreux.}

\pro{\programme (forme algébrique) Soit $z=(x,y)\in\C$. Alors $z=x+iy$. On dit que $x+iy$ est la \textit{forme algébrique} de $z$.}

\preuve{$(x,y)=(x,0)+(0,y)=x+(0,1)\times y=x+iy$.}

Cette écriture permet d'effectuer facilement des calculs sur les complexes en utilisant uniquement les règles de calcul que l'on connaît des réels auxquelles on ajoute la règle $i^2=-1$.

\ex{Retrouvons la multiplication des complexes : soit $z=(x,y)\in\C$ et $z'=(x',y')\in\C$. $z=x+iy$ et $z'=x'+iy'$. Donc $zz'=(x+iy)(x'+iy')=xx'+xiy'+iyx'+i^2yy'=xx'-yy'+i(xy'+x'y)$. C'est bien le résultat attendu.}

\ex{Retrouvons l'inverse d'un complexe. Soit $z=x+iy\in\C^*$. Alors $\dfrac{1}{z}=\dfrac{1}{x+iy}=\dfrac{x-iy}{(x+iy)(x-iy)}=\dfrac{x-iy}{x^2+y^2}=\dfrac{x}{x^2+y^2}+i\left(-\dfrac{y}{x^2+y^2}\right)$ ce qui est le résultat attendu.}

\section{Représentation graphique et objets complexes}

Dans toute cete partie, soit $P$ un plan et $R=(O,{i},{j})$ un repère \textit{orthonormé} de $P$. Dans $P$, nous savons placer des points dont les coordonnées sont sous la forme d'un couple de réels. Soit un complexe $z=(x,y)$. $z$ étant lui-même un couple de réels (muni d'opérations particulières, mais un couple de réels quand même), on peut lui associer le point de $P$ de coordonnées $(x,y)$. Idem pour les vecteurs.

\nota{Dans cette partie, on note :
\begin{itemize}
\item Pour tout point $M\in P$, on note $M_x$ l'abscisse de $M$ et $M_y$ son ordonnée.
\item La distance euclidienne \fons{\dis}{\R^2\times \R^2}{\R_+}{(A,B)}{\sqrt{(B_y-A_y)^2+(B_x-A_x)^2}}.
\end{itemize}}

\defi{\programme (points associés aux complexes) \begin{itemize}
\item A tout complexe $z=x+iy\in\C$ on peut associer le point $M\in P$ de coordonnées $M(x,y)$.
\item Réciproquement, à tout point $M(x,y)$ du plan on peut associer le complexe $z=x+iy$.
\end{itemize}
On dit alors que $z$ est l'\textit{affixe} de $M$.}
\rems{\item Attention : il n'est pas question de sommer ou encore multiplier des points du plan sous prétexte qu'ils sont associés à des complexes ! En effet, un point est un élément de $\R^2$ et n'est donc pas muni des opérations valables pour les complexes. Si on doit faire des opérations, c'est donc entre leurs affixes et \textit{seulement} leurs affixes.
\item L'affixe de $O$ est donc $0$. Il faut s'habituer à ce que l'affixe de certains points soient associés à des réels.}

\defi{\programme (vecteurs associés aux complexes) \begin{itemize}
\item A tout complexe $z=x+iy\in\C$ on peut associer le vecteur $v$ du plan affine de coordonnées $(x,y)$.
\item Réciproquement, à tout vecteur $v=(x,y)$ du plan affine on peut associer le complexe $z=x+iy$.
On dit alors que $z$ est l'\textit{affixe} de $v$.
\end{itemize}}

\defi{\programme (parties réelle et imaginaire) Soit $z=(x,y)\in\C$. On dit que $x$ est la \textit{partie réelle} de $z$ que l'on note $\re(z)$. On dit que $y$ est la \textit{partie imaginaire} de $z$ que l'on note $\im(z)$.}

\pro{\programme Soit $z\in\C$ et $M$ le point d'affixe $z$.

\begin{itemize}
\item Si $\im(z)=0$ alors $M$ est sur l'axe des abscisses.
\item Si $\re(z)=0$ alors on dit que $z$ est un \textit{imaginaire pur}. $M$ est alors sur l'axe des ordonnées.
\end{itemize}}
\preuve{Immédiat.}

\defi{\programme (conjugué) Soit $z=a+ib\in\C$. On appelle \textit{conjugué} de $z$ le complexe $a-ib$ et on le note $\cjg{z}$.}

\pro{Soit $z\in\C$, $M$ et $M'$ les points d'affixes respectives $z$ et $\cjg{z}$. Alors $M$ et $M'$ sont symétriques par rapport à l'axe des abscisses.}

\preuve{Considérons que $z=a+ib$. Alors $M(a,b)$ et $M'(a,-b)$. Ces points ont donc la même abscisse et se situent donc sur la droite d'équation $x=a$. Soit $H(a,0)$. On a $\dis(H,M)=\dis(H,M')=|b|$, d'où le résultat.}

\defi{\programme (module, argument) Soit $z=x+iy\in\C$ et $M$ le point d'affixe $z$.
\begin{itemize}
\item On appelle \textit{module} de $z$ la quantité $\dis(O,M)$. On le note $|z|$.
\item On appelle \textit{argument} de $z$ l'angle orienté $({i},\vr{OM})$. On le note $\arg(z)$. Si $M$ et $O$ sont confondus, \ie $z=0$, on décide par convention que $\arg(z)=0$.
\end{itemize}}

\rems{\item On a donc ici $\dis(O,M)=\sqrt{x^2+y^2}$.
\item On note le module comme pour la valeur absolue d'un réel car si $z\in\C$ est un réel, alors le module de $z$ et la valeur absolue de $z$ valent la même chose. Le module est donc une généralisation de la valeur absolue pour les nombres complexes. Par conséquent, dans tout ce chapitre, quand on voit le symbole $|\cdot|$ il est vivement conseillé de considérer \textit{a priori} qu'il s'agit d'un module et non d'une valeur absolue, cela évitera d'écrire des bêtises.
\item L'argument étant un angle orienté, si $\theta$ est une mesure de $\arg(z)$ alors $A=\ens{\theta+2k\pi,k\in\Z}$ est l'ensemble de toutes les mesures de $\arg(z)$. Il existe un unique $\alpha\in A$ tel que $\alpha\in]-\pi,\pi]$, on l'appelle \textit{mesure principale} de $\arg(z)$.}

\pro{\label{argu} (comment trouver un argument) Soit $z\in\C$, $M$ le point d'affixe $z$ et supposons $|z|\neq 0$. Posons $\Theta=\arccos\left(\dfrac{\re(z)}{|z|}\right)$. Selon le demi-plan où se situe $M$, on en déduit une mesure de $\arg(z)$ :

\begin{itemize}
\item Si $M$ est au-dessus de l'axe des abscisses ($\im(z)\ge 0$), $\arg(z)=\Theta$.
\item Si $M$ est en-dessous de l'axe des abscisses ($\im(z)< 0$), $\arg(z)=-\Theta$.
\end{itemize}
De plus, cette mesure de $\arg(z)$ est sa mesure principale.}

\preuve{Considérons que $z=x+iy$. Supposons $|z|\neq 0$. Notons $\theta=\arg(z)$. Soit $M'$ l'intersection entre la demie-droite $[OM)$ et le cercle de centre $O$ et de rayon 1. En résvolvant un système d'équations on trouve que pour tout $x\neq 0$, $|M'_x|=\dfrac{|x|}{|z|}$. De plus on a $\theta=({i},\vr{OM})=({i},\vr{OM'})$ et donc $M'_x=\cos\theta$. Par conséquent pour tout $x\neq 0$ on a $|\cos\theta|=\dfrac{|x|}{|z|}$. Si $x>0$ alors $\cos\theta>0$ donc $\cos\theta=\dfrac{x}{|z|}$ et si $x<0$ alors $\cos\theta<0$ donc on a également $\cos\theta=\dfrac{x}{|z|}$. Enfin si $x=0$ alors $\cos\theta=0$ donc on a aussi $\cos\theta=\dfrac{x}{|z|}$. Donc pour tout $x\in\R$ on a $\cos\theta=\dfrac{x}{|z|}$. Nous avons donc $\arccos(\cos\theta))=\arccos\left(\dfrac{x}{|z|}\right)$. C'est le moment de rappeler l'énorme piège : il est faux de dire que pour tout $\theta\in\R$, $\arccos(\cos(\theta))=\theta$. En effet, \fone{\arccos}{[-1,1]}{[0,\pi]} donc cette égalité est vraie seulement si $\theta\in[0,\pi]$. De plus, si $\theta\in[-\pi,0]$ alors $\arccos(\cos(\theta))=-\theta$ (car $x\mapsto \arccos(\cos(x))$ est paire car $x\mapsto \cos(x)$ l'est elle-même). En conséquence, si $y\ge 0$ alors $\theta\in  [0;\pi]$ et donc on a effectivement $\theta=\arccos\left(\dfrac{x}{|z|}\right)$. Si $y<0$ alors $\theta\in]-\pi,0[$ donc on a $-\theta=\arccos\left(\dfrac{x}{|z|}\right)$, d'où le résultat. Enfin, pour tout $y\in\R$ on a $\theta\in]-\pi,\pi]$, $\theta$ est donc la mesure principale de $\arg(z)$.}

\ex{Soit $z=-3-7i$. Alors $|z|=\sqrt{(-3)^2+(-7)^2}=\sqrt{58}\approx 7.6$. Donc $\Theta=\arccos\left(\dfrac{-3}{\sqrt{58}}\right)$ et puisque $\im(z)<0$, alors on en déduit que $\arg(z)=-\arccos\left(-\dfrac{3}{\sqrt{58}}\right)\approx -1.98$ qui est la mesure principale de $\arg(z)$.}

\lemme{Soit $z\in\C^*$ et $\lambda\in\R$.
\begin{itemize}
\item Si $\lambda>0$ alors $\arg(\lambda z) = \arg(z)$
\item Si $\lambda=0$ alors $\arg(\lambda z)=0$
\item Si $\lambda <0$ alors $\arg(\lambda z) = \arg(z)+\pi$.
\end{itemize}}

\rem{Si $z=0$ alors $\arg(\lambda z)=\arg 0 = 0$ et $\arg(z)+\pi=\pi$ ce qui invalide le troisième point. D'où il est important que $z\neq 0$.}

\preuve{Le cas $\lambda=0$ est immédiat. Supposons $\lambda\neq 0$. Soit $M$ le point d'affixe $z$. Soit $M'$ le point d'affixe $\lambda z$. $\vr{OM}$ et $\vr{OM'}$ sont colinéaires. Si $\lambda>0$ alors ils ont en plus le même sens et donc $\arg(\lambda z)=({i},\vr{OM'})=({i},\vr{OM})=\arg(z)$. Si $\lambda<0$ ils sont de sens contraire et donc $\arg(\lambda z)=({i},\vr{OM'})=({i},\vr{OM})+\pi=\arg(z)+\pi$.}

\pro{A tout couple $(\rho,\theta)\in\R_+\times\R$ on peut associer un unique complexe $z$ tel que $|z|=\rho$ et $\arg(z)=\theta$ et on a alors $z=\rho(\cos\theta+i\sin\theta)$.}

\preuve{Analyse : soit $(\rho,\theta)\in\R_+\times\R$ et supposons l'existence d'un $z\in\C$ tel que $|z|=\rho$ et $\arg(z)=\theta$. Soit $M$ le point d'affixe $z$. $|z|=\rho$ donc $M$ est sur le cercle $C_\rho$ de centre $O$ et de rayon $\rho$. On sait de plus que $\arg(z)=\theta$. Soit $M'(\cos\theta,\sin\theta)$. Alors $\arg(z)=({i},\vr{OM'})$ donc nécessairement $M\in[OM')$. Donc $M\in C_\rho\cap [OM')$. Ce dernier ensemble est réduit à un seul élément, donc si il existe un tel complexe, il est unique. Synthèse : soit  $z=\rho(\cos\theta+i\sin\theta)$. Alors on vérifie facilement que $z$ vérifie $|z|=\rho$ et $\arg(z)=\theta$ en utilisant le lemme précédent. La synthèse nous donne l'existence, l'analyse l'unicité.}

\pro{\programme (forme trigonométrique) Pour tout $z\in\C$ il existe $\theta\in\R$ et un unique $\rho\in\R_+$ tel que $|z|=\rho$ et $\arg(z)=\theta$. On a alors $z=\rho(\cos\theta+i\sin(\theta))$. On appelle cette écriture \textit{forme trigonométrique} de $z$.}

\preuve{Soit $z=x+iy\in\C$. On a $|z|=\sqrt{x^2+y^2}$ donc le module de $z$ est unique. Posons $\rho=|z|$. D'après la proposition \ref{argu} il existe $\theta\in\R$ tel que $\arg(z)=\theta$. Enfin, d'après la proposition précédente on a bien $z=\rho(\cos\theta+i\sin(\theta))$.}

\rem{Attention, il est faux de dire que pour tout $z\in\C$ il existe un unique couple $(\rho,\theta)\in\R_+\times\R$ tel que $|z|=\rho$ et $\arg(z)=\theta$, car il y a une infinité de mesures du même angle. En revanche si on réduit à la mesure principale, l'unicité est vraie cette fois : pour tout $z\in\C$ il existe un unique couple $(\rho,\theta)\in\R_+\times ]-\pi,\pi]$ tel que $|z|=\rho$ et $\arg(z)=\theta$.}

\defi{\programme (forme exponentielle) Soit $(\rho,\theta)\in\R_+\times\R$ et soit $z$ l'unique complexe tel que $|z|=\rho$ et $\arg(z)=\theta$. Alors on note $z=\rho e^{i\theta}$ que l'on nomme \textit{forme exponentielle} de $z$.}

\rem{Nous justifierons plus tard en quoi cette écriture est cohérente.}

\nota{(notations diverses)

\begin{itemize}
\item On note $\mathbf{U}=\{z\in\C,|z|=1\}=\{e^{i\theta},\theta\in\R\}$
\item On note $j=e^{i\dfrac{2\pi}{3}}$. On a $j\in\mathbf{U}$.
\end{itemize}}

\rem{Soit $C$ l'ensemble des points d'affixes dans $\U$. Alors $C$ est le cercle de centre $O$ et de rayon 1.}

\subsection{Commandes GeoGebra et XCas}
Mise en garde : il semblerait qu'une grande partie des commandes XCas concernant les complexes fonctionnent mal voire pas du tout avec la version de bureau pour Ubuntu. Si c'est le cas pour le lecteur, deux solutions~: soit utiliser la \href{https://www.xcasenligne.fr/giac_online/demoGiacPhp.php}{version en ligne} qui marche très bien, soit ne faire les calculs complexes qu'avec GeoGebra.

\begin{tabular}{|c|c|c|} \hline
Notation du cours & GeoGebra & XCas \\ \hline
Le point $M(2,-3)$ & \verb+M=(2,-3)+ & \verb+M:=point(2,-3)+ \\ \hline
$M_x$ où $M\in P$ & \verb+x(M)+ & \verb+abscissa(M)+ \\ \hline
$M_y$ où $M\in P$ & \verb+y(M)+ & \verb+ordinate(M)+ \\ \hline
Le vecteur $v=(6,-1)$ & \verb+v=vector((6,-1))+ & \verb+v:=[6,-1]+ \\ \hline
$z=4-3i$ & \verb+c=4-3i+ (la notation $z$ est réservée) & \verb+z:=4-3i+ \\ \hline
Affixe du point $M$ & ? & \verb+affix(M)+ \\ \hline
Affixe du vecteur $v$ & ? & \verb+affix(v)+ \\ \hline
$\re(z)$ & \verb+real(c)+ ou \verb+x(c)+ & \verb+re(z)+ \\ \hline
$\im(z)$ & \verb+imaginary(c)+ ou \verb+y(c)+ & \verb+im(z)+ \\ \hline
$\cjg{z}$ & \verb+conjugate(c)+ & \verb+conj(z)+ \\ \hline
$|z|$ & \verb+|c|+ ou \verb+abs(c)+ & \verb+abs(z)+ \\ \hline
$\arg(z)$ & \verb+arg(c)+ & \verb+arg(z)+ \\ \hline
$7e^{i\dfrac{\pi}{3}}$ & \verb+7*e^(i*pi/3)+ & \verb+7*e^(i*pi/3)+ \\ \hline
\end{tabular}

Il semblerait que GeoGebra ait parfois du mal avec la notation exponentielle : donc toujours vérifier que le complexe qu'il crée est cohérent. En cas de problème, entrer la forme trigonométrique qui est équivalente. 

\subsection{Exercice récapitulatif}

Programmer deux procédures :

\begin{itemize}
\item \verb+cxalg(t tableau de deux réels)+ qui affiche dans l'ordre, où $a=t[0]$, $b=t[1]$ et $z=a+ib$ :
\begin{enumerate}
\item $|z|$,
\item $\arg(z)$ en mesure principale,
\item $\cjg{z}$ dans un format exportable sur XCas,
\item la forme exponentielle de $z$ dans un format exportable sur XCas. 
\end{enumerate}
\item \verb+cxtri(t tableau de deux réels, le premier étant positif)+ qui affiche dans l'ordre, où $\rho=t[0]$, $\theta=t[1]$ et $z=\rho e^{i\theta}$ :
\begin{enumerate}
\item $\re(z)$
\item $\im(z)$
\item la forme algébrique de $z$ dans un format exportable sur XCas,
\item $\cjg{z}$ dans un format exportable sur XCas,
\item la forme exponentielle de $z$ dans un format exportable sur XCas. 
\end{enumerate}
\end{itemize}

Programmer les quatre fonctions :

\begin{enumerate}
\item \verb+cxsca(t1, t2 tableaux de deux reels chacun, s reel)+ qui modifie les éléments de $t_2$ comme suit : $t_2[0]=s\re(z)$ et $t_2[1]=s\im(z)$ où $z=t_1[0]+it_1[1]$.
\item \verb+cxsum(t1, t2, t3 tableaux de deux reels chacun)+ qui modifie les éléments de $t_3$ comme suit : $t_3[0]=\re(z_1+z_2)$ et $t_3[1]=\im(z_1+z_2)$ où $z_1=t_1[0]+it_1[1]$ et $z_2=t_2[0]+it_2[1]$.
\item \verb+cxpro(t1, t2, t3 tableaux de deux reels chacun)+ qui modifie les éléments de $t_3$ comme suit : $t_3[0]=\re(z_1z_2)$ et $t_3[1]=\im(z_1z_2)$ où $z_1=t_1[0]+it_1[1]$ et $z_2=t_2[0]+it_2[1]$.
\item \verb+cxquo(t1, t2, t3 tableaux de deux reels chacun)+ qui modifie les éléments de $t_3$ comme suit : $t_3[0]=\re(z_1/z_2)$ et $t_3[1]=\im(z_1/z_2)$ où $z_1=t_1[0]+it_1[1]$ et $z_2=t_2[0]+it_2[1]$. On supposera que $z_2\neq 0$.
\end{enumerate}

\section{Règles de calcul et propriétés}
La plupart des propositions ci-dessous étant simples à démontrer par calcul direct, nous ne signalerons la preuve que de celles moins évidentes. Dans toute cette partie, sauf mention contraire, $z,z'\in\C$ et $\lambda\in\R$.

\pro{(sur $i$) pour tout $k\in\N$,

\begin{tabular}{|c|c|c|} \hline
$n$ & $i^n$ & $i^{-n}$ \tend
$4k$ & $1$ & $1$ \tend
$4k+1$ & $i$ & $-i$ \tend
$4k+2$ & $-1$ & $-1$ \tend
$4k+3$ & $-i$ & $i$ \tend
\end{tabular}}

\pro{(sur les vecteurs) Soit $u,v$ deux vecteurs d'affixes respectives $z_u$ et $z_v$. Soit $A$, $B$ deux points de $P$ d'affixes $z_A$ et $z_B$.

\begin{itemize}
\item Le vecteur $u+\lambda v$ a pour affixe $z_u+\lambda z_v$.
\item Le vecteur $\vr{AB}$ a pour affixe $z_B-z_A$.
\end{itemize}}

\pro{(sur $\re$ et $\im$)

\begin{itemize}\label{reim}
\item $z=z'\eqv \re(z)=\re(z')$ et $\im(z)=\im(z')$
\item $\re(z)=\dfrac{z+\cjg{z}}{2}$
\item $\im(z)=\dfrac{z-\cjg{z}}{2i}$
\item $|\re(z)|\le |z|$
\item $|\im(z)|\le |z|$

\end{itemize}}

\pro{(sur le conjugué)

\begin{itemize}
\item $\cjg{\cjg{z}}=z$
\item $\cjg{\lambda z} = \lambda\cjg{z}$
\item $\cjg{z+z'}=\cjg{z}+\cjg{z'}$
\item $\cjg{zz'}=\cjg{z}\cjg{z'}$
\item Si $z'\neq 0$, $\cjg{\left(\dfrac{z}{z'}\right)}=\dfrac{\cjg{z}}{\cjg{z'}}$
\item \programme $z\cjg{z}=|z|^2$
\end{itemize}}

\pro{(sur le module)

\begin{enumerate}
\item $|z|=0\eqv z=0$
\item $|\cjg{z}|=|z|$
\item (inégalité triangulaire) $|z+z'|\le |z|+|z'|$
\item $|\lambda z| = |\lambda| |z|$
\item $|zz'|=|z||z'|$
\item Si $z\neq 0$ alors pour tout $k\in\Z$, $|z^k|=|z|^k$
\item Si $z'\neq 0$, $\left|\dfrac{z}{z'}\right|=\dfrac{|z|}{|z'|}$
\end{enumerate}}

\rem{Si $z,z'\neq 0$ alors le cas d'égalité de l'IT, \ie $|z+z'|=|z|+|z'|$, se produit \ssi $\arg z =\arg z'$.}

\preuve{\begin{itemize}
\item Montrons l'inégalité triangulaire (IT). Posons $d=(|z|+|z'|)^2-|z+z'|^2$. Remarquons que montrer l'IT équivaut à montrer que $d\ge 0$. $d = |z|^2+|z'|^2+2|zz'|-(z+z')(\cjg{z}+\cjg{z'})=2|zz'|-z\cjg{z'}-z'\cjg{z}=2(|zz'|-\re(z\cjg{z'}))$. Posons $Z=z\cjg{z'}$. Alors $d=2(|Z|-\re(Z))$ or $\re(Z)\le |\re(Z)|\le |Z|$ donc $d\ge 0$.
\item Montrons la proposition 6. Soit $k\in\N$. Alors la proposition est immédiate par récurrence en utilisant la proposition 5. Ensuite, $|z^{-k}|=\left|\dfrac{1}{z^k}\right|$. D'après la proposition 7, $\left|\dfrac{1}{z^k}\right|=\dfrac{1}{|z^k|}=\dfrac{1}{|z|^k}=|z|^{-k}$. Nous avons donc montré la proposition pour $k$ entier négatif.
\end{itemize}}



\pro{(sur l'argument)

\begin{enumerate}
\item Soit $z\neq0$. \begin{itemize}
\item Si $\lambda>0$ alors $\arg(\lambda z) = \arg(z)$
\item Si $\lambda=0$ alors $\arg(\lambda z)=0$
\item Si $\lambda <0$ alors $\arg(\lambda z) = \arg(z)+\pi$
\end{itemize}

\item Si $z,z'\neq 0$, $\arg{zz'}=\arg{z}+\arg{z'}$
\item Si $z\neq 0$, $\arg z^{-1}=-\arg z$
\item Si $z\neq 0$ alors pour tout $k\in\Z$, $\arg z^k=k\arg z$
\item Si $z,z'\neq 0$, $\arg\dfrac{z}{z'}=\arg z - \arg z'$
\item $\arg \cjg{z}=-\arg z$.
\end{enumerate}}

\rem{Montrons l'importance que tous ces complexes doivent être non nuls. Soit $z=0$ et $z'=i$. Alors :

\begin{itemize}
\item $\arg{zz'}=\arg{0}=0$ et $\arg z + \arg z' = 0+\dfrac{\pi}{2}=\dfrac{\pi}{2}$ : invalide la proposition 2.
\item L'inverse de $0$ n'existe pas donc invalide la proposition 3.
\item Pour la même raison, invalide la proposition 4 si $k$ est négatif.
\end{itemize}}

\preuve{\begin{enumerate}
\item Nous l'avons montré dans la section précédente.
\item Il existe $\rho,\rho'\in\R_+^*$ et $\theta,\theta'\in\R$ tel que $z=\rho(\cos\theta+i\sin\theta)$ et $z'=\rho'(\cos\theta'+i\sin\theta')$. En développant $zz'$ et en appliquant une formule de trigonométrie, on trouve $zz'=\rho\rho'[\cos(\theta+\theta')+i\sin(\theta+\theta')]$. Or $\rho\rho'>0$ donc d'après la proposition 1, $\arg zz'=\arg[\cos(\theta+\theta')+i\sin(\theta+\theta')]=\theta+\theta'=\arg z+\arg z'$.
\item D'une part $\arg(zz^{-1})=\arg 1=0$ et d'autre part par la proposition 2 $\arg(zz^{-1})=\arg z + \arg z^{-1}$ donc $\arg z^{-1} = -\arg z$.
\item Soit $k\in\N$. La proposition est immédiate par récurrence en utilisant la proposition 2. Ensuite $\arg(z^{-k})=\arg((z^k)^{-1})$. D'après la proposition 3 $\arg((z^k)^{-1})=-\arg z^k=-k\arg z$, on a donc montré la proposition pour $k$ entier négatif.
\item $\arg \dfrac{z}{z'}=\arg(zz'^{-1})=\arg z + \arg z'^{-1}=\arg z - \arg z'$.
\item Soit $M$ et $M'$ les points d'affixes respectives $z$ et $\cjg{z}$. $M$ et $M'$ sont symétriques par rapport à l'axe des abscisses, d'où le résultat.
\end{enumerate}

}

\pro{(sur la forme exponentielle)
Soit $\rho,\rho'\in\R_+$ et $\theta,\theta'\in\R$ tel que $z=\rho e^{i\theta}$ et $z'=\rho' e^{i\theta'}$.

\begin{enumerate}
\item $zz'=\rho\rho' e^{i(\theta+\theta')}$
\item Si $z'\neq 0$, $\dfrac{z}{z'}=\dfrac{\rho}{\rho'}e^{i(\theta-\theta')}$
\item Si $z\neq0$ alors pour tout $k\in\Z$, $z^k=\rho^k e^{ik\theta}$
\item $\cjg z = \rho e^{-i\theta}$
\item $z=z' \eqv \rho = \rho'$ et il existe $k\in\Z$ tel que $\theta = \theta'+2k\pi$
\end{enumerate}}

\preuve{\begin{enumerate}
\item Si $z$ ou $z'$ est nul, alors $zz'=0$ et donc la proposition est vérifiée. Considérons $z,z'\neq 0$. On a $|zz'|=|z||z'|=\rho\rho'$ et $\arg(zz')=\theta+\theta'$.
\item Si $z=0$ alors $\dfrac{z}{z'}=0$ et donc la proposition est vérifiée. Considérons $z\neq0$. On a $\left|\dfrac{z}{z'}\right|=\dfrac{|z|}{|z'|}=\dfrac{\rho}{\rho'}$ et $\arg\dfrac{z}{z'}=\theta-\theta'$.
\item $|z^k|=|z|^k=\rho^k$ et $\arg z^k = k\arg z = k\theta$.
\item $|\cjg{z}|=|z|=\rho$ et $\arg \cjg{z} = -\arg z = -\theta$.
\item 

\begin{itemize}
\item ($\Leftarrow$) On a $\re(z)=\rho\cos\theta=\rho'\cos\theta'=\re(z')$ et $\im(z)=\rho\sin\theta=\rho'\sin\theta'=\im(z')$. Donc $z=z'$ (voir propriétés sur $\re$ et $\im$).
\item ($\Rightarrow$) Si $z=z'=0$ le résultat est immédiat. Supposons $z,z'\neq 0$. $z=z'$ donc $\dfrac{z}{z'}=1$. D'après le second point, on a en outre $\dfrac{z}{z'}=\dfrac{\rho}{\rho'}e^{i(\theta-\theta')}$. Le module d'un complexe étant unique, alors $\abs{\dfrac{z}{z'}}=\abs{\dfrac{\rho}{\rho'}}=1$ d'où $\rho=\rho'$. $\arg\dfrac{z}{z'}=\arg 1 = 0$ et en outre $\arg\dfrac{z}{z'}=\theta-\theta'$. L'argument d'un complexe étant unique à $2\pi$ près, il existe $k\in\Z$ tel que $0=\theta-\theta'+2k\pi$.
\end{itemize}
\end{enumerate}}

\rem{Cette dernière série de propositions montre en fait la cohérence de la forme exponentielle. En effet, nous verrons sur le chapitre de l'exponentielle qui est, par définition, la fonction \fone{\exp}{\R}{\R_+^*} tel que sa dérivée égale elle-même et $e^0=1$ que pour tout $a,b\in\R$, $e^a e^b = e^{a+b}$ et que $\dfrac{e^a}{e^b}=e^{a-b}$. Donc, en quelque sorte, on peut étendre le comportement de l'exponentielle sur les complexes. Mais \textit{attention}, sans jamais oublier qu'il ne s'agit que d'une \textit{analogie}, la véritable fonction exponentielle n'étant définie que sur les réels ! En résumé, on peut effectuer des calculs sur la forme exponentielle comme si on avait affaire à la véritable fonction exponentielle, sans oublier que ce n'est jamais qu'une analogie.}

\cor{(formule de Moivre) Pour tout $n\in\N$ on a $(e^{i\theta})^n=e^{in\theta}=\cos(n\theta)+i\sin(n\theta)$.}

\preuve{C'est le cas particulier du troisième point de la proposition précédente avec $\rho=1$ et $k$ positif.}

\pro{(identité d'Euler) $e^{i\pi}+1=0$}

\rem{Cette jolie formule, en réalité pas très utile dans ce chapitre et ni difficile à montrer, est pourtant importante dans la culture mathématique. En effet, de nombreux matheux sont forts admiratifs (pour ne pas dire qu'ils vouent un culte) de cette formule à plusieurs titres :

\begin{itemize}
\item Elle fait intervenir les quantités $e$, $i$, $\pi$, $0$ et $1$ qui sont les représentants de base de leurs domaines respectifs : l'analyse réelle, l'analyse complexe, la trigonométrie et l'algèbre.
\item Elle établi un pont entre tous les domaines cités ci-dessus.
\item Elle fait intervenir les trois opérations de base : l'addition, la multiplication et l'exponentiation.
\item Elle est sobre et courte, donc esthétique.
\end{itemize}}

\preuve{Immédiat en passant à la forme trigonométrique.}

\pro{(formules d'Euler) Pour tout $\theta\in\R$,

\begin{itemize}
\item $\cos \theta=\dfrac{e^{i\theta}+e^{-i\theta}}{2}$
\item $\sin \theta =\dfrac{e^{i\theta}-e^{-i\theta}}{2i}$
\end{itemize}}

\preuve{C'est un cas particulier des deux premiers points de la proposition \ref{reim} en prenant $z=e^{i\theta}$.}

\exo{Soit $\theta\in\R$. En utilisant ces formules et le binôme de Newton, linéariser $\sin^4\theta$, \cad écrire cette quantité sans multiplications de fonctions trigonométriques, puis calculer une primitive de $x\mapsto \sin^4 x$ (on admettra la dérivée d'une composée de fonctions). On peut vérifier en tapant {\fontfamily{cmtt}\selectfont lineariser\_trigo(sin(theta)**4)} sur XCas.}

\pro{(sur $\U$) Soit $z,z'\in\U$.

\begin{enumerate}
\item $zz'\in\U$.
\item $1\in\U$.
\item $z^{-1}\in\U$ et $z^{-1}=\cjg z$
\end{enumerate}}

\rem{Ces propositions nous disent que $\U$ muni de la multiplication est un groupe dont le neutre est 1 (l'associativité des complexes étant toujours vérifiée, elle l'est en particulier aussi dans $\U$).}

\section{\texorpdfstring{Racine $n$-ième réelle}{Racine n-ieme reelle}}

On pose dans toute cette partie $n\in\N^*$.

\subsection{Pour les positifs}
\label{puirat}

\defi{Pour tout $r\in\R_+$, on appelle \textit{racine} $n$\textit{-ième} de $r$ l'unique réel positif $x$ tel que $x^n=r$. On le note $\sqrt[n]{r}$.}

\preuve{Si $r=0$, alors $x=0$ est une solution. Si $x\neq0$ alors $x^n\neq 0$ donc $x$ n'est pas solution, donc il y a existence et unicité de $x$ pour $r=0$. Supposons $r\neq0$. Prouvons l'existence et l'unicité de $x$. Supposons l'existence d'un tel $x$. Alors nécessairement $x\neq 0$. On a $x^n=r$ \ie $\ln(x^n)=\ln r$ \ie $x=\exp\parent{\dfrac{\ln r}{n}}$. Donc s'il y a existence de $x$, $x$ est unique. Posons alors $x=\exp\parent{\dfrac{\ln r}{n}}$. On vérifie facilement que $x^n=r$. Conclusion il y a bien existence et unicité de $x$.}

\defi{On peut donc définir la fonction \textit{racine} $n$\textit{-ième} ainsi : \fons{\sqrt[n]{\cdot}}{\R_+}{\R_+}{x}{\sqrt[n]{x}}}

\pro{(propriétés) \begin{enumerate}[i)]
\item Elle est bijective et sa réciproque est la fonction polynomiale \fonsn{\R_+}{\R_+}{x}{x^n}
\item (caractérisation exponentielle) si $x\in\R_+\sauf{0}$, $\sqrt[n]{x}=\exp\parent{\dfrac{\ln x}{n}}$
\item Elle est continue sur $\R_+$, dérivable sur $\R_+^*$ et pour tout $x\in\R_+^*$, $(\sqrt[n]{x})'=\dfrac{\sqrt[n]{x}}{nx}$.
\item Elle est strictement croissante sur $\R_+$.
\item $\limc{x}{+\infty}\sqrt[n]{x}=+\infty$
\end{enumerate}}

\preuve{\begin{enumerate}[i)]
\item Soit $x\in\R_+$. Par définition, $(\sqrt[n]{x})^n=x$. Si $x=0$ on a $\sqrt[n]{x^n}=0=x$. Supposons $x\neq0$. On a, toujours par définition, $(\sqrt[n]{x^n})^n=x^n$ \cad $\parent{\dfrac{\sqrt[n]{x^n}}{x}}^n=1$ \cad $\dfrac{\sqrt[n]{x^n}}{x}=1$ \cad $\sqrt[n]{x^n}=x$.
\item Voir preuve précédente.
\item La réciproque d'une fonction continue est continue. La dérivée se trouve en utilisant la caractérisation exponentielle (dérivée d'une composée).
\item Utiliser la caractéristaion exponentielle.
\item Idem.
\end{enumerate}}

\pro{(calculs) Soit $x\in\R_+$.

\begin{enumerate}[i)]
\item $\sqrt[1]{x}=x$
\item $\sqrt[n]{0}=0$
\item $\sqrt[n]{1}=1$
\item $\sqrt[2]{x}=\sqrt{x}$ (racine carrée usuelle)
\item $(\sqrt[n]{x})^n=x$
\item $\sqrt[n]{x^n}=x$
\end{enumerate}}

\preuve{Elles sont immédiates.}

\nota{\begin{itemize}
\item Pour tout $x\in\R_+\sauf{0}$ on note $x^{1/n}=\sqrt[n]{x}$.
\item Pour tout $x\in\R_+\sauf{0}$, $p\in\Z$, $q\in\N^*$, on note $x^{p/q}=(\sqrt[q]{x})^p$.
\end{itemize}}

\rem{Ces notations sont justifiées par le théorème suivant.} 

\theo{(puissances rationnelles) Soit $x\in\R_+\sauf{0}$. Soit $p,q\in\Q$.

\begin{itemize}
\item $x^p x^q=x^{p+q}$
\item $(x^p)^q=(x^q)^p=x^{pq}$
\item $\dfrac{x^p}{x^q}=x^{p-q}$
\end{itemize}}

\rems{\item On peut donc étendre les propriétés des puissances (que jusqu'à aujourd'hui on connaissait pour des puissances entières) pour des puissances rationnelles, sans jamais oublier qu'elles ne sont valables en général que lorsque $x$ est positif !
\item On exclut $0$ pour éviter le cas $0^0$.}

\preuve{En utilisant la caractérisation exponentielle.}

\exo{Réaliser un programme réalisant le calcul de $x^q$ pour $q\in\Q$ et $x>0$. On pourra utiliser la caractérisation exponentielle.}

\subsection{Extension aux négatifs}

\pro{Soit $r \in ]-\infty,0[$ et $(\epsilon )$ l'équation $x^n=r$ d'inconnue $x\in\R$.
\begin{itemize} 
\item Si $n$ est pair, $( \epsilon )$ n'admet aucune solution réelle.
\item Si $n$ est impair, $( \epsilon )$ admet une unique solution qui vaut $-\sqrt[n]{|r|}$.
\end{itemize}}

\preuve{
\begin{itemize}
\item Soit $x$ solution de $(\epsilon)$. Si $x$ est pair alors $x^n\ge 0$, absurde puisque $r<0$.
\item Supposons $n$ impair. Alors $(-1)^n=-1$. Posons $x=-\sqrt[n]{|r|}$. Alors $x^n=-(\sqrt[n]{|r|})^n=-|r|=r$ donc $x$ est solution. Soit $x'\in\R$ solution de $(\epsilon)$. Alors $\dfrac{x^n}{x'^n}=1$ \cad $\dfrac{x}{x'}=1$ \cad $x=x'$. Il y a donc existence et unicité de la solution.
\end{itemize}}

\nota{Soit $r\in ]-\infty,0[$ et $n$ impair. On note alors $\sqrt[n]{r}$ l'unique solution réelle de $(\epsilon )$.}

\defi{Pour $n$ impair, on peut donc étendre l'ensemble de définition de la fonction racine $n$-ième à la fonction \fonsn{\R}{\R}{x}{\sqrt[n]{x}}.}

\pro{Pour $n$ impair :

\begin{enumerate}[i)]
\item Cette fonction est bijective et sa réciproque est \fonsn{\R}{\R}{x}{x^n}.
\item (caractérisation exponentielle) 
\begin{itemize}
\item Si $x<0$, $\sqrt[n]{x}=-\exp\parent{\dfrac{\ln |x|}{n}}$
\item Si $x=0$, $\sqrt[n]{x}=0$
\item Si $x>0$, $\sqrt[n]{x}=\exp\parent{\dfrac{\ln x}{n}}$
\end{itemize}
\item Cette fonction est continue, dérivable sur $\R^*$ et sa dérivée vaut $\dfrac{\sqrt[n]{|x|}}{n|x|}$ si $x<0$ et $\dfrac{\sqrt[n]{x}}{nx}$ si $x>0$.
\item Elle est strictement croissante sur $\R$.
\item Elle est impaire, \cad que pour tout $x\in\R$, $\sqrt[n]{-x}=-\sqrt[n]{x}$ et que sa représentation graphique possède une symétrie centrée sur l'origine.
\item $\limc{x}{-\infty}=-\infty$ et $\limc{x}{+\infty}=+\infty$.
\end{enumerate}}

\preuve{Ses propositions sont facilement déductibles des autres.}

\pro{On ne peut pas étendre la notation en puissance ni le théorème des puissances rationnelles pour $x<0$ même si $n$ est impair.}

\preuve{On sait que $\sqrt[3]{-1}=-1$. Si nous pouvions étendre le théorème, alors nous aurions par exemple $\sqrt[3]{-1}=(-1)^{1/3}=(-1)^{2/6}=\cro{(-1)^2}^{1/6}=1$, c'est absurde.}

\section{\texorpdfstring{Racine $n$-ième complexe}{Racine n-ieme complexe}}

Dans toute cette partie, on pose $n\in\N^*$.

\defi{On définit l'ensemble $\U_n=\ens{e^{i\dfrac{2k\pi}{n}},k\in\lint 0,n\lint}$}

\ex{\begin{itemize}
\item $\U_1=\ens{1}$
\item $\U_2=\ens{1,-1}$
\item $\U_3=\ens{1,j,\cjg{j}}$
\item $\U_4=\ens{1,i,-1,-i}$
\item $\U_5=\ens{1,e^{i\dfrac{2\pi}{5}},e^{i\dfrac{4\pi}{5}},e^{i\dfrac{6\pi}{5}},e^{i\dfrac{8\pi}{5}}}$
\item $\U_6=\ens{1,e^{i\dfrac{\pi}{3}},j,-1,\cjg{j},e^{i\dfrac{5\pi}{3}}}$
\end{itemize}}



\pro{Pour tout $z,z'\in \U_n$,

\begin{enumerate}[i)]
\item $z\in \U$
\item $zz'\in \U_n$
\item $1\in \U_n$
\item $z^{-1}\in \U_n$ et $z^{-1}=\cjg{z}$
\end{enumerate}}

\rem{Ces propositions nous disent que $(\U_n,\times)$ est un \textit{groupe} et qu'il est en plus un \textit{sous-groupe} de $\U$ (dont on a déjà montré qu'il est un groupe).}

\preuve{

\begin{enumerate}[i)]
\item Tout élément $z\in\U_n$ est de la forme $e^{i\theta},\theta\in\R$ donc $z\in\U$.
\item Soit $z,z'\in\U_n$. Alors il existe $k,k'\in\lint 0,n\lint$ tel que $z=e^{i\dfrac{2k\pi}{n}}$ et $z'=e^{i\dfrac{2k'\pi}{n}}$. On a donc $zz'=e^{i\dfrac{2(k+k')\pi}{n}}$. On a $k+k'\ge 0$ donc si $k+k'<n$ alors on a bien $zz'\in\U_n$. Supposons que $k+k'\ge n$. Il existe donc $k''\in\lint 0,n\lint$ tel que $k+k'=n+k''$. On a alors $zz'=e^{i\parent{2\pi + \dfrac{2k''\pi}{n}}}=e^{i\dfrac{2k''\pi}{n}} e^{i2\pi}=e^{i\dfrac{2k''\pi}{n}}$ donc finalement $zz'\in\U_n$.

\item Pour $k=0$, $e^{i\dfrac{2k\pi}{n}}=e^{i0}=1$.

\item Soit $k\in\lint 0,n\lint$ et $z=e^{i\dfrac{2k\pi}{n}}\in\U_n$. Posons alors $k'=n-k\in\lint 0,n\lint$ et $z'=e^{i\dfrac{2k'\pi}{n}}\in\U_n$. Alors on vérifie facilement que $z'=z^{-1}$ et que $z'=\cjg{z}$.
\end{enumerate}}

\pro{(caractérisation géométrique de $\U_n$) On note $P_n$ l'ensemble des points d'affixes dans $\U_n$.

\begin{itemize} 
\item $P_n$ forme les sommets d'un polygône régulier à $n$ côtés inscrit dans le cercle de centre l'origine et de rayon 1.
\item $(1,0)\in P_n$
\item $(-1,0)\in P_n \eqv n$ est impair.
\end{itemize}}

\preuve{Posons $I_n=\lint 0,n\lint$. Les points de $P_n$ sont inscrits dans le cercle de centre l'origine et de rayon 1. Donc pour montrer qu'ils forment un polygône régulier à $n$ côtés, il faut :
\begin{enumerate}
\item Montrer qu'il contient bien $n$ points distincts.
\item Montrer que l'angle entre deux points successifs par rapport au centre est constant.
\item Montrer que la distance entre deux points successifs est constante.
\end{enumerate}

\begin{enumerate}
\item On pose la suite complexe $(O_k)_{k\in I_n}$ la suite à $n$ termes définie pour tout $k\in I_n$ par $O_k=e^{i\dfrac{2k\pi}{n}}$ et soit $(M_k)_{k\in I_n}$ la suite de points définie pour tout $k\in I_n$ comme suit : $M_k$ est le point d'affixe $O_k$. Remarquons tout d'abord que tous les éléments de $\U_n$ sont des termes de $(O_k)$, l'intérêt de $(O_n)$ c'est d'ordonner ces éléments, \cad que les points de $(M_k)$ sont successifs (nous allons le montrer). Posons $(A_k)_{k\in I_n}$ la suite telle que pour tout $k\in I_n$, $A_k$ vaut la mesure de $\arg{O_k}$ dans $[0,2\pi[$, \cad $\dfrac{2k\pi}{n}$. $(A_k)$ est strictement croissante, donc d'une part les points de $P_n$ sont distincts et d'autre part les points de $(M_k)$ sont successifs.
\item Soit maintenant $(d_k)_{k\in I_n}$ la suite définie pour tout $k\in\lint 0 ,n-2\rint$ par $d_k=A_{k+1}-A_k$ et par $d_{n-1}=A_0+2\pi-A_{n-1}$. $(d_k)$ représente l'angle entre deux points successifs par rapport au centre. On montre facilement que pour tout $k\in I_n, d_k=\dfrac{2\pi}{n}$, $(d_k)$ est donc constante.
\item Soit $T_n=\ens{OM_kM_{k+1},k\in\lint 0,n-2\rint}\cup OM_{n-1}M_0$, \cad que $T_n$ est l'ensemble des triangles que l'on peut former avec les points de $P_n$. Soit $t\in T_n$. Alors $t$ possède deux côtés de longueur 1 et l'angle entre les deux vaut $\dfrac{2\pi}{n}$. Or si deux triangles possèdent 2 côtés égaux et un même angle entre les deux sont égaux, alors leur troisième côté est aussi égal. Tous les côtés des triangles de $T_n$ sont donc égaux.  
\end{enumerate}

Les deux autres points de la proposition sont faciles à démontrer.}

\theo{Soit $z\in\C$. $z^n=1\eqv z\in\U_n$. On dit qu'un tel $z$ est \textit{une racine} $n$\textit{-ième de l'unité} et on appelle $\U_n$ \textit{l'ensemble des racines} $n$\textit{-ième de l'unité.}}

\rem{Nous venons donc de résoudre notre première équation complexe : l'équation $z^n=1$ d'inconnue $z\in\C$. Et nous pouvons constater quelque chose de remarquable : cette équation, si on cherche seulement les solutions dans $\R$, n'en a qu'une seule : $1$ alors que dans les complexes, il y en a $n$ distinctes ! Ainsi, les équations, lorsqu'on les résout dans les complexes, peuvent avoir des solutions englobant bien sûr les solutions réelles mais qui a des solutions propres.}

Sur XCas, pour résoudre cette équation par exemple pour $n=6$, taper la commande \verb+csolve(z^6=1,z)+ (de manière générale, la commande \verb+csolve+ permet de résoudre des équations dans $\C$).

\preuve{\begin{itemize}
\item ($\Leftarrow$) Soit $z\in\U_n$. Soit $k\in\lint 0,n\lint$ tel que $z=e^{i\dfrac{2k\pi}{n}}$. Alors $z^n=e^{i2k\pi}=1$.
\item ($\Rightarrow$) Il existe $\rho\in\R_+$ et $\theta\in [0,2\pi[$ tel que $z=\rho e^{i\theta}$. Donc $z^n=\rho^n e^{in\theta}$ (voir les propriétés sur la forme exponentielle). Supposons $z^n=1=e^{i0}$. D'après les propriétés de la forme exponentielle, on a d'une part $\rho^n = 1$ \ie $\rho = \sqrt[n]{1}=1$ et d'autre part il existe $k\in\Z$ tel que $n\theta=2k\pi$. Or $n\theta\in[0,2\pi n[$ donc nécessairement $k\in\lint 0,n \lint$ et $\theta=\dfrac{2k\pi}{n}$. Il existe donc $k\in\lint 0,n \lint$ tel que $z=e^{i\dfrac{2k\pi}{n}}$ et donc $z\in\U_n$.
\end{itemize}}

\nota{Pour tout $c\in\C$ on note $c\U_n=\ens{cz,z\in\U_n}$.}

\lemme{Soit $z\in\C$ et $c\in\C^*$. Alors $c z\in \U_n \eqv z\in c^{-1}\U_n$}

\preuve{$c z\in\U_n \eqv \exists  z'\in\U_n, c z = z' \eqv \exists  z'\in\U_n, z = c^{-1}z' \eqv z\in c^{-1}\U_n$}

\pro{Soit $z\in\C$. Pour tout $r\in\R_+$, $z^n=r\eqv z\in \sqrt[n]{r}\U_n$.}

\preuve{Si $r=0$ c'est immédiat. Supposons $r\neq 0$. $z^n=r \eqv r^{-1} z^n=1 \eqv \parent{\sqrt[n]{r^{-1}} z}^n=1 \eqv \sqrt[n]{r^{-1}} z\in \U_n \eqv z\in\sqrt[n]{r}\U_n$. }

\rem{L'équation $z^n=r$, pour $r\in\R_+$, possède pour seule solution réelle $\sqrt[n]{r}$ comme nous le savons déjà. Ainsi, dans les complexes, on retrouve encore des solutions qui n'existent pas dans $\R$. Remarquez que puisque $1\in\U_n$, $\sqrt[n]{r}$ fait toujours partie des solutions complexes (qui est ici un réel). C'est rassurant puisque une équation complexe doit toujours englober les solutions réelles.}

\ex{Pour $z\in\C$, résolvons $z^4=81$. $\sqrt[4]{81}=3$ et $\U_4=\ens{1,i,-1,-i}$ donc l'ensemble des solutions est $3\U_4=\ens{3,3i,-3,-3i}$.}
Nous pouvons vérifier avec la commande XCas \verb+csolve(z^4=81,z)+ qui renvoie effectivement cet ensemble de solutions.

\subsection{Des solutions purement imaginaires}
Soit $(\epsilon)$ l'équation $z^n=c$ où $c\in\C$ d'inconnue $z\in\C$ et soit $S$ l'ensemble des solutions de $(\epsilon)$. On dit que $S$ est l'\textit{ensemble des racines} $n$\textit{-ième de} $c$ et que tout élément de $S$ est une \textit{racine} $n$\textit{-ième de} $c$.

\theo{\label{tracines} Posons $\rho=|c|$, $\theta=\arg c$ et $z_0=\sqrt[n]{\rho}e^{i\dfrac{\theta}{n}}$. Alors :
\begin{enumerate}[i)]
\item $z_0\in S$
\item $S=z_0\U_n$
\end{enumerate}}

\preuve{Si $c=0$ c'est immédiat. Supposons $c\neq 0$. \begin{enumerate}[i)]
\item Immédiat en calculant $z_0^n$.
\item On veut montrer que $z\in S \eqv z\in z_0\U_n$.

\begin{itemize} 
\item ($\Leftarrow$) Soit $z\in z_0\U_n$. Alors il existe $z'\in \U_n$ tel que $z=z_0 z'$. Donc $z^n=z_0^n z'^n=c\times 1 = c$ donc $z\in S$.
\item ($\Rightarrow$) Soit $z\in S$. Donc $z^n=c$ \cad $c^{-1}z^n=1$. Or on sait que $z_0^n=c$ donc $\parent{z_0^n}^{-1}=c^{-1}$ donc $\parent{z_0^{-1}}^n=c^{-1}$. Par conséquent, $c^{-1}z^n=1 \eqv \parent{z_0^{-1}}^n z^n = 1 \eqv \parent{z_0^{-1}z}^n = 1 \eqv z_0^{-1}z\in\U_n \eqv z\in z_0\U_n$.
\end{itemize}
\end{enumerate}}

\rem{Pour rappel, l'équation $x^n=r$, pour $r<0$ et $n$ pair n'a aucune solution réelle. Ce théorème nous permet d'en trouver dans $\C$.}

\ex{Résolvons $z^6=-15625$ pour $z\in\C$. $|-15625|=15625$ et $\arg (-15625)=\pi$ donc $z_0=\sqrt[6]{15625}e^{i\dfrac{\pi}{6}}=5e^{i\dfrac{\pi}{6}}$ est une solution particulière. On a $\U_6=\ens{1,e^{i\dfrac{\pi}{3}},j,-1,\cjg{j},e^{i\dfrac{5\pi}{3}}}$. Donc l'ensemble des solutions, après simplification, est $\ens{5e^{i\dfrac{\pi}{6}},5i,5e^{i\dfrac{5\pi}{6}},5e^{i\dfrac{-5\pi}{6}},-5i,5e^{i\dfrac{-\pi}{6}}}$. A noter, et c'est ce qu'on attendait, qu'aucune de ces solutions n'est réelle. On remarque également que beaucoup de solutions se ressemblent, ce qu'on va constater ci-dessous.}

\subsection{Techniques pour aller plus vite}

\pro{Si $c\in\R$ et $z\in S$ alors $\cjg{z}\in S$.}

\rems{\item Donc si $c\in\R$ on divise le travail par 2 : dès qu'on a trouvé une moitié des solutions, on trouve l'autre en passant simplement au conjugué.
\item Attention : c'est faux en général si $c\not\in\R$. Par exemple si on résout $z^2=i$ alors on va trouver $S=\ens{e^{i\dfrac{\pi}{4}},e^{i\dfrac{-3\pi}{4}}}$ or ces deux solutions ne sont pas conjuguées.}

\preuve{Soit $z\in S$. Alors il existe $z'\in\U_n$ tel que $z=z_0 z'$. Donc $\cjg{z}=\cjg{z_0}\cjg{z'}$. On sait que $\cjg{z'}\in\U_n$. Montrons que $\cjg{z_0}\in S$. On a $z_0=\sqrt[n]{|c|}e^{i\theta /n}$ où $\theta=0$ si $c\ge0$ et $\theta=\pi$ si $c<0$. Supposons $c\ge0$. Alors $z_0=\sqrt[n]{c} e^{i 0/n}=\sqrt[n]{c}$. $z_0\in\R$ donc $\cjg{z_0}=z_0$ et donc $\cjg{z_0}^n=z_0^n=c$ d'où $\cjg{z_0}\in S$. Si $c<0$ alors $z_0=\sqrt[n]{-c} e^{i \pi/n}$. Donc $\cjg{z_0}=\sqrt[n]{-c}e^{-i\pi/n}$ et donc $\cjg{z_0}^n=(-c) e^{-i\pi}=c$ d'où $\cjg{z_0}\in S$.}

\pro{\label{sneg} Si $n$ est pair et si $z\in S$ alors $-z\in S$}

\preuve{Il existe $k\in\N^*$ tel que $n=2k$. $(-z)^n=(-z)^{2k}=z^{2k}=z^n=c$ donc $-z\in S$.}




\section{Équations du second degré}

\subsection{Racines carrées d'un complexe}

\defi{\programme Soit $c\in\C$. On appelle \textit{racines carrées} de $c$ les racines 2-ième de $c$.}

Selon si $c$ est sous forme exponentielle/trigonométrique ou algébrique on utilisera l'une ou l'autre des deux propositions suivantes pour le calcul des racines carrées de $c$.

\pro{Soit $\rho\in\R_+$, $\theta\in\R$ et $c=\rho e^{i\theta}$.

\begin{enumerate}[i)]
\item Les \textit{racines carrées} de $c$ sont $z_0=\sqrt{\rho}e^{i\parent{\dfrac{\theta}{2}}}$ et $z_1=\sqrt{\rho}e^{i\parent{\dfrac{\theta}{2}+\pi}}$
\item  $z_1=-z_0$
\end{enumerate}}

\preuve{\begin{enumerate}[i)]
\item Cas particulier du théorème \ref{tracines} pour $n=2$.
\item C'est la proposition \ref{sneg}.
\end{enumerate}}

\pro{\programme \label{rcarree} Soit $c=x+iy\in\C$.

\begin{enumerate}[i)]
\item Une des racines carrées de $c$, notée $z_0=a+ib$ avec $a\ge0$, est solution du sytème d'équations suivant :

\sys{a^2+b^2 &=& |c| \\
a^2-b^2 &=& x \\
2ab &=& y}

\item La seconde solution, notée $z_1$, vérifie $z_1=-z_0$.
\end{enumerate}}

\preuve{\begin{enumerate}[i)]
\item Notons tout d'abord que d'après le point 2, l'une des racines carrées a nécessairement sa partie réelle positive. $|z_0^2|=|c|$ donc $|z_0|^2=|c|$ donc $a^2+b^2=|c|$ (première ligne). De plus $z_0^2=c$ donc $(a+ib)^2=x+iy$ donc en développant et en identifiant en trouve les deux dernières lignes. Nous avons montré que $z_0$ \textit{vérifie} ce système, montrons maintenant que $z_0$ est \textit{solution} du système, \cad qu'elle suffit effectivement à trouver $a$ et $b$. En additionant les deux premières lignes on trouve $|a|=a=\sqrt{\dfrac{x+|c|}{2}}$ et par la première ligne, $|b|=\sqrt{\dfrac{|c|-x}{2}}$. La troisième équation nous donne alors le signe de $b$ : si $y\ge0$ alors $b\ge0$ et si $y\le0$ alors $b\le0$. On vérifie que dans les deux cas, on a alors bien $z_0^2=c$.
\item C'est la proposition \ref{sneg}.
\end{enumerate}}

\ex{Trouvons les racines carrées de $z=-4+3i$. Notons $z_0=a+ib$ celle telle que $a \ge 0$. $|z|=5$ donc $z_0$ est solution du système suivant :

\sys{a^2+b^2 &=& 5 \\
a^2-b^2 &=& -4 \\
2ab &=& 3}

L'addition des deux premières lignes nous donne $|a|=a=\dfrac{\sqrt{2}}{2}$. La première ligne nous donne $|b|=\sqrt{\dfrac{9}{2}}=\dfrac{3\sqrt{2}}{2}$ or d'après la troisième ligne $b\ge0$ donc $b=\dfrac{3\sqrt{2}}{2}$. Les racines carrées de $z$ sont donc $z_0=\dfrac{\sqrt{2}}{2}+i\dfrac{3\sqrt{2}}{2}$ et $z_1=-z_0=-\dfrac{\sqrt{2}}{2}-i\dfrac{3\sqrt{2}}{2}$. Nous pouvons par exemple vérifier avec XCas que cela est correct.}

\rem{Le seul complexe ayant une seule racine carrée au lieu de deux, c'est $0$.}

\subsection{Généralités sur les fonctions polynomiales}

\defi{(fonction polynomiale) Soit \fone{P}{\C}{\C} et $n\in\N$. On dit que $P$ est une \textit{fonction polynomiale} \textit{(complexe) de degré} $n$ s'il existe $(c_k)_{k\in\lint 0,n \rint}$ une famille de complexes telle que $c_n\neq 0$ et telle que pour tout $z\in\C,P(z)=\sum\limits_{k=0}^n c_k z^k$.

\begin{itemize}
\item Pour tout $k\in\lint 0,n \rint$ on dit que $c_k$ est le \textit{coefficient de degré} $k$ \textit{de} $P$ et que $c_k z^k$ est le \textit{monôme de degré} $k$ \textit{de} $P$. 
\item On dit que $z$ est une \textit{racine} de $P$ si $P(z)=0$.
\item On note $\deg P$ le degré de $P$.
\item On note $\poly$ l'ensemble des fonctions polynomiales et $\polyd{n}$ l'ensemble des fonctions polynomiales de  degré \textit{au plus} $n$.
\end{itemize}}
\rems{\item $\polyd{0}$ est l'ensemble des fonctions complexes constantes.
\item On dit souvent \textit{polynôme} à la place de \textit{fonction polynomiale}, terme plus rigoureux.}

A partir de maintenant et sauf mention contraire, nous posons $P\in\poly$ avec $\deg P = n\in\N$.

\pro{Soit $Q\in\poly$ avec $\deg Q = m\in\N$ et $\lambda\in\R^*$. Soit \fone{R}{\C}{\C} et $z\in\C$.

\begin{enumerate}[i)]
\item Si $R(z)=\lambda P(z)$ alors $R\in\poly$ et $\deg R = n$.
\item Si $R(z)=P(z)+Q(z)$ alors $R\in\polyd{\max(n,m)}$.
\item Si $R(z)=P(z)Q(z)$ et que $P(z),Q(z)\neq 0$ alors $R\in\poly$ et $\deg R=n+m$.
\end{enumerate}}
\preuve{Immédiat en revenant à la définition.}

\pro{(égalité de polynômes) Soit $Q\in\poly$ tel que $\deg Q = n$. Alors pour tout $z\in\C$, $P(z)=Q(z)$ \ssi pour tout $k\in\lint 0,n\rint$, le coefficient de degré $k$ de $P$ et de $Q$ sont égaux. On dit alors que $P$ et $Q$ sont \textit{égaux} et on note $P=Q$.}

\preuve{
\begin{itemize}
\item ($\Leftarrow$) Immédiat.
\item ($\Rightarrow$) Soit $z\in C$ et supposons que $P(z)=Q(z)$. Soit $(c_k)_{k\le n}$ la famille des coefficients de $P$ et $(q_k)_{k\le n}$ celle de $Q$, avec $c_n,q_n\neq 0$. $P(z)=Q(z)\eqv \somme{k=0}{n} c_k z^k = \somme{k=0}{n} q_k z^k \eqv \somme{k=0}{n} (c_k-q_k) z^k = 0$. En particulier, pour $z=0$, $P(0)=Q(0)\eqv c_0-q_0=0\eqv c_0 = q_0$. Donc $P(z)=Q(z)\eqv\somme{k=1}{n} (c_k-q_k) z^k = 0 $. En particulier, pour $z=0$, $P(0)=Q(0)\eqv c_1-q_1=0\eqv c_1 = q_1$. Donc $P(z)=Q(z)\eqv\somme{k=2}{n} (c_k-q_k) z^k = 0 $. En réitérant ce procédé, on obtient finalement que pour tout $k\in\lint 0,n\rint, c_k=q_k$.
\end{itemize}}


\theo{(fondamental de l'algèbre, ou de D'Alembert-Gauss) Si $n\neq0$ alors $P$ possède au moins une racine.}

\rems{\item Ce théorème, extrêmement important dans l'histoire des mathématiques tant pour la conception de sa preuve, les mathématiciens mis en jeu (D'Alembert, Euler, Lagrange, Gauss, Galois) que pour ces conséquences nombreuses, est admis.
\item Contre-exemple pour les polynômes réels : le polynôme $x\mapsto x^2+1$ ne possède aucune racine réelle.}

\theo{(de factorisation polynomiale) Supposons $n\neq0$. Alors $r$ est une racine de $P$ \ssi il existe $Q\in\polyd{n-1}$ tel que pour tout $z\in\C$, $P(z)=(z-r)Q(z)$.}

\preuve{\begin{itemize}
\item ($\Leftarrow$) Immédiat.
\item ($\Rightarrow$) Soit $(c_k)_{k\in\lint 0,n \rint}$ avec $c_n\neq 0$ une famille de complexes tel que $P(z)=\somme{k=0}{n} c_k z^k$.

\begin{itemize}
\item Supposons $r=0$. Alors d'une part $P(0)=0$ et d'autre part $P(0)=c_0$ donc $c_0=0$. On vérifie alors facilement que le polynôme $Q\in\polyd{n-1}$ défini pour tout $z\in\C$ par $Q(z)=\somme{k=0}{n-1} c_{k+1} z^k$ vérifie $(z-r)Q(z)=P(z)$ pour tout $z\in\C$.

\item Supposons $r\neq 0$. Soit $(q_k)_{k\in\lint 0,n \lint}$ une famille de complexes définie par récurrence comme suit :

\begin{enumerate}[i)]
\item $q_0=-\dfrac{c_0}{r}$
\item pour tout $k\in\lint 1, n\lint$, $q_k=\dfrac{q_{k-1}-c_k}{r}$
\end{enumerate}
Vérifions alors que le polynôme $Q\in\polyd{n-1}$ définie pour tout $z\in\C$ par $Q(z)=\somme{k=0}{n-1} q_k z^k$ vérifie $(z-r)Q(z)=P(z)$ pour tout $z\in\C$. Soit $z\in\C$. On calcule :

\chaine{& & (z-r)Q(z) \\
&=& (z-r)\somme{k=0}{n-1}q_k z^k \\
&=& (z-r)\parent{q_0+\somme{k=1}{n-1}q_k z^k} \\
&=& zq_0-rq_0+\somme{k=1}{n-1} q_k z^{k+1}-r\somme{k=1}{n-1} q_k z^k \\
&=& zq_0-rq_0-q_0z+q_{n-1}z^n+\somme{k=1}{n-1} q_{k-1}z^k-r\somme{k=1}{n-1} q_k z^k \\
&=& c_0+q_{n-1}z^n+\somme{k=1}{n-1} c_k z^k}

Les coefficients de $z\mapsto (z-r)Q(z)$ et de $P$ de degré compris dans $\lint 0, n\lint$ sont donc identiques. Le coefficient de degré $n$ de $z\mapsto (z-r)Q(z)$ égale $q_{n-1}$. Montrons que $q_{n-1}=c_n$.

\begin{itemize}
\item D'une part, $P(r)=0\eqv c_n r^n+\somme{k=0}{n-1} c_k r^k = 0\eqv c_n= -\dfrac{1}{r^n}\somme{k=0}{n-1} c_k r^k$.
\item D'autre part, par définition, $q_{n-1}=\dfrac{q_{n-2}-c_{n-1}}{r}$. Or $q_{n-2}=\dfrac{q_{n-3}-c_{n-2}}{r}$ donc par substitution, $q_{n-1}=\dfrac{q_{n-3}-c_{n-2}-r c_{n-1}}{r^2}$. En calculant ainsi successivement $q_{n-k}$, $k\in\lint 2,n\rint$ et en substituant dans la valeur de $q_{n-1}$, on trouve $q_{n-1}=\dfrac{q_0-c_1-r c_2 - ... -r^{n-2} c_{n-1}}{r^{n-1}}$, \cad en multipliant à droite par $\dfrac{r}{r}$ : $q_{n-1}=\dfrac{-c_0-r c_1 - r^2 c_2-...-r^{n-1} c_{n-1}}{r^n}$ \cad $q_{n-1}=-\dfrac{1}{r^n}\somme{k=0}{n-1} c_k r^k$. Finalement on a bien $q_{n-1}=c_n$.
\end{itemize}
Conclusion : pour tout $k\in\lint 0, n\rint$, le coefficient de degré $k$ de $z\mapsto (z-r)Q(z)$ et de $P(z)$ sont identiques, donc $P(z)=(z-r)Q(z)$.
\end{itemize}
\end{itemize}}

\rems{\item De plus, la preuve nous donne un algorithme pour déterminer $Q$ avec une simplification : au lieu de calculer $q_{n-1}$ avec la définition de récurrence, on a directement $q_{n-1}=c_n$.
\item En pratique, si on ne se souvient pas de l'algorithme, on peut supposer l'existence d'un tel $Q$ avec $(q_k)_{k\le n-1}$ ses coefficients, développer $(z-r)Q(z)$ et par identification avec $P$ déterminer chaque $q_k$. Si cette méthode fonctionne très bien, elle nécessite de résoudre des sytèmes d'équations donc potentiellement plusieurs calculs pour trouver chaque $q_k$, la rendant donc beaucoup plus chronophage qu'en utilisant l'algorithme pour lequel chaque $q_k$ ne nécessite qu'un seul calcul (plus le degré de $P$ est grand et plus la différence de temps entre ces méthodes sera sensible).
\item Contre-exemple pour les polynôme réels : le polynôme $x\mapsto x^2+1$ n'est pas factorisable.}

\ex{Considérons $P\in\poly$ défini pour tout $z\in\C$ par $P(z)=-3z^3+iz^2+(-9+2i)z-2-5i$. On a $P(-i)=0$ donc $r=-i$ est une racine de $P$. Soit $(c_k)_{k\le 3}$ les coefficients de $P$ et soit $(q_k)_{k\le 2}$ une famille de complexes défini par $q_0=-\dfrac{c_0}{r}=-\dfrac{-2-5i}{-i}=-5+2i$, $q_1=\dfrac{q_0-c_1}{r}=\dfrac{-5+2i-(-9+2i)}{-i}=4i$ et $q_2=c_n=-3$. Soit $z\in\C$. Alors on a $P(z)=(z-r)(q_2 z^2 + q_1 z + q_0)=(z+i)(-3z^2+4iz-5+2i)$. Nous avons donc factorisé $P$.}

\exo{Considérons $P\in\poly$ défini pour tout $z\in\C$ par $P(z)=iz^4+z^3+(2+5i)z^2+(-4-10i)z-12+3i$. Montrer que $3i$ est une racine de $P$ puis en utilisant l'algorithme factoriser $P$ comme le produit d'un polynôme de degré 1 et d'un polynôme de degré 3. On écrira les coefficients sous forme algébrique.}

\cor{Supposons $n\neq 0$.
\begin{itemize}
\item $P$ se factorise comme un produit de $n$ polynômes de degré 1. On dit que $P$ est \textit{scindé}.
\item $P$ admet $n$ racines, non nécessairement distinctes.
\end{itemize}}

\preuve{\begin{itemize}
\item Soit $z\in\C$. La preuve se fait par récurrence forte (voir remarque ci-dessous). Soit $(c_k)_{k\le n}$ les coefficients de $P$. Pour $n=1$ on a $P(z)=c_1 z + c_0$ : $P$ est donc déjà factorisé comme un produit d'un polynôme de degré 1. Soit $n\in\N^*$ et supposons que tout polynôme de degré $d\le n$ se factorise comme un produit de $d$ polynômes de degré 1. Soit $Q\in\poly$ et $\deg Q = n+1$. $n+1\neq0$ donc $Q$ admet une racine $r$ et donc il existe $R\in\polyd{n}$ tel que $Q(z)=(z-r)R(z)$. Soit $d=\deg R$. $d\le n$ donc $R$ se factorise comme un produit de $d$ polynômes de degré 1. C'est-à-dire qu'il existe deux familles de complexes $(a_k)_{k\in\lint 1,d\rint}$ et $(b_k)_{k\in\lint 1,d\rint}$, les $a_k$ étant tous non nul, tel que $Q(z)=(z-r)\produit{k=1}{d}(a_k z+b_k)$. Or $\deg Q=n+1$ donc nécessairement $d+1=n+1$ \ie $d=n$, donc $Q(z)=(z-r)\produit{k=1}{n}(a_k z+b_k)$ donc $Q$ se factorise comme un produit de $n+1$ polynômes de degré 1.
\item Soit $z\in\C$. $P$ est scindé donc il existe $(a_k)_{k<n}$ et $(b_k)_{k<n}$ deux familles de complexes, avec les $a_k$ tous non nuls, tel que $P(z)=\produit{k=0}{n-1}(a_k z + b_k)=\produit{k=0}{n-1}\cro{a_k\parent{z+\dfrac{b_k}{a_k}}}=\alpha\produit{k=0}{n-1}\parent{z+\dfrac{b_k}{a_k}}$ où $\alpha=\produit{k=0}{n-1} a_k$. Pour tout $k<n$, $-\dfrac{b_k}{a_k}$ est une racine de $P$ et donc $P$ admet $n$ racines. A noter que les racines ne sont en effet pas forcément distinctes, considérer par exemple le polynôme $P$ tel que pour tout $z\in\C$, $P(z)=(z-i)^2$. Ce polynôme possède 2 racines identiques : $i$.
\end{itemize}}

\rem{La récurrence forte correspond au raisonnement suivant : soit $P(n)$ un prédicat sur les naturels. Si on a :

\begin{itemize}
\item $P(0)$ est vrai et :
\item $\forall n\in\N, (\forall m\le n, P(m))\Longrightarrow P(n+1)$
\end{itemize}
alors pour tout $n\in\N$, $P(n)$ est vrai. La différence alors la récurrence simple, c'est que l'hypothèse de récurrence ne porte pas uniquement sur un rang $n$ mais sur \textit{tous les rangs inférieurs ou égaux à }$n$. La récurrence forte implique évidemment la récurrence simple.}

\lemme{Soit $Q,R\in \poly$ avec $\deg Q=1$ et $\deg R=n\in\N^*$ et soit $q_1,r_n$ les coefficients de degré 1 et $n$ respectivement de $Q$ et de $R$. Soit $P\in\poly$ tel que pour tout $z\in\C$, $P(z)=Q(z)R(z)$ et soit $p_{n+1}$ le coefficient de degré $n+1$ de $P$. Alors $p_{n+1}=q_1 r_n$.}

\preuve{Soit $(q_k)_{k\le 1}$ et $(r_k)_{k\le n}$ les coefficients de $Q$ et $R$. Soit $z\in\C$. $P(z)=Q(z)R(z)=(q_1 z+ q_0)\somme{k=0}{n} r_k z^k=\somme{k=0}{n}(q_1 r_k z^{k+1}+q_0 r_k z^k)$. On a donc bien $p_{n+1}=q_1 r_n$.}

\cor{(forme scindée) Supposons $n\neq 0$. Alors il existe $n'\le n$ tel que $P$ possède $n'$ racines distinctes $(r_k)_{k< n'}$ et seulement elles. Il existe également une famille d'entiers naturels non nuls $(m_k)_{k<n'}$ tel que pour tout $z\in\C$, $P(z)=c_n\produit{k=0}{n'-1}(z-r_k)^{m_k}$ où $c_n$ est le coefficient de degré $n$. On dit que $P$ est écrit sous forme \textit{scindée}. Pour tout $k<n'$ on dit que $r_k$ est de \textit{multiplicité} $m_k$ et en particulier si $m_k=2$ alors on dit que $r_k$ est une \textit{racine double}.}

\preuve{Soit $z\in\C$. $P$ est scindé donc il existe $(a_k)_{k<n}$ et $(b_k)_{k<n}$ deux familles de complexes, avec les $a_k$ tous non nuls, tel que $P(z)=\produit{k=0}{n-1}(a_k z + b_k)=\produit{k=0}{n-1}\cro{a_k\parent{z+\dfrac{b_k}{a_k}}}=\alpha\produit{k=0}{n-1}\parent{z+\dfrac{b_k}{a_k}}$ où $\alpha=\produit{k=0}{n-1} a_k$. Posons $(\rho_k)_{k<n}$ la famille de complexes définie pour $k<n$ par $\rho_k=-\dfrac{b_k}{a_k}$. Autrement dit, $(\rho_k)$ est la famille de toutes les racines de $P$ et $P(z)=\alpha\produit{k=0}{n-1}(z-\rho_k)$. En utilisant par récurrence le lemme précédent, on a $\alpha=c_n$. La famille $(\rho_k)$ contient éventuellement des doublons, donc il existe $n'\le n$ tel que $P$ possède $n'$ racines distinctes $(r_k)_{k< n'}$ et seulement elles et une famille d'entiers naturels non nuls $(m_k)_{k<n'}$ tel que $P(z)=c_n\produit{k=0}{n'-1}(z-r_k)^{m_k}$.}

\rems{\item Attention : il est fréquent d'oublier le $c_n$ mais il est pourtant indispensable~: si on avait toujours $c_n=1$ alors le coefficient de degré $n$ vaudrait constamment $1$ ce qui n'est évidemment pas toujours le cas.
\item Cette écriture des polynômes est la plus «~forte~» qui soit, dans le sens où elle est la forme la plus factorisée possible, et elle nous donne immédiatement toutes les racines du polynôme ainsi que leur multiplicité.}

\exo{Pour tout $z\in\C$ considérons que $P(z)=2iz^6+(6+6i)z^5+(18-6i)z^4+(-2-26i)z^3-30z^2+24iz+8$. Montrer que les seules racines de $P$ sont $-2$, $i$ et $1$. Déterminer leur multiplicité puis écrire $P$ sous forme scindée.}

On pourra vérifier sur XCas en utilisant la commande \\ \verb&cfactor(2*i*z^6+(6+6*i)*z^5+(18-6*i)*z^4+(-2-26*i)*z^3-30*z^2+24*i*z+8)&. \\


\pro{Si :

\begin{enumerate}[i)]
\item $n$ est impair,
\item les coefficients de $P$ sont réels,
\end{enumerate}
Alors $P$ possède au moins une racine réelle.}
\preuve{Soit $(r_k)_{k\le n}$ les coefficients de $P$, tous réels, avec $r_n\neq 0$. Considérons la fonction polynomiale \textit{réelle} \fons{P_0}{\R}{\R}{x}{\somme{k=0}{n} r_k x^k}. Comme $P_0$ est la restriction de $P$ dans $\R$, toute racine de $P_0$ est une racine de $P$. Supposons $r_n>0$, le raisonnement pour $r_n<0$ étant similaire. Soit $x\in\R^*$. Alors $P_0(x)=x^n\parent{r_n+\somme{k=0}{n-1} \dfrac{r_k x^k}{x^n}}=x^n\parent{r_n+\somme{k=0}{n-1} \dfrac{r_k}{x^{n-k}}}$. Pour tout $k\in\lint 0,n\lint$, $n-k>0$, donc $\limc{x}{+\infty} x^{n-k}=+\infty$, donc $\limc{x}{+\infty} P_0(x)=\limc{x}{+\infty} r_n x^n=+\infty$. De même, $\limc{x}{-\infty} P_0(x)=\limc{x}{-\infty} r_n x^n$. Or $n$ est impair donc $\limc{x}{-\infty} P_0(x)=-\infty$. On a donc $\limc{x}{+\infty} P_0(x)=+\infty$ et $\limc{x}{-\infty} P_0(x)=-\infty$ et comme $P_0$ est continue sur $R$, par le théorème des valeurs intermédiaires, il existe $r\in\R$ tel que $P_0(r)=0$. Conclusion, $P$ admet $r$ pour racine.}

\subsection{Équations du second degré}

\defi{\programme (équation polynomiale de degré $n$) On appelle \textit{équation (polynomiale) de degré $n\in\N$} toute équation $(\epsilon):P(z)=0$ d'inconnue $z\in\C$ et où $P\in\poly$ avec $\deg P=n$. $P$ est appelé \textit{polynôme associé} à $(\epsilon)$. En particulier, si $n=2$ on dit que $(\epsilon)$ est une \textit{équation du second degré}.}
\rem{Résoudre une telle équation équivaut donc à trouver les racines de $P$.}

\pro{Soit $(\epsilon)$ une équation de degré $n$, $P$ son polynôme associé et $S$ l'ensemble des solutions de $(\epsilon)$. Si $n=0$ et si $P$ est constant à 0 alors $S=\C$. Si $n=0$ et si $P$ n'est pas constant à 0 alors $S=\emptyset$. Si enfin $n>0$ alors $1\le\card S\le n$.}
\preuve{C'est une conséquence directe de tout ce qu'on a vu dans la partie précédente.}

Dans toute la suite nous posons $(\epsilon):az^2+bz+c=0$ une équation du second degré, où $a,b,c\in\C$ avec $a\neq0$. On note $S$ ses solutions.

\lemme{\programme (identités remarquables) Soit $a,b\in\C$. Alors :

\begin{itemize}
\item $(a+b)^2=a^2+2ab+b^2$
\item $(a+b)(a-b)=a^2-b^2$
\end{itemize}}

\preuve{Par simple calcul.}

\pro{\programme On pose $\Delta=b^2-4ac$, appelé \textit{discriminant}, et $\delta$ une des racines carrées de $\Delta$. Alors $S=\ens{\dfrac{-b\pm \delta}{2a}}$}
\preuve{Elle est tout à fait similaire à la preuve pour les réels en passant par la forme canonique~:

\chaine{& & az^2+bz+c = 0 \\
&\eqv& z^2+\dfrac{b}{a}z+\dfrac{c}{a} = 0 \\ \\
&\eqv& \parent{z+\dfrac{b}{2a}}^2-\dfrac{b^2}{4a^2}+\dfrac{c}{a}=0 \\
&\eqv& \parent{z+\dfrac{b}{2a}}^2-\dfrac{\Delta}{4a^2}=0 \\ 
&\eqv& \parent{z+\dfrac{b}{2a}}^2-\parent{\dfrac{\delta}{2a}}^2=0 \\

&\eqv& \parent{z+\dfrac{b}{2a}+\dfrac{\delta}{2a}}\parent{z+\dfrac{b}{2a}-\dfrac{\delta}{2a}}=0 \\
&\eqv& z=\dfrac{-b\pm\delta}{2a}  }}

\rem{Puisque les deux racines carrées de $\Delta$ sont opposées, on comprend pourquoi on peut prendre n'importe laquelle des deux pour $\delta$.}

\pro{(lien entre les solutions) La somme des deux solutions de $(\epsilon)$ égale $-\dfrac{b}{a}$.}
\preuve{Par calcul direct.}

\rem{En pratique, cette proposition sert à calculer la seconde solution connaissant la première, si ce quotient est simple à calculer (\cad si c'est un réel ou un imaginaire pur), autrement cela n'a pas beaucoup d'intérêt.}

\ex{Résolvons l'équation du second degré $(1+i)z^2+(5-7i)z-24+2i=0$ pour $z\in\C$. Ici $-\dfrac{b}{a}$ n'est pas simple à calculer, on utilisera donc la formule classique pour les deux solutions. Posons $\Delta=(5-7i)^2-4(1+i)(-24+2i)=80+18i$. On trouve, par exemple via la méthode décrite dans la proposition \ref{rcarree} qu'une racine carrée de $\Delta$ est $\delta=9+i$. Donc $S=\ens{\dfrac{-(5-7i)+(9+i)}{2(1+i)},\dfrac{-(5-7i)-(9+i)}{2(1+i)}}=\ens{3+i,-2+5i}$. On en déduit donc l'écriture scindée du polynôme $P$ correspondant : pour tout $z\in\C$, $P(z)=(1+i)\cro{z-(3+i)}\cro{z-(-2+5i)}$.}

Pour vérifier sur XCas on pourra utiliser la commande \verb&csolve((1+i)*z^2+(5-7*i)*z-24+2*i=0,z)&. \\

\pro{(sur la nature des solutions) On note $z_0,z_1$ les solutions de $(\epsilon)$.
\begin{itemize}
\item $z_0=z_1\eqv \Delta=0$
\item $z_0=\cjg{z_1}\Longrightarrow \dfrac{b}{a},\dfrac{c}{a}\in\R$
\item $z_0,z_1\in\R\Longrightarrow \dfrac{b}{a},\dfrac{c}{a}\in\R$
\item \programme Si $a,b,c\in\R$ et $\Delta<0$ alors $z_0,z_1\in\C$ et $z_0=\cjg{z_1}$.
\end{itemize}}

\preuve{\begin{itemize}
\item $z_0=z_1\eqv \dfrac{-b+\delta}{2a}=\dfrac{-b-\delta}{2a} \eqv \delta=0 \eqv \Delta = 0$.
\item Supposons $z_0=\cjg{z_1}$. Alors $az^2+bz+c=a(z-z_0)(z-\cjg{z_0})=a\cro{z^2-z(z_0+\cjg{z_0})+z_0\cjg{z_0}}=a\cro{z^2-2\re(z_0) z + |z_0|^2}$ donc $\dfrac{b}{a},\dfrac{c}{a}\in\R$.
\item Supposons $z_0,z_1\in\R$. Alors $az^2+bz+c=a(z-z_0)(z-z_1)=a[z^2-(z_0+z_1)z+z_0 z_1]$ donc $\dfrac{b}{a},\dfrac{c}{a}\in\R$.
\item Supposons $\Delta<0$. $\Delta\in\R$ donc une racine carrée de $\Delta$ est $\delta=i\sqrt{-\Delta}$. On a donc $S=\ens{\dfrac{-b+i\sqrt{-\Delta}}{2a},\dfrac{-b-i\sqrt{-\Delta}}{2a}}$, ces deux solutions sont complexes et conjuguées.
\end{itemize}}
\rem{La réciproque des deux points du milieu est fausse. Par exemple l'équation $z(z-1)=0$ vérifie bien $\dfrac{b}{a},\dfrac{c}{a}\in\R$ et pourtant $z_0\neq \cjg{z_1}$. L'équation $z^2+1=0$ vérifie bien $\dfrac{b}{a},\dfrac{c}{a}\in\R$ et pourtant $z_0,z_1\not\in\R$. }

\subsection{Pour la culture}

Pour ce qui est de la résolution des équations de degré supérieure à 2, il existe des méthodes mais qui ne sont pas du niveau lycée :

\begin{itemize}
\item Pour le degré 3, la \href{https://fr.wikipedia.org/wiki/M\%C3\%A9thode_de_Cardan}{méthode de Cardan} (1501-1576) permet de résoudre les équations de la forme $z^3+pz^2+q=0$ avec $p,q\in\R$ d'inconnue $z\in\C$.
\item Pour le degré 4, la \href{https://fr.wikipedia.org/wiki/M\%C3\%A9thode_de_Ferrari}{méthode de Ferrari} (1522-1565) permet de résoudre les équations de la forme $z^4+pz^3+qz^2+r=0$ avec $p,q,r\in\R$ d'inconnue $z\in\C$.
\item Le \href{https://fr.wikipedia.org/wiki/Th\%C3\%A9or\%C3\%A8me_d\%27Abel_(alg\%C3\%A8bre)}{théorème d'Abel} (1802-1829) nous dit qu'il n'existe aucune méthode permettant de résoudre systématiquement les équations de degré supérieur ou égal à 5 par radicaux (comprendre avec les opérations usuelles). Attention : cela ne veut pas dire que les solutions de ces équations sont toujours impossible à écrire par radicaux (par exemple l'équation $z^5-1=0$ comprend 1 comme solution évidente), cela signifie simplement qu'il n'y a pas de méthode générale.
\item Presque en même temps, \href{https://fr.wikipedia.org/wiki/Th\%C3\%A9orie_de_Galois#Th\%C3\%A9orie_des_\%C3\%A9quations_alg\%C3\%A9briques}{Galois (1811-1832) améliore le résultat} précédent en exhibant une condition nécessaire et suffisante pour qu'une équation polynomiale soit résoluble par radicaux.
\end{itemize}

Pour avoir un large panorama de l'histoire de la théorie des équations polynomiales, consulter \href{https://fr.wikipedia.org/wiki/Th\%C3\%A9orie_des_\%C3\%A9quations_(histoire_des_sciences)}{cette page}.

Avec les outils que l'on possède maintenant, on peut résoudre les équations de degré 3 si on connait une racine : on peut alors exprimer le polynôme comme le produit d'un polynôme de degré 1 et un autre de degré 2 dont on sait déterminer les racines.  Par conséquent, si en terminale vous tombez sur une équation polynomiale de degré 3 à résoudre sans indication (ce qui n'arrivera à peu près jamais), c'est forcément qu'il y a une solution évidente à chercher dans $\ens{0,\pm 1, \pm i}$.

Enfin, à noter que pour les équations polynomiales réelles on pouvait s'aider de la représentation graphique du polynôme pour chercher les racines. Mais dans les complexes ce n'est plus possible : pour représenter une fonction $\C\rightarrow\C$ il nous faudrait un espace de dimension 4 (c'est balot) et si on veut absolument la représenter dans un espace de dimension 3 il nous faudrait forcément un indicateur supplémentaire pour combler la dimension manquante, par exemple en utilisant des couleurs, mais les graphiques ne sont alors pas franchement lisibles.

\section{Transformations du plan}
\subsection{Le plan complexe}
Jusqu'à présent, pour représenter graphiquement un complexe, nous devions lui associer un point d'affixe ce complexe puis représenter ce point dans le plan usuel euclidien (on dit aussi plan affine). Il existe un plan spécialement conçu pour la représentation des complexes pour s'éviter cette gymnastique : le plan complexe.

\defi{\programme (plan complexe) On dit que $\mathscr{P}$ est un \textit{plan complexe} si :

\begin{enumerate}[i)]
\item $\mathscr{P}$ est un plan, \ie un espace de dimension 2,
\item il est muni d'un repère orthonormé direct $\mathscr{R}=(O,{u},{v})$ : $O$ étant appelé \textit{origine} de $\pc$, ${u}$ et ${v}$ les \textit{vecteurs de base} de $\pc$ vérifiant $||{u}||=||{v}||=1$ et $({u},{v})=\dfrac{\pi}{2}$.
\item chacun de ses éléments, appelés \textit{points}, est la représentation unique d'un nombre complexe.
\end{enumerate}}
\rem{On évite absolument d'appeler les vecteurs de base ${i}$ et ${j}$ pour ne pas confondre avec les complexes $i$ et $j$.}

A partir de maintenant considérons $\mathscr{P}$ un plan complexe muni du repère orthonormé $\mathscr{R}=(O,{u},{v})$.

\pro{soit $M\in\pc$. Alors il existe un unique couple $(\lambda,\mu)\in\R^2$ tel que $\vr{OM}=\lambda{u}+\mu{v}$.}

\preuve{Supposons qu'il existe un couple $(\lambda',\mu')\in\R^2$ différent de $(\lambda,\mu)$ tel que $\vr{OM}=\lambda'{u}+\mu'{v}$. Alors $(\lambda-\lambda'){u}=(\mu'-\mu){v}$ donc ${u}$ et ${v}$ sont colinéaires : c'est absurde puisque $\mathscr{R}$ est orthonormé.}

\defi{\programme (terminologie) Soit $M\in\pc$ et soit l'unique couple $(\lambda,\mu)\in\R^2$ tel que $\vr{OM}=\lambda{u}+\mu{v}$. Soit $z=\lambda+i\mu\in\C$.

\begin{itemize}
\item On dit que $M$ est la \textit{représentation} ou l'\textit{image} de $z$ dans $\pc$.
\item On dit que $z$ est l'\textit{affixe} de $M$.
\item On dit que $\lambda$ et $\mu$ sont les \textit{coordonnées} de $M$ dans $\pc$ et on note $M(\lambda,\mu)$.
\item On dit que $\lambda$ est l'\textit{abscisse} de $M$ dans $\pc$.
\item On dit que $\mu$ est l'\textit{ordonnée} de $M$ dans $\pc$.
\end{itemize}}

\pro{Tout complexe $z$ possède une image dans $\pc$, d'abscisse $\re z$ et d'ordonnée $\im z$.}
\preuve{Immédiate.}
\rem{Dans les lycées on note souvent $M(z)$ l'image d'un complexe $z$, notation que je ne cautionne pas car on dirait ici que $M$ est à la fois un point et une fonction.}

\defi{(image d'un ensemble) Soit $E\subseteq\C$. On appelle \textit{image de} $E$ \textit{dans} $\pc$ l'ensemble des images des éléments de $E$.}

\defi{\programme (axes)
\begin{itemize}
\item On appelle \textit{axe réelle} l'image de l'ensemble $\ens{z\in\C|\im(z)=0}$ dans $\pc$. On le note $(O,{u})$.
\item On appelle \textit{axe des imaginaires} l'image de l'ensemble $\ens{z\in\C|\re(z)=0}$ dans $\pc$. On le note $(O,{v})$.
\end{itemize}}

En pratique, la représentation graphique de $\pc$ obéit à certains codes pour ne pas confondre avec la représentation du plan affine classique. Ce qui ne change pas :

\begin{itemize}
\item On représente les axes réel et imaginaire sous forme de lignes droites perpendiculaires, avec une flèche au bout (à droite pour l'axe réel, en haut pour l'axe des imaginaires).
\item On représente l'origine $O$ à l'intersection des axe et les vecteurs ${u}$ et ${v}$ en prenant bien garde à ce qu'ils aient la même norme graphiquement.
\item La graduation de l'axe réel est identique à celle du plan euclidien pour l'axe des abscisses.
\end{itemize}

Ce qui change :

\begin{itemize}
\item En bout de flèche de l'axe réel on indique «~$\re$~» ou «~axe réel~» au lieu de $x$.
\item En bout de flèche de l'axe des imaginaires on indique «~$\im$~» ou «~axe des imaginaires~» au lieu de $y$.
\item La graduation de l'axe des imaginaires change : par exemple au lieu d'indiquer -2, -1, 0, 1, 2 on indique $-2i$, $-i$, $0$, $i$, $2i$.
\item La représentation des points peut changer. Prenons par exemple le point $M$ d'affixe $z=3-2i$ que l'on souhaite représenter. Il y a deux manières de s'y prendre :
\begin{itemize}
\item Soit on fait comme dans le plan euclidien, \cad qu'on place le point (avec une croix n'est-ce pas) et on écrit «~$M$~» ou si on veut préciser ses coordonnées : «~$M(3,-2)$~».
\item Soit on place le point (avec une croix n'est-ce pas) et on écrit «~$z$~» ou si on veut préciser sa valeur : «~$z=3-2i$~».
\end{itemize}
Pour des questions de lisibilité, il est fortement recommandé, dans une même représentation, de ne choisir qu'une seule de ces conventions et de s'y tenir.
\end{itemize}

Voici quelques exemples de représentations pas terribles (pourquoi le sont-elles ?) : \href{https://lh3.googleusercontent.com/-E_K0FzhfNqk/WdXl6fBzcqI/AAAAAAAABEc/igcFOQDExdM8WPya6kJPqdzsdfqbeODrQCLcBGAs/s0/plan-complexe.png}{ici}, \href{https://www.ilemaths.net/img/forum_img/0566/forum_566046_1.jpg}{ici}, \href{http://www.maeckes.nl/Tekeningen/Complexe\%20vlak\%20.png}{ici} et \href{https://stileex.xyz/wp-content/uploads/2019/03/representation-geometrique-conjugue-nombre-complexe-e1553947642885-640x445.png}{ici}.

Voici un exemple de représentation qui me semble excellente et que l'on peut attendre sur papier (normal, c'est moi qui l'ait réalisée, je n'en ai pas trouvé dans les premiers résultats de Google Images) :

\includegraphics[scale=0.5]{figures/pdf/plan_complexe-eps-converted-to.pdf}

Évidemment, sur un ordinateur c'est beaucoup plus pénible à faire que sur papier, sur un forum d'entraide par exemple personne ne s'amuse à faire une représentation aussi élaborée que la précédente. Cependant c'est bien de savoir vers quoi on doit tendre dans l'idéal.

\subsection{Les transformations du plan}

On considère dans cette partie $\mathscr{P}$ un plan complexe muni du repère orthonormé $\mathscr{R}=(O,{u},{v})$.

\defi{(transformation du plan) On appelle \textit{transformation du plan} $\pc$ toute application \fonen{\pc}{\pc} bijective (\ie qui admet une application réciproque).}

\rem{Une transformation du plan est donc une application qui transforme tout point de $\pc$ en un autre et telle qu'on puisse «~revenir en arrière~».}

\defi{\begin{itemize}
\item A toute transformation \fone{\tau}{\pc}{\pc} on associe une application \fone{t}{\C}{\C} bijective qui à tout $z\in\C$ associe l'affixe de $\tau(M)$ où $M$ est l'image de $z$ dans $\pc$. On dit que $t$ est \textit{l'application associée à la transformation} $\tau$.
\item Réciproquement, à toute application \fone{t}{\C}{\C} bijective on associe une transformation \fone{\tau}{\pc}{\pc} qui à tout $M\in\pc$ associe l'image de $t(z)$ où $z$ est l'affixe de $M$. On dit que $\tau$ est \textit{la transformation associée à l'application} $t$.
\end{itemize}}

\rem{Dans les faits, il est beaucoup plus commode de décrire une transformation en étudiant son application associée. C'est ce que nous allons faire ici pour étudier les transformations les plus classiques.}

\defi{(translation) Soit $w$ un vecteur de $\pc$ d'affixe $z_w$. On appelle \textit{translation de vecteur} $w$ la transformation $\mathscr{T}_{w}$ associée à l'application \fons{t_{z_w}}{\C}{\C}{z}{z+z_w}.}

\pro{(interprétation géométrique) Soit $w$ un vecteur de $\pc$, $M\in\pc$ et $M'=\mathscr{T}_w (M)$. Alors $\vr{MM'}=w$.}

\preuve{Soit $z$ l'affixe de $M$ et $z_w$ celui de $w$. L'affixe de $\vr{MM'}$ égale $t_{z_w}(z)-z=z+z_w-z=z_w$ d'où $\vr{MM'}=w$.}

\rem{Cette transformation consiste donc à déplacer les points dans la même direction, le même sens et à la même distance.}

\defi{(homothétie) Soit $\Omega\in\pc$ d'affixe $\omega$ et $k\in\R^*$. On appelle \textit{homothétie de centre} $\Omega$ \textit{et de rapport} $k$ la transformation $\mathscr{H}_{\Omega,k}$ associée à l'application \fons{h_{\omega,k}}{\C}{\C}{z}{k(z-\omega)+\omega}.}

\pro{(interprétation géométrique) Soit $\Omega,M\in\pc$, $k\in\R^*$ et $M'=\mathscr{H}_{\Omega,k} (M)$. Alors $\vr{\Omega M'}=k\vr{\Omega M}$.}
\preuve{Soit $\omega,z$ les affixes respectives de $\Omega$ et de $M$. L'affixe de $\vr{\Omega M'}$ égale $h_{\omega,k}(z)-\omega=k(z-\omega)+\omega-\omega=k(z-\omega)$ d'où $\vr{\Omega M'}=k\vr{\Omega M}$.}

\rems{\item Si $k=0$ tous les points du plan seraient envoyés sur $\Omega$ et l'application ne serait alors pas bijective, c'est pourquoi on l'exclut.

\item L'homothétie consiste donc à agrandir ou rétrécir les figures selon un centre donné. Si $|k|<1$ il s'agit d'un rétrécissement, si $|k|>1$ d'un agrandissement. Si $k=\pm 1$ il n'y a aucun changement concernant les distances. \href{https://upload.wikimedia.org/wikipedia/commons/thumb/a/a9/Homothetic_transformation.svg/800px-Homothetic_transformation.svg.png}{On peut observer ici} l'image $a_1 b_1 c_1$ d'un triangle $abc$ par une homothétie de centre $O$ et de rapport $k>1$ (pour calculer la valeur précise de $k$ il faudrait calculer $\dfrac{Oa_1}{Oa}$ par exemple).}

\defi{(rotation) Soit $\Omega\in\pc$ d'affixe $\omega$ et $\theta\in\R$. On appelle \textit{rotation de centre $\Omega$ et d'angle $\theta$} la transformation $\mathscr{R}_{\Omega,\theta}$ associée à l'application \fons{r_{\omega,\theta}}{\C}{\C}{z}{(z-\omega)e^{i\theta}+\omega}.}

\pro{(interprétation géométrique) Soit $\Omega\neq M\in\pc$, $\theta\in\R$ et $M'=\mathscr{R}_{\Omega,\theta} (M)$. Alors $(\vr{\Omega M},\vr{\Omega M'})=\theta$ et $||\vr{\Omega M}||=||\vr{\Omega M'}||$.}

\preuve{Montrons la première partie. Soit $z,\omega$ les affixes respectives de $M$ et de $\Omega$. Soit $A\in\pc$ tel que $\vr{OA}=\vr{\Omega M}$ et $A'\in\pc$ tel que $\vr{OA'}=\vr{\Omega M'}$. Soit $a=z-\omega$ l'affixe de $A$ et $a'=(z-\omega)e^{i\theta}+\omega-\omega=(z-\omega)e^{i\theta}$ celui de $A'$. On a donc $(\vr{\Omega M},\vr{\Omega M'})=(\vr{OA},\vr{OA'})=(\vr{OA},{u})+({u},\vr{OA'})=-({u},\vr{OA})+({u},\vr{OA'})=\arg a' - \arg a=\arg\dfrac{a'}{a}=\arg e^{i\theta}=\theta$. Montrons la seconde partie. $||\vr{\Omega M}||=||\vr{O A}||=|a|=|z-\omega|=|(z-\omega)e^{i\theta}|=||\vr{O A'}||=||\vr{\Omega M'}||$.}

\rems{\item Dans le cas particulier où $\theta=\pi$ on dit que $\mathscr{R}_{\Omega,\pi}$ est une \textit{symétrie centrée en} $\Omega$ (ce qui consiste à faire faire un demi-tour à tous les points autour de $\Omega$). A noter également qu'une symétrie centrale possède une définition par homothétie, en effet : $\mathscr{R}_{\Omega,\pi}=\mathscr{H}_{\Omega,-1}$.

\item Même si ce n'est pas obligatoire, il est préférable d'exprimer $\theta$ dans une mesure commode, soit en mesure principale (le mieux selon moi) soit dans $[0,2\pi[$.}

\defi{(projection orthogonale) Soit $\Delta$ une droite de $\pc$ dirigée par le vecteur ${w}\neq{0}$ d'affixe $z_w=a+ib$ et passant par $D\in\pc$ d'affixe $d=\alpha+i\beta$. On appelle \textit{projection orthogonale sur} $\Delta$ l'application \fons{p_\Delta}{\C}{\C}{z}{\dfrac{a^2 c_1 + abc_2+b^2 c_3}{a^2+b^2}} où : \sys{c_1 &=& \re z + i\beta \\
c_2 &=& \im z-\beta+i(\re z -\alpha) \\
c_3 &=& \alpha +i\im z}.}

\rem{Attention : cette application n'est pas bijective, par conséquent elle n'admet pas de transformation associée.}

\pro{\label{intgeo} (interprétation géométrique) Soit $\Delta$ une droite de $\pc$, $M\in\pc$ d'affixe $z$ et $P$ l'image de $p_\Delta (z)$. Alors $P\in\Delta$ et $\Delta\perp (MP)$.}

\rem{Cas particuliers facilement vérifiables :

\begin{itemize}
\item Si $\Delta=(O,{u})$, \cad si $b=\beta=0$ alors pour tout $z\in\C$, $p_\Delta (z)=\re z$ (c'est une projection sur l'axe des abscisses).
\item Si $\Delta=(O,{v})$, \cad si $a=\alpha=0$ alors pour tout $z\in\C$, $p_\Delta (z)=\im z$ (c'est une projection sur l'axe des ordonnées).
\item Si $M\in\Delta$, alors $p_\Delta(z)=z$ où $z$ est l'affixe de $M$ (il est inchangé).
\end{itemize}}

\lemme{(appartenance à une droite) Soit $\Delta$ une droite de $\pc$ dirigée par le vecteur ${w}\neq{0}$ d'affixe $z_w=a+ib$ et passant par $D\in\pc$ d'affixe $d=\alpha+i\beta$. Soit $P\in\pc$ d'affixe $z=x+iy$. Alors $P\in\Delta\eqv a(y-\beta)=b(x-\alpha)$.}

\preuve{$P\in\Delta$ \ssi ${w}$ et $\vr{DP}$ sont colinéaires. Or $\vr{DP}$ a pour affixe $x-\alpha+i(y-\beta)$, d'où le résultat.}

\lemme{(perpendicularité de deux droites) Soit $\Delta,\Delta'$ deux droites de $\pc$ dirigées par les vecteurs ${w},{w'}\neq{0}$ et d'affixes $z_w=a+ib$ et $z_w'=a'+ib'$. Alors $\Delta\perp\Delta'\eqv aa'+bb'=0$}

\preuve{$\Delta\perp\Delta'\eqv{w}\cdot{w'}=0$ d'où le résultat.}

\preuve{(de la proposition \ref{intgeo}). 

\begin{itemize}
\item Montrons que $P\in\Delta$. Soit $z$ l'affixe de $M$ et $p=p_\Delta (z)$ celui de $P$. Alors en appliquant le premier lemme en posant $x=\re p$ et $y=\im p$ (en laissant faire les calculs à XCas), on vérifie ce résultat.
\item Montrons que $\Delta\perp (MP)$. Supposons $M\neq P$ (autrement le résultat est immédiat). $(MP)$ est dirigée par $\vr{MP}\neq{0}$ d'affixe $p-z$. Alors en appliquant le second lemme en posant $a'=\re (p-z)$ et $b'=\im (p-z)$ (en laissant faire les calculs à XCas), on vérifie ce résultat.
\end{itemize}}

\defi{(réflexion) Soit $\Delta$ une droite de $\pc$. On appelle \textit{réflexion} ou \textit{symétrie d'axe} $\Delta$ la transformation $\mathscr{S}_\Delta$ associée à l'application \fons{s_\Delta}{\C}{\C}{z}{h_{p,-1}} où $p=p_\Delta (z)$.}

\rem{Comme nous l'avons vu, on peut remplacer $h_{p,-1}$ par $r_{p,\pi}$.}

\pro{(interprétation géométrique) Soit $\Delta$ une droite de $\pc$, $M\in\pc$ et $M'=\mathscr{S}_\Delta (M)$. Alors $\vr{MM'}=2\vr{MP}$ et $\Delta\perp\vr{MM'}$.}
\preuve{On a $\vr{PM'}=-\vr{PM}$ (interprétation de l'homothétie). D'où $\vr{MM'}=\vr{MP}+\vr{PM'}=2\vr{MP}$. Ce résultat nous dit de plus que $(MM')$ et $(MP)$ sont parallèles. Or nous savons que $\Delta\perp (MP)$ d'où $\Delta\perp\vr{MM'}$.}

\subsection{Classes de transformations}

Selon leurs propriétés, on peut classer les transformations en différentes catégories. Dans cette partie nous considérerons \fone{\tau}{\pc}{\pc} une transformation. Nous munissons $\pc$ de $\dis$ la distance euclidienne entre deux points de $\pc$. Soit dans cette partie $A,B,C\in\pc$ et $A'=\tau (A),B'=\tau (B),C'=\tau (C)$

\defi{(isométrie) On dit que $\tau$ est une \textit{isométrie} si elle conserve les distances, \cad si $\dis(A',B')=\dis(A,B)$.}

\defi{(déplacement) On dit que $\tau$ est un \textit{déplacement} si :

\begin{enumerate}[i)]
\item elle conserve les distances,
\item elle conserve les angles orientés, \cad $(\vr{A'B'}, \vr{A'C'})=(\vr{AB}, \vr{AC})$.
\end{enumerate}}

\rem{Un déplacement est donc une isométrie.}

\defi{(similitude) On dit que $\tau$ est une \textit{similitude} si elle vérifie l'une ou l'autre des conditions équivalentes suivantes.
\begin{itemize}
\item Elle conserve les rapports de distance, c'est-à-dire qu'il existe $k\in]0,+\infty[$, appelé \textit{rapport de $\tau$} tel que $\dis(A',B')=k\dis(A,B)$.
\item Elle conserve les angles géométriques, \ie les angles non orientés, \ie $(\vr{A'B'}, \vr{A'C'})=\pm(\vr{AB}, \vr{AC})$. Si $\tau$ conserve les angles orientés, on dit que $\tau$ est une \textit{similitude directe}, autrement que $\tau$ est une \textit{similitude indirecte}.
\end{itemize}}

\preuve{Montrons que ces deux conditions sont équivalentes.

\begin{itemize}

\item ($\Rightarrow$) Supposons que $\tau$ conserve les rapports de distances. Il existe donc $k\in\left] 0,+\infty\right[$ tel que $\mathrm{d}(A',B')=k \mathrm{d}(A,B)$, $\mathrm{d}(B',C')=k \mathrm{d}(B,C)$ et $\mathrm{d}(C',A')=k \mathrm{d}(C,A)$. Donc les triangles $A'B'C'$ et $ABC$ sont semblables leurs côtés étant proportionnels. Donc leurs angles géométriques sont conservés d'où $(\overrightarrow{A'B'},\overrightarrow{A'C'})=\pm(\overrightarrow{AB},\overrightarrow{AC})$

\item ($\Leftarrow$) Supposons que $\tau$ conserve les angles géométriques. De même, $A'B'C'$ et $ABC$ sont semblables leurs angles géométriques étant égaux et il existe donc $k\in\left] 0,+\infty\right[$ tel que $\mathrm{d}(A',B')=k \mathrm{d}(A,B)$

\end{itemize}}

\rem{Les isométries sont des similitudes. Ainsi, si on note $I$ l'ensemble des isométries, $D$ celui des déplacements et $S$ celui des similitudes, nous avons $D\subset I\subset S$. }

\pro{(stabilité des classes) Soit $\tau,\upsilon$ deux transformations appartenant à la même classe (parmi les trois précédentes). Alors $\tau\circ\upsilon$ appartient également à cette classe.}

\preuve{Soit $A,B,C\in\pc$, $A'=\upsilon(A),B'=\upsilon(B),C'=\upsilon(C)$, $A''=\tau(A'),B''=\tau(B'),C''=\tau(C')$.

\begin{itemize}
\item (pour les isométries) $\dis (A',B')=\dis (A,B)$ car $\upsilon$ est une isométrie d'une part et $\dis (A'',B'')=\dis (A',B')$ car $\tau$ est une isométrie d'autre part donc $\dis (A'',B'')=\dis (A,B)$ donc $\tau\circ\upsilon$ est une isométrie.

\item (pour les déplacements) Tout déplacement est une isométrie, donc $\tau\circ\upsilon$ est une isométrie donc conserve les distances. $(\vr{A'B'}, \vr{A'C'})=(\vr{AB}, \vr{AC})$ car $\upsilon$ est un déplacement d'une part et $(\vr{A''B''}, \vr{A''C''})=(\vr{A'B'}, \vr{A'C'})$ d'autre part car $\tau$ est un déplacement. D'où $(\vr{A''B''}, \vr{A''C''})=(\vr{AB}, \vr{AC})$ donc $\tau\circ\upsilon$ est un déplacement.

\item (pour les similitudes) Soit $t,u\in\R^*$ les rapports respectifs de $\tau$ et $\upsilon$. $\dis(A',B')=u\dis(A,B)$ car $\upsilon$ est une similitude d'une part et $\dis(A'',B'')=t\dis(A',B')$ car $\tau$ est une similitude d'autre part. D'où $\dis(A'',B'')=tu\dis(A,B)$ avec $tu\in\R^*$ donc $\tau\circ\upsilon$ est une similitude.

\end{itemize}}

\pro{(classes des transformations usuelles)

\begin{tabular}{|c|c|c|c|c|} \hline
 & Translations & Homothéties & Rotations & Réflexions \\ \hline
Déplacements & \cmark & \xmark & \cmark & \xmark \\ \hline
Isométries & \cmark & \xmark & \cmark & \cmark \\ \hline
Similitudes & \cmark & \cmark & \cmark & \cmark \\ \hline
\end{tabular}}

\rem{Attention de ne pas mal interpréter le symbole \xmark . Il signfie, par exemple, qu'en général, une homothétie n'est pas un déplacement. En revanche, il existe des homothéties qui sont effectivement des déplacements, par exemple les homothéties de rapport $k=\pm 1$ (en fait ce sont les seules). Le symbole \cmark, au contraire indique que \textit{toutes} les homothéties sont des similitudes par exemple.}

\preuve{Immédiat en utilisant les interprétations géométriques de chaque transformation.}

\theo{(des similitudes fondamentales) Les quatre transformations étudiées (translations, homothéties, rotations, réflexions) sont appelées \textit{similitudes fondamentales}. Soit $\sigma$ une similitude. Alors ou bien $\sigma$ est une similitude fondamentale, ou bien $\sigma$ est la composition de deux différentes d'entre elles.}
\preuve{Admise.}

\exo{Nous considérons \href{https://zupimages.net/up/19/28/aylp.png}{cette image} que nous munissons d'un repère $\mathscr{R}=(O,{u},{v})$ tel que :
\begin{itemize}
\item $O$ coïncide avec le coin sud-ouest de l'image,
\item $||{u}||=||{v}||=1\text{px}$,
\item l'axe $(O,{u})$ est horizontal par rapport à l'écran.
\end{itemize}

Après application d'une similitude $\sigma$ à cette image, nous obtenons \href{https://zupimages.net/up/19/28/cqzp.png}{cette seconde image} (muni du même repère). Écrire $\sigma$ comme une composition de similitudes fondamentales (théoriquement comme nous l'avons dit, il est possible de n'en utiliser que deux, mais dans cet exercice le nombre est illimité). }

\section{Suites complexes}

Chapitre non au programme de terminale, je fais une section sur les suites complexes car cela se passe de façon très similaire aux suites réelles.

\subsection{Généralités}

\defi{(suites complexes) Une suite complexe est une suite à termes complexes.}

Dans toute la suite, on pose \suite{Z} une suite complexe. On pose également \suite{X} et \suite{Y} les suites réelles définies pour tout $n\in\N$ par $X_n=\re (Z_n)$ et $Y_n=\im (Z_n)$ respectivement. Ces suites vont nous servir à faire le lien entre les suites complexes et réelles.

\defi{(boule) Soit $c\in\C$ et $r\in\R_+$. On appelle \textit{boule (ouverte) centrée en $c$ et de rayon $r$}, notée $B(c,r)$, l'ensemble $\ens{z\in\C, |z-c|<r}$.}

\rems{\item Graphiquement, l'image de $B(c,r)$ est le disque ouvert de centre l'image de $c$ et de rayon $r$. On a besoin de la notion de boule afin d'avoir des définitions analogues à celles des suites réelles.
\item Il existe aussi les boules \textit{fermées}, la différence est simplement que l'inégalité est large au lieu d'être stricte (et du coup l'image d'une boule fermée est un disque fermé).
\item On peut se demander pourquoi on n'appelle pas cet objet «~disque~» au lieu de «~boule~». C'est parce que la boule est une notion beaucoup plus générale de topologie, et il m'a semblé qu'il est intéressant de s'habituer dès maintenant à ce que ce qu'on appelle boule en topologie n'est pas forcément représenté par une boule. La boule de certains espaces sont parfois représentées par des losanges !}

\defi{(bornée) On dit que $(Z_n)$ est bornée si il existe $c\in\C,r\in\R_+$ tel que pour tout $n\in\N$, $Z_n\in B(c,r)$.}

\rems{\item Graphiquement, cela signifie que $(Z_n)$ est bornée si l'image de tous ses termes sont compris dans un disque.

\item Il n'y a pas l'équivalent de la notion de majoration pour les suites complexes (car on ne peut pas comparer deux complexes comme on compare deux réels). Cela signifie aussi qu'il n'y pas de notion de croissance.}

\pro{(centrage du disque à l'origine) $(Z_n)$ est bornée \ssi il existe $\rho\in\R_+$ tel que pour tout $n\in\N$, $Z_n\in B(0,\rho)$.}

\rem{Cela signifie que si $(Z_n)$ est bornée, alors on peut inclure les images de ses termes dans un disque centré à l'origine.}

\preuve{\begin{itemize}
\item ($\Leftarrow$) Immédiat.
\item ($\Rightarrow$) Supposons que $(Z_n)$ est bornée. Alors il existe $c\in\C,r\in\R_+$ tel que pour tout $n\in\N$, $Z_n\in B(c,r)$. Posons $\rho=|c|+r$. Soit $n\in\N$. Montrons que $Z_n\in B(0,\rho)$. $|Z_n-0|=|Z_n-c+c|\le |Z_n-c|+|c|<r+|c|=\rho$, d'où $Z_n\in B(0,\rho)$.
\end{itemize}}

\pro{(lien avec les suites réelles) $(Z_n)$ est bornée (au sens complexe) \ssi $(X_n)$ et $(Y_n)$ sont bornées (au sens réel).}

\preuve{\begin{itemize}

\item ($\Rightarrow$) Rappel d'une propriété du module : pour tout $z\in\C$, $|\re z|\le |z|$ et $|\im z|\le |z|$. Supposons $(Z_n)$ bornée. Il existe donc $c\in\C,r\in\R_+$ tel que pour tout $n\in\N$, $Z_n\in B(c,r)$. Soit $n\in\N$. Alors $r>|Z_n-c|\ge |\re (Z_n-c)|=|X_n-\re c|$. Donc $|X_n-\re c|<r$ \cad $\re c-r<X_n < \re c + r$, donc $(X_n)$ est bornée. De même, $(Y_n)$ est bornée.

\item ($\Leftarrow$) Petit lemme : pour tout $a,b\in\R_+,\sqrt{a+b}\le \sqrt{a}+\sqrt{b}$. Supposons $(X_n)$ et $(Y_n)$ bornées. Soit $M\in\R_+$ tel que pour tout $n\in\N$,  $-M<X_n,Y_n<M$. Posons $c=0$ et $r=2M+1$. Soit $n\in\N$. Montrons que $Z_n\in B(c,r)$, \ie $|Z_n-0|<r$. $|Z_n|=\sqrt{X_n^2+Y_n^2}\le \sqrt{X_n^2}+\sqrt{Y_n^2}= |X_n|+|Y_n|\le 2M<2M+1$ d'où $|Z_n|<r$. Donc $(Z_n)$ est bornée.
\end{itemize}}

\pro{(lien avec le module) Soit \suite{M} la suite réelle et positive définie pour tout $n\in\N$ par $M_n=|Z_n|$. $(Z_n)$ est bornée \ssi $(M_n)$ est majorée.}

\preuve{\begin{itemize}
\item ($\Leftarrow$) Supposons $(M_n)$ majorée. Il existe donc $\rho\in\R_+$ tel que pour tout $n\in\N$, $M_n<\rho$. Soit $n\in\N$. Montrons que $Z_n\in B(0,\rho)$. $|Z_n-0|=M_n<\rho$ donc $Z_n\in B(0,\rho)$ donc $(Z_n)$ est bornée.
\item ($\Rightarrow$) Supposons que $(Z_n)$ est bornée. Alors il existe $\rho\in\R_+$ tel que pour tout $n\in\N$, $Z_n\in B(0,\rho)$.  Soit $n\in\N$. Alors $|Z_n-0|<\rho$ \ie $M_n<\rho$ d'où $(M_n)$ est majorée par $\rho$.
\end{itemize}}

\defi{(convergence) Soit $l\in\C$. On dit que $(Z_n)$ \textit{converge vers $l$} si pour tout réel $r>0$, il existe $n_r\in\N$ tel que pour tout entier $n\ge n_r$, $Z_n\in B(l,r)$. On note alors $\limc{n}{+\infty} Z_n = l$ ou encore $Z_n\underset{n\rightarrow +\infty}{\longrightarrow} l$.}

\rem{Graphiquement, $(Z_n)$ converge vers $l$ si pour tout $r>0$, tous les termes de $(Z_n)$ sont dans le disque image de $B(l,r)$ à partir d'un certain rang.}

\defi{(divergence) 

\begin{itemize}
\item On dit que $(Z_n)$ \textit{converge} si il existe $l\in\C$ tel que $(Z_n)$ converge vers $l$.
\item On dit que $(Z_n)$ \textit{diverge} si elle ne converge pas.
\end{itemize}}

\rem{Puisqu'il n'y a pas de notion de majoration, il n'y a pas de notion de divergence vers $\pm\infty$ comme pour les suites réelles.}

\theo{(lien avec les suites réelles) Soit $l\in\C$.

\begin{itemize}
\item $Z_n \underset{n\rightarrow +\infty}{\longrightarrow} l$ (au sens complexe) \ssi $X_n \underset{n\rightarrow +\infty}{\longrightarrow} \re (l)$ et $Y_n \underset{n\rightarrow +\infty}{\longrightarrow} \im (l)$ (au sens réel).
\item $(Z_n)$  converge (au sens complexe) \ssi $(X_n)$ et $(Y_n)$ convergent (au sens réel).
\item $(Z_n)$ diverge (au sens complexe) \ssi $(X_n)$ ou $(Y_n)$ diverge (au sens réel).
\end{itemize}}

\preuve{\begin{itemize}
\item 
\begin{itemize}
\item ($\Rightarrow$) Supposons que $(Z_n)$ converge vers $l\in\C$. Soit un réel $r>0$. Posons un entier $n_r$ tel que pour tout entier $n\ge n_r$, $Z_n\in B(l,r)$. Soit un entier $n\ge n_r$. $(Z_n)$ est donc bornée à partir du rang $n_r$ donc $(X_n)$ et $(Y_n)$ sont bornées à partir du rang $n_r$ et comme nous l'avons vu dans la preuve du lien entre suite complexe bornée et suite réelle bornée, $\re l - r < X_n<\re l +r$ et $\re l - r < Y_n<\re l +r$, d'où $(X_n)$ et $(Y_n)$ convergent respectivement vers $\re l$ et $\im l$.
\item ($\Leftarrow$) Supposons maintenant que $(X_n)$ et $(Y_n)$ convergent respectivement vers $\re l$ et $\im l$. Soit un réel $R>0$ et posons $r=\dfrac{R}{2}$. Posons $n_r\in\N$ tel que pour tout entier $n\ge n_r, |X_n-\re l|<r$ et $|Y_n-\re l| < r$. Soit un entier $n\ge n_r$. Montrons que $Z_n\in B(l,R)$, \ie $|Z_n-l|<R$. $|Z_n-l|=\sqrt{(X_n-\re l)^2+(Y_n-\im l)^2}\le |X_n-\re l| + |Y_n-\im l|<2r=R$. Donc $Z_n\in B(l,R)$ et donc $(Z_n)$ converge vers $l$.
\end{itemize}
\item Immédiat.
\item Immédiat.
\end{itemize}}

\rem{Ce théorème nous permet d'étudier le comportement des suites complexes en utilisant tout l'arsenal dont on dispose pour les suites réels. }

\pro{Si $(Z_n)$ converge alors $(Z_n)$ est bornée. La réciproque est fausse.}

\preuve{Supposons que $(Z_n)$ converge. Alors $(X_n)$ et $(Y_n)$ convergent, or on sait que toute suite réelle convergente est bornée. Donc $(X_n)$ et $(Y_n)$ sont bornées et par conséquent $(Z_n)$ est bornée. Contre-exemple pour la réciproque : supposons que pour tout $n\in\N$, $Z_n=(-1)^n$. Alors $(Y_n)$ est constante à 0 donc elle est bornée, les valeurs de $(X_n)$ alternent entre 1 et -1 donc elle est bornée, donc $(Z_n)$ est bornée, pourtant $(Z_n)$ ne converge pas car $(X_n)$ diverge.}

\pro{(lien avec le module) Soit \suite{M} la suite réelle et positive définie pour tout $n\in\N$ par $M_n=|Z_n|$.

\begin{itemize}
\item Si $(Z_n)$ converge vers $l\in\C$ alors $(M_n)$ converge vers $|l|$. La réciproque est fausse.
\item$\limc{n}{+\infty} M_n = 0 \eqv \limc{n}{+\infty} Z_n = 0$.
\end{itemize}}

\preuve{\begin{itemize}
\item Supposons que $Z_n$ converge vers $l\in\C$. Alors $\limc{n}{+\infty} X_n = \re l$ et $\limc{n}{+\infty} Y_n = \im l$. Or pour tout $n\in\N$, $M_n=\sqrt{X_n^2+Y_n^2}$. Par la caractérisation séquentielle des limites \ref{caraseq}, \suite{(X_n^2)} et \suite{(Y_n^2)} convergent respectivement vers $(\re l)^2$ et $(\im l)^2$. Par sommation, \suite{(X_n^2+Y_n^2)} converge vers $(\re l)^2+(\im l)^2$ et enfin par la caractérisation séquentielle, $M_n$ converge vers $\sqrt{(\re l)^2+(\im l)^2}=|l|$. Pour la réciproque, il suffit de considérer que pour tout $n\in\N$, $Z_n=(-1)^n$. $(M_n)$ est constant à 1 et pourtant $(Z_n)$ diverge.
\item Le sens $\Leftarrow$ est immédiat d'après le point précédent. Supposons que $(M_n)$ converge vers 0.  Supposons par l'absurde que $(X_n)$ ou $(Y_n)$ diverge. Alors $(Z_n)$ diverge, donc ne converge pas vers 0, \cad qu'il existe un réel $r>0$ tel que pour tout $n_s\in\N$, il existe un entier $n_\sigma\ge n_s$ tel que $|Z_n-0|=M_n>r$. Nous venons exactement d'écrire la traduction mathématique de la proposition «~$(M_n)$ ne converge pas vers 0~», ce qui est absurde. Donc $(X_n)$ et $(Y_n)$ convergent et notons $\alpha$ et $\beta$ leur limite respective. Par la caractérisation séquentielle et par sommation des limites, \suite{(\sqrt{X_n^2+Y_n^2})}, \ie $(M_n)$, converge vers $\sqrt{\alpha^2+\beta^2}$. Mais on a supposé que $(M_n)$ converge vers 0 donc par unicité de la limite, on a $\sqrt{\alpha^2+\beta^2}=0$ \cad $\alpha^2+\beta^2=0$ \cad $\alpha=\beta=0$. Donc $(Z_n)$ converge vers 0. 
\end{itemize}}

\subsection{Représentation avec XCas}

\subsubsection{Sous forme explicite}

Soit \suite{Z} la suite définie pour tout $n\in\N$ par $Z_{n}=\dfrac{10}{n+1}{e^{i\dfrac{2n\pi}{7}}}$. Pour représenter graphiquement les 8 premiers termes de $(Z_n)$ avec XCas, on peut taper successivement les deux commandes suivantes. \\

\verb&z(n):=10/(n+1)*e^(i*(2*n*pi)/7)& \\
\verb&for(n:=0;n<8;n:=n+1) {point(z(n))}& \\

Attention cependant : la fenêtre du graphique peut ne pas zoomer automatiquement, il faut le faire manuellement. De plus, ces commandes, si elles marchent très bien sur XCas en ligne, ne semblent pas fonctionner avec le XCas de bureau.

\subsubsection{Sous forme récurrente}

Soit \suite{Z} la suite définie par $Z_0=2-i$ et telle que pour tout $n\in\N$, $Z_{n+1}=\dfrac{Z_n}{n+i}$.  Pour le coup, GeoGebra semble mieux s'en sortir que XCas. Prendre le tableur de GeoGebra, dans A1 mettre $n$ (la colonne $A$ sera réservée aux rangs) et dans B1 mettre $Z_n$. Dans A2 mettre $0$ et dans B2 mettre $2-i$. Dans A3 mettre \verb&=A2+1& et dans B3 mettre \verb&=B2/(A2+1)&. Ensuite étirer A3 et B3 jusqu'à avoir le nombre de termes souhaité, par exemple juqu'à la ligne 10. En principe les images sont représentés dans le plan, sinon sélectionner B3:B10, clique droit, Show Object.

\subsection{Une jolie application des suites complexes}

Pour tout $c\in\C$, on considère \suite{Z} la suite complexe, dépendante de $c$, définie par récurrence comme suit.

\sys{Z_0 &=& 0 \\
Z_{n+1} &=& Z_{n}^2+c,\forall n\in\N}

Nous nous intéressons aux valeurs de $c$ pour lesquelles $(Z_n)$ est bornée. Quelques exemples.

\begin{itemize}
\item Si $c=0$ alors $(Z_n)$ est constante à 0, elle est donc bornée.
\item Si $c=1$, alors $(Z_n)$ est une suite réelle non majorée, elle est n'est donc pas bornée.
\item Si $c=\dfrac{1}{4}$, alors à l'aide du théorème du point fixe \ref{ptfixe} on montre que $(Z_n)$ converge vers $\dfrac{1}{2}$ donc $(Z_n)$ est bornée.
\item Si $c=i$ alors on montre par récurrence que pour tout $n\ge 2$, si $n$ est pair alors $Z_n=-1+i$ et si $n$ est impair alors $Z_n=-i$. Donc $(Z_n)$ est bornée.
\item Si $c=1+i$, $(Z_n)$ n'est visuellement pas bornée.
\item Si $c=-1-\dfrac{1}{4}i$, $(Z_n)$ est visuellement bornée (graphiquement on a deux îlots de points).
\end{itemize}

On note $\mathscr{M}$ l'ensemble des $c\in\C$ pour lesquels $(Z_n)$ est bornée. Ainsi $0\in\mathscr{M}$, $1\notin\mathscr{M}$, $\dfrac{1}{4}\in\mathscr{M}$, $i \in\mathscr{M}$, $1+i\notin\mathscr{M}$, $-1-\dfrac{1}{4}i\in\mathscr{M}$. \\

On appelle $\mathscr{M}$ «~ensemble de Mandelbrot~» du nom du mathématicien \href{https://fr.wikipedia.org/wiki/Beno\%C3\%AEt_Mandelbrot}{Benoît Mandelbrot} (1924-2010). L'image de $\mathscr{M}$ dans un plan complexe, représentée ci-dessous, révèle une figure extrêmement célèbre dans le monde des mathématiques : il s'agit d'une fractale avec un nombre immense de propriétés que je ne saurais résumer correctement ici, la \href{https://fr.wikipedia.org/wiki/Ensemble_de_Mandelbrot}{page Wikiépédia dédiée} est une excellente approche.

\includegraphics[scale=0.5]{figures/pdf/mandelbrot-eps-converted-to.pdf}

Ce n'est pas lui qui a découvert $\mathscr{M}$ mais Gaston Julia (1893-1978) et Pierre Fatou (1878-1929). En revanche, on lui doit les représentations graphiques qu'il a réalisées. C'est aussi lui qui est l'inventeur du concept de fractale, si utile pour modéliser notre monde.

\subsection{Ensembles de Julia}

Nous considérons, pour tout $c,\alpha\in\C$, la \suite{Z} la suite complexe, définie par récurrence comme suit.

\sys{Z_0 &=& \alpha \\
Z_{n+1} &=& Z_{n}^2+c,\forall n\in\N}

Pour tout $c\in\C$, note $J_c$ l'ensemble des $\alpha\in\C$ pour lesquels $(Z_n)$ est bornée. On l'appelle «~ensemble de Julia~». Attention, même si on a l'impression, $\mathscr{M}$ n'est pas un ensemble de Julia particulier. En revanche, si $c\in\mathscr{M}$ alors $J_c$ est topologiquement connexe, \ie fait d'une seule pièce, donc même si l'ensemble de Mandelbrot n'est pas un ensemble de Julia il y a un lien fort entre ces ensembles. Les images des ensembles de Julia \href{https://fr.wikipedia.org/wiki/Ensemble_de_Julia}{peuvent offrir de très esthétiques représentations}. Le domaine des mathématiques étudiant ces structures est la \textit{dynamique holomorphe}.
\chapter{Géométrie dans l'espace}

Dans ce chapitre, on identifie l'espace de dimension 3 à l'ensemble $\R^3$, c'est-à-dire à l'ensemble des triplets de réels.

\defi{Les éléments de $\R^3$ sont appelés \textit{points}.}

\section{Vecteurs de l'espace et produit scalaire}

Essentiellement, nous verrons que les vecteurs de l'espace ne sont rien d'autre qu'une extension des vecteurs du plan avec une dimension (donc coordonnée) supplémentaire. Beaucoup de preuves sont faisables rapidement par simple calcul, nous ne les signaleront pas.

\defi{(vecteurs de $\R^3$) Les vecteurs de $\R^3$ sont des triplets de réels vérifiant les deux propriétés suivantes.

\begin{itemize}
\item Pour tout vecteurs $v=(x,y,z)$ et $v'=(x',y',z')$, $v+v'$ est un vecteur valant le triplet $(x+x',y+y',z+z')$.
\item Pour tout vecteur $v=(x,y,z)$ et $\lambda\in\R$, $\lambda v$ est un vecteur valant le triplet $(\lambda x, \lambda y, \lambda z)$.
\end{itemize}}

\rem{Attention : bien que les vecteurs et les points sont des triplets de réels, ce ne sont pas les mêmes objets et ils n'ont pas les mêmes propriétés. Ainsi, quand on écrit $M\in\R^3$ on sous-entend que $M$ est un point et non un vecteur. Si on veut absolument écrire quelque chose d'équivalent pour les vecteurs, alors il faut noter par exemple $E$ les vecteurs de $\R^3$, et alors quand on écrit $v\in E$ on sous-entend que $v$ est un vecteur de $\R^3$.}

Avec XCas, on crée le vecteur $u=(7,0,-5)$ avec la commande \verb&u:=[7,0,-5]&. Toutes les opérations sur les vecteurs sont possibles avec XCas.

\nota{Soit $v=(x,y,z)$ un vecteur. On peut aussi noter $v=\parent{\begin{matrix}x \\
y \\
z
\end{matrix}}$}

\rem{Rien n'oblige à mettre des flèches sur le nom des vecteurs. Toutefois, c'est assez conventionnel au lycée.}

\pro{(propriétés des vecteurs) Soit $v=(x,y,z)$, $w=(x',y',z')$, $t=(x'',y'',z'')$ trois vecteurs et $\lambda,\mu\in\R$.

\begin{itemize}
\item $(0,0,0)+v=v+(0,0,0)=v$. On appelle $(0,0,0)$ le \textit{vecteur nul} et on le note ${0}$.
\item (associativité) $(v+w)+t=v+(w+t)$
\item $v-v={0}$
\item (commutativié) $v+w=w+v$
\item (distributivité à droite) $(\lambda+\mu)v=\lambda v + \mu v$
\item (distributivité à gauche) $\lambda (v+w) = \lambda v + \lambda w$
\item (associativité mixte) $(\lambda\mu)v = \lambda (\mu v)$
\end{itemize}}

\defi{(colinéarité) Deux vecteurs $v$ et $w$ sont dits colinéaires si il existe $\lambda\in\R$ tel que $v=\lambda w$.}

\pro{\label{procroix} Soit $u=(u_1,u_2,u_3)$ et $v=(v_1,v_2,v_3)$ deux vecteurs. $u$ et $v$ sont colinéaires \ssi $u_1v_2-u_2v_1=0$, $u_1v_3-u_3v_1=0$ et $u_2v_3-u_3v_2=0$.}

\preuve{\doubleimp{Il existe $\lambda\in\R$ tel que $v=(\lambda u_1,\lambda u_2, \lambda u_3)$. On vérifie alors facilement que $u_1v_2-u_2v_1=0$, $u_1v_3-u_3v_1=0$ et $u_2v_3-u_3v_2=0$.}

{Le résultat est évident pour $v={0}$ donc supposons $v\neq{0}$. $u_1v_2-u_2v_1=0$, $u_1v_3-u_3v_1=0$ et $u_2v_3-u_3v_2=0$ donc il existe $a,b,c\in\R$ tel que $u_1=av_1$, $u_2=av_2$, $u_1=bv_1$, $u_3=b v_3$,  $u_2=cv_2$, $u_3=c v_3$. On a donc :

\sys{av_1 &=& bv_1 \\
av_2 &=& c v_2 \\
bv_3 &=& c v_3}

Puisque $v\neq{0}$, alors $v_1,v_2$ ou $v_3$ est non nul. Supposons donc par exemple $v_1\neq 0$ (les autres cas étant similaires). Alors par la première ligne, on a $a=b$. Mais alors on a $u_1=av_1$, $u_2=av_2$ et $u_3=a v_3$, d'où $u=av$, donc $u$ et $v$ sont colinéaires.}}

\rem{Aussi fou que cela puisse paraître, il faut bien les trois conditions $u_1v_2-u_2v_1=0$, $u_1v_3-u_3v_1=0$ et $u_2v_3-u_3v_2=0$ pour garantir la colinéarité. Si on ne suppose seulement que $u_1v_2-u_2v_1=0$ et $u_1v_3-u_3v_1=0$ par exemple, ça ne marche pas en général : les vecteurs $\vre{0}{0}{1}$ et $\vre{0}{1}{1}$ ne sont pas colinéaires mais respectent pourtant ces deux conditions.}

\defi{(produit scalaire) Soit $u=(x,y,z)$ et $v=(x',y',z')$ deux vecteurs. On appelle \textit{produit scalaire de} $u$ \textit{et} $v$ le réel $xx'+yy'+zz'$. On le note $u\cdot v$}

Avec XCas, on calcule le produit scalaire de $u$ et $v$ par la commande \verb&dot(u,v)&.

\pro{(propriétés du produit scalaire) Soit $u,v,w$ trois vecteurs et $\lambda\in\R$.

\begin{itemize}
\item $u\cdot u \ge 0$
\item $u\cdot u = 0\eqv u={0}$
\item $u\cdot v = v \cdot u$
\item $(u + \lambda v)\cdot w = u\cdot w + \lambda (v\cdot w)$
\end{itemize}}

\nota{On note $u\cdot u$ souvent $u^2$.}

\defi{(norme) Soit $v=(x,y,z)$ un vecteur. On appelle \textit{norme (euclidienne) de} $v$ le réel positif $\sqrt{v^2}=\sqrt{x^2+y^2+z^2}$. On le note $||v||$. Ainsi, $v^2=||v||^2$.}

\rem{En effet, on voit dans le supérieur que cette norme est définie à partir du produit scalaire.}
Avec XCas, on calcule la norme de $u$ et $v$ par la commande \verb&norm(u,v)&.

\lemme{\label{cauchyscwharz} (inégalité de Cauchy-Schwarz) Soit $u,v$ deux vecteurs. Alors $|u\cdot v| \le ||u||\times ||v||$. Autrement écrit, $(u\cdot v)^2 \le u^2 v^2$.}

\rem{Dans les faits, cette proposition sert à démontrer l'inégalité triangulaire de la norme comme nous le verrons dans la proposition suivante. La démonstration de cette inégalité, très ingénieuse, vaut également le coup d'oeil.}

\preuve{Soit $u,v$ deux vecteurs et soit \fons{P}{\R}{\R}{t}{(u+tv)^2}. Soit $t\in\R$. On sait que $P(t)\ge 0$ (premier point de la proposition précédente). De plus, on a  :

\chaine{P(t) &=& (u+tv)^2 \\
&=& (u+tv)\cdot (u+tv) \\
&=& u\cdot (u+tv) + (tv)\cdot (u+tv) \\
&=& u\cdot u + u\cdot (tv) + (tv)\cdot u + (tv)\cdot (tv) \\
&=& u^2 + t(u\cdot v) + t(v\cdot u) + t^2 v^2 \\
&=& v^2 t^2 + 2(u\cdot v) t + u^2 \\
P(t) &=& ||v||^2 t^2 + 2(u\cdot v) t + ||u||^2}

Ainsi, $P$ est une fonction polynomiale réelle de degré 2, positive. Son discriminant $\Delta$ est donc négatif (parce que sinon, $P$ admettrait 2 racines et ne serait donc pas toujours positif). Ainsi :

\chaine{& & \Delta \le 0 \\
&\eqv& 4(u\cdot v)^2-4||v||^2||u||^2 \le 0 \\
&\eqv& (u\cdot v)^2-||u||^2||v||^2 \le 0 \\
&\eqv& (u\cdot v)^2 \le ||u||^2||v||^2 \\
&\eqv& |u\cdot v| \le ||u||\times ||v||}}

\lemme{(identités remarquables) Soit $u,v$ deux vecteurs.

\begin{enumerate}
\item $(u+v)^2=u^2+2(u\cdot v) + v^2$
\item $(u+v)\cdot (u-v) = u^2-v^2$
\end{enumerate}}

\pro{(propriétés de la norme)  Soit $u,v$ deux vecteurs de $\R^3$.

\begin{enumerate}[i)]
\item $v={0}\eqv ||v||=0$
\item Pour tout $\lambda\in\R$, $||\lambda v||=|\lambda|\times ||v||$.
\item (inégalité triangulaire) $||u+v||\le ||u||+||v||$
\end{enumerate}}

\preuve{Nous montrons l'inégalité triangulaire.

\chaine{||u+v||^2 &=& (u+v)^2 \\
&=& u^2+2(u\cdot v) + v^2}

Par l'inégalité de Cauchy-Schwarz, on a donc :

\chaine{& & ||u+v||^2 \le u^2+2||u||\times||v||+ v^2 \\
&\eqv& ||u+v||^2 \le ||u||^2+2||u||\times ||v||+ ||v||^2 \\
&\eqv& ||u+v||^2 \le (||u||+||v||)^2 \\
&\eqv& ||u+v|| \le ||u||+||v||}}

\defi{(orthogonalité) Soit $u,v$ deux vecteurs. On dit que $u$ \textit{et} $v$ \textit{sont} \textit{orthogonaux} si $u\cdot v = 0$.}
\rem{${0}$ est orthogonal avec tous les vecteurs, et c'est le seul.}

\pro{Soit $u,v$ deux vecteurs. Alors si $u',v'$ sont deux vecteurs respectivement colinéaires à $u$ et $v$, $u'$ et $v'$ sont orthogonaux.}

\pro{(génération de vecteurs orthogonaux) Soit $u=(x,y,z)$ un vecteur. Alors le vecteur $n=(-y,x,0)$ est orthogonal à $u$. De plus, tout vecteur colinéaire à $n$ est orthogonal à $u$.}

\section{Lien entre vecteurs et points}

\defi{Soit $A=(x,y,z)$ et $B=(x',y',z')$ deux points de $\R^3$. On appelle \textit{vecteur} $\vr{AB}$ le vecteur $(x'-x,y'-y,z'-z)$.}

\rem{On associe donc chaque couple de points à un unique vecteur. Attention, la réciproque n'est pas vrai : à tout vecteur est associé une infinité de couples de points.}

\pro{(propriétés fondamentales) Soit $A,B,C\in\R^3$.

\begin{itemize}
\item (relation de Chasles) $\vr{AB}+\vr{BC}=\vr{AC}$
\item (translation) Pour tout vecteur $u$, il existe un unique point $D$ tel que $u=\vr{AD}$
\end{itemize}}


\section{Bases, repères et points}

\defi{(base) Soit $B=(v_n)_{n<3}$ une famille de trois vecteurs. On dit que $B$ \textit{est une base de} $\R^3$ si l'équation $x v_0+y v_1+z v_2={0}$ d'inconnues $x,y,z\in\R$ ne possède qu'une unique solution : $(x,y,z)=(0,0,0)$.}

\ex{Soit $v_0=(1,2,3)$, $v_1=(1,1,1)$ et $v_2=(-2,-4,-6)$. Montrons que $B=(v_0,v_1,v_2)$ n'est pas une base de $\R^3$. Résolvons l'équation $x v_0 + y v_1 + z v_2 ={0}$, \cad le système suivant.

\sys{x + y -2z = 0 \\
2x + y -4z = 0 \\
3x + y -6z = 0}

Nous pouvons remarquer que les lignes 1 et 3 sont équivalentes, donc nous pouvons supprimer l'une des deux. Avec les deux lignes restantes, nous trouvons que l'ensemble des solutions est $\ens{(2\lambda,0,\lambda),\lambda\in\R)}$. Cela signifie que cette équation possède une infinité de triplets solutions et que donc $(0,0,0)$ n'est pas l'unique solution. Donc $B$ n'est pas une base de $\R^3$.}

Sur XCas, voici comment vérifier que $B$ est une base ou non : \\

\verb&solve([x*[1,2,3]+y*[1,1,1]+z*[-2,-4,-6]=[0,0,0]],[x,y,z])& \\

Si XCas ne renvoie pas \verb+(0 0 0)+ c'est que $B$ n'est pas une base.

\rem{De manière générale, si deux des trois vecteurs sont colinéaires, alors $B$ n'est pas une base car cela signifie que deux lignes du système sont équivalentes. Mais attention, la réciproque est fausse : si les vecteurs sont deux à deux non colinéaires, ils peuvent tout de même ne pas former une base (prendre par exemple $v_0=(1,0,0),v_1=(2,3,0),v_2=(2,2,0)$). Géométriquement, pour que trois vecteurs non colinéaires deux à deux forment une base, il faut en plus qu'ils soient non coplanaires (\ie ils ne doivent pas être dans un même plan).} 

\pro{(base canonique) La famille de vecteurs $\parent{\vre{1}{0}{0},\vre{0}{1}{0},\vre{0}{0}{1}}$ est une base de $\R^3$. On l'appelle \textit{base canonique de} $\R^3$. On note ces vecteurs respectivement $e_1,e_2,e_3$ (notation surtout utilisée dans le supérieur que nous allons conserver ici, parce que pourquoi pas).}

\preuve{Immédiat.}

\defi{(repère) On appelle \textit{repère de} $\R^3$ tout couple $(O,B)$ où $O$ est un point de $\R^3$ appelé \textit{origine du repère} et où $B$ est une base de $\R^3$ appelée \textit{base du repère}.}

\defi{(repères usuels) Soit $R=(O,B)$ un repère de $\R^3$ avec $B=(v_0,v_1,v_2)$.

\begin{itemize}
\item On dit que $R$ est \textit{normé} si $||v_0||=||v_1||=||v_2||=1$.
\item On dit que $R$ est \textit{orthogonal} si les vecteurs de $B$ sont deux à deux orthogonaux.
\item On dit que $R$ est \textit{orthonormé} (ou \textit{orthonormal}) si $R$ est normé et orthogonal.
\end{itemize}}

\rem{Lorsque la norme d'un vecteur vaut 1, on dit qu'il est \textit{unitaire}.}

\pro{Pour tout point $O$ de $\R^3$, $R=(O,(e_1,e_2,e_3))$ est un repère orthonormé. On appelle en particulier $((0,0,0),(e_1,e_2,e_3))$ le \textit{repère canonique}.}

Nous considérons dans la suite $\mathscr{R}=(O,(v_0,v_1,v_2))$ un repère de $\R^3$ quelconque.

\pro{\label{decompvec} Pour tout vecteur $u$, il existe un unique triplet $(x,y,z)\in\R^3$ tel que $u=x v_0 + y v_1 + z v_2$.}

\preuve{Prouvons l'unicité. Supposons par l'absurde qu'il existe un triplet $(x',y',z')\in\R^3$ différent de $(x,y,z)$ solution. Alors on à la fois  $u=x v_0 + y v_1 + z v_2$ et  $u=x' v_0 + y' v_1 + z' v_2$ d'où $(x-x')v_0 + (y-y')v_1 + (z-z')v_2={0}$. Posons $X=x-x'$, $Y=y-y'$ et $Z=z-z'$. Alors $Xv_0 + Yv_1 + Zv_2={0}$, or puisque $(v_0,v_1,v_2)$ est une base, on a nécessairement $X=Y=Z=0$. D'où $(x,y,z)=(x',y',z')$~: c'est absurde. L'existence se prouve en résolvant de manière générale le système (sur XCas par exemple) : le fait que $(v_0,v_1,v_2)$ est une base va garantir qu'aucune ligne ne sera équivalente à aucun moment de la résolution.}

\defi{Soit $A\in\R^3$. D'après la proposition précédente, il existe un unique triplet $(x,y,z)\in\R^3$ tel que $\vr{OA}=x v_0 + y v_1 + z v_2$. On dit que $x,y,z$ sont \textit{les coordonnées de $A$ dans le repère} $\mathscr{R}$ et on note $M_\mathscr{R}(x,y,z)$.}

\pro{On a $O_\mathscr{R}(0,0,0)$.}

\pro{Soit $M=(x,y,z)$ un point. Si $\mathscr{R}$ est le repère canonique, alors $M_\mathscr{R} (x,y,z)$, ce qu'on note de manière raccourcie $M(x,y,z)$.}

\rem{Donc, quand on écrit «~soit $M(2,6,-7)$~» cela signifie que les coordonnées de $M$ dans le repère canonique sont $2,6$ et $-7$.}

\ex{\label{exrep} Soit $v_0=(1,2,3)$, $v_1=(-1,0,5)$ et $v_3=(1,-1,0)$ trois vecteurs. On suppose que $B=(v_0,v_1,v_2)$ est une base. Soit $O(7,1,2)$ un point et $\mathscr{R}=(O,B)$. Enfin, soit $A(4,3,-5)$ un point. Trouvons les coordonnées de $A$ dans $\mathscr{R}$. Notons-les $x,y,z$. Alors $\vr{OA}$ vérifie $\vr{OA}=x v_0 + y v_1 + z v_2$, \cad le système suivant.

\sys{x -y + z &=& 4-7 \\
2x -z &=& 3-1 \\
3x + 5y &=& -5-2}

On résout le système, on trouve $(x,y,z)=\parent{-\dfrac{2}{3},-1,-\dfrac{10}{3}}$. Conclusion, $A_\mathscr{R}\parent{-\dfrac{2}{3},-1,-\dfrac{10}{3}}$.}

On peut aussi résoudre directement sur XCas (ce que je n'ai pas fait pour résoudre cet exo, bien entendu...) : \\

\verb&solve([x*[1,2,3]+y*[-1,0,5]+z*[1,-1,0]=[4-7,3-1,-5-2]],[x,y,z])&

\pro{(changement d'origine) Soit $A$ un point tel que $A_\mathscr{R}(x,y,z)$ Soit $O'_\mathscr{R}(\alpha,\beta,\gamma)$ et $\mathscr{R}'=(O',B)$. Alors $A_{\mathscr{R}'}(x-\alpha,y-\beta,z-\gamma)$.}

\preuve{On a $\vr{O'A}=\vr{O'O}+\vr{OA}=\vr{OA}-\vr{OO'}$. On sait que $\vr{OA}=x v_0 + y v_1 + z v_2$ et que $\vr{OO'}=\alpha v_0 + \beta v_1 + \gamma v_2$ d'où $\vr{O'A}=(x-\alpha) v_0 + (y-\beta) v_1 + (z-\gamma) v_2$ et donc $A_{\mathscr{R}'}(x-\alpha,y-\beta,z-\gamma)$.}

\nota{On note $\dis$ la distance euclidienne entre deux points. Autrement dit, soit $A(x,y,z)$ et $B(x',y',z')$. Alors $\dis(A,B)=\sqrt{(x'-x)^2+(y'-y)^2+(z'-z)^2}$.}

\pro{Soit $A(x,y,z)$ et $B(x',y',z')$. Alors $\dis(A,B)=||\vr{AB}||$.}

\pro{Soit $A_\mathscr{R}(x,y,z)$ et $B_\mathscr{R}(x',y',z')$. Si $\mathscr{R}$ est orthonormé, alors $\dis(A,B)=\sqrt{(x'-x)^2+(y'-y)^2+(z'-z)^2}$.}

\rem{Cette proposition est utile car si on dispose des coordonnées de deux points dans un repère orthonormé quelconque, alors on peut calculer leur distance.}

\preuve{Supposons que $A(\alpha,\beta,\gamma)$ et $B(\alpha',\beta',\gamma')$ (les coordonnées de $A$ et $B$ dans le repère canonique). $\mathscr{R}$ et le repère canonique sont tous deux orthonormés. Cela signifie  qu'ils sont égaux à une translation et une ou plusieurs rotations près. Or, de même que dans le plan, les translations et rotations de l'espace sont des isométries, \cad que les distances sont conservées. Cela signifie que $\alpha'-\alpha=x'-x$, $\beta'-\beta=y'-y$ et $\gamma'-\gamma=z'-z$. Or par définition, $\dis(A,B)=\sqrt{(\alpha'-\alpha)^2+(\beta'-\beta)^2+(\gamma'-\gamma)^2}$. Donc $\dis(A,B)=\sqrt{(x'-x)^2+(y'-y)^2+(z'-z)^2}$.}

\section{Objets de l'espace}

On se place dans toute la suite le repère canonique $\mathscr{C}$.

\defi{(paramétrage). Soit $E$ un ensemble de points. Supposons qu'il existe trois fonctions \fone{x,y,z}{\R^n}{\R} avec $n\in\N^*$, tel que $E=\ens{\parent{(x(t_1,...,t_n),y(t_1,...,t_n),z(t_1,...,t_n)},\forall t_1,...,t_n\in\R}$.
\begin{itemize}
\item On dit qu'on a \textit{paramétré} $E$.
\item On dit que $t_1,...,t_n$ sont les \textit{paramètres} de ce paramétrage.
\item On dit que $x,y,z$ sont les \textit{composantes} de ce paramétrage.
\end{itemize}

 }

\rem{En somme, paramétrer un ensemble de points, c'est juste donner leurs trois coordonnées sous forme de fonctions, éventuellement de plusieurs variables. Attention, ici il faut bien comprendre que $x,y,z$ contrairement à d'habitude sont ici des fonctions.}

\nota{Soit $E$ un ensemble de points paramétré. Soit \fone{f,g,h}{\R^n}{\R} avec $n\in\N^*$, tel que pour tout $t_1,...,t_n\in\R$, on a :

\sys{x(t_1,...,t_n) &=& f(t_1,...t_n) \\
y(t_1,...,t_n) &=& g(t_1,...t_n) \\
z(t_1,...,t_n) &=& h(t_1,...t_n)}

On appelle ce système \textit{système d'équations paramétriques de $E$} ou simplement \textit{équation paramétrique de $E$}.}

\rem{Nous allons dans la suite pouvoir paramétrer certains objets de l'espace, ces notions qui peuvent paraître un peu obscures s'éclairciront. L'intérêt est de pouvoir déterminer d'un coup les coordonnées de tous les points de $E$.}

\subsection{Plans et droites}

\subsubsection{Définitions des plans}

\defi{(plan) Soit $u,v$ deux vecteurs non colinéaires et $M\in\R^3$. On appelle \textit{plan dirigé par $u$ et $v$ et passant par $M$} l'ensemble $\ens{A\in\R^3,\exists \lambda,\mu\in\R,\vr{MA}=\lambda u + \mu v}$. On dit que $u$ et $v$ sont \textit{des vecteurs directeurs de} $\mathscr{P}$.}

\rem{Un plan $\plan$ admet une infinité de vecteurs directeurs. En particulier, si $u$ et $v$ dirigent $\plan$, alors soit $u'\neq{0}$ colinéaire à $u$ et $v'\neq{0}$ colinéaire à $v$, alors $u'$ et $v'$ dirigent $\plan$.}

\pro{(paramétrage du plan) Soit $u=(u_1,u_2,u_3)$ et $v=(v_1,v_2,v_3)$ deux vecteurs non colinéaires. Soit $M(m_1,m_2,m_3)$. Soit enfin $\mathscr{P}$ le plan dirigé par $u$ et $v$ et passant par $M$. Alors voici un système d'équations paramétriques de $\mathscr{P}$, pour tout $t_1,t_2\in\R$~:

\sys{x(t_1,t_2) &=& t_1 u_1 + t_2 v_1 + m_1 \\
y(t_1,t_2) &=& t_1 u_2 + t_2 v_2 + m_2 \\
z(t_1,t_2) &=& t_1 u_3 + t_2 v_3 + m_3 \\}}

\rem{Ainsi, si on pose $t_1=2$ et $t_2=-1$, ce système paramétrique nous donne le point $(2u_1 - v_1 + m_1,2 u_2 - v_2 + m_2,2 u_3 - v_3 + m_3)$ qui appartient donc à $\mathscr{P}$. Et en faisant ainsi parcourir $t_1$ et $t_2$ sur $\R$, on obtient tous les points de $\mathscr{P}$.}

\preuve{Pour montrer qu'un système d'équations paramètre un ensemble $E$, il faut d'une part prendre un point de $E$ et montrer que ses coordonnées vérifient nécessairement le système, puis réciproquement prendre un point dont les coordonnées vérifient le système et montrer qu'alors ce point est dans $E$ (il s'agit en fait d'un raisonnement par double inclusion).

\begin{itemize}
\item Soit $A(x,y,z)\in\mathscr{P}$. Montrons alors que les coordonnées de $A$ vérifient nécessairement le système. Il existe $\lambda,\mu\in\R$ tel que $\vr{MA}=\lambda u + \mu v$, \cad tel que  $\vre{x-m_1}{y-m_2}{z-m_3} = \vre{\lambda u_1 + \mu v_1}{\lambda u_2 + \mu v_2}{\lambda u_3 + \mu v_3}$, \cad tel que :

\sys{x &=& \lambda u_1 + \mu v_1 + m_1 \\
y &=& \lambda u_2 + \mu v_2 + m_2 \\
z &=& \lambda u_3 + \mu v_3 + m_3 \\}

Donc $x=x(\lambda,\mu)$, $y=y(\lambda,\mu)$ et $z=z(\lambda,\mu)$ : les coordonnées de $A$ vérifient le système.

\item Soit $t_1,t_2\in\R$ et $A(x(t_1,t_2),y(t_1,t_2),z(t_1,t_2))$ (on a donc pris $A$ tel que ses coordonnées vérifient le système). On a alors $\vr{MA}=\vre{x(t_1,t_2)-m_1}{y(t_1,t_2)-m_2}{z(t_1,t_2)-m_3}=\vre{t_1 u_1 + t_2 v_1 + m_1-m_1}{t_1 u_2 + t_2 v_2 + m_2-m_2}{t_1 u_3 + t_2 v_3 + m_3-m_3}=\vre{t_1 u_1 + t_2 v_1}{t_1 u_2 + t_2 v_2}{t_1 u_3 + t_2 v_3}=t_1 u + t_2 v$ d'où $A\in\mathscr{P}$.
\end{itemize}}

\pro{(équation cartésienne d'un plan) Soit $a,b,c,d\in\R$ avec $a,b,c$ non tous nuls et soit $E=\ens{(x,y,z)\in\R^3|ax+by+cz+d=0}$. Alors $E$ est un plan. Notons-le $\mathscr{P}$. On dit que $ax+by+cz+d=0$ est \textit{une équation cartésienne de} $\mathscr{P}$ et on note $\mathscr{P}:ax+by+cz+d=0$.}

\rem{Sémantiquement, attention de ne pas confondre les phrases «~$a,b,c$ sont \textit{non tous} nuls~» qui signifie que $(a,b,c)\neq (0,0,0)$ (\cad qu'au moins un des trois est non nul) et «~$a,b,c$ sont \textit{tous non} nuls~» qui signifie que $a\neq0, b\neq0$ et $c\neq 0$ (\cad que les trois sont non nuls).}
 
\preuve{\begin{itemize} 
\item Supposons $c\neq 0$. Montrons que $E$ est dans un plan. Soit $A(x,y,z)\in E$. Ainsi, $ax+by+cz+d=0$. Posons $\lambda=x$ et $\mu=y$. Alors $a\lambda+b\mu+cz+d=0$ d'où $z=-\dfrac{a\lambda+b\mu+d}{c}$. Nous avons ainsi :

\sys{x &=& \lambda \\
y &=& \mu \\
z &=& -\dfrac{a\lambda+b\mu+d}{c}}

Donc les coordonnées de $A$ vérifient le système paramétrique du plan $\mathscr{P}$ dirigé par les vecteurs non colinéaires $u=\parent{1,0,-\dfrac{a}{c}}$ et $v=\parent{0,1,-\dfrac{b}{c}}$ et passant par $M\parent{0,0,-\dfrac{d}{c}}$, avec $t_1=\lambda$ et $t_2=\mu$. Donc $A\subset\mathscr{P}$ et donc $E\subset\mathscr{P}$. Réciproquement, montrons que $\mathscr{P}\subset E$. Soit $B\in\mathscr{P}$. Alors il existe $t_1,t_2\in\R$ tel que $B(x(t_1,t_2),y(t_1,t_2),z(t_1,t_2))$. On vérifie alors par le calcul que $ax(t_1,t_2)+by(t_1,t_2)+cz(t_1,t_2)+d=0$.
\item Supposons $c=0$ et $b\neq 0$. Alors on démontre exactement de la même façon en posant cette fois $\lambda=x$ et $\mu=z$.
\item Supposons $c=0$ et $b=0$. Alors $a\neq 0$. Alors on démontre exactement de la même façon en posant cette fois $\lambda=y$ et $\mu=z$.
\end{itemize}}

\rems{\item La démonstration nous donne en plus une façon de trouver des vecteurs directeurs et un point appartenant à un plan dont on connait une équation cartésienne.
\item Si $a=b=c=0$, Alors $E$ devient $\ens{(x,y,z)\in\R^3|d=0}$, \cad soit $\R^3$ en entier si $d=0$ soit $\emptyset$ si $d\neq 0$, mais ces deux ensembles ne sont pas des plans. C'est pourquoi il faut que $a,b,c$ soient non tous nuls.
\item \textbf{Attention au gros piège} : quand on écrit $\mathscr{P}:2x-y+5=0$ par exemple, il faut bien comprendre que dans l'espace, cet objet \textit{n'est pas} une droite (alors que dans le plan, nous serions effectivement en présence d'une équation cartésienne de droite). Il s'agit bien d'un \textit{plan}. Pour s'en convaincre, il faut revenir à la définition : $\mathscr{P}:2x-y+5=0$ signifie que $\mathscr{P}=\ens{(x,y,z)\in\R^3|2x-y+5=0}$ : on voit donc bien que $z$ n'a pas disparu : c'est juste que $c=0$ ici. D'où l'importance cruciale de savoir si on travaille dans le plan ou dans l'espace. D'ailleurs, sur GeoGebra 3D, on peut constater que si on tape $2x-y+5=0$ le logiciel représente bien un plan.}

\lemme{Soit $\plan,\plan'$ deux plans et $A,B,C$ trois points distincts et non alignés. Si $A,B,C\in\plan$ et $A,B,C\in\plan'$ alors $\plan=\plan'$.}

\preuve{Puisque $A,B,C$ sont distincts et non alignés alors $\vr{AB}$ et $\vr{AC}$ sont non colinéaires. Donc $\plan$ et $\plan'$ sont deux plans dirigés par $\vr{AB}$ et $\vr{AC}$ et passant par $A$. Donc $\plan=\plan'$}

\pro{Tout plan possède une équation cartésienne.}

\preuve{Soit $\plan$ un plan dirigé par les vecteurs $u=(u_1,u_2,u_3)$ et $v=(v_1,v_2,v_3)$ et passant par $M(m_1,m_2,m_3)$. Alors par le système d'équations paramétriques de $\plan$, on a que les points $A(x(0,1),y(0,1),z(0,1))$ et $B((x(1,0),y(1,0),z(1,0))$ appartiennent à $\plan$. Raisonnons par analyse-synthèse.
\begin{description}
\item[Analyse] Supposons donc l'existence d'une équation cartésienne de $\plan$, \cad qu'il existe $a,b,c,d\in\R$ avec $a,b,c$ non tous nuls tel que $\mathscr{P}:ax+by+cz+d=0$. Comme $M,A,B$ sont dans $\plan$ alors leurs coordonnées vérifient :

\sys{am_1+bm_2+cm_3+d &=& 0 \\
ax(0,1)+by(0,1)+cz(0,1)+d &=& 0 \\
ax(1,0)+by(1,0)+cz(1,0)+d &=& 0}

Posons les vecteurs $w_1=\vre{m_1}{x(0,1)}{x(1,0)}$, $w_2=\vre{m_2}{y(0,1)}{y(1,0)}$, $w_3=\vre{m_3}{z(0,1)}{z(1,0)}$ et $w_4=\vre{-d}{-d}{-d}$. Alors le système d'équations précédent est équivalent à l'équation $w_4=aw_1+bw_2+cw_3$. Or on vérifie que $(w_1,w_2,w_3)$ est une base de $\R^3$, donc par la proposition \ref{decompvec}, $(a,b,c)$ est l'unique triplet de réels satisfaisant cette équation et donc le système précédent. Ainsi, nous venons de prouver que si $\plan$ admet une équation cartésienne $\mathscr{P}:ax+by+cz+d=0$, alors nécessairement $(a,b,c)$ est l'unique triplet solution de l'équation $w_4=aw_1+bw_2+cw_3$.

\item[Synthèse] Soit $d\in\R$. Posons $(a,b,c)$ l'unique triplet de réels solution de $w_4=aw_1+bw_2+cw_3$. Soit $\plan':ax+by+cz+d=0$. Montrons alors que $\plan'=\plan$. Puisque $(a,b,c)$ est l'unique triplet de réels solution de $w_4=aw_1+bw_2+cw_3$ alors on a :

\sys{am_1+bm_2+cm_3+d &=& 0 \\
ax(0,1)+by(0,1)+cz(0,1)+d &=& 0 \\
ax(1,0)+by(1,0)+cz(1,0)+d &=& 0}

C'est-à-dire que $M,A,B\in\plan'$. Mais on a également $M,A,B\in\plan$. Comme $M,A,B$ sont distincts et non alignés, alors par le lemme, $\plan'=\plan$.

\end{description}}

Nous disposons ainsi de trois manières de définir les plans de l'espace. Selon ce que l'on veut faire, telle ou telle définition sera plus pratique, il faut donc savoir jongler entre les trois. Par exemple, l'équation cartésienne est très pratique pour savoir si un point appartient à un plan. L'équation paramétrique est plus efficace en revanche pour générer les points du plan en faisant varier les paramètres. Enfin, la première définition est pratique pour montrer des choses d'ordre géométrique.

Sur GeoGebra, on peut représenter le plan des trois manières. Par exemple, soit $\plan$ un plan dirigé par $u$ et $v$ et passant par $M$. Soit $ax+by+cz+d=0$ une équation cartésienne de $\plan$ et soit $X(t),Y(t)$ et $Z(t)$ les composantes du paramétrage de $\plan$, pour tout $t\in\R$. Alors voici trois commandes successives pour représenter $\plan$ sur GeoGebra de ces trois manières : \\

\verb&P=Plane(M,u,v)& \\
\verb&P:ax+by+cz+d=0& \\
\verb&P=curve(X(t),Y(t),Z(t),t,-100,100)& \\

Enfin, si on connait trois point $A,B,C$ distincts et non alignés de $\plan$, on peut le représenter via la commande : \\

\verb&P=Plane(A,B,C)& \\


\pro{Tout plan possède une infinité d'équations cartésiennes.}

\rem{C'est pourquoi on dit \textit{une} équation cartésienne et non \textit{l}'équation cartésienne.}

\preuve{Soit $\plan:ax+by+cz+d=0$. Soit $\lambda\in\R^*$. Alors $ax+by+cz+d=0\eqv \lambda(ax+by+cz+d)=0$, donc $\lambda ax+\lambda by+\lambda cz+\lambda d=0$ est une équation cartésienne de $\plan$.}

\nota{(plans particuliers) \\\begin{itemize}
\item On note $(Oxy)$ le plan dirigé par $e_1$ et $e_2$ et passant par $O(0,0,0)$. Une équation cartésienne de $(Oxy)$ est $z=0$.
\item On note $(Oxz)$ le plan dirigé par $e_1$ et $e_3$ et passant par $O(0,0,0)$. Une équation cartésienne de $(Oxz)$ est $y=0$.
\item On note $(Oyz)$ le plan dirigé par $e_2$ et $e_3$ et passant par $O(0,0,0)$. Une équation cartésienne de $(Oyz)$ est $x=0$.
\end{itemize}}

\preuve{Nous traitons le premier cas, les deux autres étant similaires. Premièrement, $e_1$ et $e_2$ sont non colinéaires, donc $(Oxy)$ est bien un plan. On montre facilement que $A(1,0,0)$ et $B(0,1,0)$ sont dans $(Oxy)$. Soit $\plan:z=0$. On montre facilement que $A,B,O\in\plan$, or $A,B,O$ sont distincts et non alignés, donc $(Oxy)=\plan$, conclusion $z=0$ est bien une équation cartésienne de $(Oxy)$.}

\subsubsection{Définitions des droites}

Comme nous l'avons expliqué, il n'est pas possible de définir les droites de l'espace par une équation cartésienne directement (car une équation cartésienne, dans l'espace, est toujours un plan). En revanche, le paramétrage des droites de l'espace est tout à fait possible et similaire à celui des plans.

\defi{(droite) Soit $u\neq{0}$ un vecteur et $M$ un point. On appelle \textit{droite dirigée par $u$ et passant par $M$} l'ensemble $\ens{A\in\R^3|\exists \lambda\in\R, \vr{MA}=\lambda u}$. Si on note $\dt$ cette droite, on dit que $u$ est un \textit{vecteur directeur de $\dt$.}}

\pro{(propositions basiques) Soit $\dt$ la droite dirigée par $u$ et passant par $M$. Soit $A,B\in\dt$ distincts.

\begin{enumerate}[i)]
\item $\dt$ est dirigée par tout vecteur $v\neq{0}$ colinéaire à $u$.

\item $\dt$ est dirigée par $\vr{AB}$.

\item $\dt$ est la droite dirigée par $u$ et passant par $A$.

\item Soit $\dt'$ une droite. Si $A,B\in\dt'$ alors $\dt=\dt'$.
\end{enumerate}}

\preuve{Nous ne montrons que le premier point, la démonstration des autres étant très similaires. Si le lecteur souhaite s'entraîner à démontrer les autres, il est conseillé de les faire dans l'ordre.\\
Soit $k\in\R^*$ et $v=k u$. Soit $\dt'$ la droite dirigée par $v$ et passant par $M$. Soit $A\in\dt$. Alors il existe $\lambda\in\R, \vr{MA}=\lambda u$. Posons $\mu=\dfrac{\lambda}{k}$. Alors $\vr{MA}=\mu k u=\mu v$ d'où $A\in\dt'$. Donc $\dt\subset \dt'$. Soit maintenant $B\in\dt'$. Alors il existe $\lambda\in\R, \vr{MB}=\lambda v=\lambda k u$. Posons $\mu=\lambda k$. Alors $\vr{MB}=\mu u$ d'où $B\in\dt$. Donc $\dt'\subset \dt$. Conclusion, $\dt=\dt'$.}

\pro{(paramétrage de la droite) Soit $\dt$ la droite dirigée par $u=(u_1,u_2,u_3)$ et passant par $M(m_1,m_2,m_3)$. Alors voici un système d'équations paramétriques de $\dt$, pour tout $t\in\R$ :

\sys{x(t) &=& u_1 t + m_1 \\
y(t) &=& u_2 t + m_2 \\
z(t) &=& u_3 t + m_3}}

\preuve{La preuve est absolument similaire à celle du paramétrage du plan.}

\ex{Soit $u=\vre{1}{-2}{3}$ et $M(1,-4,-7)$ et $\dt$ la droite dirigée par $u$ et passant par $M$. Les points $A\parent{\dfrac{3}{2},-5,-\dfrac{11}{2}}$ et $B\parent{10,-22,21}$ sont-ils dans $\dt$ ? Voici un paramétrage de $\dt$ :

\sys{x(t) &=& t+1 \\
y(t) &=& -2t-4 \\
z(t) &=& 3t-7}

Supposons que $A\in\dt$. Alors il existe $t\in\R$ tel que $x(t)=\dfrac{3}{2}$ \cad tel que $t+1=\dfrac{3}{2}$. Donc nécessairement $t=\dfrac{1}{2}$. Puisque $A\in\dt$, on doit avoir $y\parent{\dfrac{1}{2}}=-5$ et c'est le cas. Enfin, on doit avoir $z\parent{\dfrac{1}{2}}=-\dfrac{11}{2}$ et c'est encore le cas, conclusion on a bien $A\in\dt$. Supposons que $B\in\dt$. Alors il existe $t\in\R$ tel que $x(t)=10$, donc nécessairement $t=9$. On doit donc avoir $y(9)=-22$ et c'est le cas. Enfin, on doit avoir $z(9)=21$ mais ce n'est pas le cas car $z(9)=20$. Conclusion $B\not\in\dt$.}

Sur GeoGebra, pour représenter une telle droite, on peut utiliser la commande suivante : \\

\verb&Curve(t+1,-2*t-4,3*t-7,t,-100,100)& \\

\exo{Soit le plan $\plan$ dirigé par $e_2$ et $e_3$ et passant par $O'(4,5,-2)$. Soit $\dt$ la droite de $\plan$ dont une équation cartésienne dans $\plan$ munit du repère orthonormé $\rep'=(O',(e_2,e_3))$ est $4x-5y+1=0$. Déterminer un système d'équations paramétriques de $\dt$ dans l'espace (munit du repère canonique). Pour vérifier, voici un système d'équations paramétriques possible de $\dt$ (notons qu'il est possible que vous ne trouviez pas le même système mais qu'il soit tout de même juste, un système paramétrique n'est jamais unique) :

\sys{x(t) &=& 4 \\
y(t) &=& 5t+5 \\
z(t) &=& 4t-\dfrac{9}{5}}}

\subsubsection{Objets et caractéristiques liés aux droites et aux plans}

\defi{(points coplanaires) Soit $(E_k)_{k<n}$ une famille de $n$ ensemble de points. On dit que qu'ils sont \textit{coplanaires} s'il existe un plan $\plan$ tel que pour tout $k\in\lint 0,n\lint $, $E_k\subset\plan$.}

\rem{Trois points ou moins sont forcément coplanaires.}

\defi{(sécant) Soit $E$ et $F$ deux ensembles de points. On dit que \textit{$E$ et $F$ sont sécants} si $E\cap F\neq \emptyset$.}

Soit dans la suite $\dt$ et $\dt'$ deux droites dirigées par $u$ et $u'$.

\defi{(parallélisme de droites) On dit que \textit{$\dt$ et $\dt'$ sont parallèles} si $u$ dirige $\dt'$. On note alors $\dt // \dt'$.}

\defi{(orthogonalité de droites) On dit que \textit{$\dt$ et $\dt'$ sont orthogonales} si $u$ et $u'$ sont orthogonaux.}

\defi{(perpendicularité de droites) On dit que \textit{$\dt$ et $\dt'$ sont perpendiculaires} si elles sont sécantes et orthogonales. On note alors $\dt\perp\dt'$.}

\rem{Attention, même si dans la suite on verra que la notion d'orthogonalité est représentée souvent par le symbole $\perp$, dans le cas des droites il signifie réellement la perpendicularité. Il n'y a pas de raccourci pour dire que deux droites sont orthogonales. Toutefois, si dans un exercice on doit beaucoup utiliser cette notion, on a tout à fait le droit de créer une notation pour l'occasion, par exemple $\llcorner$ (ce n'est qu'une suggestion).}

Soit dans toute la suite $\plan$ un plan dirigé par $u=(u_1,u_2,u_3)$ et $v=(v_1,v_2,v_3)$ et passant par $M(m_1,m_2,m_3)$.

\defi{(vecteur parallèle) On dit que \textit{$w$ est un vecteur parallèle à $\plan$} s'il existe $\lambda,\mu\in\R$ tel que $w=\lambda u + \mu v$. On note alors $w // \plan$.}

\defi{(vecteurs coplanaires) Soit $(v_k)_{k<n}$ une famille de $n$ vecteurs. On dit que ces vecteurs sont \textit{coplanaires} s'il existe un plan $\plan$ tel que pour tout $k\in\lint 0,n\lint $, $v_k // \plan$.}

\rem{Deux vecteurs ou moins sont forcément coplanaires.}

\pro{Soit $w_1,w_2$ deux vecteurs parallèles à $\plan$ non colinéaires. Alors $\plan$ est dirigé par $w_1$ et $w_2$.}

\preuve{Soit $k_1,k_2,k_3,k_4\in\R$ tel que $w_1=k_1 u + k_2 v$ et $w_2=k_3 u + k_4 v$. Soit $A\in\plan$. Donc il existe $t_1,t_2\in\R,\vr{MA}=t_1 u + t_2 v$. Posons $\lambda,\mu$ les solutions du système :

\sys{\lambda k_1 + \mu k_3 &=& t_1 \\
\lambda k_2 + \mu k_4 &=& t_2}

A noter que ce système possède bien une unique solution car $w_1$ et $w_2$ sont non colinéaires. Ainsi, nous avons $\vr{MA}=t_1 u + t_2 v = (\lambda k_1 + \mu k_3) u + (\lambda k_2 + \mu k_4) v = \lambda w_1 + \mu w_2$. Donc $\plan =\ens{A\in\R^3|\exists \lambda,\mu\in\R, \vr{MA}=\lambda w_1+\mu w_2}$, d'où $\plan$ est dirigé par $w_1$ et $w_2$.}

\defi{(droite parallèle à un plan) Soit $\dt$ une droite dirigée par $n$. On dit que \textit{$\dt$ est parallèle à $\plan$} si $n//\plan$. On note $\dt // \plan$.}

\defi{(vecteur normal) On dit que \textit{$n$ est un vecteur normal de $\plan$} si $n$ est orthogonal à $u$ et à $v$. On note $n\perp\plan$.}

\pro{\label{normcol} Soit $n$ un vecteur normal de $\plan$. Alors $n'$ est un vecteur normal de $\plan$ \ssi $n$ et $n'$ sont colinéaires.}

\preuve{\begin{itemize}
\item ($\Leftarrow$) Il existe $\lambda\in\R$ tel que $n'=\lambda n$. On vérifie facilement que $u\cdot n' = 0 = v\cdot n'$ donc $n'$ est un vecteur normal de $\plan$.
\item ($\Rightarrow$) Nous allons faire une démonstration géométrique. Soit $U,V$ les points de $\plan$ tel que $\vr{MU}=u$ et $\vr{MV}=v$. Soit $N,N'$ les points tel que $\vr{MN}=n$ et $\vr{MN'}=n'$. Donc $(MN)$ et $(MN')$ sont perpendiculaires à $(MU)$ et à $(MV)$. Or dans l'espace, si deux droites sont perpendiculaires à deux mêmes droites non parallèles, alors elles sont parallèles entre elles. Donc $(NM)$ et $(NM')$ sont parallèles, donc $n$ et $n'$ sont colinéaires.
\end{itemize}}

\defi{(droite orthogonale à un plan) Soit $\dt$ une droite dirigée par $n$. On dit que \textit{$\dt$ est orthogonale à $\plan$} si $n\perp\plan$. On note $\dt\perp\plan$.}

\lemme{$\parent{\vre{a}{b}{c},\vre{a'}{b'}{c'},\vre{a''}{b''}{c''}}$ est une base de $\R^3$ \ssi $\parent{\vre{a}{a'}{a''},\vre{b}{b'}{b''},\vre{c}{c'}{c''}}$ est une base de $\R^3$.}

\preuve{Ce résultat est admis car il nécessite l'utilisation des matrices. Il provient du fait qu'une matrice $M$ est inversible \ssi la transposée de $M$ est inversible (\cad la matrice où les lignes et les colonnes de $M$ sont échangées).}

\pro{Soit $n\neq{0}$ un vecteur normal de $\plan$ et $w$ un vecteur. Alors $w//\plan$  \ssi $n\cdot w = 0$.}

\preuve{\begin{itemize}
\item ($\Rightarrow$) Il existe $\lambda,\mu\in\R$ tel que $w=\lambda u + \mu v$. On a $n\cdot w = n\cdot (\lambda u + \mu v) = \lambda (n\cdot u) + \mu (n\cdot v)=0$.
\item ($\Leftarrow$) Posons $w=(w_1,w_2,w_3)$. Supposons que $n\cdot w=0$. On sait également que $n\cdot u=0$ et $n\cdot v = 0$ donc :

\sys{n_1 w_1 + n_2 w_2 + n_3 w_3 &=& 0 \\
n_1 u_1 + n_2 u_2 + n_3 u_3 &=& 0 \\
n_1 v_1 + n_2 v_2 + n_3 v_3 &=& 0}

C'est-à-dire $n_1\vre{w_1}{u_1}{v_1}+n_2\vre{w_2}{u_2}{v_2}+n_3\vre{w_3}{u_3}{v_3}={0}$. Or $(n_1,n_2,n_3)\neq{0}$ donc $\parent{\vre{w_1}{u_1}{v_1},\vre{w_2}{u_2}{v_2},\vre{w_3}{u_3}{v_3}}$ n'est pas une base de $\R^3$. Donc d'après le lemme, $(w,u,v)$ n'est pas non plus une base de $\R^3$. Donc il existe $(a,b,c)\neq(0,0,0)$ tel que $aw+bu+cv={0}$. Montrons maintenant que $a\neq 0$. Si $a=0$, alors on a $bu+cv={0}$, mais comme $(a,b,c)\neq(0,0,0)$ alors $b$ ou $c$ est non nul, et donc $u$ et $v$ sont colinéaires, ce qui est absurde puisqu'ils dirigent un plan. Donc $a\neq0$. Finalement, $w=-\dfrac{bu+cv}{a}$ donc $w // \plan$.
\end{itemize}}

\pro{Soit $\plan:ax+by+cz+d=0$.

\begin{enumerate}[i)]
\item $\vre{a}{b}{c}\perp \plan$
\item $\vre{0}{-c}{b},\vre{-c}{0}{a},\vre{-b}{a}{0} // \plan$.
\end{enumerate}}

\preuve{\begin{enumerate}[i)]
\item Supposons $c\neq 0$. Alors $A\parent{0,0,-\dfrac{d}{c}},B\parent{1,0,-\dfrac{d+a}{c}},C\parent{0,1,-\dfrac{d+b}{c}}\in\plan$. $\vr{AB}$ et $\vr{AC}$ ne sont pas colinéaires donc dirigent $\plan$ et on vérifie qu'ils sont orthogonaux à $\vre{a}{b}{c}$, d'où ce dernier est normal à~$\plan$. Si $c=0$ alors on suppose $b\neq0$ et la démonstration est similaire, enfin si $c=b=0$ on a $a\neq 0$ et on montre le résultat.

\item Le produit scalaire de chacun d'eux avec $\vre{a}{b}{c}$ est nul, donc par la proposition précédente ils sont tous parallèles à $\plan$.
\end{enumerate}}

\defi{(plans parallèles) Soit $\plan'$ un plan. On dit que \textit{$\plan$ et $\plan'$ sont parallèles} si $u$ et $v$ dirigent $\plan'$. On note $\plan //\plan'$.}

\pro{Soit $\plan'$ un plan. $\plan // \plan'$ \ssi il existe un vecteur normal de $\plan$ qui est un vecteur normal de $\plan'$.}

\preuve{\begin{itemize}
\item ($\Rightarrow$) Soit $\plan'$ un plan parallèle à $\plan$. Soit $n$ un vecteur normal de $\plan$. Alors $n\cdot u = n\cdot v = 0$ or $u$ et $v$ dirigent $\plan'$ donc $n$ est normal à $\plan'$.
\item ($\Leftarrow$) Soit $n$ un vecteur normal de $\plan$ et de $\plan'$. On a alors $u // \plan'$ et $v // \plan'$, or $u$ et $v$ sont non colinéaires donc ils dirigent $\plan'$. Donc par définition, $\plan' // \plan$.
\end{itemize}}

\pro{Soit $\plan:ax+by+cz+d=0$ et $\plan':a'x+b'y+c'z+d'=0$. $\plan // \plan'$ \ssi $\vre{a}{b}{c}$ et $\vre{a'}{b'}{c'}$ sont colinéaires.}

\preuve{\begin{itemize}
\item ($\Rightarrow$)  $\vre{a'}{b'}{c'}$ un vecteur normal de $\plan'$. $\plan // \plan'$ donc $\vre{a}{b}{c}$ est aussi un vecteur normal de $\plan'$. Donc d'après la proposition \ref{normcol}, $\vre{a}{b}{c}$ et $\vre{a'}{b'}{c'}$ sont colinéaires.
\item ($\Leftarrow$) $\vre{a}{b}{c}$ et $\vre{a'}{b'}{c'}$ sont colinéaires donc d'après la proposition \ref{normcol}, $\vre{a'}{b'}{c'}$ est normal à $\plan$. Donc d'après la proposition précédente, $\plan // \plan'$.
\end{itemize}}

\defi{(plans perpendiculaires) Soit $\plan'$ un plan. On dit que \textit{$\plan$ et $\plan'$ sont perpendiculaires} si il existe une droite $\dt\subset\plan'$ tel que $\dt\perp\plan$. On note alors $\plan\perp\plan'$.}

\rem{Dans cette partie on utilise beaucoup les notions d'orthogonalité et de perpendicularité. Pour ne pas confondre les deux, retenir ceci : la perpendicularité de deux objets \textit{implique toujours qu'ils sont sécants}, alors que l'orthogonalité non.}

\pro{Soit $\plan'$ un plan, $n$ normal à $\plan$ et $n'$ normal à $\plan'$. $\plan\perp\plan'$ \ssi $n\perp n'$.}
\preuve{Immédiat par la définition.}

\subsubsection{Intersections}

\pro{(intersections de plans) Soit $\plan'$ un plan.

\begin{enumerate}[i)]
\item Si $\plan // \plan'$ et s'ils sont sécants alors $\plan\cap\plan'=\plan=\plan'$.
\item Si $\plan // \plan'$ et si $M\not\in\plan'$ alors ils ne sont pas sécants.
\item Si $\plan$ et $\plan'$ ne sont pas parallèles alors $\plan\cap\plan'$ est une droite.
\end{enumerate}}

\preuve{

\begin{enumerate}[i)]
\item Supposons que $\plan // \plan'$ et que $\plan\cap\plan'\neq\emptyset$. Il existe donc $A\in\R^3$ tel que $\plan$ et $\plan'$ passent par $A$. D'autre part $u$ et $v$ dirigent $\plan$ et $\plan'$ puisqu'ils sont parallèles, et ainsi $\plan$ et $\plan'$ possèdent un paramétrage identique, donc $\plan=\plan'$.

\item Supposons que $\plan // \plan'$ et que $M\not\in\plan'$. Soit $\plan:ax+by+cz+d=0$ et $\plan':ax+by+cz+e=0$ deux équations cartésiennes de ces plans (c'est possible car $\plan // \plan'$). Si $e=d$ alors $\plan=\plan'$ et donc $M\in\plan'$ ce qui est absurde, donc $e\neq d$. Soit $A(a_1,a_2,a_3)\in\plan$. Donc $aa_1+ba_2+ca_3+d=0$. Mais si $A\in\plan'$, alors on a $aa_1+ba_2+ca_3+e=0$ et alors on aurait $e=d$, ce qui est absurde. Donc $A\not\in\plan$, conclusion $\plan\cap\plan'=\emptyset$.

\item Supposons que $\plan$ et $\plan'$ ne sont pas parallèles. Soit $\plan:ax+by+cz+d=0$ et $\plan':a'x+b'y+c'z+d'=0$. $\plan\cap\plan'$ est l'ensemble des $x,y,z\in\R$ tel que :

\sys{ax+by+cz+d &=& 0 \\
a'x+b'y+c'z+d' &=& 0}

Notons $S$ ce système. Comme $\vre{a}{b}{c}$ et $\vre{a'}{b'}{c'}$ ne sont pas colinéaires, alors $ab'-a'b\neq0$ ou $ac'-a'c\neq 0$ ou $bc'-b'c\neq 0$, supposons donc $ab'-a'b\neq0$ (les autres cas étant similaires). Si ($a=0$ et ($a'=0$ ou $b=0$)) ou ($b'=0$ et ($a'=0$ ou $b=0$)) alors $ab'-a'b=0$. Donc par contraposée, on a nécessairement ($a\neq0$ ou ($a'\neq0$ et $b\neq0$)) et ($b'\neq0$ ou ($a'\neq0$ et $b\neq0$)). Donc on a nécessairement :

\begin{enumerate}[i)]
\item $a\neq0$ et $b'\neq 0$, ou :
\item $a\neq0$ et $a'\neq0$ et $b\neq0$, ou :
\item $a'\neq0$ et $b\neq0$ et $b'\neq 0$, ou :
\item $a'\neq0$ et $b\neq0$.
\end{enumerate}

Nous allons ici supposer le cas numéro 2, les autres cas étant similaires. En résumé, voici nos suppositions pour cette démonstration :

\sys{ab'-a'b &\neq& 0 \\
a &\neq& 0 \\
a'&\neq& 0 \\
b &\neq& 0}

Appelons $L1$ et $L2$ les deux lignes de $S$. Multiplions $L1$ par $a'$ et $L2$ par $-a$, $S$ équivaut donc à :

\sys{aa'x+ba'y+ca'z+da' &=& 0 \\
-a'ax-b'ay-c'az-d'a &=& 0}

Substituons $L2$ par $L2+L1$, et divisons $L1$ par $a'$. $S$ équivaut donc à : 

\sys{ax+by+cz+d &=& 0 \\
(ba'-b'a)y+(ca'-c'a)z+da'-d'a &=& 0}

Comme $ab'-a'b\neq0$, alors on peut isoler $y$ dans L2 nous donnant ainsi $y=\dfrac{ad'-a'd+z(ac'-a'c)}{ab'-a'b}$.

En substituant $y$ dans $L1$, $S$ équivaut à :

\sys{ax &=& -b\dfrac{ad'-a'd+z(ac'-a'c)}{ab'-a'b}-cz-d \\
y &=& \dfrac{ad'-a'd+z(ac'-a'c)}{ab'-a'b}}

Puisque $a\neq0$ on peut isoler $x$ et $S$ équivaut à :

\sys{x &=& \dfrac{1}{a}\parent{-b\dfrac{ad'-a'd+z(ac'-a'c)}{ab'-a'b}-cz-d} \\
y &=& \dfrac{ad'-a'd+z(ac'-a'c)}{ab'-a'b}}

Par conséquent, en posant $z=\lambda$, $\plan\cap\plan'$ est l'ensemble des points $(x,y,z)$ tel que :

\sys{x &=& \dfrac{1}{a}\parent{-b\dfrac{ad'-a'd+\lambda(ac'-a'c)}{ab'-a'b}-c\lambda-d} \\
y &=& \dfrac{ad'-a'd+\lambda(ac'-a'c)}{ab'-a'b} \\
z &=& \lambda}

Or il s'agit d'un système paramétrique de droite. Conclusion, $\plan\cap\plan'$ est une droite.

\end{enumerate}}

\pro{(intersections de droites) Soit $\dt$, $\dt'$ deux droites dont la première passe par $A$.

\begin{itemize}
\item Si $\dt // \dt'$ et si elles sont sécantes alors $\dt\cap\dt' = \dt = \dt'$.
\item Si $\dt // \dt'$ et si $A\not\in \dt'$ alors elles ne sont pas sécantes.
\item Si $\dt$ et $\dt'$ ne sont pas parallèles et si elles sont sécantes alors $\dt\cap\dt'$ est réduit à un unique point.
\end{itemize}}
\preuve{En s'inspirant du théorème précédent.}

\pro{(intersections d'une droite et d'un plan) Soit $\dt$ une droite passant par $A$.

\begin{itemize}
\item Si $\dt//\plan$ et s'ils sont sécants alors $\dt\cap\plan=\dt$.
\item Si $\dt//\plan$ et si $A\not\in\plan$ alors ils ne sont pas sécants.
\item Si $\dt$ n'est pas parallèle à $\plan$ alors $\dt\cap\plan$ est réduit à un unique point.
\end{itemize}}
\preuve{Idem.}

Quand on demande les \textit{positions relatives} de deux plans, deux droites ou d'une droite et d'un plan (question très classique de terminale), on demande en fait s'ils sont orthogonaux, perpendiculaires, parallèles, sécants, et la nature de leur intersection.

\ex{\label{explans} Soit $u=(1,-8,6)$, $v=(1,2,3)$, $u'=(-2,-5,1)$ et $v'=(62,-49,-61)$ quatre vecteurs et $A(1,1,1)$ et $B(0,-4,-2)$ deux points. Soit $\plan,\plan'$ les plans dirigés respectivement par $u$ et $v$ et par $u'$ et $v'$ et passant respectivement par $A$ et $B$.  Étudier la position relative de $\plan$ et $\plan'$ et déterminer un paramétrage de l'intersection des deux si elle existe.}

La première question que l'on se pose, c'est si ces plans sont parallèles. Pour cela, nous allons trouver un vecteur normal à $\plan$ et voir s'il est aussi normal à $\plan'$. Nous verrons par la suite qu'il existe une technique pour avoir immédiatement un vecteur normal à un plan connaissant ses vecteurs ditecteurs. En attendant, posons $n=(x,y,z)$ un vecteur normal de $\plan$. Résolvons simplement le système :

\sys{n\cdot u &=& 0 \\
n\cdot v &=& 0}

Avec XCas : \verb&solve([dot([x,y,z],u)=0,dot([x,y,z],v)=0],[x,y,z])& \\

On obtient un ensemble paramétré comme solution, et on choisit une valeur particulière d'un paramètre, par exemple $-36$ (pour supprimer les fractions), et on trouve que le vecteur $n=(-36,3,10)$ est normal à $\plan$. Mais $n\cdot u'=67\neq0$, donc $n\not\perp\plan'$ donc $\plan$ et $\plan'$ ne sont pas parallèles. Donc $\plan\cap\plan'$ est une droite, appelons-la $\dt$. Pour trouver un paramétrage de $\dt$, le plus simple est certainement de déterminer d'abord une équation cartésienne de $\plan$ et de $\plan'$. Il nous faut donc d'abord un vecteur normal de $\plan'$, en utilisant la même technique on trouve par exemple le vecteur $n'=(59,-10,68)$. Ainsi, il existe $d,d'\in\R$ tel que $\plan:-36x+3y+10z+d=0$ et $\plan':59x-10y+68z+d'=0$. En exploitant le fait que $A\in\plan$ et $B\in\plan'$, on trouve la valeur de $d$ et $d'$, et ainsi on obtient $\plan:-36x+3y+10z+23=0$ et $\plan':59x-10y+68z+96=0$. Donc $(x,y,z)\in\dt$ \ssi :

\sys{-36x+3y+10z+23 &=& 0 \\
59x-10y+68z+96 &=& 0}

On résout donc avec XCas : \verb&S:=solve([-36*x+3*y+10z+23=0,59*x-10*y+68*z+96=0],[x,y,z])&. On obtient un paramétrage sous forme de matrice. D'ailleurs, une commande pratique pour extraire un élément d'une matrice est la commande \verb&at&, par exemple \verb&at(S,[0,1])& pour avoir le second élément. Ainsi, un paramétrage de $\dt$, pour tout $t\in\R$, est :

\sys{x(t) &=& t \\

y(t) &=& \dfrac{1519}{152}t - \dfrac{302}{152}  \\\\

z(t) &=& \dfrac{183}{304}t - \dfrac{259}{152}}

Si on simplifie en faisant un changement de variable pour avoir un système plus sympathique (c'est le bon moment pour dire que le PPCM se trouve avec la commande \verb&lcm&), on obtient :

\sys{ x(t) &=& 304t \\

y(t) &=& 3038t - \dfrac{302}{152} \\\\

z(t) &=& 183t - \dfrac{259}{152}}


Dernière question que l'on peut se poser, c'est si $\plan\perp\plan'$. On calcule $n\cdot n'=-1474\neq0$, donc non.

Conclusion de l'exercice : \textcolor{brown}{$\plan$ et $\plan'$ ne sont ni parallèles ni perpendiculaires. Leur intersection est une droite qui a pour paramétrage le précédent.}




\theo{(du toit, première version) Soit $\plan'$ un plan. Supposons que $\plan$  et $\plan'$ ne sont pas parallèles et notons la droite $\Delta=\plan\cap\plan'$. Soit $\dt\subset\plan$ et $\dt'\subset\plan'$ deux droites. Si $\dt // \dt'$ alors $\dt // \Delta$ et $\dt' // \Delta$.}

\preuve{...}

\theo{(du toit, seconde version) Soit $\plan'$ un plan. Supposons que $\plan$  et $\plan'$ ne sont pas parallèles et notons la droite $\Delta=\plan\cap\plan'$. Soit $\dt$ une droite. Si $\dt//\plan$ et $\dt//\plan'$ alors $\dt // \Delta$.}

\preuve{...}

\subsection{Notions avancées : angles, produit vectoriel}

\subsubsection{Angles}

Les angles vectoriels ne possèdent pas d'orientation dans l'espace comme c'est le cas pour les angles vectoriels du plan. Il n'existe par conséquent qu'une notion d'angle géométrique dans l'espace.

Pour rappel, dans l'espace, une définition possible du produit scalaire de deux vecteurs $u$ et $v$ est $u\cdot v=||u||\times ||v|| \cos(u,v)$. Nous allons définir de manière similaire les angles vectoriels dans l'espace. \textit{A noter qu'il y a plusieurs conventions possibles pour définir les angles dans l'espace}, je choisis ici celle qui me semble la plus simple à manipuler.

\defi{(angles vectoriels) Soit $u$ et $v$ deux vecteurs non nuls (de l'espace). On appelle \textit{angle entre $u$ et $v$}, noté $\widehat{u,v}$, la quantité $\arccos\parent{\dfrac{|u\cdot v|}{||u||\times||v||}}$.}

\rem{Cette notation est personnelle, je l'utilise pour insister sur le fait qu'il n'y pas d'orientation.}

\preuve{Quoi prouver ? En effet, \fone{\arccos}{[-1,1]}{\cro{0,\pi}}, il nous faut donc prouver que $\dfrac{|u\cdot v|}{||u||\times||v||}\in [-1,1]$. Puisque $\dfrac{|u\cdot v|}{||u||\times||v||}\ge 0$ cela revient à montrer que $\dfrac{|u\cdot v|}{||u||\times||v||}\le 1$, c'est-à-dire que $|u\cdot v|\le ||u||\times||v||$. Mais c'est exactement l'inégalité de Cauchy-Schwarz (lemme \ref{cauchyscwharz}) qui est valide pour tout $u,v$.}

\pro{(propriétés de base) Soit $u,v,w$ trois vecteurs. \begin{itemize}
\item $\widehat{u,v}\in\cro{0,\dfrac{\pi}{2}}$
\item $\widehat{u,v}=\dfrac{\pi}{2}$ \ssi $u$ et $v$ sont orthogonaux.
\item $\widehat{u,v}=0$ \ssi $u$ et $v$ sont colinéaires.
\item $\widehat{u,v}=\widehat{v,u}$
\item Si $w$ est colinéaire à $u$, alors $\widehat{u,v}=\widehat{w,v}$
\item En général, $\widehat{u,v}+\widehat{v,w}\neq \widehat{u,w}$ (pas de relation de Chasles).
\end{itemize}}

\preuve{Elles sont toutes faciles à montrer en utilisant la définition d'angle et la définition de \fone{\arccos}{[-1,1]}{\cro{0,\pi}}. Il n'y en a qu'une qui n'est pas évidente : c'est le sens $\Rightarrow$ du point 3 que nous allons donc montrer. Supposons donc $\widehat{u,v}=0$. Donc $\dfrac{|u\cdot v|}{||u||\times||v||}=1$. Donc $|u\cdot v|=||u||\times||v||$. Pour rappel, l'inégalité de Cauchy-Schwarz (lemme \ref{cauchyscwharz}) nous dit que $|u\cdot v|\le ||u||\times||v||$ : il nous faut donc montrer que le cas d'égalité de l'inégalité de Cauchy-Schwarz implique que $u$ et $v$ sont colinéaires (c'est en fait une équivalence). Pour rappel, nous avions posé la fonction \fons{P}{\R}{\R}{t}{(u+tv)^2} dont nous avions prouvé qu'elle est une fonction polynomiale du second degré. Puis nous avions montré que $\Delta\le0 \eqv |u\cdot v|\le ||u||\times||v||$ où $\Delta$ est le discriminant de ce polynôme. Ainsi, puisque $|u\cdot v|= ||u||\times||v||$ alors $\Delta=0$, donc $P$ admet une unique racine double $t_0\in\R$. Donc $P(t_0)=0$ donc $(u+t_0v)^2=0$ donc $u+t_0v = {0}$ et finalement $u=-t_0v$ : $u$ et $v$ sont colinéaires.}

\rem{Il peut sembler très curieux que l'angle entre deux vecteurs ne puisse pas dépasser les $90\degree$. Mais voyez les vecteurs comme deux droites sécantes : il y a deux angles possibles que forment ces droites, et l'un d'eux est forcément inférieur ou égal à $90\degree$. C'est celui-ci qu'on prend comme mesure d'angle.}

\defi{(angle entre deux droites) Soit $\dt$ et $\dt'$ dirigées par $u$ et $u'$. On appelle \textit{angle entre $\dt$ et $\dt'$} l'angle $\widehat{u,u'}$. On le note $\widehat{\dt,\dt'}$.}

\rems{\item C'est pour cela que dans la définition d'angle, le produit scalaire est en valeur absolue. En effet, s'il n'y en avait pas, alors il y aurait plusieurs mesures possibles d'un angle entre deux droites.
\item A noter qu'on peut parler d'angle entre deux droites même si elles ne sont pas sécantes.}

\defi{(angle entre un plan et une droite) Soit $\dt$ une droite dirigée par $u$ et $\plan$ un plan dont $n$ est un vecteur normal. On appelle \textit{angle entre $\dt$ et $\plan$} la quantité $\dfrac{\pi}{2}-\widehat{u,n}$. On le note $\widehat{\dt,\plan}$.}

\rem{Faire un dessin permettra facilement de comprendre pourquoi on prend le complémentaire de l'angle avec la normale.}

\defi{(angle entre deux plans) Soit $\plan$ et $\plan'$ deux plans ayant pour vecteurs normaux respectivement $n$ et $n'$. On appelle \textit{angle entre $\plan$ et $\plan'$} l'angle $\widehat{n,n'}$. On le note $\widehat{\plan,\plan'}$.}

En revanche, il n'existe pas de notion d'angle entre trois points comme dans le plan, car ce que mesure l'angle (pas plus grand que 90 degrés) ne correspondrait pas à aucune intuition géométrique (du moins avec la définition d'angle que nous avons décidé d'adopter dans ce cours).

Avec XCas, on peut créer une fonction \fone{\text{ang}}{\R^3\times\R^3}{\R_+} qui à tout couple de vecteurs associe leur angle comme suit : \verb&ang:=(u,v)->approx(acos(abs(dot(u,v))/(norm(u)*norm(v)))&. Si on veut en degrés : \verb&angd:=(u,v)->approx(acos(abs(dot(u,v))/(norm(u)*norm(v)))*(180/pi))&. Ainsi, on a par exemple $\widehat{\vre{42}{-7}{-9},\vre{8}{0}{-11}}\approx 42.7\degree$.

Sur GeoGebra, il existe une commande \verb&angle& mais attention, ils n'utilisent pas la même convention que celle que je donne dans ce cours : selon les situations parfois nos définitions vont donner le même résultat, et parfois GeoGebra trouvera le complémentaire de ce que donne notre définition.

\subsubsection{Produit vectoriel}

Le produit vectoriel est une opération entre deux vecteurs qu'on pourrait qualifier de cousine au produit scalaire : contrairement à lui, le résultat est un vecteur. Alors que le produit scalaire détecte les vecteurs orthogonaux, lui détecte les vecteurs colinéaires. Le produit vectoriel permet de fournir très facilement un vecteur orthogonal à deux autres, et donc il permet de générer facilement des bases de $\R^3$. Enfin, les propriétés sur sa norme sont utiles.

\defi{(produit vectoriel) Soit $u=\vre{a}{b}{c}$ et $v=\vre{a'}{b'}{c'}$ deux vecteurs quelconques. On appelle \textit{produit vectoriel de $u$ et de $v$}, noté $u\wedge v$, le vecteur $\vre{bc'-b'c}{ca'-c'a}{ab'-a'b}$.} 

\rem{Pour s'en souvenir, il suffit de faire les produits en croix de $u$ et de $v$ en commençant avec la seconde ligne et en descendant d'une ligne à chaque fois.}

Avec XCas, on obtient le produit vectoriel avec la commande \verb&cross(u,v)&.

\pro{(propriétés de base) Soit $u,v,w$ trois vecteurs et $\lambda\in\R$.

\begin{itemize}
\item (distributivité) $u\wedge (v+w)=u\wedge v + u\wedge w$
\item (multiplication par un scalaire) $\lambda(u\wedge v) = (\lambda u)\wedge v = u \wedge (\lambda v)$
\item (antisymétrie) $u\wedge v = -v\wedge u$
\item En général, $u\wedge (v\wedge w)\neq (u\wedge v)\wedge w$ (pas associatif)
\item En général, $u\wedge v \neq v\wedge u$ (pas commutatif)
\end{itemize}}

\rem{Conséquence des deux derniers points : on ne peut pas faire les opérations dans l'ordre qu'on veut avec le produit vectoriel : il faut être vigilant.}

\preuve{Tout se montre par le calcul.}

\lemme{Pour tout $x\in [0,1], \sin(\arccos(x))=\sqrt{1-x^2}$}

\preuve{Soit $y\in \cro{0,\dfrac{\pi}{2}}$. On a $\sin(y)^2+\cos(y)^2=1$ donc $|\sin(y)|=\sqrt{1-\cos(y)^2}$. Or $\sin(y)\ge0$ donc $\sin(y)=\sqrt{1-\cos(y)^2}$. Soit $x\in [0,1]$. On a donc $\arccos(x)\in\cro{0,\dfrac{\pi}{2}}$ et par conséquent, $\sin(\arccos(x))=\sqrt{1-\cos(\arccos(x))^2}$. Or pour tout $x\in[-1,1]$, $\cos(\arccos(x))=x$ donc finalement, $\sin(\arccos(x))=\sqrt{1-x^2}$.}

\pro{(propriétés fondamentales) Soit $u,v$ deux vecteurs et $w=u\wedge v$.

\begin{enumerate}[i)]
\item $w$ est orthogonal à $u$ et à $v$.
\item $w={0}$ \ssi $u$ et $v$ sont colinéaires.
\item $||w||=||u||\times||v||\times \sin(\widehat{u,v})$.
\item Si $u$ et $v$ sont non colinéaires, alors $(u,v,w)$ est une base de $\R^3$.
\end{enumerate}}

\rem{En toute généralité on doit mettre une valeur absolue au sinus dans le troisième point, mais avec notre définition de l'angle, puisque $\widehat{u,v}\in\cro{0,\dfrac{\pi}{2}}$ alors $\sin(\widehat{u,v})\ge 0$.}

\preuve{\begin{enumerate}[i)]
\item Immédiat par calcul.
\item C'est la proposition \ref{procroix}.
\item $\sin(\widehat{u,v})=\sin\cro{\arccos\parent{\dfrac{|u\cdot v|}{||u||\times||v||}}}$. Puisque $\dfrac{|u\cdot v|}{||u||\times||v||}\in[0,1]$, alors par le lemme, $\sin(\widehat{u,v})=\sqrt{1-\dfrac{(u\cdot v)^2}{||u||^2\times||v||^2}}$. Montrer le point équivaut à montrer $||w||^2=||u||^2\times||v||^2\times \sin(\widehat{u,v})^2$, \cad $||w||^2=||u||^2\times ||v||^2-(u\cdot v)^2$. Il ne reste plus qu'à calculer pour montrer que cela est effectivement vérifié.
\item $u$ et $v$ ne sont pas colinéaires donc sont coplanaires. Soit donc $\plan$ un plan dirigé par $u$ et $v$. $w$ est normal à $\plan$ et est non nul, donc il n'existe pas de combinaison linéaire de $u$ et de $v$ donnant $w$, cela signifie que $(u,v,w)$ est une base de $\R^3$.
\end{enumerate}}

Avec ces propriétés, on peut trouver géométriquement le résultat d'un produit vectoriel. Soit $u$ et $v$ deux vecteurs et $w=u\wedge v$. Soit $O(0,0,0)$, $U$ tel que $\vr{OU}=u$ et $V$ tel que $\vr{OV}=v$.

\begin{itemize}
\item (direction) soit $\dt$ une droite perpendiculaire à $(OU)$ et $(OV)$ (si $u$ et $v$ ne sont pas colinéaires, il n'y en a qu'une possible). Cette droite nous donne la direction de $w$.
\item (norme) soit $S$ la sphère de centre $O$ et de rayon $||w||=||u||\times||v||\times \sin(\widehat{u,v})$. $\dt$ possède deux intersections avec $\dt$ : l'une des deux sera "la pointe" de $w$.
\item (sens) On utilise la \href{https://fr.wikipedia.org/wiki/R\%C3\%A8gle_de_la_main_droite}{règle de la main droite} pour déterminer laquelle de ces deux intersections est la bonne. Appelons-la $W$.
\end{itemize}
Finalement, $\vr{OW}=w$.

\ex{Dans l'exercice \ref{explans}, nous devions trouver des vecteurs normaux aux plans $\plan$ dirigé par $u=(1,-8,6)$ et $v=(1,2,3)$ et $\plan'$ dirigé par  $u'=(-2,-5,1)$ et $v'=(62,-49,-61)$. En résvolant une équation nous avions trouver pour $\plan$ le vecteur normal $n=(-36,3,10)$ et pour $\plan'$ le vecteur normal $n'=(59,-10,68)$. Avec le produit vectoriel, on obtient directement $u\wedge v = n$ et $u'\wedge v'=6n'$.}

\pro{(produits vectoriels des vecteurs canoniques)

\begin{itemize}
\item $e_1\wedge e_2 = e_3$
\item $e_2\wedge e_3 = e_1$
\item $e_3\wedge e_1 = e_2$
\item $e_1\wedge e_3 = -e_2$
\item $e_2\wedge e_1 = -e_3$
\item $e_3\wedge e_2 = -e_1$
\end{itemize}}

\rem{Pour s'en souvenir : ça donne celui qui reste, quand on va de gauche à droite c'est positif, sinon c'est négatif.}

\pro{Soit $\rep = (O,(u,v,w))$ un repère orthogonal. Alors $w$ et $u\wedge v$ sont colinéaires.}
\preuve{$w$ et $u\wedge v$ sont tous deux orthogonaux à $u$ et $v$, donc sont colinéaires.}

\defi{(repère orthogonal direct) Soit $\rep = (O,(u,v,w))$ un repère orthogonal. On dit que $\rep$ \textit{est un repère direct} si $w$ et $u\wedge v$ sont de même sens, \cad si il existe $\lambda\in\R_+$ tel que $w=\lambda (u\wedge v)$. Autrement $\rep$ est dit \textit{indirect}.}

\cor{\begin{itemize} Soit $O\in\R^3$.
\item $(O,(e_1,e_2,e_3)),(O,(e_2,e_3,e_1))$ et $(O,(e_3,e_1,e_2))$ sont des repères directs.
\item $(O,(e_1,e_3,e_2)),(O,(e_2,e_1,e_3))$ et $(O,(e_3,e_2,e_1))$ sont des repères indirects.
\end{itemize}}

\subsubsection{Projeté orthogonal}

\defi{(projeté orthogonal sur une droite) Soit $A\in\R^3$ et $\dt$ une droite. On appelle \textit{projeté orthogonal de $A$ sur $\dt$} l'unique point $H$ vérifiant :

\begin{enumerate}[i)]
\item $H\in\dt$
\item $(HA)\perp \dt$
\end{enumerate}

On le note $p_\dt(A)$.}

\defi{(projeté orthogonal sur un plan) Soit $A\in\R^3$ et $\plan$ un plan. On appelle \textit{projeté orthogonal de $A$ sur $\plan$} l'unique point $H$ vérifiant :

\begin{enumerate}[i)]
\item $H\in\plan$
\item $(HA)\perp\plan$
\end{enumerate}

On le note $p_\plan(A)$.}




\subsection{Sphère et coordonnées sphérique}
Dans cette partie, nous ne ferons pas de démonstration. C'est simplement à titre culturel.

\subsubsection{Sphère}

\defi{(sphère) On appelle \textit{sphère de centre $\Omega$ et de rayon $R$} l'ensemble $\ens{A\in\R^3,\parallel\vr{\Omega A}\parallel=R}$. On la note $\sph(\Omega,R)$}

\defi{(boule) On appelle \textit{boule de centre $\Omega$ et de rayon $R$} l'ensemble $\ens{A\in\R^3,\parallel\vr{\Omega A}\parallel\le R}$. On la note $\mathscr{B}(\Omega,R)$}

\pro{(équation cartésienne d'une sphère) Soit $\Omega(a,b,c)$ et $R\in\R_+$. Alors $\sph(\Omega,R)=\ens{(x,y,z)\in\R^3,(x-a)^2+(y-b)^2+(z-c)^2=R^2}$. On dit que $(x-a)^2+(y-b)^2+(z-c)^2=R^2$ est une \textit{équation cartésienne} de $\sph(\Omega,R)$.}

\rem{C'est finalement une extension dans l'espace de ce qu'on connaissait des cercles dans le plan. De même que les plans, cette équation cartésienne est très pratique pour savoir si un point donné est dans une sphère, en revanche elle ne l'est pas pour générer facilement les points de la sphère. Nous verrons comment paramétrer la sphère ultérieurement.}

\subsubsection{Coordonnées sphériques}
Il y a une façon beaucoup plus pratique de se repérer sur une sphère : en utilisant les coordonnées sphériques. \href{https://fr.wikipedia.org/wiki/Coordonn\%C3\%A9es_sph\%C3\%A9riques#Conventions}{Il y a plusieurs conventions possibles pour se faire}, nous allons utiliser ici la plus courante en mathématiques : la convention \textit{rayon-longitude-colatitude}. Les angles dans l'espace que nous avons définis sont compris dans $\cro{0,\dfrac{\pi}{2}}$, cela ne suffira pas pour ce que nous allons faire, il faudra donc passer par des angles orientés dans des plans. Nous nous plaçons dans toute la suite dans le repère canonique. 

\theo{Soit $M\in\R^3\neq (0,0,0)$, $H=p_{(Oxy)}(M)$ et $h=\dfrac{\vr{OH}}{\parallel\vr{OH}\parallel}$. \\ Il existe un unique triplet $(\rho,\theta,\phi)\in \R_+^*\times [0,2\pi[\times [0,\pi]$ vérifiant :

\begin{enumerate}[i)]
\item $\parallel\vr{OM}\parallel=\rho$
\item Dans le plan $(Oxy)$ repéré par $(O,(e_1,e_2))$, $(e_1,h)=\theta$
\item Dans le plan dirigé par $e_3$ et $h$ repéré par $(O,(e_3,h))$ , $(e_3,\vr{OM})=\phi$
\end{enumerate}

On dit alors que :

\begin{itemize}
\item $(\rho,\theta,\phi)$ sont les \textit{coordonnées sphériques de }$M$,
\item $\rho$ est appelé le \textit{rayon},
\item $\theta$ la \textit{longitude},
\item $\phi$ la \textit{colatitude}.
\end{itemize}
Par convention, les coordonnées sphériques du point $(0,0,0)$ sont $(0,0,0)$.}
\rems{\item Si $M(a,b,c)$ alors $H(a,b,0)$.}

\includegraphics[scale=0.5]{figures/pdf/spherique-eps-converted-to.pdf}

\pro{Réciproquement, pour tout triplet $(\rho,\theta,\phi)\in \R_+^*\times [0,2\pi[\times [0,\pi]\cup (0,0,0)$, il existe un unique point $M\in\R^3$ tel que $(\rho,\theta,\phi)$ sont les coordonnées sphériques de $M$.}
\rem{Cela signifie qu'on a une bijection entre les coordonnées sphériques et cartésiennes : nous pouvons passer de l'un à l'autre.}

\pro{(paramétrage de la sphère en coordonnées sphériques) Soit $R\in\R_+^*$. Alors un paramétrage de $\sph(O,R)$ en coordonnées spéhériques est, pour tout $t_1\in[0,2\pi[$, $t_2\in[0,\pi]$ :

\sys{\rho (t_1,t_2) &=& R \\
\theta (t_1,t_2) &=& t_1 \\
\phi (t_1,t_2) &=& t_2}}

\pro{(relations de passage aux coordonnées cartésiennes) Soit $M\in\R^3\neq (0,0,0)$ ayant pour coordonnées sphériques $(\rho,\theta,\phi)$. Soit $(x,y,z)$ les coordonnées cartésiennes de $M$. Alors :

\sys{x &=& \rho\sin\phi\cos\theta \\
y &=& \rho\sin\phi\sin\theta \\
z &=& \rho\cos\phi}}

\preuve{Nous allons exposer comment retrouver ces relations. Remarquons tout d'abord que $H$ et $M$ ont la même abscisse et la même ordonnée, par conséquent nous allons constamment travailler dans le triangle rectangle $OHM$.

\begin{itemize}
\item Plaçons-nous donc dans le plan dirigé par $e_3$ et $h$ repéré par $(O,(e_3,h))$. La trigonométrie nous donne $\sin\parent{\dfrac{\pi}{2}-\phi}=\dfrac{z}{\rho}$ d'où $\color{brown}{z=\rho\cos\phi}$. D'autre part, nous avons $\cos\parent{\dfrac{\pi}{2}-\phi}=\dfrac{OH}{\rho}$ d'où $OH=\rho\sin\phi$.

\item Plaçons-nous à présent dans le plan $(Oxy)$ repéré par $(O,(e_1,e_2))$. Alors on a immédiatement $h=(\cos\theta,\sin\theta)$. Or $\vr{OH}=OH h$ d'où $\color{brown}{x=\rho\sin\phi\cos\theta}$ et $\color{brown}{y=\rho\sin\phi\sin\theta}$.
\end{itemize}}

\pro{(paramétrage de la sphère en coordonnées cartésiennes) Soit $\Omega(a,b,c)$ et $R\in\R_+^*$. Un paramétrage de $\sph(\Omega,R)$ en coordonnées cartésiennes, pour tout $t_1\in[0,2\pi[$, $t_2\in[0,\pi]$, est :

\sys{x(t_1,t_2) &=& R\sin t_2\cos t_1 + a \\
y(t_1,t_2) &=& R\sin t_2\sin t_1 + b \\
z(t_1,t_2) &=& R\cos t_2 + c}}

\rem{Si le seul but est de générer des points, dans la pratique on est pas obligé de limiter $t_1$ et $t_2$, qui sont des angles, aux ensembles $[0,2\pi[$ et $[0,\pi]$ : on peut leur faire parcourir $\R$ entier. Simplement, on risque de générer plusieurs fois le même point avec des paramètres différents : selon ce que l'on veut faire, ça peut être embêtant ou pas.}

\subsubsection{Applications en astronomie}

\begin{description}
\item[Repérage terrestre] Soit $T$ le centre de la Terre, $R$ son rayon, $N$ le pôle nord (géographique), $G$ l'observatoire de Greenwich, $\plan$ le plan dirigé par $\vr{TN}$ et $\vr{TG}$. Soit $G'$ le point de l'équateur tel que $\vr{TG'}//\plan$ et tel qu'il soit dans le même hémisphère que $G$. Soit $\vec{k}=\vr{TN}$, $\vec{i}=\vr{TG'}$ et $\vec{j}=\vec{k}\wedge \vec{i}$. Nous munissons ainsi la Terre du repère orthonormé direct $(T,(\vec{i},\vec{j},\vec{k}))$. Soit $P$ sur la Terre et $(\rho,\theta,\phi)$ les coordonnées sphériques de $P$ avec en convention \href{https://fr.wikipedia.org/wiki/Coordonn\%C3\%A9es_sph\%C3\%A9riques#Conventions}{\textit{rayon-longitude-latitude}}.  On appelle \textit{coordonnées géographiques de $P$} le couple $(\theta,\phi)$ (le rayon n'a aucune importance puisque tout point terrestre possède un rayon de 1 dans ce repérage). Dans la pratique, ces coordonnées sont exprimées dans le système DMS (degrés/minutes/secondes). Dans celui-ci, une minute d'angle vaut 1/60 degré et une seconde d'angle vaut 1/60 minute d'angle (soit 1/3600 degré), ce qui nous offre une précision à environ $3\times 10^{-4}$ degré près. Par exemple, l'église de Villefranche-sur-Saône a pour latitude $45\degree 59' 25"$, soit $\approx 45\degree 59.4167'$ soit $\approx 45.9903\degree$ (cela ne sert à rien de renseigner plus de 4 décimales). De plus, pour éviter d'avoir des latitudes négatives on ajoute le suffixe $N$ (nord) ou $S$ (sud). Et pour éviter d'avoir des longitudes supérieures à $180\degree$ on ajoute le suffixe $E$ (est) ou $O$ (ouest). Pour l'église de Villefranche, nous obtenons ainsi les coordonnées géographiques $(45\degree 59' 25" N, 4\degree 43' 13"E)$. Ainsi, $N$ a pour coordonnées terrestres $(0\degree 00'00" E,90\degree 00'00" N)$.

\item[Repérage céleste équatorial] De même que sur la Terre, il va nous falloir un repère pour repérer les objects du ciel. Le repère terrestre est bien entendu inadpaté : nous allons en définir un autre. La Terre possède un axe de rotation passant par son centre $T$ et passant par $N$. La droite $[TN)$ est appelée \textit{pôle nord céleste} et $[TS)$ \textit{pôle sud céleste}. A noter que l'étoile \href{https://fr.wikipedia.org/wiki/Alpha_Ursae_Minoris}{Alpha Ursae Minoris} se situe presque sur le pôle nord céleste, c'est donc d'abord en repérant cette étoile que les astronomes amateurs (de l'hémisphère nord) en déduisent plus précisément la direction du pôle nord céleste. C'est pour cette raison que tout le monde connait plutôt Alpha Ursae Minoris sous le nom d'\textit{étoile polaire}. Je recommande à présent de visualiser \href{https://upload.wikimedia.org/wikipedia/commons/thumb/6/6a/Coordonnees_equatoriales.svg/603px-Coordonnees_equatoriales.svg.png}{cette image} en lisant les explications qui suivent. Nous savons que la Terre tourne autour du Soleil, mais du point de vue d'un observateur situé au centre de la Terre, c'est le Soleil qui tourne autour de lui suivant une trajectoire quasi-circulaire. Ce cercle est appelé \textit{écliptique}. Soit à présent la sphère $\sph$ de centre $T$ et passant par le centre du Soleil. L'écliptique appartient à cette sphère. On appelle \textit{équateur céleste} la projection de l'équateur terrestre sur $\sph$. L'écliptique et l'équateur céleste s'intersectent sur $\sph$ en deux points. Au cours de sa trajectoire, le Soleil passe donc par ces points. L'un des deux, en passant de l'hémisphère nord à l'hémisphère sud, et l'autre de l'hémisphère sud à l'hémisphère nord : ce dernier est appelé \textit{point vernal} et on le note $\gamma$. Nous avons à présent tout ce qu'il nous faut pour créer un repère céleste. Notons $N'$ l'intersection entre le pôle nord céleste et $\sph$. Posons $\vec{k}=\vr{TN'}$, $\vec{i}=\vr{T\gamma}$ et $\vec{j}=\vec{k}\wedge \vec{i}$. Nous munissons ainsi l'espace du repère orthonormé direct $(T,(\vec{i},\vec{j},\vec{k}))$. Soit $P$ un point de l'espace et $(\rho,\alpha,\delta)$ les coordonnées sphériques de $P$ en convention \textit{rayon-longitude-latitude}. On dit que $(\alpha,\delta)$ sont les \textit{coordonnées équatoriales} de $P$. Dans ce contexte, $\alpha$ est appelé \textit{ascension droite} plutôt que longitude et $\delta$ \textit{déclinaison} plutôt que latitude (mais dans les faits c'est la même chose). A noter que $\rho$ n'est pas précisé, car un observateur cherche rarement à connaître la distance qui le sépare des objets du ciel : pour observer un objet les deux autres coordonnées suffisent. D'ailleurs, à noter que $||\vec{i}||=||\vec{j}||=||\vec{k}||=1\text{UA}$ (unité astronoique). Pour exprimer l'ascension droite $\alpha$ on utilise le système \textit{heures/minutes/secondes}: l'horaire, vérifiant $24h=360\degree$, la minute horaire et la seconde horaire ce qui nous offre une précision à $4\times 10^{-3}$ degré près. Par exemple, $247.481\degree$ vaut environ 16h29m55s. A noter que si on fait la conversion dans l'autre sens, on tombe sur environ $247.479\degree$. Pas d'arnaque, c'est simplement qu'une seconde vaut environ $4\times {(10^{-3})}\degree$ comme nous l'avons expliqué : c'est normal qu'il y ait des décalages. Quant à $\delta$, il est simplement exprimé en système DMS. C'est ainsi que sur Stellarium (excellent logiciel fournissant des cartes du ciel en temps réel, qu'on peut coordonner avec un repérage équatorial notamment), nous voyons que l'étoile Alkaid a pour coordonnées équatoriales (13h47m32.21s, +49$\degree$18'47.0"). Si on convertie tout en simples degrés,  on obtient pour coordonnées équatoriales (206.88421$\degree$, 49.31306$\degree$). Pour voir tout ce repérage, \href{https://upload.wikimedia.org/wikipedia/commons/6/66/Ra_and_dec_demo_animation_small.gif}{voici un GIF}.

\item[Coordonnées horizontales (ou alt-azimutales)] Contrairement au repérage équatorial, les coordonnées alt-azimutales d'un objet céleste varie en fonction de l'heure et du lieu de l'observateur. Si ce repérage est très pratique à mettre en place pour un observateur (le repérage étant local), il n'est pas stable contrairement au repérage équatorial. Soit $O$ un point terrestre. C'est l'observateur. La droite $[TO)$ est nommée \textit{zénith}. Soit $\plan$ le plan tangeant à la Terre passant par $O$. Ce plan est appelé \textit{horizon}. Soit $A=p_\plan(N)$. Soit $Z$ le point tel que $\vr{TO}$ et $\vr{OZ}$ soient colinéaires et de même sens et tel que $OZ=ON$. Soit enfin $\vec{j}=\vr{OZ}\wedge\vr{ON}$. On munit ainsi l'espace du repère orthonormé direct $(O,(\vr{ON},\vec{j},\vr{OZ}))$. Soit $P$ un point de l'espace et soit $(\rho,\zeta,h)$ les coordonnées sphériques de $P$ en convention \textit{rayon-longitude-latitude}. On dit que $(\zeta,h)$ sont les \textit{coordonnées horizontales} ou \textit{alt-azimutales} de $P$. Dans ce contexte, $\zeta$ est appelé \textit{azimut} au lieu de longitude et $h$ \textit{hauteur} au lieu de latitude. Comme ce repérage est local et ne convient qu'à l'observateur, et que celui-ci ne peut observer des objets de latitude négative, la hauteur est compris dans $[0\degree,90\degree]$. Les deux coordonnées sont exprimées dans le système DMS.

\end{description}

\subsubsection{Cercle dans l'espace}

\defi{Soit $M\in\R^3$, $u$ et $v$ deux vecteurs non colinéaires et $r\in\R_+$. On appelle \textit{cercle de centre $M$, de rayon $r$ et dirigé par $u$ et $v$} l'intersection entre le plan $\plan$ dirigé par $u$ et $v$ et passant par $M$ et $\sph(M,r)$. On le note $\mathscr{C}(\plan,r)$.}
On pourrait avoir l'idée, pour trouver un paramétrage d'un tel cercle, d'écrire les paramétrages du plan et de la sphère puis de résoudre le système d'équations. Ça ne marchera jamais car les paramètres de la sphère sont dans des fonctions trigonométriques. Le plus simple, c'est donc de se placer dans un repère bien choisi de $\plan$, d'écrire l'équation paramétrique du cercle dans le plan, puis en effectuant un changement de repère, trouver les coordonnées du cercle dans le repère canonique. 

\ex{Soit $u=\vre{2}{-1.5}{3}$ et $v=\vre{-1}{-1}{0}$ et $M(3,-3,2)$ et $r=2$. Trouvons un paramétrage de $\mathscr{C}(\plan,r)$. D'abord, il faut nous munir d'un repère pratique pour travailler dans $\plan$ : nous allons construire un repère orthonormé direct. Tout d'abord, posons $\vec{i}=\dfrac{u}{\parallel u \parallel}$ (on rend $u$ unitaire). Ce sera le premier vecteur de notre futur repère. Posons maintenant $\vec{k}=\dfrac{u\wedge v}{\parallel u\wedge v\parallel}$. $\vec{k}$ est donc un vecteur normal à $\plan$ et unitaire : ce sera un autre vecteur de notre futur repère. Enfin, posons $\vec{j}=\vec{k}\wedge\vec{i}$. Soit $\mathscr{R}=(M,(\vec{i},\vec{j},\vec{k}))$ : ce dernier est donc par construction un repère orthonormé direct. Dans $\mathscr{R}$, nous avons un paramétrage de $\mathscr{C}(\plan,r)$ facile, pour tout $t\in[0,2\pi[$ :

\sys{x_\mathscr{R}(t) &=& 2\cos t \\
y_\mathscr{R}(t) &=& 2\sin t \\
z_\mathscr{R}(t) &=& 0}

C'est-à-dire que le point paramétré $P_\mathscr{R}(x_\mathscr{R}(t),y_\mathscr{R}(t),z_\mathscr{R}(t))$ parcourt $\mathscr{C}(\plan,r)$.

Il nous faut maintenant trouver les coordonnées de $P$ dans le repère canonique. Ainsi, $\vr{MP_\mathscr{R}}=x_\mathscr{R}(t)\vec{i}+y_\mathscr{R}(t)\vec{j}+z_\mathscr{R}(t)\vec{k}=x_\mathscr{R}(t)\vec{i}+y_\mathscr{R}(t)\vec{j}$. Soit le vecteur $m=(3,-3,2)=\vr{OM}$. On a $\vr{OP_\mathscr{R}}=\vr{OM}+\vr{MP_\mathscr{R}}=m+x_\mathscr{R}(t)\vec{i}+y_\mathscr{R}(t)\vec{j}$. Par définition, en calculant les composantes de ce vecteur, nous aurons les coordonnées de $P$ dans le repère canonique et donc un paramétrage du cercle.

Un tour de moulinette dans XCas (de toute manière j'ose espérer que le lecteur aura depuis longtemps compris que c'est inutile de faire ces calculs ideux à la main) et il nous trouve un paramétrage approximatif de $\mathscr{C}(\plan,r)$ dans le repère canonique, pour tout $t\in[0,2\pi[$ :

\sys{x(t) &=&3+ 1.02\cos t-1.33\sin t \\
y(t) &=& -3-0.77\cos t-1.49\sin t\\
z(t) &=& 2+1.54\cos t+0.14\sin t}}
\chapter{Fonctions}

\section{Définitions}

Cette section donne des notions sur les fonctions non utilisées au lycée mais beaucoup dans le supérieur, elle se destine donc avant tout aux élèves désireux de poursuivre des études de mathématiques. Il s'agit essentiellement de donner au futur étudiant du vocabulaire précis pour parler correctement de ces objets.

\subsection{Définitions générales}

\defi{(application/fonction) Soit $E$ et $F$ deux ensembles. On appelle $f$ une \textit{application} ou une \textit{fonction de $E$ dans $F$} un objet qui à tout élément de $E$ associe un unique élément de $F$. On la note \fone{f}{E}{F}. On dit que $E$ est \textit{l'ensemble de départ} de $f$ et $F$ \textit{l'ensemble d'arrivée} de $f$.}
\rem{Pour représenter schématiquement, on utilise souvent des patatoïdes. A noter que si il y a une flèche pour tous les éléments de $E$, ce n'est pas nécessairement le cas pour $F$. De plus, il est parfaitement possible qu'à deux éléments de $E$ soient associés le même élément de $F$ (flèches du milieu dans le schéma suivant).}

\includegraphics[scale=0.5]{figures/pdf/application-eps-converted-to.pdf}

Dans la suite et sauf mention contraire, on considère  \fone{f}{E}{F} une application quelconque.

\defi{(image d'un élément) Soit $x\in E$. Par définition, il existe un unique $y\in F$ tel que $x$ est associé à $y$ par $f$. On dit que $y$ est \textit{l'image de $x$ par $f$} et on le note $f(x)$.}

\defi{(antécédent) Soit $y\in F$ et $A$ l'ensemble des éléments de $E$ tel que pour tout $x\in A$, $f(x)=y$. Alors les éléments de $A$ sont appelés \textit{antécédents de $y$ par $F$.}}

\nota{On note \fons{f}{E}{F}{x}{f(x)} pour signifier que tout élément $x\in E$ est associé à $f(x)\in F$.}

\defi{(composée) Soit $G$ un ensemble et \fone{g}{F}{G}. On appelle \textit{composée de $g$ avec $f$}, notée $g\circ f$, la fonction \fons{g\circ f}{E}{G}{x}{f(g(x))}.}

\rem{Attention à ne pas se tromper de sens : c'est bien $g\circ f$ et non $f\circ g$. En effet, ici parler de $f\circ g$ n'aurait aucun sens (dessiner des patatoïdes pour s'en convaincre).}

\defi{(ensemble de définition) On dit que $E$ est \textit{l'ensemble de définition de $f$}. On le note souvent $D_f$.}

\rem{Il arrive souvent que l'ensemble de définition d'une fonction usuelle ne soit pas précisé. Dans ce cas, on prendra toujours le plus grand ensemble de définition possible. Par exemple, si on travaille avec une fonction homographique, on prendra pour ensemble de définition $\R$ privé des points où le dénominateur s'annule.}

\defi{(image d'une application) On appelle \emph{image de $f$}, notée $\im f$, l'ensemble des images de tous les éléments de $E$.\label{imappli}}
\rem{$\im f\subseteq F$ mais a priori, $\im f\neq F$ en général.}

\ex{Soit $f:\R\to\R,x\mapsto x^2$. Alors $\im f=\R^+$.}

\defi{(image directe) Soit $A\subset E$. On appelle \emph{image directe de $A$ par $f$}, notée $f(A)$, l'ensemble des images des éléments de $A$. C'est un sous-ensemble de $\im f$.}

\rem{En particulier, $f(E)=f(D_f)=\im f$.}

\includegraphics[scale=0.5]{figures/pdf/imagedirecte-eps-converted-to.pdf}

\ex{Soit $f:\R\to\R,x\mapsto x^2$ et $A=]3;10[$. Alors $f(A)=]9;100[$.}

\defi{(image réciproque) Soit $B\subset F$. On appelle \emph{image réciproque de $B$ par $f$}, notée $f^{-1}(B)$, l'ensemble des antécédents de tous les éléments de $B$. C'est un sous-ensemble de $E$.}

\ex{Soit $f:\R\to\R,x\mapsto x^2$, $B=\{3;4\}$, $C=\{-5\}$, $D=[-3;3[$. Alors $f^{-1}(B)=\{-2;-\sqrt{3};\sqrt{3};2\}$, $f^{-1}(C)=\emptyset$, $f^{-1}(D)=]-\sqrt{3};\sqrt{3}[$.}

\defi{(égalité) Soit $g:E'\to F'$. On dit que \emph{$f$ égale $g$} et on note $f=g$, si $E=E'$, $F=F'$ et si pour tout $x\in E$, $f(x)=g(x)$.\label{egalitefonction}}

\rem{Dans la même veine, on peut définir des fonctions implicitement. Par exemple, soit \fone{f,g,h}{\R}{\R}. On peut écrire $h=f+g$ pour dire de façon raccourcie que pour tout $x\in\R$, $h(x)=f(x)+g(x)$. Ça marche avec $h=fg$, la valeur absolue $h=|f|$, l'inverse $h=\dfrac{1}{f}$... Mais attention de ne pas abuser de ces raccourcis : d'une part il y a des pièges, pour écrire $\dfrac{1}{f}$ il faut s'assurer que $f$ ne s'annule pas. Ensuite, il faut que le raccourci soit extrêmement clair, il ne doit y avoir aucune difficulté pour traduire la notation raccourcie, c'est pourquoi je conseille vivement de ne pas dépasser le cadre de la somme, du produit et de l'inverse, en gros. Enfin, ce n'est jamais qu'un raccourci, qui va très bien pour les brouillons, \textit{mais pas dans les contrôles ou examens}.}

\defi{(restriction) Soit $G\subset E$. Alors on appelle \emph{restriction de $f$ à $G$}, notée $f_{|G}$, la fonction $f_{|G}:G\to F,x\mapsto f(x)$.}
\rem{Il s'agit donc simplement de réduire l'ensemble de départ.}

\defi{(prolongement) Soit $E',F'$ deux ensembles et $g:E'\to F'$. On dit que $g$ est un \emph{prolongement de $f$} si $E\subseteq E'$, $F\subseteq F'$ et si pour tout $x\in E,f(x)=g(x)$.}

\ex{Soit $f:\R\to\R,x\mapsto |x|$, $g:\R\to\R,x\mapsto -x$ et $I=]-\infty;0]$. Alors $g$ est un prolongement de $f_{|I}$.}

\defi{(courbe représentative) Supposons que $E,F\subset\R$ et soit $\plan$ un plan. On appelle \textit{courbe représentative de $f$} l'ensemble de points $\ens{(x,y)\in\plan,y=f(x)}$.}

\subsection{Injection, surjection, bijection}
Ces trois notions sont omniprésentes en analyse et permettent entre autres de voir proprement la notion de fonction réciproque.

\defi{(injection) On dit que $f$ est \emph{injective} si pour tout $x,y\in E$ tels que $f(x)=f(y)$, alors $x=y$.}

\rems{\item En termes de patatoïdes, ça veut dire qu'il n'y a qu'une flèche au maximum qui arrive sur chaque élément de $F$.
\item Dans le cas où $E,F\subseteq\R$, cela se traduit graphiquement par le fait que toute droite d'équation $y=k$ avec $k\in F$ admet \emph{au plus} un point d'intersection avec la courbe représentative de $f$.}

\includegraphics[scale=0.5]{figures/pdf/injective-eps-converted-to.pdf}

\defi{(surjection) On dit que $f$ est \emph{surjective} si pour tout $y\in F$, il existe $x\in R$ tel que $y=f(x)$.}

\rems{\item En termes de patatoïdes, cela se traduit par le fait que tous les éléments de $F$ ont au moins une flèche.
\item Dans le cas où $E,F\subseteq\R$, cela se traduit graphiquement par le fait que toute droite d'équation $y=k$ avec $k\in F$ admet \emph{au moins} un point d'intersection avec la courbe représentative de $f$.
\item Si $f$ est surjective alors $F=\im f$}

\includegraphics[scale=0.5]{figures/pdf/surjective-eps-converted-to.pdf}

\defi{(bijection) Soit $E$ et $F$ deux ensembles, $f:E\to F$ une application. On dit que $f$ est \emph{bijective} si elle est injective \emph{et} surjective.}

\rem{\begin{itemize}
\item En termes de patatoïdes, cela se traduit par le fait que tous les éléments de $F$ possèdent exactement une flèche.
\item Dans le cas où $E,F\subseteq\R$, cela se traduit graphiquement par le fait que toute droite d'équation $y=k$ avec $k\in F$ admet \emph{un unique} point d'intersection avec la courbe représentative de $f$.
\item On peut obtenir une définition alternative de la bijection en combinant celles d'injection et de surjection. Nous laissons le soin au lecteur de la déterminer.
\end{itemize}}

\includegraphics[scale=0.5]{figures/pdf/bijection-eps-converted-to.pdf}

\exo{Pour chaque fonction ci-dessous, déterminer graphiquement (et formellement pour les plus courageux) si elle est injective, surjective, bijective (la réponse est donnée en blanc entre les crochets) :
\begin{enumerate}[(i)]
\item \fons{f}{\N}{\R}{x}{x} [\textcolor{white}{inj., non surj., non bij.}]
\item \fons{g}{\R}{\R}{x}{x^2} [\textcolor{white}{non inj., non surj., non bij.}]
\item \fons{h}{\R}{[-1;1]}{x}{\cos(x)} [\textcolor{white}{non inj., surj., non bij.    }]
\item \fons{r}{]-\infty;0[}{]0;+\infty[}{x}{-\dfrac{1}{x}} [\textcolor{white}{inj., surj., bij.}]
\end{enumerate}}

\exo{Soit \fons{f}{E}{F}{x}{\lfloor x\rfloor} où $E,F\subset\R$. Trouver $E$ et $F$ (il n'y a pas forcément qu'une seule bonne réponse) tel que :

\begin{enumerate}
\item $f$ soit non injective et non surjective,
\item $f$ soit injective et non surjective,
\item $f$ soit non injective et surjective,
\item $f$ soit bijective.
\end{enumerate}}

\theo{\label{frec} (fonction réciproque) Supposons $f$ bijective. Alors il existe une unique fonction $f^{-1}:F\to E$, appelée \emph{fonction réciproque de $f$}, telle que pour tout $x\in E,(f^{-1}\circ f)(x)=x$ et pour tout $y\in F,(f\circ f^{-1})(y)=y$.\label{reciproque}}

\preuve{\textsc{Existence}

$f$ étant bijective, pour tout $y\in F$ il existe un unique $x\in E$ tel que $f(x)=y$. Pour tout $y\in F$, on note $\overline{y}$ l'un d'eux et alors $f(\overline{y})=y$. On peut donc définir la fonction \fonl{g}{F}{E}{y}{\overline{y}}, montrons que celle-ci est une réciproque de $f$. Soit $y\in F$, alors on a $f(g(y))=f(\overline{y})=y$. Soit $x\in E$, alors on a $g(f(x))=\overline{f(x)}$. Or par définition $f(\overline{f(x)})=f(x)$ et par injectivité de $f$, $\overline{f(x)}=x$. Donc $g(f(x))=x$. Conclusion, $g$ est bien une réciproque de $f$.

\textsc{Unicité}

Soit deux fonctions réciproques de $f$ : \fone{g}{F}{E} et \fone{h}{F}{E}. Soit $y\in F$. On a $f(g(y))=y$ et $f(h(y))=y$ donc $f(g(y))=f(h(y))$ et par injectivité de $f$, $g(y)=h(y)$. Donc $g=h$ et par conséquent $f$ n'admet qu'une unique fonction réciproque.}

\rem{\begin{itemize}
\item Cette implication est en fait une équivalence.
\item Attention : $f^{-1}$ peut soit faire référence à l'image réciproque, soit à l'application réciproque de $f$. Mais contrairement à la première notion, il est obligatoire d'établir la bijectivité avant de pouvoir parler de la seconde.
\item Dans le cas où $E,F\subseteq\R$, les courbes représentatives des deux fonctions $f$ et $f^{-1}$ sont symétriques par rapport à la droite $y=x$ dans un repère orthonormé. \end{itemize}}

\pro{Supposons $f$ bijective et \fone{f^{-1}}{F}{E} la fonction réciproque de $f$. Alors $f^{-1}$ est bijective et $f$ est la réciproque de $f^{-1}$.}

\preuve{Elle est laissée au lecteur curieux. Il faut bien reprendre les définitions et vérifier chaque hypothèse.}

\exo{Pour chacune des fonctions bijectives suivantes (le vérifier graphiquement), déterminer leur fonction réciproque puis tracer leurs courbes.
\begin{enumerate}[(i)]
\item \fons{f_1}{\R}{\R}{x}{x}
\item \fons{f_2}{]-\infty;0[}{]-\infty;0[}{x}{\dfrac{1}{x}}
\item \fons{f_3}{[0;+\infty[}{[0;+\infty[}{x}{x^2}
\item \fons{f_4}{[0;\pi]}{[-1;1]}{x}{\cos(x)}
\end{enumerate}}

\pro{(bijection monotone) Soit $E,F\subseteq\R$ et $f:E\to F$ une application. Si $f$ est strictement monotone sur $E$ et $F=\im(f)$ alors $f$ est bijective.\label{bijmono}}

\begin{proof}
La monotonie \emph{stricte} de $f$ assure son injectivité et la condition $F=\im(f)$ sa surjectivité donc par définition, $f$ est bijective.
\end{proof}

\rem{\begin{itemize}
\item Attention : cela peut paraître contre-intuitif mais la condition $F=\im(f)$ n'est pas systématiquement assurée ! Ne pas oublier qu'en général on a juste $F\supseteq\im(f)$ (revoir la définition~\ref{imappli}).
\item Cette proposition, combinée avec le théorème de l'image d'un intervalle par une fonction continue que nous verrons ultérieurement, permet de démontrer la bijectivité de beaucoup de fonctions de base.
\end{itemize}}

\ex{Montrons que la fonction $f_3$ de l'exercice précédent est bijective. Soit $I=[0;+\infty[$. Nous savons que $f_3$ est continue et strictement croissante (donc monotone) sur $I$. D'après le théorème mentionné dans la remarque précédente, $\im(f_3)$ est un intervalle. Nous avons $\Lim{x\to +\infty}x^2=+\infty$ et de plus pour tout $x\in I,x^2\ge 0$ et $f_3(0)=0$. Par conséquent, $\im(f_3)=[0;+\infty[$ et finalement par la proposition~\ref{bijmono}, $f_3$ est bijective. $\square$}

\subsection{Périodicité, parité}

\defi{\programme (périodique) Soit \fone{f}{D_f}{\R} avec $D_f\subset\R$. On dit que \textit{$f$ est périodique} s'il existe $T\in\R_+$ tel que pour tout $x\in\R$ tel que $x+T\in D_f$, $f(x+T)=f(x)$. On dit que $T$ est \textit{une période} de $f$.}

\rem{Si $T$ est une période de $f$, alors pour tout $k\in\N^*$, $Tk$ est une période de $f$. Si $t$ est la plus petite période de $f$, alors $\ens{tk,k\in\N^*}$ est l'ensemble de \textit{toutes} les périodes de $f$.}

\ex{Les fonctions trigonométriques usuelles : $\cos,\sin,\tan$ sont périodiques de plus petite période $2\pi$. La fonction partie fractionnaire est périodique de plus petite période $1$.}

\defi{\programme (pair, impair) Soit \fone{f}{D_f}{\R} avec $D_f\subset\R$.

\begin{itemize}
\item On dit que \textit{$f$ est paire} si pour tout $x\in\R$, $f(x)=f(-x)$.
\item On dit que \textit{$f$ est impaire} si pour tout $x\in\R$, $f(-x)=-f(x)$.
\end{itemize}}

\pro{Soit \fone{f}{D_f}{\R} avec $D_f\subset\R$.

\begin{itemize}
\item $f$ est paire \ssi sa courbe représentative admet $x=0$ pour axe de symétrie.
\item $f$ est impaire \ssi sa courbe représentative admet une symétrie centrée sur l'origine.
\end{itemize}}

\ex{$\cos$ et $x\mapsto x^2$ sont paires. $\sin$, l'identité et $x\mapsto x^3$ sont impaires.}

\pro{Soit \fone{f}{D_f}{\R} avec $D_f\subset\R$ et $k\in\R$.

\begin{itemize}
\item La courbe représentative de $f$ admet $x=k$ pour axe de symétrie \ssi la fonction $x\mapsto f(x-k)$ est paire.
\item La courbe représentative admet une symétrie centrée au point $(k,0)$ \ssi la fonction $x\mapsto f(x-k)$ est impaire.
\end{itemize}}

\ex{$x\mapsto (x+3)^2$ admet $x=-3$ pour axe de symétrie. $x\mapsto (x-4)^3$ admet une symétrie centrée au point $(4,0)$.}

\theo{(décomposition en parties paire et impaire) Soit \fone{f}{D_f}{\R} avec $D_f\subset\R$ tel que si $x\in D_f$ alors $-x\in D_f$. Alors il existe deux uniques fonctions \fone{f_p,f_i}{D_f}{\R} tel qu'on a simultanément :

\begin{enumerate}[i)]
\item $f_p$ est paire,
\item $f_i$ est impaire,
\item $f=f_p + f_i$
\end{enumerate}

$f_p$ est alors appelée \textit{partie paire de $f$} et $f_i$ \textit{partie impaire de $f$}.}

\pro{Soit \fone{f}{D_f}{\R} avec $D_f\subset\R$ tel que si $x\in D_f$ alors $-x\in D_f$ et soit $f_p,f_i$ les parties respectivement paire et impaire de $f$. Alors :

\begin{itemize}
\item pour tout $x\in\R$, $f_p(x)=\dfrac{f(x)+f(-x)}{2}$
\item pour tout $x\in\R$, $f_i(x)=\dfrac{f(x)-f(-x)}{2}$
\end{itemize}}

\exo{Déterminer les parties paires et impaires des fonctions usuelles quand c'est possible (pas la fonction racine par exemple). Donner une fonction dont ni la partie paire ni la partie impaire ne soit constante à $0$.}

\subsection{Trois fonctions utiles}

Nous allons pour terminer ces généralités définir trois applications très courantes en analyse.

\defi{(fonction identité) Soit $E$ un ensemble. On appelle \textit{fonction identité de $E$} la fonction \fons{\id_E}{E}{E}{x}{x}}.

\rem{\begin{itemize}
\item En clair c'est la fonction qui "ne fait rien".
\item Lorsqu'il n'y a pas d'ambiguité sur l'espace de travail, parfois on note cette fonction seulement $\id$.
\item Cette application est bijective (quel que soit $E$).
\end{itemize}}

\pro{Soit $E,F$ deux ensembles et \fone{f}{E}{F}. Alors $f\circ\id_E=f$ et $\id_F\circ f=f$.}

\preuve{C'est immédiat en utilisant la définition.}

\exo{En utilisant la fonction identité et la définition~\ref{egalitefonction}, réécrire le théorème~\ref{reciproque}.}

\defi{(fonction caractéristique) Soit $E$ un ensemble et $A\estdans E$. On appelle \emph{fonction caractéristique de $A$} la fonction \fone{\chi_A}{E}{\{0;1\}} qui à $x$ associe $1$ si $x\in A$ et $0$ sinon.}

\rem{\begin{itemize}
\item Ne pas oublier que même si cela n'est pas précisé dans la notation, cette fonction dépend de l'espace ambiant $E$.
\item On la rencontre aussi sous le nom \textit{fonction indicatrice}, la notation correspondante est alors $\mathbf{1}_A$. 
\end{itemize}}

\ex{Soit $E=\R$. Alors $\chi_{\Z}(x)=1$ si et seulement si $x$ est un entier, et $\chi_\Z(x)=0$ si et seulement si $x$ est un réel non entier.}

\exo{Soit $E$ un ensemble et $A,B\estdans E$. Déterminer $\chi_{A\cup B}$ et $\chi_{A\cap B}$ en fonction de $\chi_{A}$ et $\chi_{B}$.}

\defi{(symbole de Kronecker) Soit $A,B$ deux ensembles. On appelle \emph{symbole de Kronecker} la fonction \fone{\delta}{A\times B}{\{0;1\}} qui à $(i,j)$ associe 1 si $i=j$ et 0 sinon.}

\rems{\item On note conventionnelement $\delta_{ij}$ ou bien $\delta_i^j$ au lieu de $\delta(i,j)$.
\item La plupart du temps $i$ et $j$ sont des indices et donc $A,B\estdans\N$ et en général $A=B$.}

\exo{Pour $n\in\N^*$, définir la matrice identité $\I_n$ en utilisant le symbole de Kronecker.}

\section{Limite d'une fonction}

Nous conseillons au lecteur de relire et de bien se représenter la définition de limite de suites, car la définition formelle de limite de fonction est très similaire (la limite de suite en est un cas particulier en réalité). Grossièrement, la limite d'une fonction en un point est ce vers quoi tend la fonction en ce point, et ce point peut éventuellemnt être situé aux extrémités de là où la fonction est définie (notamment $\pm\infty$). Si cette approximation est en générale celle que retiennent les élèves en terminale et si elle permet de s'en sortir dans beaucoup de cas, mathématiquement on a évidemment besoin d'aller au-delà de cette intuition car beaucoup de choses découlent de cette notion fondamentale~: continuité, dérivée, primitive...

\subsection{Préliminaires}

\nota{(l'infini en tant qu'élément) On note $a=+\infty$ pour dire qu'il n'existe pas de réel plus grand que $a$. On note $a=-\infty$ pour dire qu'il n'existe pas de réel plus petit que $a$.}
\rem{\textbf{Attention} : cette notation n'a bien sûr pas grand sens mathématiques : à strictement parler, l'infini ne peut pas être un élément. Voyez donc cette notation comme un raccourci qui nous permettra de rendre plus commode la notion de limite.}

\defi{(droite réelle achevée) On appelle \textit{droite réelle achevée} l'ensemble $\R\cup\ens{-\infty,+\infty}$. On le note $\overline{\R}$.}

\rem{Il s'agit donc simplement de $\R$ auquel on ajoute ses extrémités, ce qui est rendu possible par la notation précédente.}

\defi{(intervalle de $\overline{\R}$) Soit $a,b\in\overline{\R}$ avec $a\le b$. On appelle \textit{intervalle de $\overline{\R}$} l'un des 4 ensembles suivant.

\begin{itemize}
\item $[a,b]$
\item $[a,b[$
\item $]a,b]$
\item $]a,b[$
\end{itemize}}

\rem{En particulier, si $a=b$, alors $[a,b]$ est un intervalle réduit à un seul élément, on dit que c'est un \textit{singleton} et on le note $\ens{a}$. Toujours si $a=b$, $]a,b[$ est aussi un intervalle qui ne contient aucun élément, c'est donc l'\textit{ensemble vide} noté $\emptyset$.}

\defi{(intervalle de $\R$) On dit que $I$ est \textit{un intervalle de $\R$} si $I$ est un intervalle de $\overline{\R}$ et si $\pm\infty\not\in I$.}

\rem{C'est la définition d'intervalle à laquelle nous étions habitués jusqu'à présent. Dans la suite, on utilisera des intervalles de $\overline{\R}$ et de $\R$, soyez donc attentifs duquel on parle car la nuance est essentielle.}

\defi{(adhérence d'un intervalle) Soit $I$ un intervalle de $\overline{\R}$. On appelle \textit{adhérence de $I$}, noté $\overline{I}$, l'intervalle $I$ union ses extrémités.}

\rems{\item En particulier, la droite réelle achevée n'est rien d'autre que l'adhérence de $\R$ et c'est pourquoi c'est cohérent de l'écrire $\overline{\R}$.
\item L'adhérence d'un intervalle est lui-même un intervalle.}

\ex{$\overline{]-\infty,-6]}=[-\infty,-6]$ et $\overline{]3,\pi[}=[3,\pi]$.}

\defi{(adhérence d'une union d'intervalles) Soit $A,B$ deux intervalles de $\overline{\R}$ et $U=A\cup B$. On appelle \textit{adhérence de $U$}, notée $\overline{U}$, l'ensemble $\overline{A}\cup\overline{B}$.}

\rem{Par récurrence, on peut donc définir l'adhérence d'une union de $n$ intervalles~: c'est l'union des adhérences de chacun d'eux.}

\ex{$\overline{]-\infty,-6]\cup]3,\pi[}=\overline{]-\infty,-6]}\cup\overline{]3,\pi[}=[-\infty,-6]\cup [3,\pi]$.}

\subsection{Définitions}

Dans toute la suite, nous considérons $U$ une union d'intervalles de $\R$ (et non de $\overline{ \R}$) et \fone{f}{U}{\R}.

\defi{\programme (limite en un point) Soit $a\in\overline{U}$.

\begin{itemize}
\item Soit $l\in\overline{\R}$. On dit que $f$ admet pour limite $l$ en $a$ si pour tout $\epsilon>0$, il existe $\delta>0$ tel que pour tout $x\in U$, si $|x-a|<\delta$ alors $|f(x)-l|<\epsilon$. \textit{\textbf{Attention} : si $a$ ou $l$ valent $\pm\infty$, il y a aussi une définition de limite de $f$ en $a$ qu'il convient d'adapter par rapport à celle qu'on vient de donner, voir ci-dessous.}

\item On dit que $f$ admet une limite en $a$ s'il existe $l\in\overline{\R}$ tel que $f$ admet pour limite $l$ en $a$.
\end{itemize}}

Intuitivement, $f$ admet pour limite $l$ en $a$ si pour tout rayon $\epsilon$, il existe un tuyau vertical de rayon $\delta$ et entourant $a$ tel que les images des éléments de ce tuyau sont tous dans le tuyau horizontal de rayon $\epsilon$ et entourant $l$. \\

\textbf{Attention} : si $a=\pm\infty$, la quantité $|x-a|$ n'a aucun sens, si bien qu'on doit adapter la définition : on dit que $f$ admet pour limite $l$ en $+\infty$ si pour tout $\epsilon>0$, il existe $\delta>0$ tel que pour tout $x\in U$, si $x>\delta$ alors $|f(x)-l|<\epsilon$. De même, si $l=\pm\infty$, il faut alors adapter la quantité $|f(x)-l|$. \textcolor{blue}{Ainsi, selon que $a$ et $l$ soient réels ou valent $\pm\infty$, cela nous donne au total \textbf{9 définitions} différentes de la limite. Nous conseillons au lecteur de s'exercer à les écrire et de vérifier ci-dessous (nous écrivons les définitions en notation logique pour être plus synthétique).}

\begin{enumerate}[i)]
\item $a\in\R$, $l\in\R$ : $\forall\epsilon>0,\exists\delta>0,\forall x\in U, |x-a|<\delta\Rightarrow |f(x)-l|<\epsilon$
\item $a\in\R$, $l=-\infty$ : \spoiler{$\forall\epsilon\in\R,\exists\delta>0,\forall x\in U, |x-a|<\delta\Rightarrow f(x)<\epsilon$}
\item $a\in\R$, $l=+\infty$ : \spoiler{$\forall\epsilon\in\R,\exists\delta>0,\forall x\in U, |x-a|<\delta\Rightarrow f(x)>\epsilon$}

\item $a=-\infty$, $l\in\R$ : \spoiler{$\forall\epsilon>0,\exists\delta\in\R,\forall x\in U, x<\delta\Rightarrow |f(x)-l|<\epsilon$}
\item $a=-\infty$, $l=-\infty$ : \spoiler{$\forall\epsilon\in\R,\exists\delta\in\R,\forall x\in U, x<\delta\Rightarrow f(x)<\epsilon$}
\item $a=-\infty$, $l=+\infty$ : \spoiler{$\forall\epsilon\in\R,\exists\delta\in\R,\forall x\in U, x<\delta\Rightarrow f(x)>\epsilon$}

\item $a=+\infty$, $l\in\R$ : \spoiler{$\forall\epsilon>0,\exists\delta\in\R,\forall x\in U, x>\delta\Rightarrow |f(x)-l|<\epsilon$}
\item $a=+\infty$, $l=-\infty$ : \spoiler{$\forall\epsilon\in\R,\exists\delta\in\R,\forall x\in U, x>\delta\Rightarrow f(x)<\epsilon$}
\item $a=+\infty$, $l=+\infty$ : \spoiler{$\forall\epsilon\in\R,\exists\delta\in\R,\forall x\in U, x>\delta\Rightarrow f(x)>\epsilon$}
\end{enumerate}

\rems{\textbf{Conseil important} : il s'agit probablement de la définition la plus abstraite et la plus difficile à digérer de toute la scolarité pré-BAC. Il est de mon point de vue \textit{essentiel} de comprendre intuitivement, par le dessin, ce que cela signifie. Autrement nous sommes condamnés à apprendre par coeur cette définition formelle sans réellement la comprendre. Ne surtout pas hésiter à regarder des vidéos pour voir des illustrations, de s'entraîner sur beaucoup d'exemples (avec corrections) pour réellement s'imprégner de cette définition.

\item On appelle cette définition de la limite \textit{définition epsilon-delta}. Il y a en effet d'autres façons (notamment topologiques) de définir la limite.}

\includegraphics[scale=0.5]{figures/pdf/limite-eps-converted-to.pdf}

\exo{\textbf{(important)} Écrire la négation de «~$f$ admet pour limite $l$ en $a$~» et de «~$f$ admet une limite en $a$~» \spoiler{pour le premier~: «~$f$ n'admet pas pour limite $l$ en $a$ si il existe $\epsilon>0$ tel que pour tout $\delta>0$, il existe $x\in U$ tel que $|x-a|<\delta$ et $|f(x)-l|\ge\epsilon$~». Pour le second~: «~$f$ n'admet pas de limite en $a$ si pour tout $l\in\R\cup\ens{\pm\infty}$, il existe $\epsilon>0$ tel que pour tout $\delta>0$, il existe $x\in U$ tel que $|x-a|<\delta$ et $|f(x)-l|\ge\epsilon$~».}}

\exo{Toute suite peut être identifiée à une fonction de $\N$ dans $\R$. Prenons donc le cas particulier où $U=\N$ ($\N$ est bien une union d'intervalles, ici de singletons) et $l\in\R$. Soit \suite{u} définie pour tout $n\in\N$ par $u_n=f(n)$. écrire la définition de «~$\limc{x}{+\infty} f(x)=l$~» et constater qu'elle est équivalente à la définition de «~$(u_n)$ converge vers $l$~». Ainsi, la convergence de suite n'est qu'un cas particulier d'une limite de fonction.}

\nota{\programme Si $f$ admet pour limite $l$ en $a$, on note $\limc{x}{a} f(x) = l$ ou encore $f(x)\underset{x\rightarrow a}{\longrightarrow} l$. }

\rems{\item Par contre en classe on privilégiera la première, la seconde n'est utilisée qu'à partir du supérieur.
\item \textit{Sur un brouillon}, on utilise souvent la notation encore plus concise $f\underset{a}{\longrightarrow} l$.}

Concrètement, pour \textit{chercher} si une fonction a pour limite $l$ (qu'on soupçonne d'être la bonne) en $a$, il est conseillé de suivre les 4 étapes suivantes :

\begin{enumerate}
\item Questionner le sens de la question. Cela consiste à vérifier que $a\in\overline{U}$ (autrement la question ne se pose pas).

\item Écrire le but. Cela consiste à écrire \textit{formellement} la définition de $\limc{x}{a}f(x)=l$ dans notre cas précis.

\item Chercher graphiquement un $\delta$ qui semble convenir (peu importe si ce n'est pas le bon dans l'immédiat). Attention : $\delta$ ne doit dépendre \textit{que} de $\epsilon$ (et s'il ne dépend pas de $\epsilon$, par exemple s'il est constant, c'est louche).

\item Écrire la démonstration en suivant le but et en posant le $\delta$ précédent. Soit ce $\delta$ convient et c'est fini, soit ce n'est pas le bon, et dans ce cas il est généralement facile de rectifier. Reprendre alors la démonstration avec le nouveau $\delta$ jusqu'à en trouver un qui marche. Si les problèmes persistent malgré tout, c'est \textit{peut-être} un indicateur que ce qu'on essaye de prouver est en fait faux !
\end{enumerate}

Et ensuite, pour \textit{rédiger} la démonstration, il est conseillé de suivre les 4 étapes suivantes :

\begin{enumerate}
\item (sens) Questionner le sens de la question. Cela consiste à vérifier que $a\in\overline{U}$ (autrement la question ne se pose pas).

\item (but) Écrire le but. Cela consiste à écrire \textit{formellement} la définition de $\limc{x}{a}f(x)=l$ dans notre cas précis.

\item (démonstration) Écrire la démonstration.

\item (conclusion) Écrire la conclusion. Cela consiste à vérifier que la démonstration a permis d'affirmer que le but (voir point 2) est bien respecté.
\end{enumerate}


\ex{Montrons que $\limc{x}{0}\sqrt{x}=0$.

\begin{enumerate}
\item (sens) Ici $D_f=\R_+$ et $\overline{D_f}=[0,+\infty]$ donc $0\in\overline{D_f}$ et donc cela a du sens de demander la limite en $0$.

\item (but) Soit $\epsilon>0$. On veut prouver qu'il existe $\delta>0$ tel que pour tout $x\ge0$, si $|x|<\delta$ alors $|\sqrt{x}|<\epsilon$. 

\item (démonstration) Posons $\delta=\epsilon^2$. Soit $x\ge0$ et supposons $|x|<\delta$, \cad $x<\delta$ \cad $x<\epsilon^2$. Alors $\sqrt{x}<|\epsilon|$ \cad $|\sqrt{x}|<\epsilon$.

\item (conclusion) Nous avons bien prouvé que pour tout $\epsilon>0$, il existe $\delta>0$ tel que pour tout $x\ge0$, si $|x|<\delta$ alors $|\sqrt{x}|<\epsilon$. Conclusion, $\limc{x}{0}\sqrt{x}=0$.
\end{enumerate}}

\ex{Conjecturer la valeur de $\limc{x}{-\infty}x^2$ puis le démontrer. Il semble que $\limc{x}{-\infty}x^2=+\infty$.

\begin{enumerate}
\item (sens) Ici $D_f=\R$ et $\overline{D_f}=\overline{\R}$ donc $-\infty\in\overline{D_f}$ et donc cela a du sens de demander la limite en $-\infty$.

\item (but) Soit $\epsilon\in\R$. On cherche $\delta\in\R$ tel que pour tout $x\in\R$, si $x<\delta$ alors $x^2>\epsilon$.

\item (démonstration) Constatons pour commencer que si $\epsilon<0$ alors on a $x^2>\epsilon$ pour tout $x\in\R$, donc tous les $\delta$ marchent. Montrons maintenant le résultat pour $\epsilon\ge 0$. Posons $\delta=-\sqrt{\epsilon}$. Soit $x\in\R$ et supposons $x<\delta$ \cad $x<-\sqrt{\epsilon}$. Alors comme tout est négatif, on a $x^2>(-\sqrt{\epsilon})^2$ \cad $x^2>\epsilon$.

\item (conclusion) Nous avons bien montré que pour tout $\epsilon\in\R$, il existe $\delta\in\R$ tel que pour tout $x\in\R$, si $x<\delta$ alors $x^2>\epsilon$. Conclusion, $\limc{x}{-\infty}x^2=+\infty$.
\end{enumerate}}

\ex{Montrons que $\limc{x}{0}\dfrac{1}{x}\neq+\infty$. 

\begin{enumerate}
\item (sens) Ici $D_f=\R^*$ et $\overline{D_f}=\overline{\R}$ donc $0\in\overline{D_f}$ et donc cela a du sens de demander la limite en $0$.

\item (but)  Nous voulons montrer (voir exercice sur la négation de limite) qu'il existe $\epsilon\in\R$ tel que pour tout $\delta>0$, il existe $x\in\R^*$ tel que $|x|<\delta$ et $\dfrac{1}{x}\le\epsilon$. Pour ne pas se tromper, il conseillé d'écrire la définition de «~$f$ a pour limite $+\infty$ en $0$~» puis de la nier.

\item (démonstration) Posons $\epsilon=0$ et soit $\delta>0$. Posons $x=-\dfrac{\delta}{2}$. On a bien $x\in\R^*$. De plus, $|x|=\dfrac{\delta}{2}<\delta$ et $\dfrac{1}{x}=-\delta\le\epsilon$.

\item (conclusion) Nous avons ainsi bien montré qu'il existe $\epsilon\in\R$ tel que pour tout $\delta>0$, il existe $x\in\R^*$ tel que $|x|<\delta$ et $\dfrac{1}{x}\le\epsilon$. Conclusion, $\limc{x}{0}\dfrac{1}{x}\neq+\infty$.

\end{enumerate}
On comprend d'ailleurs qu'on peut parfaitement adapter cette démonstration pour montrer que $\limc{x}{0}\dfrac{1}{x}\neq-\infty$}


\ex{\label{exrat} Déterminer \textit{graphiquement} si $\limc{x}{+\infty}\dfrac{x^2}{2x^2-1}$ existe et si oui, donner sa valeur. La méthode préconisée est la suivante : commencer par écrire le domaine de définition de la fonction considérée. Ici, $D_f=\R\backslash\ens{\pm\dfrac{\sqrt{2}}{2}}$. Ensuite, calculer l'adhérence de ce domaine. Ici, $\overline{D_f}=\overline{\R}$. Ensuite, est-ce que le point où on cherche la limite est dans cette adhérence ? Si non, cela n'a aucun sens de chercher cette limite : elle n'existe pas. Ici nous cherchons la limite en $+\infty$ et $+\infty\in\overline{\R}$ : donc cela a un sens de chercher l'existence de cette limite. A présent, on observe la courbe représentative : s'il ne semble pas y avoir de limite claire, on part du principe que la limite n'existe pas. S'il semble y avoir une limite, on la conjecture. Ici, il semble bien que $1/2$ soit la limite. Ensuite on écrit la définition de $\limc{x}{+\infty}\dfrac{x^2}{2x^2-4+1}=1/2$ : pour tout $\epsilon>0$, il existe $\delta\in\R$ tel que pour tout $x\in D_f$, si $x>\delta$ alors $\abs{\dfrac{x^2}{2x^2-4+1}-\dfrac{1}{2}}<\epsilon$. On vérifie avec quelques valeurs de $\epsilon$ que cette définition est effectivement vérifiée et à partir de là on conclue. Ici, cela fonctionne, donc la limite existe (graphiquement) et $\limc{x}{+\infty}\dfrac{x^2}{2x^2-1}=\dfrac{1}{2}$.}

\exo{\label{exlim} Déterminer \textit{graphiquement} si les limites suivantes existent (et si oui, donner leur valeur).

\begin{enumerate}[i)]
\item $\limc{x}{-\infty} x^2$ \spoiler{oui : $+\infty$}
\item $\limc{x}{0} \sqrt{x}$ \spoiler{oui : $0$}
\item $\limc{x}{-1} \sqrt{x}$ \spoiler{non------}
\item $\limc{x}{0} f(x)$ où $f(0)=1$ et $f(x)=\dfrac{\sin x}{x}$ si $x\neq0$ \spoiler{oui : $1$}
\item $\limc{x}{0} \dfrac{\sin x}{x}$ \spoiler{oui : $1$}
\item $\limc{x}{0} f(x)$ où $f(0)=2$ et $f(x)=\dfrac{\sin x}{x}$ si $x\neq0$ \spoiler{non------}
\item $\limc{x}{0} \dfrac{1}{x}$ \spoiler{non------}
\item $\limc{x}{0} \dfrac{|x|}{x}$ \spoiler{non------}
\item $\limc{x}{+\infty} \cos(x)$ \spoiler{non------}
\item $\limc{x}{1} \arccos(x)$ \spoiler{oui : $0$}
\item $\limc{x}{\pi/2} \tan(x)$ \spoiler{non------}
\item $\limc{x}{\pi/2} |\tan(x)|$ \spoiler{oui : $+\infty$}
\item $\limc{x}{0} f(x)$ où \fons{f}{\ens{0}}{\R}{x}{1} \spoiler{oui : $1$}
\end{enumerate} }

Avec XCas, on calcule la limite comme suit (\href{http://www-fourier.ujf-grenoble.fr/~parisse/giac/doc/fr/cascmd_fr/cascmd_fr317.html#sec418}{documentation ici}) : \\

\verb&limit(x^2,x,-inf)& \\

Si on ajoute un 1 à la fin, cela signifie qu'on veut calculer la limite à droite et -1 à gauche, par exemple : \\

\verb&limit(tan(x),x,pi/2,1)& \\

\textit{Mais attention cependant} : il faut toujours avoir du recul avec ce que donne XCas. Ainsi, il trouve $\limc{x}{-1}\sqrt{x}=i$ alors que cette limite n'existe pas (dans les réels), il trouve $\limc{x}{0}\dfrac{1}{x}=\infty$ (on ne sait même pas si c'est $\pm\infty$) alors que cette limite n'existe pas. Donc l'utiliser comme outil de vérification ou de conjecture, mais garder à l'esprit que la limite donnée par XCas n'est jamais une preuve.

\defi{\programme (asymptote)
\begin{itemize}
\item Soit $a\in\R$. Si $f$ admet pour limite $\pm\infty$ en $a$, alors on dit que la droite d'équation $x=a$ est une \textit{asymptote verticale à la courbe représentative de $f$}.

\item Soit $a=\pm\infty$. Si $f$ admet une limite finie $l$ en $a$, alors on dit que la droite d'équation $y=l$ est une \textit{asymptote horizontale à la courbe représentative de $f$}.
\end{itemize}}

\rem{Graphiquement, si une courbe représentative admet une asymptote, alors cette courbe semble «~épouser~» cette asymptote, \cad qu'à l'infini, la courbe semble se confondre avec l'asymptote.}

\defi{(limite à gauche/droite) Soit $a\in\overline{U}$.

\begin{itemize}
\item Soit $G=]-\infty,a[\cap U$ et $g=f_{|G}$. On appelle \textit{limite de $f$ à gauche de $a$} la quantité, si elle existe, $\limc{x}{a} g(x)$.

\item Soit $D=]a,+\infty[\cap U$ et $d=f_{|D}$. On appelle \textit{limite de $f$ à droite de $a$} la quantité, si elle existe, $\limc{x}{a} d(x)$.
\end{itemize}}

\rem{\item La limite à gauche de $a$, c'est donc simplement la limite de la fonction à laquelle on retire tout ce qu'il y a à partir de $a$.}

\nota{Soit $a\in\overline{U}$. Si $f$ admet pour limite $l$ à gauche de $a$, alors on note $\limc{x}{a-} f(x)=l$ ou bien $f(x)\underset{x\rightarrow a-}{\longrightarrow} l$. Pareil pour la limite à droite mais avec un «~+~».}

\exo{Déterminer si les limites à gauche et à droite des fonctions de l'exercice \ref{exlim} existent graphiquement et si oui, déterminer leur valeur.

\begin{enumerate}[i)]
\item  \spoiler{A droite : $+\infty$. A gauche : non.}
\item  \spoiler{A droite : $0$. A gauche : non.}
\item  \spoiler{A droite : non. A gauche : non.}
\item  \spoiler{A droite : $1$. A gauche : $1$.}
\item  \spoiler{A droite : $1$. A gauche : $1$.}
\item  \spoiler{A droite : $1$. A gauche : $1$.}
\item  \spoiler{A droite : $+\infty$. A gauche : $-\infty$.}
\item  \spoiler{A droite : $1$. A gauche : $-1$.}
\item  \spoiler{A droite : non. A gauche : non.}
\item  \spoiler{A droite : non. A gauche : $0$.}
\item  \spoiler{A droite : $-\infty$. A gauche : $+\infty$.}
\item  \spoiler{A droite : $+\infty$. A gauche : $+\infty$.}
\item  \spoiler{A droite : non. A gauche : non.}
\end{enumerate}}

\rem{A travers cet exercice, on remarque donc qu'une fonction peut admettre une limite à gauche et à droite de $a$ mais ne pas admettre de limite en $a$. Et à l'inverse, la dernière montre qu'une fonction peut admettre une limite en $a$ mais pas de limite ni à gauche ni à droite de $a$. C'est essentiel de le retenir : \textbf{a priori, on ne peut rien déduire d'une limite à partir d'une limite à gauche et/ou à droite et vice-versa} (sauf dans le cas très précis d'un théorème que nous verrons ultérieurement).}

\subsection{Propositions et théorèmes pratiques}

Dans la pratique, il est bien entendu hors de question de revenir constamment à la définition chaque fois qu'on doit prouver la valeur d'une limite. On a comme pour les suites une série de propositions et théorèmes qui permettent d'aller bien plus vite. Nous ne montrerons pas systématiquement les démonstrations qui sont pour beaucoup des redites de celles sur les suites. Le lecteur qui maîtrise bien le chapitre sur les suites et la définition de limite ne devrait pas avoir trop de problème à les compléter.

\theo{(caractérisation séquentielle des limites) Soit $a\in\overline{U}$ et $l\in\overline{\R}$. Alors $\limc{x}{a}f(x)=l$ \ssi pour toute suite \suite{u} à valeurs dans $U$ qui converge vers $a$, la suite \suite{f(u_n)} converge vers $l$. }
\rem{Dans le cas où $a=\pm\infty$, on remplace «~qui converge vers $a$~» par «~qui diverge vers $a$~» pour que le théorème reste valide.}

Grâce à ce théorème, la plupart des résultats valables pour les suites restent valables pour les limites et on peut de fait utiliser les démonstrations des suites pour démontrer celles sur les limites. C'est pourquoi nous le présentons en premier.

\pro{Soit $a\in\overline{U}$. Si $f$ admet une limite en $a$, alors elle est unique.}

\preuve{Similaire à la preuve que si une suite admet une limite alors elle est unique : il faut supposer qu'il existe deux limites réelles différentes $l_1$ et $l_2$ : la définition de la limite doit alors marcher pour tout rayon $\epsilon>0$. On montre alors que le rayon particulier $\epsilon=\dfrac{|l_2-l_1|}{2}$ conduit à une absurdité.}

\pro{\programme (opérations sur les limites 1) Soit $a\in\overline{U}$ et $l\in\R$.

\begin{enumerate}[i)]
\item $\limc{x}{a} f(x)=l\eqv \limc{x}{a} (f(x)-l)=0$
\item $\limc{x}{a} f(x)=l\eqv \limc{x}{a} |f(x)-l|=0$
\end{enumerate}}

\rem{Si $l=\pm\infty$, la quantité $f(x)-l$ ne voudrait rien dire, c'est pourquoi $l\in\R$ et non $l\in\overline{\R}$.}

\preuve{Montrons juste le sens $\Rightarrow$ du premier point, les autres sont dans la même veine (il est conseillé au lecteur de s'y entraîner néanmoins pour bien manipuler la définition epsilon-delta). Supposons donc $\limc{x}{a} f(x)=l$. Soit $\epsilon>0$ et soit $\delta>0$ tel que pour tout $x\in U$, si $|x-a|<\delta$ alors $|f(x)-l|<\epsilon$. Soit $x\in U$ et supposons $|x-a|<\delta$. Alors $|f(x)-l|<\epsilon$ \cad $l-\epsilon <f(x)<l+\epsilon$ \cad $0-\epsilon<f(x)-l<0+\epsilon$ \cad $|(f(x)-l)-0|<\epsilon$. Nous venons d'écrire la définition de $\limc{x}{a} (f(x)-l)=0$.}

\pro{\programme \label{oplim2}(opérations sur les limites 2) Soit \fone{f,g}{U}{\R} deux fonctions, $\lambda\in\R$ et $a\in\overline{U}$. Alors on peut (ou pas) déterminer $\limc{x}{a}(\lambda f(x))$, $\limc{x}{a}(f(x)+g(x))$ et $\limc{x}{a}\dfrac{f(x)}{g(x)}$ suivant les mêmes règles que pour les suites : voir \ref{oplim}.}

\pro{\label{limimg} Si $a\in U$ et si $\limc{x}{a} f(x) = l$ alors $l=f(a)$.}
\rem{Attention : $a\in U$ et non $a\in\overline{U}$ ! Contre-exemple : $\limc{x}{0} \dfrac{\sin x}{x} = 1$ et pourtant $1\neq f(0)$ ($f(0)$ n'est pas défini).}

\preuve{Sous ces hypothèses, supposons par l'absurde que $l\neq f(a)$. La définition de limite doit marcher pour tout rayon $\epsilon>0$. Mais le rayon particulier $\epsilon=|f(a)-l|$ conduit à une absurdité.}

\pro{Soit $a\in\overline{U}$. S'il existe $l,l'\in\overline{\R}$ avec $l\neq l'$ tel que $\limc{x}{a-}=l$ et $\limc{x}{a+}=l'$, alors $f$ n'admet pas de limite en $a$.}

\ex{$\limc{x}{0}\dfrac{1}{x}$ n'existe pas puisque $\limc{x}{0-}\dfrac{1}{x}=-\infty$ et $\limc{x}{0+}\dfrac{1}{x}=+\infty$.}

\theo{\label{limgd} Soit $a\in U$ et supposons qu'il existe $\alpha,\beta\in\R$ tel que $a\in]\alpha,\beta[\subset U$ et soit $l\in\R$. Alors $\limc{x}{a} f(x)=l$ \ssi on a simultanément :

\begin{enumerate}[i)]
\item $\limc{x}{a-}=l$
\item $\limc{x}{a+}=l$
\item $f(a)=l$
\end{enumerate}}

\rems{\item Intuitivement, ce théorème nous dit que si la fonction est définie sur un intervalle ouvert contenant $a$, alors il y a équivalence entre limite en $a$ et limites à droite/gauche de $a$. Constatez que cela marche pour les limites des exercices précédents.

\item \textit{Toutes} les hypothèses sont importantes.}

\exo{Pour chaque hypothèse de ce théorème, la retirer et trouver un contre-exemple.}

\preuve{\doubleimp{Supposons que $\limc{x}{a} f(x)=l$. La dernière proposition nous donne $f(a)=l$. Si $\limc{x}{a-}=l'\neq l$ alors le rayon $\epsilon=|l-l'|$ conduirait à une absurdité. Idem si $\limc{x}{a+}=l'\neq l$.}

{Supposons maintenant $\limc{x}{a-}=\limc{x}{a+}=f(a)=l$. On a donc $a,l\in\R$. Ainsi, on peut écrire les définitions des limites à gauche et à droite : soit $\epsilon>0$. Il existe $\alpha,\beta>0$ tel que :

\sys{\forall x\in U, a-\alpha < x < a &\Rightarrow& |f(x)-l|<\epsilon \\
\forall x\in U, a < x < a+\beta &\Rightarrow& |f(x)-l|<\epsilon}

Soit maintenant $\delta=\min(\alpha,\beta)$. On a alors immédiatement par le système précédent que pour tout $x\in U\sauf{a}$, si $|x-a|<\delta$ alors $|f(x)-l|<\epsilon$. Si maintenant $x=a$ alors $|f(x)-l|=|f(a)-l|=0$ qui est toujours strictement inférieur à $\epsilon$. Donc il existe $\delta>0$ tel que pour tout $x\in U$, si $|x-a|<\delta$ alors $|f(x)-l|<\epsilon$ ce qui est la définition de $\limc{x}{a} f(x)=l$. }}

\ex{Soit \fone{f}{\R}{\R} qui à tout $x$ associe $\cos(x)$ si $x<0$ et $x^2+1$ sinon. Montrer que $\limc{x}{0} f(x)=1$.}

\theo{\programme (composition de limites) Soit $U'$ une union d'intervalles de $\R$ et \fone{g}{U'}{U}. Soit $a\in\overline{U'}$ et supposons qu'il existe $l\in\overline{U}$ tel que $\limc{x}{a} g(x)=l$ et $l'\in\overline{\R}$ tel que $\limc{x}{l}f(x)=l'$. Alors $\limc{x}{a}(f\circ g)(x)=l'$.}

\rem{Ce résultat est absolument fondamental car il permet de trouver les limites de toutes les composées de fonctions usuelles, il est par conséquent utilisé en permanence en terminale. S'il n'y avait que deux choses à retenir sur les limites, ça serait la définition et ce théorème.}

\preuve{Montrons ce résultat dans le cas où $a,l,l'\in\R$ (si l'un ou plusieurs des trois vaut $\pm\infty$, la démonstration s'adapte). Soit $\epsilon>0$. Puisque $\limc{x}{a} g(x)=l$ et $\limc{x}{l}f(x)=l'$ alors :

\sys{\exists \delta>0,\forall x'\in U, |x'-l| < \delta &\Rightarrow& |f(x')-l'| < \epsilon \\
\exists \delta'>0,\forall x\in U', |x-a| < \delta' &\Rightarrow& |g(x)-l| < \delta \\}

Soit $x\in U'$ et supposons $|x-a|<\delta'$. Alors $|g(x)-l|<\delta$. Or $g(x)\in U$ donc en posant $x'=g(x)$, on a $x'\in U$ et $|x'-l|<\delta$ d'où $|f(x')-l'|=|f(g(x))-l'|=|(f\circ g)(x)-l'|<\epsilon$. Nous venons d'écrire la définition de $\limc{x}{a}(f\circ g)(x)=l'$.}

\ex{Conjecturer la valeur de $\limc{x}{+\infty} \dfrac{1}{1-x}$ puis le démontrer. Il semble qu'elle vale $0$. Ici il est clair qu'on veut utiliser le théorème de composition.

\begin{itemize}
\item (calcul des limites séparées) On prouve facilement (par exemple en revenant à la définition) que $\limc{x}{+\infty} 1-x = -\infty$ et que $\limc{x}{-\infty} \dfrac{1}{x} = 0$.

\item (hypothèses du théorème) On veut appliquer le théorème de composition, vérifions que \textit{toutes} les hypothèses sont vérifiées. Ici, \fons{g}{\R\sauf{1}}{\R^*}{x}{1-x} et \fons{f}{\R^*}{\R}{x}{\dfrac{1}{x}}. Ici $a=+\infty$ et $U'=\R\sauf{1}$ donc $a\in\overline{U'}$. Ici $l=-\infty$ et $U=\R^*$ donc $l\in\overline{U}$. Enfin $l'=0\in\overline{\R}$. Toutes les hypothèses du théorème sont donc vérifiées.

\item (application) On a donc $\limc{x}{+\infty} \dfrac{1}{1-x}=\limc{x}{+\infty} (f\circ g)(x)=l'=0$.
\end{itemize}}

Il est bien entendu que dans la pratique on ne vérifie pas systématiquement toutes les hypothèses (à part pour des cas ambigus). En terminale et même après, calculer les limites séparément, citer le nom du théorème et appliquer suffit très bien. Cependant, en tout cas au début, c'est bien de le faire quelques fois pour vérifier qu'on sait de quoi on parle. Autrement dit, pour un bon étudiant en maths, lui demander de justifier rigoureusement pourquoi on peut appliquer le théorème ne devrait en théorie pas poser de problème. \\

Autre chose qui me tient à coeur : quand on demande de conjecturer une limite puis de la montrer, il faut de mon point de vue ne pas hésiter à \textit{user et abuser} des logiciels graphiques pour voir et conjecturer mieux. Il y a une certaine tradition du corps enseignant (pas encore totalement remise de la révolution numérique manifestement) qui voudrait que «~voir~» la limite juste en regardant la formule serait plus noble et que regarder graphiquement c'est synonyme d'abrutir les élèves. Ce sont les mêmes en général qui pestent contre les calculatrices et qui se plaignent que les élèves ne savent plus simplifier une fraction de trois kilomètres de long. Je caricature mais il y a incontestablement quelque chose de cet ordre dans le corps enseignant. Mon point de vue : on a de formidable outils à disposition, exploitons-les à fond, et utilisons le temps gagné pour réfléchir, modéliser, démontrer. Bref, pour faire des maths.

\ex{Conjecturer la valeur de $\limc{x}{+\infty} \tan\parent{\dfrac{1}{1-x}}$ puis le démontrer.

\begin{itemize}
\item D'une part $\limc{x}{+\infty} 1-x = -\infty$ et d'autre part $\limc{x}{-\infty} \dfrac{1}{x} = 0$ donc par le théorème de composition des limites, $\limc{x}{+\infty} \dfrac{1}{1-x}=0$.

\item Montrons que $\limc{x}{0}\tan(x)=0$. $\left]-\dfrac{\pi}{2},\dfrac{\pi}{2}\right[\subset D_{\tan}$ et $0\in\left]-\dfrac{\pi}{2},\dfrac{\pi}{2}\right[$. De plus $\tan(0)=0$ et $\limc{x}{0-}\tan(x)=\limc{x}{0+}\tan(x)=0$ donc d'après le théorème \ref{limgd}, $\limc{x}{0}\tan(x)=0$.

\item Finalement, par le théorème de composition des limites, $\limc{x}{+\infty} \tan\parent{\dfrac{1}{1-x}}=0$.
\end{itemize}}

\ex{Nous avons conjecturé dans l'exercice \ref{exrat} que $\limc{x}{+\infty} \dfrac{x^2}{2x^2-1}=\dfrac{1}{2}$. Le prouver. La technique pour les fonctions rationnelles (quotient de deux polynômes) est toujours la même : factoriser par le terme de plus haut degré. Commençons par poser $f$ la restriction de $x\mapsto\dfrac{x^2}{2x^2-1}$ à $]0,+\infty[$. Il est clair que  $\limc{x}{+\infty} \dfrac{x^2}{2x^2-1}=\limc{x}{+\infty} f(x)$. Pour tout $x\in]0,+\infty[$, $f(x)=\dfrac{x^2}{x^2\parent{2-\dfrac{1}{x^2}}}=\dfrac{1}{2-\dfrac{1}{x^2}}$ donc $\limc{x}{+\infty} \dfrac{x^2}{2x^2-1}=\limc{x}{+\infty}\dfrac{1}{2-\dfrac{1}{x^2}}$. On a $\limc{x}{+\infty}\dfrac{1}{x^2}=0$, $\limc{x}{0}\parent{2-\dfrac{1}{x^2}}=2$ et $\limc{x}{2}\dfrac{1}{x}=\dfrac{1}{2}$ donc par composition, $\limc{x}{+\infty} \dfrac{x^2}{2x^2-1}=\dfrac{1}{2}$.}



\theo{\programme (théorèmes de comparaisons) Soit $a\in\overline{U}$ et $l\in\overline{\R}$ et supposons $\limc{x}{a}f(x)=l$.

\begin{enumerate}[i)]
\item Si $l$ est \textit{fini}, alors il existe $m,M\in\R$ tel que $m<a<M$ et tel que $f(]m,M[)$ est bornée.

\item Si il existe $m,M\in\R$ tel que $m<a<M$ et tel que $f(]m,M[)$ est bornée, c'est-à-dire qu'il existe $\alpha,\beta\in\R$ tel que $f(]m,M[)\in]\alpha,\beta[$, alors $\alpha<l<\beta$.

\item Soit \fone{g}{U}{\R} et supposons que $\limc{x}{a}g(x)=l'\in\overline{U}$. Si il existe $m,M\in\R$ tel que $m<a<M$ et tel que dans $]m,M[$, les images de $g$ sont inférieures ou égales à celles de $f$, alors $l'\le l$. Même chose pour «~supérieures ou égales~». \textbf{Attention} : pour «~strictement~», ça ne marche pas. 
\end{enumerate}}

\rem{Intuitivement, le premier nous dit que si la limite est finie, on peut mettre un rectangle autour de $a$ tel que les images de la fonction n'y sortent pas. Pour le second, si on peut mettre un rectangle autour de $a$ tel que les images de la fonction n'y sortent pas, alors la limite ne sort pas de ce rectangle. Le troisième, si autour de $a$, $g$ est en-dessous de $f$ alors leur limite aussi.}

\theo{\programme (gendarmes, minoration, majoration) Soit \fone{f,g,h}{U}{\R} trois applications, $a\in\overline{U}$ et $l\in\overline{\R}$. Supposons qu'il existe $m,M\in\R$ tel que $m<a<M$ et tel que dans $]m,M[$, les images de $f$ sont inférieures ou égales à celles de $g$ et les images de $g$ sont inférieures ou égales à celles de $h$.

\begin{enumerate}[i)]
\item (gendarmes/sandwichs) Si $\limc{x}{a}f(x)=\limc{x}{a}h(x)=l$ alors $\limc{x}{a}g(x)=l$.

\item (minoration) Si $\limc{x}{a}f(x)=+\infty$ alors $\limc{x}{a}g(x)=+\infty$.

\item (majoration) Si $\limc{x}{a}h(x)=-\infty$ alors $\limc{x}{a}g(x)=-\infty$.
\end{enumerate}}

\rem{C'est l'équivalent de ce qu'on connaissait pour les suites.}

\section{Continuité}

Soit $U$ une union d'intervalles de $\R$ et \fone{f}{U}{\R} et $a\in U$ (pas $\overline{U}$ !). Enfin on pose $I$ un intervalle de $\R$. 

\subsection{Définition}

\defi{\programme (continuité en un point) On dit que \textit{$f$ est continue en $a$} si $\limc{x}{a}f(x)$ existe. Dans le cas contraire, on dit que \textit{$f$ est discontinue en $a$}.}

\pro{Supposons que $f$ est continue en $a$. Alors $\limc{x}{a}f(x)=f(a)$}

\rems{\item C'est ce qui correspond à l'intuition selon laquelle une fonction est continue si on peut ne pas lever le crayon.
\item Nous pouvons donc réécrire la définition de continuité en $a$ comme suit : pour tout $\epsilon>0$, il existe $\delta>0$ tel que pour tout $x\in U$, si $|x-a|<\delta$ alors $|f(x)-f(a)|<\epsilon$.}

\preuve{C'est la proposition \ref{limimg}.}

\rem{Par contre les limites à gauche et à droite de $a$ n'existent pas forcément ! Contre-exemple : la fonction racine admet une limite en $0$ donc est continue en $0$, mais elle n'admet pas de limite à gauche. Pour que la limites à gauche et à droite existent, il faut en plus qu'il existe un intervalle ouvert contenant $a$ et qui est dans $U$ (voir le théorème \ref{limgd}).}

\defi{\programme (continuité en un intervalle) On dit que \textit{$f$ est continue sur $I$} si pour tout $x\in I$, $f$ est continue en $x$.}

\rem{Par contre, la notion de \textit{continuité sur une union d'intervalles ne veut rien dire}. En effet, prenons par exemple la fonction inverse : elle est continue sur $]-\infty,0[$, elle est continue sur $]0,+\infty[$, pourtant dire qu'elle est continue sur $\R^*$ ne veut rien dire.}

\nota{On note $\mathscr{C}^0(I,\R)$ l'ensemble des fonctions à valeurs dans $\R$ continues sur $I$.}

\defi{(prolongement par continuité) Supposons qu'il existe $a\not\in U$. On dit que $f$ est \textit{prolongeable par continuité en $a$} s'il existe un prolongement $g$ de $f$ continue en $a$.}

\ex{Soit \fons{f}{\R^*}{\R}{x}{\dfrac{\sin(x)}{x}} et \fone{g}{\R}{\R} la fonction telle que $g(0)=1$ et pour tout $x\in\R^*$, $g(x)=f(x)$. Alors $g$ est un prolongement de $f$ et $g$ est continue en $0$, donc $f$ est prolongeable par continuité en $0$.}

\subsection{Opérations sur la continuité}

\pro{(composition et continuité) Soit $U'$ une union d'intervalles de $\R$ et \fone{g}{U'}{U}. Si $g$ est continue en $a$ et que $f$ est continue en $g(a)$, alors $f\circ g$ est continue en $a$.}

\pro{(opérations sur la continuité) Soit \fone{g}{U}{\R}, $\lambda\in\R$ et supposons que $f$ et $g$ sont continues en $a$. Alors les fonctions suivantes sont également continues en $a$  :


\begin{enumerate}[i)]
\item $\lambda f$
\item $|f|$
\item $f+g$
\item $fg$
\item $\dfrac{f}{g}$ si $g(a)\neq 0$
\end{enumerate}}
\rem{Et donc puisque les propositions précédentes sont vraies en un point, elles sont aussi vraies sur un intervalle $I$ (attention, dans ce cas pour le dernier point il faut que $g(x)\neq 0$ pour tout $x\in I$).}

\subsection{Continuité des fonctions usuelles}

\theo{\programme Posons :

\begin{itemize}
\item \fons{f_1}{\R}{\R}{x}{x}
\item \fons{f_2}{\R}{\R}{x}{|x|}
\item \fons{f_3}{\R}{\R}{x}{P(x)} où $P$ est un polynôme réel.
\item \fons{f_4}{I}{\R}{x}{\dfrac{P(x)}{P'(x)}} où $P,P'$ sont des polynômes réels et $I$ un intervalle ouvert ne contenant aucune racine de $P'$.
\item \fons{f_5}{\R_+}{\R}{x}{\sqrt{x}}
\item \fons{f_6}{I}{\R}{x}{\dfrac{1}{x}} où $I=]-\infty,0[$ ou $]0,+\infty[$.
\item \fons{f_7}{\R}{\R}{x}{\cos(x)}
\item \fons{f_8}{\R}{\R}{x}{\sin(x)}
\item \fons{f_9}{I}{\R}{x}{\tan(x)} où $I=\left]-\dfrac{\pi}{2}+k\pi,\dfrac{\pi}{2}+k\pi\right[$ avec $k\in\Z$.
\end{itemize}

Alors toutes ces fonctions sont continues sur leur ensemble de définition.}

\rems{\item Pour résumer, toutes les fonctions usuelles sont continues sur un intervalle.

\item $f_4$ est appellée \textit{fonction rationnelle} : le quotient de deux polynômes.}

\preuve{Pour chaque fonction, nous donnerons simplement soit un $\delta$ qui convient, soit nous expliquerons quelles propriétés nous utilisons.

\begin{itemize}
\item ($f_1$) $\delta=\epsilon$
\item ($f_2$) $|f_1|$ est continue (opérations sur la continuité).
\item ($f_3$) Pour tout $n\in\N$, $(f_1)^n$ est continue par multiplication donc pour tout $\lambda\in\R$, $\lambda(f_1)^n$ est continue et finalement par somme, $P$ est continue.
\item ($f_4$) $P$ et $P'$ sont continues sur $\R$ et $P'(x)\neq0$ pour tout $x\in I$ donc par division, on a le résultat.
\item ($f_5$) $\delta=a-(\sqrt{a}-\epsilon)^2=\epsilon(2\sqrt{a}-\epsilon)$ (faire un dessin permet de comprendre que ce delta est une intersection particulière).
\item ($f_6$) $\dfrac{1}{f_1}$ est continue par division.
\item ($f_7$) Nous admettons. Le plus simple en fait de prouver que $\cos$ est une fonction \href{https://fr.wikipedia.org/wiki/Application_lipschitzienne}{lipschitzienne} (vue dans le supérieur), or toute fonction lipschitzienne est continue.
\item ($f_8$) Pour tout $x\in\R$, $\sin(x)=\cos\parent{x-\dfrac{\pi}{2}}$ donc par composition $\sin$ est continue.
\item ($f_9$) $\dfrac{f_8}{f_7}$ est continue par division ($D_{\tan}=\R\sauf{\dfrac{\pi}{2}+2k\pi,k\in\Z}$).
\end{itemize}}

\subsection{Images d'intervalles par une fonction continue}

\theo{\programme (Théorème des Valeurs Intermédiaires) Soit $I$ un intervalle de $\R$. Si $f$ est continue sur $I$ alors $f(I)$ est un intervalle de $\R$.}

\rems{\item Autrement dit, l'image d'un intervalle par une fonction continue est un intervalle.
\item On l'appelle très souvent TVI. Ce théorème est incroyablement utilisé en analyse réelle, s'il n'y en qu'un seul à retenir, ça serait certainement lui.
\item Dans ce cas, $f(I)\subset\im f$}

\exo{Montrer que tout polynôme de degré impair admet au moins une racine. \spoiler{Notons $P$ un polynôme de degré impair. On suppose le coefficient de plus haut degré positif (l'autre cas est similaire). On commence par montrer que $\limc{x}{-\infty}P(x)=-\infty$ et que cela implique nécessairement qu'il existe $a\in\R$ tel que $P(a)<0$. De même on montre que $\limc{x}{+\infty}P(x)=+\infty$ et que cela implique nécessairement qu'il existe $b>a\in\R$ tel que $P(b)>0$. Et puisque $P$ est continue sur $\R$, alors par le TVI, $f([a,b])$, que l'on note $J$, est un intervalle, et $J\subset\im f$. Or $P(a),P(b)\in J$ et $P(a)<P(b)$ donc nécessairement, $[P(a),P(b)]\subset J$. De plus $0\in[P(a),P(b)]$ donc $0\in J$, donc $0\in\im f$ donc il existe $c\in\R$ tel que $P(c)=0$, d'où $P$ admet au moins une racine.}}



\preuve{Déjà démontré en utilisant la dérivée.}

\theo{(de Weierstrass) Soit $I$ un intervalle \textit{fermé} de $\R$. Si $f$ est continue sur $I$ alors $f(I)$ est bornée et atteint ses bornes.}

\preuve{Par le TVI, $f(I)$ est un intervalle. Pour prouver qu'il est fermé, on raisonne par l'absurde.}

\cor{Soit $I$ un intervalle \textit{fermé} de $\R$. Si $f$ est continue sur $I$ alors $f(I)$ est un intervalle fermé.}
\preuve{C'est la combinaison des deux théorèmes précédents}

\subsection{Note culturelle}

La définition de limite que nous l'avons présentée est en réalité typiquement française et porte le nom très rigoureux de \textit{limite pointée}. Il existe une autre façon de définir la limite dans la plupart des autres pays et qui ne conduit pas forcément aux mêmes résultats~: c'est la \textit{limite épointée}. La question de la pertinence d'utiliser l'une plutôt que l'autre suscite des débats dans le corps enseignant. Les lecteurs \textit{déjà complètement au point} sur les notions de limite et de continuité que nous avons présentés dans le cours (au risque de tout confondre !) peuvent jeter un coup d'oeil à \href{http://www.les-mathematiques.net/phorum/read.php?18,1424120,1426100}{ce fil de discussion} et surtout \href{https://www.math.u-psud.fr/~perrin/CAPES/analyse/fonctions/definitiondelimite.pdf}{à ce PDF} très intéressants.


\section{Dérivées}

Soit dans cette section \fone{f}{U}{\R} où $U$ est une union d'intervalles de $\R$.

\subsection{Notions de base}

\defi{(taux d'accroissement) Soit $a\in U$. On appelle \textit{taux d'accroissement de $f$ en $a$} la fonction \fons{\tau_{f,a}}{U\sauf{a}}{\R}{x}{\dfrac{f(x)-f(a)}{x-a}}.}

\rems{\item Intuitivement, le taux d'accroissement n'est rien d'autre que les coefficients directeurs des droites passant par $(a,f(a))$ et par $(x,f(x))$ avec $x\in U\sauf{a}$.

\item On trouve, selon les enseignants, une définition alternative du taux d'accroissement : \fons{\tau_{f,a}}{U\sauf{0}}{\R}{h}{\dfrac{f(a+h)-f(a)}{h}}. Cette définition a pour avantage de centrer en 0, mais sinon, cela revient complètement au même (en fait pour passer de l'une à l'autre, on fait simplement le changement de variable $x=a+h$).}

\defi{(dérivée) Soit $a\in U$. On dit que \textit{$f$ est dérivable en $a$} si $\limc{x}{a}\tau_{f,a}(x)$ existe est est finie. On appelle alors cette limite \textit{dérivée de $f$ en $a$} et on la note $f'(a)$. }

\rem{Avec la définition alternative du taux d'accroissement, cela donne : «~$f$ est dérivable en $a$ si $\limc{h}{0}\tau_{f,a}(h)$ existe~».}

\defi{(dérivabilité sur un ensemble) Soit $D\subset U$. On dit que \textit{$f$ est dérivable sur $D$} si $f$ est dérivable en tout point de $D$.}

\theo{Soit $I$ un intervalle de $\R$. Si $f$ est dérivable sur $I$ alors $f$ est continue sur $I$.}
\rem{\textbf{La réciproque est fausse ! }Contre-exemple : la fonction valeur absolue est continue sur $\R$ mais pas dérivable en $0$.}

\preuve{Soit $a\in I$. $f$ est dérivable en $a$ donc $\limc{x}{a}\tau_{f,a}(x)=f'(a)$ existe et est finie. Soit \fons{g}{U\sauf{a}}{\R}{x}{x-a}. $g$ admet pour limite $0$ en $a$. Donc par produit (proposition \ref{oplim2}), $\limc{x}{a}\parent{\tau_{f,a}(x)\times g(x)}=\limc{x}{a}\tau_{f,a}(x)\limc{x}{a}g(x)=f'(a)\times 0 = 0$. Donc $\limc{x}{a} (f(x)-f(a))=0$ et donc $\limc{x}{a}f(x)=f(a)$, conclusion $f$ est continue en $a$.}

\defi{(fonction dérivée) Soit $D$ le plus grand sous-ensemble de $U$ tel que $f$ est dérivable sur $D$. On appelle \textit{fonction dérivée de $f$} la fonction \fons{f'}{D}{\R}{x}{f'(x)}.}

\rem{\emph{Attention} : il n'y aucune raison que $D=U$. Contre-exemple : la fonction racine est définie mais pas dérivable en 0.}

\theo{\programme (dérivées usuelles)

\begin{tabular}{|c|c|c|} \hline
Fonction & Domaine de dérivabilité & Dérivée \\ \hline
$x^q$, $q\in\Q$ & au moins $]0,+\infty[$ mais peut être plus grand selon la valeur de $q$ & $qx^{q-1}$ \\ \hline
$\cos(x)$ & $\R$ & $-\sin(x)$ \\ \hline
$\sin(x)$ & $\R$ & $\cos(x)$ \\ \hline
\end{tabular}}

\rems{\item Nous renvoyons le lecteur à la partie \ref{puirat} concernant la signification de $x^q$ avec $q\in\Q$.

\item Ces trois dérivées regroupent toutes les dérivées des fonctions usuelles. En particulier, la première permet de retrouver $1/x$, les polynômes, racine carrée...}

\theo{\programme (opération sur les dérivées) Soit \fone{g}{U}{\R} et $\lambda\in\R$. Alors là où cela à un sens :

\begin{enumerate}[i)]
\item $(\lambda f)'=\lambda f'$
\item $(f+g)'=f'+g'$
\item $(fg)'=f'g+fg'$
\item $\parent{\dfrac{1}{f}}'=-\dfrac{f'}{f^2}$
\item $\parent{\dfrac{f}{g}}'=\dfrac{f'g-fg'}{g^2}$
\item $(u^n)'=nu'u^{n-1}$, $n\in\N^*$
\item $(f\circ g)'=f'\circ g \times g'$
\end{enumerate}}

\rem{En gros, ce qu'il faut absolument retenir, ces sont les dérivées des fonctions usuelles, la somme, le produit, le quotient et la composée. Tout le reste se retrouve.}

Avec XCas, on calcule une fonction dérivée comme suit : \\

\verb&diff(sqrt(x))& \\

\subsection{Grands théorèmes}

Dans cette section, $I$ est un intervalle de $\R$ et \fone{f}{I}{\R} une fonction dérivable sur $I$.

\theo{(de Rolle) Supposons qu'il existe $a,b\in I$ avec $a<b$ tel que $f(a)=f(b)$. Alors il existe $c\in ]a,b[$ tel que $f'(c)=0$.}

\includegraphics[scale=0.15]{figures/pdf/rolle-eps-converted-to.pdf}

\rem{Il y a existence mais pas forcément unicité. Par exemple prendre $f=\sin$, $a=0$ et $b=8\pi$.}

\preuve{Déjà vue.}

\theo{(des accroissements finis, ou TAF) Pour tout $a,b\in I$ avec $a<b$, il existe $c\in ]a,b[$ tel que $f'(c)=\dfrac{f(b)-f(a)}{b-a}$.}

\includegraphics[scale=0.15]{figures/pdf/taf-eps-converted-to.pdf}

\rems{\item De même, il n'y a pas forcément unicité.
\item Il est clair que le théorème de Rolle est un cas particulier de celui-ci.}

\preuve{Déjà vue.}

\theo{(lien entre monotonie et dérivée)

\begin{itemize}
\item Pour tout $x\in I$, $f'(x)= 0$ \ssi $f$ est constante sur $I$.
\item Pour tout $x\in I$, $f'(x)\ge 0$ \ssi $f$ est croissante sur $I$.
\item Pour tout $x\in I$, $f'(x)\le 0$ \ssi $f$ est décroissante sur $I$.
\item Supposons que pour tout $x\in I$, $f'(x)> 0$. Alors $f$ est strictement croissante sur $I$.
\item Supposons que pour tout $x\in I$, $f'(x)< 0$. Alors $f$ est strictement décroissante sur $I$.
\end{itemize}}

\rems{\item Ce théorème est fondamental pour étudier la monotonie des fonctions.}

\exo{Les 3 premiers points sont donc des équivalences, mais pas les deux derniers~: comment les modifier pour qu'elles le soient ?}

\preuve{Déjà vue.}

\subsection{Classes de régularité}

Soit $I$ un intervalle de $\R$. Nous avons déjà vue la notation $\mathscr{C}^0(I,\R)$ qui est l'ensemble des fonctions continues de $I$ dans $\R$. Nous allons généraliser cette notation. 

\nota{(classes de fonctions dérivables)\begin{itemize}
\item On note $\mathscr{D}(I,\R)$ l'ensemble des fonctions dérivables de $I$ dans $\R$.
\item On note $\mathscr{D}^2(I,\R)$ l'ensemble des fonctions deux fois dérivables de $I$ dans $\R$ (\cad les fonctions dérivables de $I$ dans $\R$ dont la dérivée est elle-même dérivable de $I$ dans $\R$).
\item Par récurrence, pour tout $k\in\N^*$, on note $\mathscr{D}^k(I,\R)$ l'ensemble des fonctions $k$ fois dérivables de $I$ dans $\R$.
\item Enfin, on note $\mathscr{D}^\infty(I,\R)$ l'ensemble des fonctions infiniment dérivables de $I$ dans $\R$.
\end{itemize}} 
\rem{Attention, contrairement à $\mathscr{C}^0$, il n'y a pas de $\mathscr{D}^0$.}

\nota{(dérivée $k$-ième) Soit $k\ge 2$ et soit $f\in\mathscr{D}^k(I,\R)$.

\begin{itemize}
\item On note $f'$ la dérivée de $f$.
\item On note $f''$ la dérivée de la dérivée de $f$ et on l'appelle\textit{ dérivée seconde} de $f$.
\item On note $f^{(k)}$ la dérivée $k$-ième de $f$, \cad la dérivée de la dérivée de la dérivée... de la dérivée de $f$, avec $k$ répétitions du mot «~dérivée~».
\end{itemize}}
\rem{Les parenthèses servent à ne pas confondre avec $f^k$ qui peut selon le contexte désigner $f$ puissance $k$ ou $f$ composée $k$ fois.}

\nota{(classes de fonctions continues et dérivables)

\begin{itemize}
\item On note $\mathscr{C}^0(I,\R)$ l'ensemble des fonctions continues de $I$ dans $\R$.

\item On note $\mathscr{C}^1(I,\R)$ l'ensemble des fonctions dérivables de $I$ dans $\R$ et de dérivée continue de $I$ dans $\R$.

\item Par récurrence, pour tout $k\in\N^*$, on note $\mathscr{C}^k(I,\R)$ l'ensemble des fonctions $k$ fois dérivables de $I$ dans $\R$ et dont la dérivée $k$-ième est continue de $I$ dans $\R$.

\item On note $\mathscr{C}^\infty(I,\R)$ l'ensemble des fonctions infiniment dérivables et dont toutes les dérivées sont continues de $I$ dans $\R$. 
\end{itemize}}

\rems{\item Comme toute fonction dérivable de $I$ dans $\R$ est continue, alors $\mathscr{D}(I,\R)\subset\mathscr{C}^0(I,\R)$ et $\mathscr{C}^\infty(I,\R)=\mathscr{D}^\infty(I,\R)$.

\item Pour tout $k\in\N^*$, $\mathscr{C}^k(I,\R)\subset\mathscr{D}^k(I,\R)$.

\item De manière générale, $\mathscr{C}^\infty(I,\R)\subset...\subset\mathscr{C}^k(I,\R)\subset\mathscr{D}^k(I,\R)\subset...\subset\mathscr{C}^1(I,\R)\subset\mathscr{D}^1(I,\R)\subset\mathscr{C}^0(I,\R)$.

\item L'existence de la classe $\mathscr{C}$ est justifiée car ce n'est pas parce qu'une fonction est dérivable que sa dérivée est forcément continue ! Nous verrons un exemple ultérieurement.}

Quand on demande la classe d'une fonction \fone{f}{I}{\R}, on demande généralement le plus grand $k$ tel que $f\in\mathscr{C}^k(I,\R)$.

\exo{Déterminer la classe des fonctions usuelles sur le plus grand intervalle possible.}


\section{Primitives et intégration}

\subsection{Primitives}

Dans toute la suite, $I$ est un intervalle de $\R$ et \fone{f}{I}{\R}.

\defi{\programme On appelle \textit{primitive de $f$} toute fonction \fone{F}{I}{\R}, dérivable sur $I$ et telle que pour tout $x\in I$, $F'(x)=f(x)$.}

\exo{Toute fonction n'admet pas forcément de primitive. Contre-exemple : prouver que la fonction partie entière inférieure n'admet pas de primitive sur $\R$.}

\rem{En particulier, si $f$ est dérivable sur $I$, alors $f$ est une primitive de $f'$ sur $I$.}

\pro{\programme Si $f$ est continue sur $I$ alors $f$ admet une primitive sur $I$.}

\preuve{La preuve, dépassant très largement le cadre du programme de lycée, est admise.}

\rem{Et comme on sait que tout fonction dérivable sur un intervalle est continue, toute primitive d'une telle fonction $f$ est continue sur $I$.}

\exo{Montrer que la fonction valeur absolue admet une primitive sur $\R$, en trouver une.}

\exo{\textit{Attention, la réciproque de cette proposition est fausse}. Contre-exemple : soit \fone{f}{\R}{\R} telle que $f(0)=0$ et pour tout $x\in\R^*$, $f(x)=2x\sin\parent{\dfrac{1}{x}}-\cos\parent{\dfrac{1}{x}}$. Montrer que $f$ n'est pas continue sur $\R$ mais admet pourtant pour primitive la fonction \fone{F}{\R}{\R} telle que $F(0)=0$ et telle que pour tout $x\in\R^*$, $F(x)=x^2 \sin\parent{\dfrac{1}{x}}$ (on vérifiera bien que $F$ est effectivement dérivable sur $\R$ avant de dériver).}

\lemme{Si $f$ est dérivable sur $I$ et que $f'$ est nulle sur $I$, alors $f$ est constante.}

\preuve{$f$ est dérivable donc continue sur $I$. Supposons que $f$ n'est pas constante sur $I$. Alors il existe $a,b\in I$ tel que $f(a)\neq f(b)$. Donc d'après le théorème des accroissements finis,  il existe $c\in I$ tel que $f'(c)=\dfrac{f(b)-f(a)}{b-a}\neq0$, donc $f'$ ne s'annule pas en $c$ : c'est absurde.}

\theo{Supposons que $f$ admet une primitive \fone{F}{I}{\R} sur $I$. Alors $\ens{F+k,k\in\R}$ est l'ensemble de \textit{toutes} les primitives de $f$.}


\preuve{Nous faisons une preuve par double inclusion.\begin{itemize}
\item Soit $k\in\R$ et $G=F+k$. $F$ et $k$ sont dérivables sur $I$ et par sommation, $G'=F'+k'=F'=f$ donc $F+k$ est une primitive de $f$.

\item Réciproquement, soit $G$ une primitive de $f$ et soit $D=G-F$. $G$ et $F$ sont des primitives de $f$ sur $I$ donc pour tout $x\in I$, $G'(x)=f(x)=F'(x)$ donc $D'(x)=0$. Par le lemme, $D$ est constante donc il existe $k\in\R$ tel que pour tout $x\in I$, $D(x)=k$, \cad tel que $G-F=k$, \cad tel que $G=F+k$.
\end{itemize}}

\cor{En particulier, soit $y\in\R$. Si $x\in I$, $f$ admet une \textit{unique} primitive \fone{F}{I}{\R} tel que $F(x)=y$.}

\theo{\programme (primitives usuelles)

\begin{tabular}{|c|c|c|} \hline
Fonction & $I$ & Primitive \\ \hline
$x^q$ avec $q\in \Q\sauf{-1}$ & Au moins $]0,+\infty[$ & $\dfrac{x^{q+1}}{q+1}$ \\ \hline
$\cos(x)$ & $\R$ & $\sin(x)$ \\ \hline
$\sin(x)$ & $\R$ & $-\cos(x)$ \\ \hline
\end{tabular}}
\rem{Nous verrons ultérieurement une primitive des fonctions inverse et tangente.}

\ex{Déterminer une primitive de \fons{f}{]0,+\infty[}{\R}{x}{\dfrac{1}{\sqrt[3]{x}}} sur $]0,+\infty[$. Pour tout $x\in]0,+\infty[$, $f(x)=x^{-1/3}$ donc une primitive de $x$ sur $]0,+\infty[$ est \fons{F}{]0,+\infty[}{\R}{x}{\dfrac{x^{2/3}}{2/3}} soit, en réécrivant, $\dfrac{3}{2}(\sqrt[3]{x})^2$.}

\ex{Déterminer \textit{la} primitive de \fons{f}{\R}{\R}{x}{4\sin(3x-2)} sur $\R$ qui s'annule en 0. Comme nous savons que $-\cos$ est une primitive de $\sin$, on peut essayer $x\mapsto -\cos(3x-2)$. La dérivée de cette fonction est $x\mapsto 3\sin(3x-2)$ par composition, il suffit donc de corriger le coefficient. Ainsi, une primitive de $f$ sur $\R$ est \fons{F_0}{\R}{\R}{x}{-\dfrac{4}{3}\cos(3x-2)}. Puisque toutes les primitives de $f$ diffèrent d'une constante, il existe $k\in\R$ tel que la primitive de $f$ qui s'annule en 0 est \fons{F}{\R}{\R}{x}{-\dfrac{4}{3}\cos(3x-2)+k}. Donc $F(0)=0$ d'où $-\dfrac{4}{3}\cos(-2)+k=0$ d'où $k=\dfrac{4}{3}\cos(2)$ (on rappelle que $\cos$ est paire). Conclusion, \textcolor{brown}{\fons{F}{\R}{\R}{x}{\dfrac{4}{3}\cro{\cos(2)-\cos(3x-2)}}} est la primitive de $f$ qui s'annule en 0.}

\subsection{Intégration}

Dans toute la suite, nous considérons $I$ un intervalle de $\R$, $a,b\in I$  tel que $a\le b$, \fone{f}{I}{\R} une fonction \textit{continue} et à valeurs \textit{positives} sur $[a,b]$ et enfin $C$ la courbe représentative de $f$.

\subsubsection{Généralités}

Le but est de déterminer l'aire $A$ de la surface délimitée par $C$ et les droites d'équations $x=a$, $x=b$ et $y=0$.

\includegraphics[scale=0.4]{figures/pdf/int_obj-eps-converted-to.pdf}

Pour cela, une méthode est de placer des rectangles dans cette surface pour remplir $A$. Plus il y a de rectangles, plus l'approximation est bonne. Et avec une infinité de rectangles, on obtient exactement $A$. Ci-dessous on voit comment approximer $A$ avec 3 rectangles :

\includegraphics[scale=0.4]{figures/pdf/int_rec-eps-converted-to.pdf}

\defi{(subdivision) Soit $n\in\N^*$. On appelle \textit{subdivision d'ordre $n$ de $[a,b]$} la famille $(x_k)_{k\in\lint 0,n\rint}$ telle que pour tout $k\in\lint 0,n\rint$, $x_k=a+kh$ où $h=\dfrac{b-a}{n}$.}
\rems{\item Nous avons donc : $x_0=a$, $x_n=b$ et la distance entre deux éléments successifs de la subdivision est constante et égale $h$.
\item Sur le schéma ci-dessus, nous avons illustré le cas particulier $a=2$, $b=8$ et $n=3$.}

Dans toute la suite, posons $n\in\N^*$ et $(x_k)_{k\in\lint 0,n\rint}$ la subdivision d'ordre $n$ de $[a,b]$.

\defi{(somme de Riemann) On appelle \textit{somme de Riemann de $f$ sur $[a,b]$} la suite $(S_n)_{n\in\N^*}$ définie pour tout $n\in\N^*$ par $\somme{k=0}{n-1} \cro{hf(x_k)}$.}

\rems{\item Intuitivement il s'agit simplement de la somme des aires des rectangles générés par $C$ et la subdivision d'ordre $n$ de $[a,b]$.
\item Sur le schéma ci-dessus est illustré $S_3$ qui est la somme des aires des trois rectangles verts.
\item Puisque $h$ ne dépend pas de $k$ dans la somme, alors pour tout $n\in\N^*$, $S_n=h\somme{k=0}{n-1} f(x_k)$.}

\theo{La somme de Riemann de $f$ sur $[a,b]$ converge et sa limite vaut $A$.}

\preuve{Ce théorème est admis mais il reste très intuitif.}

\ex{Calculons $A$ dans le cas où $a=1$, $b=2$ et \fons{f}{\R}{\R}{x}{x}. Notons que cela a du sens puisque $f$ est continue et positive sur $[1,2]$. D'abord, remarquons que géométriquement la surface est un trapèze dont on sait calculer l'aire : on trouve $A=\dfrac{3}{2}$. Essayons de retrouver ce résultat avec les sommes de Riemann. La somme de Riemann de $f$ sur $[1,2]$ vérifie, pour tout $n\in\N^*$, $S_n=h\somme{k=0}{n-1} f(x_k)=h\somme{k=0}{n-1} x_k=\dfrac{1}{n}\somme{k=0}{n-1} \parent{1+\dfrac{k}{n}}=1+\dfrac{1}{n^2}\somme{k=0}{n-1} k$. Il s'agit de la somme d'une suite arithmétique de raison 1 et de premier terme 0, ainsi pour tout $n\in\N^*$, $S_n=1+\dfrac{1}{n^2}\dfrac{n(n-1)}{2}=1+\dfrac{n-1}{2n}$. Ainsi $A=\limc{n}{+\infty}\parent{1+\dfrac{n-1}{2n}}=1+\limc{n}{+\infty}\dfrac{n-1}{2n}=1+\dfrac{1}{2}=\dfrac{3}{2}$. On retrouve le résultat.}

Bien entendu, il est plus intéressant de voir l'intérêt de cette somme de Riemann pour des surfaces dont on ne sait pas géométriquement calculer l'aire.

\exo{Soit \fons{f}{\R}{\R}{x}{x^2}.

\begin{enumerate}
\item Montrer par récurrence que pour tout $n\in\N^*$, $\somme{k=0}{n}k^2=\dfrac{n(n+1)(2n+1)}{6}$.
\item Justifier que $f$ est continue et positive sur $[0,1]$.
\item Déterminer une forme explicite de la somme de Riemann de $f$ sur $[0,1]$.
\item En déduire que $A=\dfrac{1}{3}$ pour $a=0$ et $b=1$.
\end{enumerate}}

\defi{\programme (intégrale) On appelle \textit{intégrale (de Riemann) de $f$ sur $[a,b]$} la limite de la somme de Riemann de $f$ sur $[a,b]$ et on la note \textcolor{blue}{$\intg{a}{b}{f(x)}$}. }

Sur XCas, on peut calculer $\intg{0}{1}{x^2}$ ainsi : \\

\verb&int(x^2,x,0,1)& \\

\rems{\item \programme Ainsi, $\intg{a}{b}{f(x)}=\limc{n}{+\infty} S_n=A$
\item La variable peut changer de nom sans que cela ne change la valeur de l'intégrale (on parle de variable muette). Ainsi $\intg{a}{b}{f(x)}=\intgv{a}{b}{f(t)}{t}$.
\item Le «~$\mathrm{d}$~» désigne en réalité ce qu'on appelle une \textit{différentielle}. Au niveau lycée, considérez simplement qu'elle désigne une \textit{différence infinitésimale des abscisses} (l'écart $h$ entre les rectangles tend vers 0).
\item Dans cette notation, il s'agit réellement d'une lettre S stylisée pour rappeler que l'intégrale est issue d'une somme de Riemann.
\item Dans le supérieur, vous découvrirez qu'il existe plusieurs autres intégrales, donc en toute rigueur on doit parler ici d'intégrale de Riemann. Cependant, quand le contexte est clair (et il le sera jusqu'en L3), on peut juste parler d'«~intégrale~».}

\ex{Les deux exercices précédents nous indiquent donc que $\intg{1}{2}{x}=\dfrac{3}{2}$ et que $\intg{0}{1}{x^2}=\dfrac{1}{3}$.}

Quelques notes sur notre cadre d'étude :

\begin{itemize}
\item Nous avons ici défini arbitrairement l'intégrale de Riemann par des rectangles qui partent à gauche des points de subdivison (on parle de \textit{méthode des rectangles à gauche}). En réalité on aussi les faire partir à droite, et même au milieu, sans que cela ne change le résultat. On peut aussi faire des trapèzes (\textit{méthode des trapèzes}) reliant des points de la courbe : la somme de Riemann définie à partir de ce dernier procédé converge plus rapidement qu'avec les rectangles. Enfin, on peut aussi, aux points de subdivision, approximer la courbe par des polynômes de degré 2 reliés entre eux (et on sait calculer l'intégrale des polynômes de degré 2 comme l'illustre un des exercices précédents). Cette méthode est appellée \textit{méthode de Simpson} et elle est bien plus rapide que les trapèzes (donc que les rectangles).

\item Nous nous sommes placés ici dans le cadre des fonctions positives sur $[a,b]$. En réalité il n'y a aucun problème à calculer l'intégrale de fonctions négatives, simplement les parties d'aires situées sous l'axe des abscisses doivent être comptées négativement. En quelque sorte il faut considérer l'aire \textit{algébrique} sous la courbe. Mais attention, si on prend en compte les négatifs, l'égalité de l'intégrale avec $A$ n'est plus respectée : autrement dit prendre en compte les négatifs est possible mais au prix de l'interprétation géométrique de l'intégrale.

\item Nous nous sommes placés ici dans le cadre de fonctions continues sur $[a,b]$. Il est en fait tout à fait possible de calculer l'intégrale de fonctions \textit{continues par morceaux} sur $[a,b]$, c'est-à-dire qu'il existe $n$ intervalles $I_1,...,I_n$ tous disjoints tel que $I_1\cup...\cup I_n=[a,b]$ et tel que la fonction est continue sur chaque $I_k$. L'intégrale sur $[a,b]$ d'une telle fonction est alors simplement la somme des intégrales de la fonction sur chaque $I_k$. Typiquement, il est ainsi possible de calculer l'intégrale de la fonction partie entière sur $[0,10]$ car elle est continue par morceaux sur cet intervalle.
\end{itemize}

\subsubsection{Propriétés}

\pro{$\intg{a}{a}{f(x)}=0$}

\preuve{Pour $a=b$, on a $S_n=\dfrac{b-a}{n}\somme{k=0}{n-1}f(x_k)=0$ d'où $\intg{a}{a}{f(x)}=\limc{n}{+\infty}0=0$.}

\pro{Si $f$ est constante à $\lambda\in\R_+$ sur $[a,b]$ alors $\intg{a}{b}{f(x)}=\lambda(b-a)$}

\preuve{Laissée au lecteur.}

\pro{\programme (relation de Chasles) Soit $c\in [a,b]$. Alors $\intg{a}{b}{f(x)}=\intg{a}{c}{f(x)}+\intg{c}{b}{f(x)}$}

\preuve{La preuve est purement géométrique : la droite $x=c$ partage la surface totale en deux parties, ces deux intégrales donnent donc deux portions d'aires dont la somme vaut $A$.}

\pro{\programme (linéarité de l'intégrale) Soit \fone{g}{I}{\R} positive et continue sur $[a,b]$ et $\lambda\in\R_+$. Alors $\intg{a}{b}{\parent{\lambda f(x)+g(x)}}=\lambda\intg{a}{b}{f(x)}+\intg{a}{b}{g(x)}$}

\preuve{La somme de Riemann sur $[a,b]$ de la fonction $x\mapsto \lambda f(x)+g(x)$ est $S_n=h\somme{k=0}{n-1}\parent{\lambda f(x_k)+g(x_k)}=h\cro{\lambda\somme{k=0}{n-1} f(x_k)+\somme{k=0}{n-1} g(x_k)}$. En passant à la limite, on a le résultat.}

\pro{Soit \fone{g}{I}{\R} positive et continue sur $[a,b]$. Si pour tout $x\in [a,b]$, $g(x)\le f(x)$, alors $\intg{a}{b}{g(x)}\le\intg{a}{b}{f(x)}$}

\preuve{La fonction $x\mapsto f(x)-g(x)$ est positive et continue sur $[a,b]$ donc on peut calculer son intégrale. Comme toute intégrale, $\intg{a}{b}{\parent{f(x)-g(x)}}\ge0$ d'où $\intg{a}{b}{f(x)}-\intg{a}{b}{g(x)}\ge0$ d'où le résultat.}

\subsubsection{Théorème fondamental de l'analyse}

\theo{\programme (Théorème Fondamental de l'Analyse) Soit $I=[a,b]$ un intervalle de $\R$ et \fone{f}{I}{\R} continue et positive sur $I$.

\begin{enumerate}[i)]
\item \fons{F}{I}{\R}{x}{\intg{a}{x}{f(x)}} est la primitive de $f$ sur $I$ qui s'annule en $a$.
\item $\intg{a}{b}{f(x)}=F(b)-F(a)$ où $F$ est une primitive de $f$ sur $I$.
\end{enumerate}}

\rem{\textit{J'ai les mots pour décrire à quel point ce résultat est beau et profondément utile à tous points de vue en analyse réelle, mais la marge est trop étroite.}}

\preuve{La preuve du premier point est exposée dans l'annexe \ref{tfa}. Montrons ici le point 2 en supposant le point 1. On sait que \fons{F_0}{I}{\R}{x}{\intg{a}{x}{f(x)}} est une primitive de $f$ sur $[a,b]$. Soit maintenant $F$ une primitive quelconque de $f$. Il existe donc $k\in\R$ tel que $F=F_0+k$. On a $F(b)-F(a)=\intg{a}{b}{f(x)}+k-\intg{a}{a}{f(x)}-k=\intg{a}{b}{f(x)}$ d'où le résultat.}

Ce théorème permet concrètement de calculer l'intégrale d'une fonction connaissant une de ses primitives. Certaines fonctions n'ont pas de primitive déterminable avec les fonctions usuelles. C'est le cas, par exemple, de la fonction \fons{f}{\R}{\R}{x}{e^{x^2}} (à connaître pour la culture). Puisqu'elle est continue sur $\R$ elle admet une primitive sur $\R$, mais avec les fonctions de bases celle-ci n'est pas déterminable. C'est ainsi qu'on a crée la \href{https://fr.wikipedia.org/wiki/Fonction_d\%27erreur}{\textit{fonction d'erreur}}, notée $\text{erf}$, qui est une primitive d'une fonction cousine de $f$. Les fonctions non calculables deviennent en fait simplement de nouvelles fonctions.

\nota{\programme On note $F(b)-F(a)$ d'une manière plus commode pour les calculs : $\cro{F(t)}_{t=a}^b$ ou encore, quand il n'y a pas d'ambiguité sur la variable, $\cro{F(t)}_{a}^b$.}

\ex{Calculer $\intg{1}{42}{\sin(x)}$. Soit $I=[1,42]$. La première étape est de déterminer une primitive de la fonction sur $I$. On sait que $-\cos$ en est une. Donc par le théorème fondamental de l'analyse (TFA), $\intg{1}{42}{\sin(x)}=\cro{-\cos(t)}_{1}^{42}=\cos(1)-\cos(42)$.}

Voici une proposition de modèle d'une rédaction qu'on pourrait attendre à retrouver dans une excellente copie de Terminale S :

\ex{Soit $I=\intg{2}{7}{\sqrt{x}}$. Justifier l'existence de $I$, trouver sa valeur exacte puis l'approximer à $10^{-2}$ près. Soit $J=[2,7]$ et \fons{f}{J}{\R}{x}{\sqrt{x}}. On sait que la fonction racine est continue et positive sur $\R_+$ donc sur $J$, par conséquent $\intg{2}{7}{f(x)}=I$ existe. Pour tout $x\in J$, $f(x)=x^{q}$ avec $q=\dfrac{1}{2}$ donc une primitive de $f$ sur $J$ est \fons{F}{J}{\R}{x}{\dfrac{x^{q+1}}{q+1}}, \cad que pour tout $x\in J$, $F(x)=\dfrac{x^{3/2}}{3/2}=\dfrac{2}{3}\sqrt{x}^3$. Ainsi par le théorème fondamental de l'analyse, $I=\cro{F(t)}_2^7=\dfrac{2}{3}\sqrt{7}^3-\dfrac{2}{3}\sqrt{2}^3=\dfrac{2}{3}\parent{\sqrt{7}^3-\sqrt{2}^3}$. Conclusion, \boxed{I=\dfrac{2}{3}\parent{\sqrt{7}^3-\sqrt{2}^3}\approx 10.46}.}


Le TFA et les propriétés précédentes (Chasles, linéarité...) peuvent à loisir être étendus aux cas suivants si besoin :

\begin{itemize}
\item Si la fonction est parfois négative sur $[a,b]$. Les propriétés et le théorème restent identiques, la seule chose qui change, comme nous l'avons déjà dit, c'est qu'au sens des sommes de Riemann, les parties d'aires situées sous l'axe des abscisses doivent être comptées négativement.

\item Si la fonction est continue par morceaux sur $[a,b]$ (voir précédemment dans le cours ce que cela signifie). De même, tout est conservé.

\item Si $b> a$. On se retrouve avec un intervalle «~inversé~», et dans ce cas tout est conservé en considérant que $\intg{a}{b}{f(x)}=-\intg{b}{a}{f(x)}$.
\end{itemize}


\section{Exponentielle et logarithme}

\subsection{Exponentielle}

\theo{\programme Il existe une unique fonction, notée $\exp$, définie et dérivable de $\mathbf{R}$ dans $\mathbf{R}$, telle que pour tout $x\in\mathbf{R},\exp'(x)=\exp(x)$ et $\exp(0)=1$.}

\begin{proof}
\textsc{\programme Unicité} \\
Commençons par montrer que $\exp$ ne s'annule pas. Soit $f:\R\rightarrow \R,x\mapsto \exp(x)\exp(-x)$. Soit $x\in\R$. $f$ est dérivable sur $\R$ et $f'(x)=0$. Donc $f$ est constante et $f(x)=f(0)=1$. Supposons par l'absurde qu'il existe $a\in\R$ tel que $\exp(a)=0$. Alors $f(a)=0$ ce qui est absurde. Donc $\exp$ ne s'annule pas. \\
Prouvons maintenant l'unicité. Soit $f:\mathbf{R}\rightarrow \mathbf{R}$ une fonction qui possède les mêmes propriétés que la fonction $\exp$. $f$ ne s'annulant pas, on peut définir $g:\R\rightarrow\R,x\mapsto \exp(x)/f(x)$. Soit $x\in\R$. $g$ est dérivable sur $\R$ et $g'(x)=0$. Donc $g$ est constante et $g(x)=g(0)=1$. Donc pour tout $x\in\R,f(x)=\exp(x)$ ce qui donne le résultat. \\
\\
\textsc{Existence} \\
La preuve de l'existence de $\exp$ n'est pas aisée. Nous en proposons dans l'annexe \ref{annexp} une très longue et astucieuse mais qui a le mérite de n'utiliser (presque) que des outils de lycée. 
\end{proof}

\pro{$\exp$ est une primitive de $\exp$ sur $\R$.}

\preuve{Immédiat.}

\pro{(inégalité de Bernoulli) Pour tout $n\in\N$, tout réel $\alpha> -1$, on a $(1+\alpha)^n\ge1+n\alpha$.\label{inbernoulli1}}
\preuve{Par simple récurrence.}
\rem{Cette inégalité, présente dans la preuve de l'existence de l'exponentielle, nous sera utile pour montrer certaines propriétés.}

\pro{Soit $x\in\R$. $\exp(x)=\Lim{n\to\infty}\left( 1+\dfrac{x}{n}\right)^n$\label{defexplim}}

\begin{proof}
La preuve est incluse dans la démonstration de l'existence de $\exp$, voir l'annexe \ref{annexp}.
\end{proof}

\rem{\begin{itemize}
\item Ce résultat, combiné avec l'inégalité de Bernoulli, permet comme nous le verrons de démontrer de manière commode les propriétés générales de l'exponentielle.
\item Cette définition par limite de suite est également une façon très commode de calculer une valeur d'exponentielle informatiquement si on ne dispose pas directement de cette fonction. Nous laissons au lecteur curieux le soin de reprogrammer la fonction exponentielle ainsi
\end{itemize}}

\pro{\programme (propriétés analytiques de $\exp$) :
\begin{enumerate}[(i)]
\item $\exp$ est continue sur $\R$.
\item $\exp$ est strictement positive sur $\R$.
\item $\exp$ est strictement croissante sur $\R$.
\item $\lim\limits_{x\to +\infty} \exp(x)=+\infty$.
\item $\lim\limits_{x\to -\infty} \exp(x)=0$.
\item $\exp$ est bijective de $\R$ dans $\R^*_+$ (attention : pas de $\R$ dans $\R$).
\end{enumerate}}

\preuve{\begin{enumerate}[(i)]
\item $\exp$ est dérivable sur $\R$ donc continue sur $\R$.
\item Nous avons déjà prouvé que $\exp$ ne s'annule pas sur $\R$ en démontrant l'unicité de $\exp$ et de plus $\exp$ est continue sur $\R$ donc par le théorème des valeurs intermédiaires, $\exp$ est soit strictement positive soit strictement négative sur $\R$. Or $\exp(0)=1$, conclusion $\exp$ est strictement positive sur $\R$.
\item Pour tout $x\in\R$, $\exp'(x)=\exp(x)>0$, donc $\exp$ est strictement croissante sur $\R$.
\item \programme Montrons que pour tout $x\in\R$, $\exp(x)\ge x+1$. Soit $x\in\R$. On a $\exp(x)=\Lim{n\to\infty}\left( 1+\dfrac{x}{n}\right)^n$. Posons $n_0\in\N^*$ tel que $n_0>x$ et soit $n\ge n_0$. Alors $\dfrac{x}{n}>-1$ donc en appliquant l'inégalité de Bernoulli, $\left( 1+\dfrac{x}{n}\right)^n\ge x+1$. En passant à la limite, $\Lim{n\to\infty}\left( 1+\dfrac{x}{n}\right)^n\ge x+1$ donc $\exp(x)\ge x+1$. Nous avons donc $\limc{x}{\infty}\exp(x)\ge\limc{x}{+\infty} x+1=+\infty$ et par conséquent $\limc{x}{+\infty}\exp(x)=+\infty$.
\item \programme Avec la proposition~\ref{defexplim} nous montrons facilement que pour tout $x\in\R$, $\exp(x)=\dfrac{1}{\exp(-x)}$. Donc $\limc{x}{-\infty}\exp(x)=\limc{x}{+\infty}\dfrac{1}{\exp(x)}=0$.
\item $\exp$ est continue sur $\R$ et d'après le théorème des valeurs intermédiaires, $\im\exp$ est un intervalle. Puisque $\exp$ est de plus strictement croissante, que $\limc{x}{-\infty}\exp(x)=0$ et que $\limc{x}{-\infty}\exp(x)=+\infty$, alors $\im(\exp)=]0;+\infty[$. Enfin, par le théorème de bijection monotone, $\exp$ est bijective de $\R$ dans $\R_+^*$.
\end{enumerate}}

\theo{\programme (règles de calcul) Soit $x,y\in\R$ et $k\in\Z$.

\begin{enumerate}
\item $\exp(x+y)=\exp(x)\exp(y)$
\item $\exp(-x)=\dfrac{1}{\exp(x)}$
\item $\exp(x-y)=\dfrac{\exp(x)}{\exp(y)}$
\item $\exp(x)^k=\exp(kx)$
\end{enumerate}}

\preuve{\begin{enumerate}[i)]
\item Pour tout $y\in\R$, posons la fonction \fons{f}{\R}{\R}{x}{\dfrac{\exp(x+y)}{\exp(x)}}. $f$ est dérivable sur $\R$ et pour tout $x\in\R$, $f'(x)=0$, donc $f$ est constante sur $\R$. Or $f(0)=\exp(y)$ donc pour tout $x,y\in\R$, $\dfrac{\exp(x+y)}{\exp(x)}=\exp(y)$.

\item D'une part $\exp(x-x)=\exp(x)\exp(-x)$ et d'autre part $\exp(x-x)=1$ d'où $\exp(x)\exp(-x)=1$.

\item Immédiat par les deux points précédents.

\item On montre le résultat pour $k\in\N$ par récurrence en utilisant le premier point, puis on étend à $k\in\Z$ en utilisant le second.
\end{enumerate}}

\nota{On note $e$ le réel $\exp(1)$. Il vaut $2.72$ au centième près.}

\rem{Cette constante fait partie des constantes fondamentales en mathématiques. Son utilité va transparaître dans le théorème suivant. Il est utile de connaître son approximation au centième. Enfin, à noter que $e$, comme $\pi$, est un nombre \textit{transcendant}, \cad qu'il n'existe aucun polynôme à coefficients entiers dont l'une racine est $e$. Tout transcendant est irrationnel, mais la réciproque n'est pas vrai. Par exemple le nombre d'or $\dfrac{1+\sqrt{5}}{2}$ est irrationnel mais pas transcendant puisqu'il est l'une des racines du polynôme $x^2-x-1$ comme chacun sait, lol.}


\theo{\programme Pour tout $x\in\R$, $\exp(x)=e^x$.}

\rems{\item C'est-à-dire que \textit{exponentielle de $x$ }égale \textit{le réel $e$ exposant $x$}.

\item Ce théorème est fondamental car il montre que l'exponentielle marche exactement comme les puissances. Cela n'a en fait rien d'un hasard : comme nous l'expliquerons, les puissances sont rigoureusement définis justement par l'exponentielle (et le logarithme que nous verrons ultérieurement). Donc en réalité, c'est plutôt les puissances qui marchent comme l'exponentielle. Ce n'est pas pour rien si les mots \textit{exposant} et \textit{exponentielle} ont la même origine.

\item Avec cette écriture, les règles de calcul deviennent totalement naturelles : $e^{x+y}=e^x e^y$, $(e^x)^n=e^{nx}$ etc.

\item On préfère très largement l'écriture $e^x$ au lieu de $\exp(x)$ pour des raisons évidentes de commodité. Mais attention de ne pas confondre : la fonction exponentielle continue à se noter $\exp$. $e$ est un réel et pas une fonction !  Hors de question par exemple d'écrire que «~$e$ est strictement croissante sur $\R$~» : tout le monde comprend mais ce n'est pas rigoureux, ça serait comme dire que 42 est croissante sur $\R$. A force d'utiliser cette écriture, on finit trop rapidement par l'oublier. De même, toujours garder en tête que les écritures $(e^x)'$ ou encore $e'^x$ sont des abus de notation, l'écriture correcte étant $\exp'(x)$.

\item Cette écriture nous rappelle largement tout ce qu'on a vu sur les complexes avec la notation $e^{i\theta}$. Là encore ce n'est pas un hasard : on peut montrer qu'on peut étendre la définition de l'exponentielle sur l'ensemble des complexes, et que toutes les propriétés sont alors conservées.}

\preuve{\begin{itemize}
\item Montrons le résultat pour $x\in\Z$. On a alors $\exp(x)=\exp(1\times x)=(\exp(1))^x=e^x$ d'après les règles de calcul.
\item Montrons le résultat pour $x\in\Q$. Il existe $p\in\Z$ et $n\in\N^*$ tel que $x=\dfrac{p}{n}$. Nous avons donc $\exp(x)=\exp\parent{\dfrac{p}{n}}=\cro{\exp\parent{\dfrac{1}{n}}}^p$ (règles de calcul). Or d'une part $\exp\parent{n\times\dfrac{1}{n}}=\cro{\exp\parent{\dfrac{1}{n}}}^{n}$ et d'autre part $\exp\parent{n\times\dfrac{1}{n}}=e$ d'où $\cro{\exp\parent{\dfrac{1}{n}}}^{n}=e$ d'où $\exp\parent{\dfrac{1}{n}}=e^{1/n}$. Finalement, $\exp(x)=\cro{\exp\parent{\dfrac{1}{n}}}^p=e^{p/n}=e^x$.
\item Pour $x\in\R\sauf{\Q}$, \cad pour $x$ irrationnel, c'est une \textit{définition}. C'est-à-dire qu'on \textit{pose} $e^x=\exp(x)$. Cela peut sembler complètement artificiel, mais les puissances irrationnelles \textit{sont définies} par l'exponentielle (et le logarithme). 
\end{itemize}
Pour résumer : si $x$ est un entier, le résultat se montre par ce qu'on connait des puissances entières qui sont des multiplications par récurrence; si $x$ est rationnel, le résultat se montre par ce qu'on connait des racines $n$-ièmes; enfin pour $x$ irrationnel, c'est une définition.}


\lemme{\programme (croissance comparée) Soit $n\in\N$.

\begin{enumerate}[i)]
\item $\limc{x}{+\infty}\dfrac{e^x}{x^n}=+\infty$
\item $\limc{x}{-\infty}x^ne^x=0$

\end{enumerate}}
\preuve{Sous forme d'exercice : soit \fons{f}{\R}{\R}{x}{e^x-\dfrac{x^2}{2}.}

\begin{enumerate}
\item En étudiant les dérivées première et seconde de $f$, montrer que $f(x)>0$ pour tout $x>0$.
\item En déduire la première limite pour $n=1$.
\item En posant $y=-x$, montrer que la seconde limite pour $n=1$, si elle existe, égale $\limc{y}{+\infty}-\dfrac{1}{e^y/y}$ puis conclure pour $n=1$.
\item En remarquant que $\dfrac{e^x}{x^n}=\parent{\dfrac{1}{n}\dfrac{e^{x/n}}{x/n}}^n$ pour tout $x\in\R$, en déduire la première limite pour $n\in\N$.
\item Enfin, toujours en posant $y=-x$, montrer que si elle existe, $\limc{x}{-\infty}x^ne^x=\limc{y}{+\infty}(-y)^ne^{-y}$. Puis en distinguant $n$ pair et impair, conclure.
\end{enumerate}}

\theo{($\exp$ bat tout le monde) Soit $R$ une fonction rationnelle (\cad un quotient de polynômes de degrés quelconques), $a,b$ les coefficients de plus haut degré respectivement du numérateur et du dénominateur et $s$ le signe de $\dfrac{a}{b}$.

\begin{enumerate}[i)]
\item $\limc{x}{+\infty}R(x)e^x=s\infty$
\item $\limc{x}{-\infty}R(x)e^x=0$
\end{enumerate}}
\rem{Au lycée, c'est la proposition précédente (croissance comparée) qui est enseignée, mais à mon sens ce qui est vraiment utile et le plus facile à mémoriser, c'est ce théorème.}

\preuve{Soit $n,m$ les degrés respectifs du numérateur et du dénominateur.
\begin{enumerate}[i)]
\item $R$ étant une fonction rationnelle, on sait que $\limc{x}{+\infty}R(x)=\dfrac{a}{b}\limc{x}{+\infty}x^{n-m}$. Si $n> m$ le résultat est immédiat, sinon on applique la première limite du lemme précédent.
\item De même, $\limc{x}{-\infty}R(x)=\dfrac{a}{b}\limc{x}{-\infty}x^{n-m}$. Si $n< m$ le résultat est immédiat, sinon on applique la seconde limite du lemme précédent. 
\end{enumerate}}


\pro{$\limc{x}{0} \dfrac{e^x-1}{x}=1$}
\preuve{C'est la dérivée de $\exp$ en 0.}

\subsection{Logarithme népérien}

\defi{\programme (logarithme népérien) La fonction \fone{\exp}{\R}{]0,+\infty[} étant bijective, elle admet une unique fonction réciproque \fone{\exp^{-1}}{]0,+\infty[}{\R}. On l'appelle \textit{logarithme népérien} et on la note $\ln$.}

Nous conseillons au lecteur de relire le théorème \ref{frec} ainsi que les propositions qui suivent pour se remémorer la définition et les propriétés des fonctions réciproques.

\pro{\programme \begin{itemize}
\item Pour tout $x\in\R$, $\ln(e^x)=x$
\item Pour tout $x\in\R^*_+$, $e^{\ln(x)}=x$
\end{itemize}}

\preuve{Par définition de ce qu'est une fonction réciproque.}

\theo{\programme (propriétés analytiques de $\ln$) 

\begin{enumerate}[i)]
\item $\ln$ est continue sur $\R^*_+$
\item $\ln$ est dérivable sur $\R^*_+$ et pour tout $x\in\R^*_+$, $\ln'(x)=\dfrac{1}{x}$.
\item $\ln$ est strictement croissante sur $\R^*_+$
\item $\limc{x}{0}\ln(x)=-\infty$
\item $\limc{x}{+\infty}\ln(x)=+\infty$
\item \begin{enumerate}
\item $\ln(1)=0$
\item Si $x\in]0,1[$ alors $\ln(x)<0$
\item Si $x\in ]1,+\infty[$ alors $\ln(x)>0$
\end{enumerate}



\end{enumerate}}
\rem{En effet, ce n'est pas parce que la dérivée de l'exponentielle égale elle-même qu'il en est de même pour sa réciproque.}

\preuve{\begin{enumerate}[i)]
\item Soit $a\in\R^*_+$. Il existe un unique $b\in\R$ tel que $a=e^b$ (par bijectivité de $\exp$). D'où $\limc{x}{a}{\ln(x)}=\limc{x}{b}\ln(e^x)=\limc{x}{b}x=b$, donc $\ln$ admet une limite finie en tout point $a\in\R^*_+$ donc $\ln$ est continue sur $\R^*_+$.
\item Soit $a\in\R^*_+$. Soit $x\in\R^*_+\sauf{a}$. Alors $\tau_{\ln,a}(x)=\dfrac{\ln x-\ln a}{x-a}$. Posons $y=\ln x$ et $b=\ln a$. Alors $\tau_{\ln,a}(x)=\dfrac{y-b}{e^y-e^b}$. Donc $\limc{x}{a}\tau_{\ln,a}(x)=\limc{y}{b}\dfrac{y-b}{e^y-e^b}=\limc{y}{b}\dfrac{1}{\tau_{\exp,b}(y)}=\dfrac{1}{\exp'(b)}=\dfrac{1}{a}$. Conclusion $\ln$ est dérivable sur $\R^*_+$ et pour tout $x\in\R^*_+$, $\ln'(x)=\dfrac{1}{x}$.
\item Pour tout $x\in\Rep$, $\ln'(x)=\dfrac{1}{x}>0$, d'où $\ln$ est strictement croissante sur $\Rep$.
\item $\limc{x}{0}\ln(x)=\limc{x}{-\infty}\ln e^x=\limc{x}{-\infty}x=-\infty$.
\item $\limc{x}{+\infty}\ln(x)=\limc{x}{+\infty}\ln e^x=\limc{x}{+\infty}x=+\infty$.
\item \begin{enumerate}
\item $\ln(1)=\ln e^0=0$
\item $\ln$ étant continue, strictement croissante sur $\Rep$, et sachant que $\ln(1)=0$, alors par le théorème des valeurs intermédiaires, $\ln x <0$ pour tout $x\in]0,1[$.
\item Idem.
\end{enumerate}
\end{enumerate}}

\pro{La fonction \fons{F}{\Rep}{\R}{x}{x\ln x - x} est une primitive de $\ln$ sur $\Rep$.}

\preuve{Il suffit de dériver et de constater qu'on retrouve la fonction logarithme.}

\rem{Pour retrouver cette primitive par le calcul, il faut utiliser la technique de l'intégration par parties, qui n'est pas au programme de lycée.}

\theo{\programme (règles de calcul) Soit $x,y\in\Rep$ et $n\in\N$.

\begin{enumerate}[i)]
\item $\ln (xy)=\ln x +\ln y$
\item $\ln \parent{\dfrac{x}{y}}=\ln x -\ln y $
\item $\ln(x^n)=n\ln x$
\end{enumerate}}

\preuve{ Il existe $a,b\in\R$ tel que $e^a=x$ et $e^b=y$.

\begin{enumerate}[i)]
\item $\ln(xy)=\ln(e^a e^b)=\ln(e^{a+b})=a+b=\ln x + \ln y$
\item $\ln \parent{\dfrac{x}{y}}=\ln \parent{\dfrac{e^a}{e^b}}=\ln(e^{a-b})=a-b=\ln x - \ln y$
\item C'est le premier point par récurrence.
\end{enumerate}}

\rem{Le logarithme étant la réciproque d'exponentielle, il est logique que les règles de calcul soient inversés.}

\exo{Montrer que le point 3 s'étend pour $n\in\Q$, \cad que pour tout $n\in\Q$, $x\in\Rep$, $\ln(x^n)=n\ln x$.}

\theo{\programme (le logarithme est battu par tout le monde) Soit $R$ une fonction rationnelle.

\begin{itemize}
\item Si $\limc{x}{+\infty}R(x)=0$ alors $\limc{x}{+\infty}R(x)\ln x=0$
\item Si $\limc{x}{0}R(x)=0$ alors $\limc{x}{0}R(x)\ln x = 0$
\end{itemize}}

C'est une simple conséquence du fait que l'exponentielle bat tout le monde. En résumé l'exponentielle va plus vite que tout le monde, et le logarithme va plus lentement que tout le monde. En effet, pour mesurer à quel point il croît lentement, $\ln(10^{80})\approx 184$ ($10^{80}$ est une estimation du nombre d'atomes dans l'univers).

\subsection{Exponentielle et logarithme en base quelconque}

Nous avons vu que l'exponentielle permettait de calculer les puissances de la forme $e^x$ avec $x\in\R$. L'idée est de généraliser afin de pouvoir calculer des puissances de la forme $a^x$ avec $a$ positif.

Dans toute la suite, on pose $a\in\Rep$.

\defi{(exponentielle en base quelconque) On appelle \textit{exponenitelle de base $a$} la fonction \fons{\exp_a}{\R}{\Rep}{x}{e^{x\ln a}}.}

\nota{Pour tout $x\in\Rep$, on note $a^x$ pour $e^{x\ln a}$.}

\rems{\item Nous avons ainsi étendu la notation puissance. Il y a donc à présent un sens à écrire par exemple $\pi^{\sqrt{3}}$ : il s'agit simplement du réel $\exp_\pi(\sqrt{3})$.
\item Ainsi, la fonction $\exp$ est juste la fonction $\exp_e$.
\item A noter que $\exp_1$ est constante à 1 sur $\R$.}

\pro{\begin{enumerate}[i)]
\item $\exp_a$ est strictement positive et dérivable sur $\R$ et pour tout $x\in\R$, $\exp_a'(x)=\ln (a) a^x$.
\item $\exp_a$ est bijective.
\item $a^0=1$
\item Les règles de calculs de l'exponentielle en base $a$ sont les mêmes que l'exponentielle.
\end{enumerate}}

\rem{\textbf{Attention}, les propriétés analytiques ne sont pas nécessairement conservées : remarquer par exemple que $\exp_{1/2}$ est décroissante et que ses limites ne correspondent pas à celles de $\exp$.}

\exo{Étudier les proporiétés analytiques de $\exp_a$ (croissance, limites) en fonction des valeurs de $a$.}

\defi{(logarithme de base $a$) On appelle \textit{logarithme de base $a$} la réciproque de \fone{\exp_a}{\R}{\Rep} et on la note \fone{\log_a}{\Rep}{\R}.}

\rems{\item Ainsi, $\ln$ n'est rien d'autre que $\log_e$.
\item \textbf{Attention} : si on écrit $\log$ sans indication de base, cela peut, selon les contextes, désigner soit $\ln$, soit $\log_{10}$, soit même $\log_2$ chez les informaticiens. C'est pourquoi, de votre côté, il est fortement conseillé de toujours préciser la base du logarithme que vous utilisez. Mais si jamais au cours d'un problème vous utilisez intensivement le logarithme décimal par exemple, rien ne vous empêche d'écrire au début de la composition «~nous noterons $\log$ la fonction $\log_{10}$~» par commodité. Mais a priori, vous ne rencontrerez jamais la notation $\log$ pour une base autre que $e$, 10 ou 2.}

\pro{Pour tout $x\in\Rep$, $\log_a(x)=\dfrac{\ln x}{\ln a}$.}

\preuve{En exercice. \spoiler{Poser \fons{f}{\Rep}{\R}{x}{\ln x/\ln a}, calculer $f\circ\exp_a$ et $\exp_a\circ f$, constater que cela fait l'identité pour les deux. On en déduit donc que $f$ est la réciproque de $\exp_a$ or la réciproque étant unique, on a l'égalité cherchée.}}

\pro{\begin{enumerate}[i)]
\item $\log_a$ est dérivable sur $\Rep$ et pour tout $x\in\Rep$, $\log_a'(x)=\dfrac{1}{x\ln a}$.
\item $\log_a(1)=0$
\item Pour tout $x\in\R$, $\log_a(a^x)=x$
\item Les règles de calcul de $\log_a$ sont les mêmes que celles de $\ln$.
\end{enumerate}}

\rem{De même que pour $\exp_a$, les propriétés analytiques de $\log_a$ ne sont pas forcément conservées.}

\subsection{Applications}

\subsubsection{Les logarithmes usuels}

Les trois logarithmes de loin les plus utilisés en sciences sont $\ln=\log_e$ car c'est la réciproque de l'exponentielle, $\log_{10}$ (qu'on appelle aussi \textit{logarithme décimal}) car très pratique pour les calculs (nous utilisons la base 10 pour écrire les nombres), enfin $\log_2$ principalement en informatique où la base 2 est omniprésente. 

Avant l'invention des calculatrices et des ordinateurs, on utilisait des tables de logarithmes qui fournissaient des approximations des logarithmes décimaux (base 10), ce qui permettait aux scientifiques de faire des approximations de calculs compliqués.

\ex{Cherchons une approximation de $10.879\times 5238.46$. L'idée est de calculer le logarithme décimal de ce produit en exploitant les propriétés de cette fonction.

\chaine{\log_{10}(10.879\times 5238.46) &=& \log_{10}(1.0879\times 10 \times 5.23846\times 10^3) \\
&=& \log_{10}(1.0879\times5.23846\times 10^4) \\
&=& \log_{10}(1.0879)+\log_{10}(5.23846)+\log_{10}(10^4) \\
&=&  \log_{10}(1.0879)+\log_{10}(5.23846)+4}

Les tables de logarithmes fournissaient une approximation de ces deux logarithmes : $\log_{10}(1.0879)\approx 0.03658$ et $\log_{10}(5.23846)\approx 0.7192$. Et ainsi :

\chaine{\log_{10}(10.879\times 5238.46) &\approx& 0.03658+0.7192+4 \\
&=& 4.75578}

Enfin, les tables permettaient de trouver le réel dont le logarithme décimal vaut $4.75578$, \cad qu'elles fournissaient $10^{4.75578}\approx 56988$. Conclusion : \fbox{$10.879\times 5238.46\approx 56988$} à comparer avec la réponse exacte : $56989.20634$.}
On voit à travers cet exemple un peu simpliste que les anciens scientifiques n'avaient à la main qu'à faire des additions, ce concept a révolutionné le calcul (on pouvait faire des calculs beaucoup plus compliqués et avec moins d'erreurs). Par ailleurs les tables fournissaient également des tables de fonctions trigonométriques.

\exo{En théorie musicale, l'échelle tempérée est caractérisée par deux axiomes :

\begin{enumerate}
\item La suite des fréquences des notes est géométrique, autrement dit, si $f$ est la fréquence d'une note et $f'$ celle de la note immédiatement précédente alors $\dfrac{f}{f'}$ est constante.
\item La fréquence d'une note donnée est multipliée par deux par rapport à celle de la même note de l'octave précédente et il y a 12 notes dans une octave.

\end{enumerate}

 On se place dans une des conventions où la note la3 a pour fréquence 440Hz. Les calculs seront arrondis au dixième de Hertz près. Les lecteurs intéressés peuvent aller sur \href{https://www.szynalski.com/tone-generator/}{ce site} pour entendre les sons associés aux fréquences calculées.
 
\begin{enumerate}
\item Calculer la fréquence du la-1 (4 octaves en-dessous du la3) : c'est l'une des plus basses notes que les instruments conventionnels puissent produire. \spoiler{27.5Hz}
\item On note \suite{f} la suite des fréquences des notes tel que $f_0$ est la fréquence du la-1. Comme nous l'avons expliqué, $(f_n)$ est géométrique : calculer sa raison $q$ et son terme général. \spoiler{Raison : $q=2^{1/12}$. Terme général : pour tout $n\in\N$, $f_n=f_0 q^n=27.5\times 2^{n/12}$}
\item Sans calcul, que vaut $f_{12\times4}$ ? \spoiler{440Hz}
\item \begin{enumerate}
\item Pour les non musiciens : calculer $f_{93}$ \spoiler{$f_{93}\approx5919.9$Hz}
\item Pour les musiciens : calculer la fréquence du do\#5. \spoiler{$f_{12\times5+4}=f_{64}\approx1108.7$Hz}
\end{enumerate}
\item Trouver le plus proche $n\in\N$ tel que $f_n\approx329.6$. Pour les musiciens : à quelle note la plus proche cette fréquence correspond-t-elle ? \spoiler{$n=12\log_2\parent{329.6/27.5}\approx 43$. C'est un mi3.}
\item Définir une fonction \fone{g}{[f_0,+\infty[}{\N} qui a toute fréquence $F\ge f_0$ associe le plus proche $n\in\N$ tel que $f_n\approx F$ (on supposera l'existence d'une fonction \fone{\text{arr}}{\R}{\Z} qui arrondi tout réel à l'entier le plus proche). Vérifier que $g(329.6)$ donne bien la réponse à la question précédente. \spoiler{Pour tout $F\ge f_0$, $g(F)=\text{arr}\cro{12\log_2\parent{F/27.5}}$}

\end{enumerate}}

\subsubsection{Échelles logarithmiques}

Parce que certaines fonctions croissent très rapidement ou très lentement, typiquement les fonctions exponentielles et logarithmes, les représentations graphiques usuelles ne sont pas les plus adaptées. C'est pourquoi il existe ce qu'on appelle les \textit{échelles logarithmiques}. L'idée est la suivante : dans les représentations classiques la graduation est linéaire, \cad que si les graduations 2 et 3 sont séparées de 1cm alors les graduations 3 et 4 sont aussi séparées de 1cm. Une \textit{échelle logarithmique de base $a\in\R^*_+$} ne rend plus les graduations linéaires, mais \textit{géométrique de raison $a$}. C'est-à-dire que si les graduations $a^2$ et $a^3$ sont séparées de 1cm alors les graduations $a^3$ et $a^4$ sont aussi séparées de 1cm. Formellement :

\defi{Une \textit{échelle logarithmique} est telle que les graduations suivent une progression géométrique de raison $a$.}

\ex{Construisons une échelle logarithmique de base 10 (la plus courante) dont on prend pour convention : la distance entre 1 et 10 est de 3cm. Dans la suite, on note $\log=\log_{10}$

\begin{itemize}
\item On peut commencer par placer les puissances de 10 : $10$ est à 3cm de 1, $100$ est à 3cm de $10$ et ainsi de suite.

\includegraphics[scale=0.5]{figures/pdf/echlog1-eps-converted-to.pdf}

\item Ce sont nos graduations principales. Disons qu'on veut diviser les espaces entre deux graduations principales par 10 valeurs. Ainsi on veut placer par exemple les graduations 1,2,...,9 afin de diviser par 10 l'espace entre 1 et 10. Il nous faut donc une fonction \fone{g}{\R^*_+}{\R} qui convertit tout réel en sa distance à la graduation 10. Ainsi on doit avoir $g(1)=0$cm, $g(10)=3$cm et ainsi de suite. Comme on sait qu'il y a progression géométrique, $g$ consiste en fait à annuler la puissance de 10. Ainsi $g(x)$ sera de la forme $\log(x)$. Bien sûr ce n'est pas encore fini : $\log(1)=0$ et $\log(10)=1$, il faut corriger. \textit{Les graduations sont linéaires par rapport au logarithme des valeurs} donc il existe $a\in\R$ tel que $g(x)=a\log(x)$. Donc \fbox{$g(x)=3\log(x)$} pour tout $x\in\R^*_+$. Ainsi $g(2)\approx0.9$cm, on peut le placer, ainsi que toutes les autres valeurs intermédiaires.

\includegraphics[scale=0.5]{figures/pdf/echlog2-eps-converted-to.pdf}

Nous avons terminé de construire notre échelle logarithmique.

\end{itemize}

}

Bien sûr dans toute représentation graphique plane il y a deux axes. Quand on utilise une échelle classique pour les deux axes, en général ils se coupent au point $(0,0)$. Mais si l'un des deux utilise une échelle logarithmique, par exemple l'axe des abscisses, il n'est pas possible que le second axe ait pour équation $x=0$ : en effet, sur une échelle logarithmique il est impossible de graduer le 0. Il faut donc le faire couper en un autre point, typiquement une graduation déjà tracée sur l'échelle logarithmique. Pour reprendre notre exemple précédent, l'axe des ordonnées pourrait être $x=1$ (classique).

\defi{(repère semi-logarithmique) Un repère \textit{semi-logarithmique} est tel que l'un des axes utilise une échelle logarithmique et l'autre une échelle linéaire.}

\defi{(repère $\log-\log$) Un repère \textit{$\log-\log$} est tel que les deux axes utilisent une échelle logartihmique.}

En résumé :

\begin{itemize}
\item L'échelle logarithmique est utile pour représenter efficacement des fonctions à croissance très rapide ou très lente.
\item Pour les fonctions à croissance rapide type exponentielle : le repère semi-logarithmique pour les ordonnées est approprié (compression des images). Dans un tel repère, toute fonction exponentielle (peu importe la base) devient une droite.
\item Pour les fonctions à croissance lente type logarithme : le repère semi-logarithmique pour les abscisses est approprié (compression des antécédents). Dans un tel repère, toute fonction logarithme (peu importe la base) devient une droite.
\end{itemize}

\exo{Dans un logiciel quelconque (spoiler : GeoGebra n'a pas cette option), représenter les 20 premiers termes de la suite \suite{f} de l'exercice de théorie musical dans un repère semi-logarithmique pour l'axe des ordonnées. Le résultat est-il étonnant ? Représenter de la façon la plus commode possible la fonction $g$ du même exercice.}

\exo{\href{https://www.ilemaths.net/img/forum_img/0416/forum_416807_1.JPG}{Voici la courbe représentative d'une certaine fonction}. Proposer une définition de cette dernière.}

\exo{Donner une fonction qu'il serait pertinent de représenter dans un repère log-log.}

\begin{appendix}

\chapter{Preuve du théorème fondamental de l'analyse}
\label{tfa}

Rappelons l'énoncé du théorème :

\theo{(Théorème Fondamental de l'Analyse) Soit $I=[a,b]$ un intervalle de $\R$ et \fone{f}{I}{\R} continue et positive sur $I$.

\begin{enumerate}[i)]
\item \fons{F}{I}{\R}{x}{\intg{a}{x}{f(x)}} est la primitive de $f$ sur $I$ qui s'annule en $a$.
\item $\intg{a}{b}{f(x)}=F(b)-F(a)$ où $F$ est une primitive de $f$ sur $I$.
\end{enumerate}}

Il s'agit ici de montrer le premier point, nous avons vu dans le cours que le second n'est en fait qu'un corollaire. Soit donc $I=[a,b]$ un intervalle de $\R$ et \fone{f}{I}{\R} continue et positive sur $I$.

\section{Définition de la fonction}

Montrons que \fons{F}{I}{\R}{x}{\intg{a}{x}{f(x)}} est une fonction bien définie. Pour tout $x\in I$, $a\le x\le b$ donc $\intg{a}{x}{f(x)}$ est bien définie.

\section{\texorpdfstring{$F$ est une primitive de $f$}{F est une primitive de f}}

Nous souhaitons montrer que $F$ est une primitive de $f$, \cad d'une part que $F$ est dérivable et d'autre part que $F'=f$. Que signifie que $F$ est dérivable sur $I$ ? Soit $c\in I$. Écrivons donc le taux d'accroissement de $F$ en $c$ : il s'agit de la fonction \fons{\tau_{f,c}}{I\sauf{c}}{\R}{x}{\dfrac{F(x)-F(c)}{x-c}}. $F$ est dérivable sur $I$ signifie que $\tau_{f,c}$ admet une limite finie en $c$ et que cette limite vaut $F'(c)$, \cad, en écrivant avec une définition epsilon-delta~:

\[ \forall\epsilon>0,\exists\delta>0,\forall x\in I\sauf{c}, |x-c|< \delta\Rightarrow \left|\dfrac{F(x)-F(c)}{x-c}-F'(c)\right|< \epsilon \]

Et puisqu'en plus, on veut montrer que $F'=f$, alors on veut montrer :


\[ \forall\epsilon>0,\exists\delta>0,\forall x\in I\sauf{c}, |x-c|<\delta\Rightarrow \left|\dfrac{F(x)-F(c)}{x-c}-f(c)\right|<\epsilon \text{     }(\star)\] 

Ainsi, si on montre $(\star)$, on montre que $F$ est une primitive de $f$.

Soit $\epsilon>0$. $f$ est continue sur $I$ donc sur $c$, donc il existe $\delta>0$ tel que pour tout $x\in I$, si $|x-c|<\delta$ alors $|f(x)-f(c)|<\epsilon$. Posons un tel $\delta$ et soit $x\in I\sauf{c}$ tel que $|x-c|<\delta$. Alors $|f(x)-f(c)|<\epsilon$. D'autre part :

\chaine{\left|\dfrac{F(x)-F(c)}{x-c}-f(c)\right| &=& \left|\dfrac{\intg{a}{x}{f(x)}-\intg{a}{c}{f(x)}}{x-c}-f(c)\right|}

Supposons que $c\in [a,x]$. Alors par la relation de Chasles :

\chaine{\left|\dfrac{\intg{a}{x}{f(x)}-\intg{a}{c}{f(x)}}{x-c}-f(c)\right| &=& \left|\dfrac{\intg{a}{c}{f(x)}+\intg{c}{x}{f(x)}-\intg{a}{c}{f(x)}}{x-c}-f(c)\right| \\

&=& \left|\dfrac{\intg{c}{x}{f(x)}}{x-c}-f(c)\right|}

Puisque $\intg{c}{x}{f(c)}=(x-c)f(c)$, alors :

\chaine{\left|\dfrac{\intg{c}{x}{f(x)}}{x-c}-f(c)\right| &=& \left|\dfrac{\intg{c}{x}{f(x)}}{x-c}-\dfrac{\intg{c}{x}{f(c)}}{x-c}\right| \\

&=& \left|\dfrac{\intg{c}{x}{\parent{f(x)-f(c)}}}{x-c}\right|}

A priori il y a un problème : le cadre de la théorie du cours ne nous autorise à utiliser la linéarité de l'intrégale que si $\lambda\ge0$ or ici $\lambda=-1$. En fait cela n'est pas grave, nous avons déjà fait remarqué dans le cours qu'on pouvait tout à fait étendre la définition et les propriétés de l'intégrale aux fonctions parfois négatives sur les bornes d'intégration, et donc l'extension marche très bien lorsque $\lambda<0$.

Or $|f(x)-f(c)|<\epsilon$, donc $f(x)-f(c)<\epsilon$, donc $\intg{c}{x}{\parent{f(x)-f(c)}}< \intg{c}{x}{\epsilon}$. Et ainsi :

\chaine{\left|\dfrac{\intg{c}{x}{\parent{f(x)-f(c)}}}{x-c}\right| &<& \left|\dfrac{\intg{c}{x}{\epsilon}}{x-c}\right| \\

&=& \left|\dfrac{(x-c)\epsilon}{x-c}\right| \\

&=& \epsilon}

Nous avons prouvé que si $c\in [a,x]$, alors $\left|\dfrac{F(x)-F(c)}{x-c}-f(c)\right|<\epsilon$. Si maintenant $c\not\in [a,x]$, alors on reprend la démonstration à la relation de Chasles en écrivant $\intg{a}{x}{f(x)}=\intg{a}{c}{f(x)}-\intg{x}{c}{f(x)}$ et grâce aux valeurs absolues qui règle le problème de signe, le reste est similaire. Ainsi, nous avons prouvé que pour tout $c\in I$,  $\left|\dfrac{F(x)-F(c)}{x-c}-f(c)\right|<\epsilon$. Nous avons donc montré $(\star)$, conclusion $F$ est bien une primitive de $f$.



\section{\texorpdfstring{Annulation en $a$}{Annulation en a}}

C'est immédiat : $F(a)=\intg{a}{a}{f(x)}=0$. \\

Nous avons prouvé le théorème fondamental de l'analyse. 
\begin{flushright}$\square$\end{flushright}
\chapter{Existence de la fonction exponentielle}
\label{annexp}

Nous présentons ici une preuve de l'existence de cette fonction. Elle est très longue et astucieuse mais accessible à un solide élève de TS. De plus, tout lecteur qui aura vue cette preuve sera largement récompensé : il disposera d'une définition de l'exponentielle qui permet facilement de démontrer tout un tas de propriétés sur elle. Tout n'est pas développé dans cette preuve, ce qui n'est pas montré sont des résultats faciles mais un peu lourds : ils sont laissés en exercice au lecteur. Cette preuve est issue essentiellement inspirée de \href{http://tsmaths.free.fr/Prepa/existenceexpo.pdf}{cette preuve}, revue et complétée.

\section{Stratégie}
Pour tout $x\in\R$ nous définissons les suites de réels, dépendant de $x$, $(u_n(x))_{n\ge\eta}$ et $(v_n(x))_{n\ge\eta}$  avec $\eta\in\N$ tel que $\eta>|x|$ (que nous noterons dorénavant simplement $(u_n)$ et $(v_n)$ pour alléger, en oubliant jamais la dépendance à $x$) définies pour tout $n\ge\eta$ par $u_n(x)=\left(1+\dfrac{x}{n}\right)^n$ et $v_n(x)=\left(1-\dfrac{x}{n}\right)^{-n}$. Nous faisons commencer la suite à partir de $\eta$, car autrement la suite $(v_n)$ pourrait ne pas être définie, exemple : $v_3(3)$ n'est pas défini. En posant ainsi $\eta$, il n'y plus de problème de définition. \\
Nous commencerons par montrer que $(u_n)$ et $(v_n)$ sont adjacentes (voir la définition \ref{adj}) et qu'elles admettent donc une limite commune dépendant de $x$, nous construirons ainsi $\exp$ comme la limite de $(u_n)$. Nous montrerons ensuite que $\exp$ ainsi définie satisfait les propriétés voulues (à savoir qu'elle est égale à sa dérivée et que $\exp(0)=1$), ce qui prouvera bien l'existence d'une telle fonction. \\
Sauf mention contraire, dans tout ce qui suit $x$ est un réel quelconque et $n$ un entier naturel non nul.

\section{\texorpdfstring{Adjacence de $(u_n)$ et $(v_n)$}{Adjacence de (u\_n) et (v\_n)}}
Montrons que $(u_n)$ et $(v_n)$ sont adjacentes. Pour rappel il y a deux choses à montrer~:
\begin{enumerate}
\item l'une de ces suites est croissante à partir d'un certain rang, l'autre décroissante à partir d'un certain rang,
\item $\lim\limits_{n\to +\infty} v_n(x)-u_n(x)=0$.
\end{enumerate}

\subsection{\texorpdfstring{Croissance de $(u_n)$}{Croissance de (u\_n)}}
C'est le premier gros morceau de la preuve, le second étant l'établissement de l'égalité de $\exp$ avec sa dérivée. \\
\\
Montrons que $(u_n)$ est croissante à partir du rang $\eta$, c'est-à-dire que pour tout $n\ge \eta$, $u_{n+1}(x)\ge u_n(x)$. \\
\\
Nous allons plusieurs fois utiliser le résultat suivant :

\pro{(inégalité de Bernoulli) Pour tout $n\in\N$, tout réel $\alpha> -1$, on a $(1+\alpha)^n\ge1+n\alpha$.\label{inbernoulli}}
Cette inégalité se montre par un raisonnement par récurrence facile. Remarque : il n'y a pas qu'une version de cette inégalité.

Soit $n\ge \eta$ et posons $f_n(x)=1+\dfrac{x}{n}$ et $g_n(x)=1-\dfrac{x}{n(n+1)\left(1+\dfrac{x}{n}\right)}$. L'existence de $f_n(x)$ est justifié par la non nullité de $n$. Prouvons l'existence de $g_n(x)$. On a $n\ge \eta>-x$ et $n>-x\imp n+x>0\imp \dfrac{n+x}{n}>0\imp 1+\dfrac{x}{n}>0$. Donc $1+\dfrac{x}{n}>0$. De plus $n(n+1)>0$. Donc $n(n+1)\left(1+\dfrac{x}{n}\right)>0$. Donc le dénominateur ne s'annule pas, ce qui prouve l'existence de $g_n(x)$. \\
\\
Nous avons :
\begin{equation}
\forall n\ge \eta, 1+\dfrac{x}{n+1}=f_n(x)g_n(x)
\label{fg}
\end{equation}
Cela se montre par exemple en calculant la différence des deux membres et en constatant qu'elle est nulle.\\
\\
Pour tout $n\ge \eta$ on a $u_{n+1}(x)=\left(1+\dfrac{x}{n+1}\right)^{n+1}$ donc par~\eqref{fg} nous avons~:
\begin{equation}
\forall n\ge \eta,u_{n+1}(x)=f_n(x)^{n+1}g_n(x)^{n+1}.
\label{defun}
\end{equation}
Montrons que pour tout $n\ge \eta$ on a $g_n(x)>0$. Soit $n\ge \eta$. \\
$g_n(x)>0\eqv \dfrac{x}{n(n+1)\left(1+\dfrac{x}{n}\right)}<1\eqv n^2+n+nx>0\eqv n+x+1>0$. \\
Or $n>-x\imp n+x>0\imp n+x+1>1>0$ donc on a bien $n+x+1>0$. Finalement on a bien $g_n(x)>0$ pour tout $n\ge \eta$. \\
\\
Soit $n\ge \eta$. Posons $\alpha_n(x)=-\dfrac{x}{n(n+1)\left(1+\dfrac{x}{n}\right)}$. On a $g_n(x)^{n+1}=(1+\alpha_n(x))^{n+1}$. \\
De plus $g_n(x)>0\eqv \alpha_n(x)>-1$ donc $\alpha_n(x)>-1$. \\
\\
Nous pouvons ainsi appliquer l'inégalité de Bernoulli au rang $n+1$ et ainsi $(1+\alpha_n(x))^{n+1}\ge 1+(n+1)\alpha_n(x)$ et donc :
\begin{equation}
\forall n\ge \eta, g_n(x)^{n+1}\ge 1+(n+1)\alpha_n(x)
\label{gp}
\end{equation}
De plus nous avons :
\begin{equation}
\forall n\ge \eta, 1+(n+1)\alpha_n(x)=\dfrac{n}{n+x}
\label{egalpha}
\end{equation}
Cela se montre facilement par calcul direct. \\
\\
Donc par~\eqref{gp} et~\eqref{egalpha}, nous avons :
\begin{equation}
\forall n\ge \eta, g_n(x)^{n+1}\ge\dfrac{n}{n+x}
\label{finalg}
\end{equation}
Recollons les morceaux. Soit $n\ge \eta$. Par~\eqref{defun} et~\eqref{finalg} nous avons $u_{n+1}(x)\ge f_n(x)^{n+1}\dfrac{n}{n+x}=f_n(x)^n\left(1+\dfrac{x}{n}\right)\left(\dfrac{n}{n+x}\right)=f_n(x)^n\left(\dfrac{x+n}{n}\right)\left(\dfrac{n}{n+x}\right)=f_n(x)^n=u_n(x)$. \\
\\
Conclusion, pour tout $n\ge \eta$ on a $u_{n+1}(x)\ge u_n(x)$ donc $(u_n)$ est croissante à partir d'un certain rang (ouf !).

\subsection{\texorpdfstring{Décroissance de $(v_n)$}{Décroissance de (v\_n)}}
Soit $n\ge \eta$.
On a montré que $f_n(x)>0$ (en prouvant l'existence de $g_n(x)$). Or $u_n(x)=f_n(x)^n$ donc $u_n(x)> 0$. \\
On a $u_{n+1}(-x)\ge u_{n}(-x)$. Puisque $u_n(-x)> 0$, on a le droit de passer à l'inverse et alors $\dfrac{1}{u_{n+1}(-x)}\le \dfrac{1}{u_{n}(-x)}$.\\
\\
Or nous avons également $v_n(x)=\dfrac{1}{u_n(-x)}$ (cela se montre directement). \\
\\
Donc pour tout $n\ge\eta$, $v_{n+1}(x)\le v_{n}(x)$ : $(v_n)$ est finalement décroissante à partir d'un certain rang.

\subsection{\texorpdfstring{Limite de la différence}{Limite de la différence}}

Soit $n\ge\eta$. Nous rappelons que $u_n(x)=\left(1+\dfrac{x}{n}\right)^n$ et $v_n(x)=\left(1-\dfrac{x}{n}\right)^{-n}$.  \\
\\
On montre par calcul que
\begin{equation}
\forall n\ge\eta,v_n(x)-u_n(x)=v_n(x)\left(1-\left[1-\left(\dfrac{x}{n}\right)^2\right]^n\right)
\label{uvdiff}
\end{equation}
En appliquant l'inégalité de Bernoulli (dont on justifiera pourquoi on a le droit de l'utiliser), on a $\left[1-\left(\dfrac{x}{n}\right)^2\right]^n\ge 1-\dfrac{x^2}{n}$. Puis, par inégalités successives jusqu'à retrouver~\eqref{uvdiff}, on trouve $v_n(x)-u_n(x)\le v_n(x)\dfrac{x^2}{n}$.\\
\\
$(v_n)$ étant décroissante à partir d'un certain rang, elle est majorée : il existe $M(x)\in\R$ (qui dépend de $x$ mais \textit{pas} de $n$, c'est tout l'intérêt) tel que pour tout $n\ge\eta$, $v_n(x)\le M(x)$.\\
Ainsi, $v_n(x)-u_n(x)\le M(x)\dfrac{x^2}{n}$. On a $\lim\limits_{n\to\infty} v_n(x)-u_n(x)\le \lim\limits_{n\to\infty} M(x)\dfrac{x^2}{n}=0$. \\
Donc $\lim\limits_{n\to\infty} v_n(x)-u_n(x)\le 0$. \\

De plus, depuis \eqref{uvdiff}, sachant que $v_n\ge 0$ et que pour tout $x\in]0,1[$, $n\in\N^*$, $x^n\in]0,1[$, on montre facilement que $v_n-u_n\ge 0$. \\
Par conséquent $\lim\limits_{n\to\infty} v_n(x)-u_n(x)\ge 0$ \\
\\
Finalement $0\le \lim\limits_{n\to\infty} v_n(x)-u_n(x)\le 0$ donc $\lim\limits_{n\to\infty} v_n(x)-u_n(x)= 0$.

\subsection{\texorpdfstring{Construction de $\exp$}{Construction de exp}}
Nous avons montré que $(u_n)$ et $(v_n)$ sont respectivement croissante et décroissante à partir d'un certain rang, et que $\lim\limits_{n\to\infty} v_n(x)-u_n(x)=0$. \\
$(u_n)$ et $(v_n)$ sont donc adjacentes et ainsi ces deux suites convergent vers la même limite. \\
A titre d'exemple, voici une représentation graphique des premières valeurs de $(u_n(2))$ et $(v_n(2))$ en prenant $\eta=3$. Nous avons fait aussi apparaître la droite d'équation $y=l$ où $l$ est la limite des deux suites.

\includegraphics[scale=0.4]{figures/pdf/adjaexpo-eps-converted-to.pdf}

Nous pouvons par conséquent légitimement définir la fonction $\exp:\R\to\R,x\mapsto \lim\limits_{n\to +\infty} u_n(x)$ (nous aurions également pu le faire avec $(v_n)$ mais l'utilisation de $(u_n)$ est plus standard).

\section{Vérification des propriétés de la fonction construite}
Il faut à présent vérifier que la fonction que nous venons de construire vérifie les propriétés attendues.

\subsection{\texorpdfstring{$\exp(0)=1$}{exp(0)=1}}
Pour tout $n\ge\eta$, $u_n(0)=1^n=1$ donc $\exp(0)=\lim\limits_{n\to\infty} u_n(0) = \lim\limits_{n\to\infty} 1=1$.

\subsection{Sa dérivée égale elle-même}
Soit $x\in\R$. Nous souhaitons montrer que $\Lim{h\to 0}\dfrac{\exp(x+h)-\exp(x)}{h}$ existe et égale $\exp(x)$.\\
\\
Soit $h\in\R$ tel que $|h|<1$. Montrons :
\begin{equation}
\exp(x+h)\ge \exp(x)(1+h)
\label{expxph}
\end{equation}
Posons $n\ge\eta$ tel que $n>1-x$. Soit $f_n(x,h)=\lp 1+\dfrac{x+h}{n}\rp$ et $g_n(x,h)=\lp 1+\dfrac{\dfrac{h}{n}}{1+\dfrac{x}{n}} \rp$. Notons qu'on a $\exp(x+h)=\Lim{n\to+\infty} f_n(x,h)^n$.\\
\\
On montre par simple calcul que :
\begin{equation}
f_n(x,h)^n=u_n(x)g_n(x,h)^n
\label{fug}
\end{equation}
Montrons :
\begin{equation}
f_n(x,h)^n\ge u_n(x)\lp 1+\dfrac{h}{1+\dfrac{x}{n}} \rp
\label{ifug}
\end{equation}
On a $n+x>1$ donc par inégalités successives on montre que $0<\dfrac{1}{1+\dfrac{x}{n}}<n$ et donc que $\la\dfrac{1}{1+\dfrac{x}{n}}\ra<n$. De plus, $|h|<1$ donc $\la \dfrac{h}{n}\ra<\dfrac{1}{n}$. En multipliant ces deux inégalités on a $\la\dfrac{\dfrac{h}{n}}{1+\dfrac{x}{n}}\ra<1$ et finalement $\dfrac{\dfrac{h}{n}}{1+\dfrac{x}{n}}>-1$.\\
\\
Cette dernière inégalité nous autorise à appliquer l'inégalité de Bernoulli à $g_n(x,h)^n$ et nous obtenons par conséquent $g_n(x,h)^n\ge 1+\dfrac{h}{1+\dfrac{x}{n}}$. En injectant ce dernier résultat à~\ref{fug} nous obtenons bien~\ref{ifug}.\\
\\
Enfin, en faisant tendre $n$ vers l'infini dans~\ref{ifug} nous obtenons exactement~\ref{expxph}.\\
\\
Nous avons bâti notre raisonnement en considérant $|h|<1$. Par conséquent, il reste valide en remplaçant $h$ par $-h$ et ainsi nous obtenons également $\exp(x-h)\ge \exp(x)(1-h)$. \\
\\
Soit $x'=x+h$. Par l'inégalité précédente nous avons $\exp(x'-h)\ge \exp(x')(1-h)$ donc $\exp(x)\ge\exp(x+h)(1-h)$ donc $\exp(x+h)\le \dfrac{\exp(x)}{1-h}$ ($|h|<1$ donc $1-h>0$). En combinant cette inégalité avec~\ref{expxph}, en soustrayant par $\exp(x)$ et enfin en divisant par $h$ on obtient :
\begin{equation}
\exp(x)\le\dfrac{\exp(x+h)-\exp(x)}{h}\le \dfrac{\exp(x)}{1-h}
\end{equation}
Finalement, en faisant tendre $h$ vers $0$ dans cette inégalité, on obtient $\Lim{h\to 0}\dfrac{\exp(x+h)-\exp(x)}{h}=\exp(x)$, ce qu'il fallait démontrer.

\section{Conclusion}
Nous avons construit une fonction $\exp:\R\to\R$ dont nous avons prouvé que $\exp(0)=1$, dérivable sur $\R$ et telle que pour tout $x\in\R,\exp'(x)=\exp(x)$. \\ \\
Ainsi nous avons bien prouvé l'existence d'une fonction $\exp$ qui vérifie toutes les propriétés voulues.
\begin{flushright}
$\square$
\end{flushright}

\end{appendix}

\end{large}
\end{document}